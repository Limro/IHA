%!TEX root = Main.tex
\documentclass[Main]{subfiles}

\begin{document}

\section*{What is pervasive computing?} 
\vspace{-15pt}
The idea of "pervasive computing" is, that the technology works for you, instead of you working on the technology.
This means, that the technology will record what happens around it, process the data (in some way) and react on it.

By pervasive computing it is usually small processors and/or sensors which registers the surroundings.
The sensors can be created for location awareness, context awareness, activity recognition.
The processor then either sends the data to a bigger processor through a network, or activate some mechanism based on the data collected.
This can all done without a user being aware of it and can be considered invisible. 
But this does not mean, that a person can't use it. 




\section*{Which "enabling" technologies does it involve?} 
\vspace{-15pt}
Through enabling technology the user can make the surrounding obey the user's will and interact with computers, sensors, lights and other equipment.
Take todays mobile phone / smart phone: 
When the user activates the screen, the display is automatically adjusted for the optimal brightness through the camera.
When the user likes to see the weather, the GPS/mobile network's location have you approximate location and can download the latest weather updates before you can click the button.
When you are reading on the phone the camera can detect your eyes still being on the display and keep the display alive as long as you look.




\section*{What is your personal experience with pervasive computing projects?}
\vspace{-15pt}
During my bachelor project, me and my partner enhanced a quadro copter with 3 sonar sensors.
The user had a remote control to command the drone in all directions, but should just one of the three sensors pick up an object within a certain range, all commands would be stopped and the drone would fly away from the object, allowing the user to command it again.









\end{document}