\documentclass[oneside, 10pt]{article}

\usepackage{tcolorbox}
\usepackage{ulem} %math
\usepackage{amsmath}
\usepackage{amsfonts}
\usepackage{amssymb}
\usepackage{graphicx}
\usepackage{enumerate}


%Create a box for theorems
%\begin{theo}[titel] %optional
%tekst
%\end{theo}
\newenvironment{theo}[1][Vigtigt]{%
\begin{tcolorbox}[colback=green!5,colframe=green!40!black,title=\textbf{#1}]
}{%
\end{tcolorbox}
}




%Create a square matrix
%\begin{ArgMat}{2}
%21 & 22 & 23 \\  
%a & b & c
%\end{ArgMat}
%
% Info: http://tex.stackexchange.com/questions/2233/whats-the-best-way-make-an-augmented-coefficient-matrix
%
\newenvironment{ArgMat}{%
$
  \left[\begin{array}{@{}*{100}{r}r@{}}
}{%
  \end{array}\right]
  $
}

\newenvironment{deter}{%
$
  \left|\begin{array}{@{}*{100}{r}r@{}}
}{%
  \end{array}\right|
  $
}


%Create multiple lines with holes
%\begin{SysEqu}
%x_1 && &- &5x_3 &+ &2x_4=& 1 \\
%x_1 &+ &x_2 &+ &x_3 && =& 4 \\
%&&&&&&0 =& 0
%\end{SysEqu}
\newenvironment{SysEqu}{%
$  \setlength\arraycolsep{0.1em}
  \begin{array}{@{}*{100}{r}r@{}}
}{%
  \end{array}$
}

%Create solution for x_1, x_n...
%\begin{solu}
%x_1 &= d \\
%x_2 &= e \\
%x_3 &= s
%\end{solu}
\newenvironment{solu}{%
$
  \setlength\arraycolsep{0.1em}
  \left\{\begin{array}{@{}*{100}{r}r@{}}
}{%
  \end{array}\right.
$
}

\usepackage{lastpage}


\newcommand{\HRule}{\rule{\linewidth}{0.8mm}}

%Tekst i fotter
\newcommand{\footerText}{\thepage\xspace /\pageref{LastPage}}
\newcommand{\ProjectName}{433 MHz styring af AeroQuad}


\chapterstyle{hangnum}




\nouppercaseheads
\makepagestyle{mystyle} 

\makeevenhead{mystyle}{}{\\ \leftmark}{} 
\makeoddhead{mystyle}{}{\\ \leftmark}{} 
\makeevenfoot{mystyle}{}{\footerText}{} 
\makeoddfoot{mystyle}{}{\footerText}{} 
\makeatletter
\makepsmarks{mystyle}{% Overskriften på sidehovedet
  \createmark{chapter}{left}{shownumber}{\@chapapp\ }{.\ }} 
\makeatother
\makefootrule{mystyle}{\textwidth}{\normalrulethickness}{0.4pt}
\makeheadrule{mystyle}{\textwidth}{\normalrulethickness}

\makepagestyle{plain}
\makeevenhead{plain}{}{}{}
\makeoddhead{plain}{}{}{}
\makeevenfoot{plain}{}{\footerText}{}
\makeoddfoot{plain}{}{\footerText}{}
\makefootrule{plain}{\textwidth}{\normalrulethickness}{0.4pt}

\pagestyle{mystyle}

%%----------------------------------------------------------------------
%
%%Redefining chapter style
%%\renewcommand\chapterheadstart{\vspace*{\beforechapskip}}
%\renewcommand\chapterheadstart{\vspace*{10pt}}
%\renewcommand\printchaptername{\chapnamefont }%\@chapapp}
%\renewcommand\chapternamenum{\space}
%\renewcommand\printchapternum{\chapnumfont \thechapter}
%\renewcommand\afterchapternum{\space: }%\par\nobreak\vskip \midchapskip}
%\renewcommand\printchapternonum{}
%\renewcommand\printchaptertitle[1]{\chaptitlefont #1}
\setlength{\beforechapskip}{0pt} 
\setlength{\afterchapskip}{0pt} 
%\setlength{\voffset}{0pt} 
\setlength{\headsep}{25pt}
%\setlength{\topmargin}{35pt}
%%\setlength{\headheight}{102pt}
%\setlength{\textheight}{302pt}
\renewcommand\afterchaptertitle{\par\nobreak\vskip \afterchapskip}
%%----------------------------------------------------------------------




%Sidehoved og -fod pakke
%Margin
\usepackage[left=2cm,right=2cm,top=2.5cm,bottom=2cm]{geometry}
\usepackage{lastpage}



%%URL kommandoer og sidetal farve
%%Kaldes med \url{www...}
%\usepackage{color} %Skal også bruges
\usepackage{hyperref}
\hypersetup{ 
	colorlinks	= true, 	% false: boxed links; true: colored links
    urlcolor	= blue,		% color of external links
    linkcolor	= black, 	% color of page numbers
    citecolor	= blue,
}



%Mellemrum mellem linjerne    
\linespread{1.5}


%Seperated files
%--------------------------------------------------
%Opret filer således:
%\documentclass[Navn-på-hovedfil]{subfiles}
%\begin{document}
% Indmad
%\end{document}
%
% I hovedfil inkluderes således:
% \subfile{navn-på-subfil}
%--------------------------------------------------
\usepackage{subfiles}

%Prevent wierd placement of figures
%\usepackage[section]{placeins}

%Standard sti at søge efter billeder
%--------------------------------------------------
%\begin{figure}[hbtp]
%\centering
%\includegraphics[scale=1]{filnavn-for-png}
%\caption{Titel}
%\label{fig:referenceNavn}
%\end{figure}
%--------------------------------------------------
\usepackage{graphicx}
\usepackage{subcaption}
\usepackage{float}
\graphicspath{{../Figures/}}

%Speciel skrift for enkelt linje kode
%--------------------------------------------------
%Udskriver med fonten 'Courier'
%Mere info her: http://tex.stackexchange.com/questions/25249/how-do-i-use-a-particular-font-for-a-small-section-of-text-in-my-document
%Eksempel: Funktionen \code{void Hello()} giver et output
%--------------------------------------------------
\newcommand{\code}[1]{{\fontfamily{pcr}\selectfont #1}}


% Følgende er til koder.
%----------------------------------------------------------
%\begin{lstlisting}[caption=Overskrift på boks, style=Code-C++, label=lst:referenceLabel]
%public void hello(){}
%\end{lstlisting}
%----------------------------------------------------------

%Exstra space
\usepackage{xspace}
%Navn på bokse efterfulgt af \xspace (hvis det skal være mellemrum
%gives det med denne udvidelse. Ellers ingen mellemrum.
\newcommand{\codeTitle}{Kodeudsnit\xspace}

%Pakker der skal bruges til lstlisting
\usepackage{listings}
\usepackage{color}
\usepackage{textcomp}
\definecolor{listinggray}{gray}{0.9}
\definecolor{lbcolor}{rgb}{0.9,0.9,0.9}
\renewcommand{\lstlistingname}{\codeTitle}
\lstdefinestyle{Code}
{
	keywordstyle	= \bfseries\ttfamily\color[rgb]{0,0,1},
	identifierstyle	= \ttfamily,
	commentstyle	= \color[rgb]{0.133,0.545,0.133},
	stringstyle		= \ttfamily\color[rgb]{0.627,0.126,0.941},
	showstringspaces= false,
	basicstyle		= \small,
	numberstyle		= \footnotesize,
%	numbers			= left, % Tal? Udkommenter hvis ikke
	stepnumber		= 2,
	numbersep		= 6pt,
	tabsize			= 2,
	breaklines		= true,
	prebreak 		= \raisebox{0ex}[0ex][0ex]{\ensuremath{\hookleftarrow}},
	breakatwhitespace= false,
%	aboveskip		= {1.5\baselineskip},
  	columns			= fixed,
  	upquote			= true,
  	extendedchars	= true,
 	backgroundcolor = \color{lbcolor},
	lineskip		= 1pt,
%	xleftmargin		= 17pt,
%	framexleftmargin= 17pt,
	framexrightmargin	= 0pt, %6pt
%	framexbottommargin	= 4pt,
}

%Bredde der bruges til indryk
%Den skal være 6 pt mindre
\usepackage{calc}
\newlength{\mywidth}
\setlength{\mywidth}{\textwidth-6pt}


% Forskellige styles for forskellige kodetyper
\usepackage{caption}
\DeclareCaptionFont{white}{\color{white}}
\DeclareCaptionFormat{listing}%
{\colorbox[cmyk]{0.43, 0.35, 0.35,0.35}{\parbox{\mywidth}{\hspace{5pt}#1#2#3}}}
\captionsetup[lstlisting]
{
	format			= listing,
	labelfont		= white,
	textfont		= white, 
	singlelinecheck	= false, 
	width			= \mywidth,
	margin			= 0pt, 
	font			= {bf,footnotesize}
}

\lstdefinestyle{Code-C} {language=C, style=Code}
\lstdefinestyle{Code-Java} {language=Java, style=Code}
\lstdefinestyle{Code-C++} {language=[Visual]C++, style=Code}
\lstdefinestyle{Code-VHDL} {language=VHDL, style=Code}
\lstdefinestyle{Code-Bash} {language=Bash, style=Code}

%Text typesetting
%--------------------------------------------------------
%\usepackage{baskervald}
\usepackage{lmodern}
\usepackage[T1]{fontenc}              
\usepackage[utf8]{inputenc}         
\usepackage[english]{babel}       

\setlength{\parindent}{0pt}
\nonzeroparskip

%\setaftersubsecskip{1sp}
%\setaftersubsubsecskip{1sp}
 


%Dybde på indholdsfortegnelse
%----------------------------------------------------------
%Chapter, section, subsection, subsubsection
%----------------------------------------------------------
\setcounter{secnumdepth}{3}
\setcounter{tocdepth}{3}


%Tables
%----------------------------------------------------------
\usepackage{tabularx}
\usepackage{array}
\usepackage{multirow} 
\usepackage{multicol} 
\usepackage{booktabs}
\usepackage{wrapfig}
\renewcommand{\arraystretch}{1.5}



%Misc
%----------------------------------------------------------
\usepackage{cite}
\usepackage{appendix}
\usepackage{amssymb}
\usepackage{url,ragged2e}
\usepackage{enumerate}
\usepackage{amsmath} %Math bibliotek


\usepackage{longtable}


\title{Advanced pervasive computing}
\author{Rasmus Bækgaard}
\date{\today}

\begin{document}

\maketitle

Generelt:
\begin{itemize}
	\item Gør maskiner så intelligente, at de kan gøre ting for os
	\item Hvis man har faldedetector kan den ikke selv afgøre om det er et fald (false positive), men flere enheder kan forstå context og undgå falske alarmer.
	\item Skal køre automatisk - vi vil ikke hele tiden spørge sensorer.
	\item Man skal ikke bruge lang tid på opsætning -- det skal være out-of-the-box og virkende.
	\item Hvordan deployer man smart en ny sensor? Helst uden konfiguration.
\end{itemize}

\section{Distributed Context Awareness}

Context:
\begin{itemize}
	\item Man tænker altid på noget i en hvis context (det kan man træne sig fra).
	\item Vigtig for den måde mennesker opfatter ting -- bør også være for maskiner.
	\item Maskiner kan få context gennem sensorer (temperatur, lyd, lys mm.)
	\item En "feature" er eks. tid (noget man får fra context) eller en masse data, samlet og kun gældende for et bestemt øjeblik.
	\item De 5 'w' -- What, What, Where, When, Why \textbf{Vigtig for eksamen}.
	\item Opdel i katagorier
	\begin{itemize}
		\item Primær og sekundær
		\item Location, Identity, Time og Activity
	\end{itemize}
	\item Alle enheder skal kunne sende sine fata til ét sted, og her kan data'en blive fortolket. Alle enheder har et stadie de er i. Man kan bruge resoning engine til at afgøre, hvad det alt sammen betyder.
\end{itemize}

Context Awareness: 
\begin{itemize}
	\item Der skal helst være 3 features: \textbf{Vigtig!}
	\begin{itemize}
		\item Præsenter information og service til brugeren
		\item Eksikver automatisk en service
		\item Tag context af information til senere brug!
	\end{itemize}
	\item To kategorier:
	\begin{itemize}
		\item Infrastruktur context awareness (eks. smart phone -- kan være besværlige).
		\item Self-contained (eks. smart phone app -- kan bruge features på en infrastruktur).
	\end{itemize}
	\item Context tagging:
	\begin{itemize}
		\item Hvis man tager blodtryk -- bevægede man sig og hvor var man mm.
	\end{itemize}
\end{itemize}

Context Models:
\begin{itemize}
	\item Man laver modeller for at forstå det, validere at det vil virker, finde strukturen mm.
	\item Attributes:
	\begin{itemize}
		\item Kan være et ID,
		\item En type og en værdi
		\item En sandsynlighed -- hvad kunne man lave?
	\end{itemize}
	\item En projekter kunne være af typen "projektor", type = smart projekter, ID = 2 og har en URL (værdi) = \dots
	\item Object oriented models:  meget tæt på det kode man normalt skriver
	\item Logic based: Alt skrive med logik og matematik (meget ITTT)
	\item Ontology based: Noget fra filosofi om at kategorisere (alt der kan beskrives som en entity) -- meget tæt på domain modeller, men kan også være meget specifikke (eks. hvordan implementeres noget).
	\item \textbf{Brug ikke AMML med ASET!} -- lav god kode, en webservice og en domain model istedet.

	\item Context reasoning og interpretation
	\begin{itemize}
		\item Brug TCP/IP stack for flere enheder der skal kunne kommunikere.
		\item Hvad betyder det, når \dots
		\item Man kan ikke reagere på ét event. Men en timeline af event vil betyde noget. \textbf{Medbring figurer fra slide (pie chart og stock chart)}
	\end{itemize}
	\begin{itemize}
		\item Lifecycle
	\end{itemize}
\end{itemize}

Enabling tech:
\begin{itemize}
	\item Z-Wave, ZigBee, Bluetooth, ENOCEAN
\end{itemize}





\newpage
\section{Context Frameworks}


Udfordringer:
\begin{itemize}
	\item Primært Java -- omskriv til Open Source .NET 
	\item De kan være meget komplekse og derfor sværre at bruge
	\item Sikkerhed er der ikke meget af og de er ikke heteorgene (er skrevet i andre sprog)
	\item Du skal selv skrive context for egne sensorer (hvilket tager lang tid). An
	\item Begrænset dokumentation og tutorials, da folk laver det til sig selv og kun til lokalnetværk (ikke internettet).
\end{itemize}

Context toolkit:
\begin{itemize}
	\item Widget er et interface, som en app kan tilgå.
	\item Interpreters er det, der kan omsætte informationen til noget brugbart/læsbart.
	\item Aggregator er det, der samler oplysninger og smider det vidre til apps
	\item Service er det, der hele tiden kører og kan holde ting opdateret
	\item Discover er det, der er rundt omrking, hvad de kan tilbyde og hvordan de bruges.
	\item \textbf{Tag slide med in om dette.}
\end{itemize}

Java context-awareness Framework
\begin{itemize}
	\item De forskillige \code{GenericFoo} kan arve flere forskellige, konkrete klasser.
	\item Kombinationer af dem alle gør, at man kan udføre en handling (åben dør for sygeplejeske til ældrehjem).
	\item \textbf{Medbring slide med JCAF}
\end{itemize}





\newpage
\section{Intelligent Environment og UbiComp UIs}

Intelligent environments:
\begin{itemize}
	\item Så småt en realitet
	\item De kan være (næsten usynlige), men behøver blot have et smartere interface
\end{itemize}

Shapes and sizes:
\begin{itemize}
	\item Leap (der registrere dine bevægelser), 
	\item SmartWatch 
	\item Microsoft Surface
\end{itemize}

UbiComp UI:
\begin{itemize}
	\item Tangible UI -- et bord der kan f.eks. se hvad der ligger på bordet og bruge de ting.
	\item En overflade, der bliver til input/output display (projekteret billede).
	\item Ambient  UI -- Lys der blot blinker der en besked eller retter det en anden vej.
	\item Context-aware UI -- Reagerer på, hvad der sker (eller ikke sker)
	\item Artificial Reality interface -- Google glass eller gennemsigtig skærm, som kan vise information om det du ser.
\end{itemize}




\newpage
\section{Intelligent Environments and Activity Classification}

\begin{itemize}
	\item Grundet for lidt funding, er der ikke blevet gjort nok på dette område
	\item Det kræver meget udstyr
	\item Specielt til plejehjem
\end{itemize}


Smart Homes:
\begin{itemize}
	\item Man skal kunne bo uafhængig af andre
	\item For yngre mennesker ($\leq 50$ år) er dette konfortable problemer (patienner, vaskemaskine, varmeforbrug).
	\item Se L4S6 for forskellige typer.
\end{itemize}


Smart Spaces:
\begin{itemize}
	\item Når bordet eller muren bliver til en skærm
\end{itemize}


User interfaces:
\begin{itemize}
	\item 
\end{itemize}


AAL Smart Homes:
\begin{itemize}
	\item 
\end{itemize}


Activity of Daily Living, ADL:
\begin{itemize}
	\item Det vi gør hver dag for os selv (spiser, sover, tage medicin, arbejder, madlavning\dots)
	\item Dette er vigtigt, da variation i dette kan være tegn på sygdom af forskellig art.
	\item Ældre mennesker husker ikke så godt og vil også få svært ved at tage sig af sig selv. Dette kan normalt registreres med hvor meget man sover (hvilket kun kræver én sensor).
	\item ADL skalaen, L4S11, beskriver hvor meget hjælp man skal have.
	\item ADL monitoring, L4S12, er de værktøjer der kan bruges. L4S13 beskriver de FW der kan bruges
	\begin{itemize}
		\item SPINE, Signal Processing in Node Environment:
		\item Sensorer man kan tage på (wearable)
		\item Bruger bluetooth eller 820.15.4
		\item Der findes en central koordinator
		\item Lavet i Java, men skulle nok kunne omskrives
		\item Ny firmware kan blive uploaded, mens man har det aktivt
		\item "In-node signal processing" giver meget lang batterilevetid.
		\item Kan også bruges til at spore grises færd (hvornår de skal føde).
	\end{itemize}
\end{itemize}


IE Challenges:
\begin{itemize}
	\item Det er en stor opgave, at træne hvert system til personligt brug i et hjem.
	\item Batteri er også meget krævende, da de skal skiftes relativt ofte
	\item Sensorer bliver mindre og billigere -- dette kan gøre det nemmere at lave denne teknologi. Nok standard i alle nye bygninger om 10-15 år.
	\item Deployment er nok det største problem, da man skal vide, hvad er det man forbinder.
	\item Sikkerhed er også et problem, da WPS kan tillade konstant-søgende systemer i ens system.
	\item Hvis man ønsker at sælge ting i HW-stores skal det være meget nemt deployable
\end{itemize}

Et system:
\begin{itemize}
	\item Context Client: App'en, der bruger sensorerne (central system)
	\item Context Sensors: De sensorer der bruges (med' box og PIR).
	\item Context Service: Det, der forbinder sensorer til clienten og beregner input
	\item Hvis man så vil tilføje et Fall detetion, så hører Bed- og bathroom sensor måske også til hos en ADL klassificering. Og det skal også styres med Light Control.
\end{itemize}

Til eksamen:
\begin{itemize}
	\item Skriv også, hvad der er på forbindelserne til et system (UML-diagram)
	\begin{itemize}
		\item Der skal være low-power system
	\end{itemize}
	\item Bring assignment 2 med ind, da den vil forklare en masse ting (service discovery, sensors osv)
	\item 
\end{itemize}











\newpage
\section{Zero configuration and service discovery}

L5S4:
\begin{itemize}
	\item Hvordan undgår man opsætning (eller tæt på?) -- zero configuration.
	\item Find automatisk hardware-enheder: router, printer, telefoner mm.
	\item Vi skal have autodiscovery.
	\item Også uden DHCP server (IPv4 Link-Local Addressing)
	\item I stedet - brug Multicast DNS.
	\begin{itemize}
		\item Send data diagram ud, de lever en hvis tid, men det virker i lokale netværk (mellem to maskiner)
		\item Det er rigtig dårligt for internet\dots
	\end{itemize}

	\item DNS Server Discovery (DNSSD) skal kunne bruges
\end{itemize}

Standardization efforts:
\begin{itemize}
	\item Alle kan forbinde til en statisk addresse, hvis ikke DHCP-serveren virker.
	\item Man kan bruge Link-Local uden DHCP (spørg om der er nogen der har en adresse -- hvis ikke er det nu din). Det giver en addresse i stil med \code{169.254.0.0/16}.

	\item UPNP tager ikke så mange porte? som Bonjour.
\end{itemize}

UPNP:
\begin{itemize}
	\item Med en webservice har du mulighed for at lave kontrol af en server (musik mm)
\end{itemize}

Bonjour:
\begin{itemize}
	\item Kræver en DNSDS daemon
	\item Implementer kun, hvis det skal bruges.
\end{itemize}

Addressering og naming:
\begin{itemize}
	\item I DNS tjekker man pr. punktum. Hvis der ikke er noget netværk findes der noget der hedder "local".
	\item Dette er link-local.
	\item Bonjour's daemon snakker gennem Multicast DNS Queries. (mDNSResponder)
	\item Naming følger S5L9
	\item Kommunikation er standard TCP/IP til start
	\item Herefter bruges daemon til kommunikation.
	\item DNS responder vil forblive det samme, da den kan få en anden IP. IP gemmes also ikke, men DNS responder's navn.
	\begin{itemize}
		\item Eksempel på L5S17-20
		\item Få navn
		\item Få service instance
		\item Lav et DNS lookup (ingen DHCP, lav MultiCast)
		\item Send multicast request og modtag IP
	\end{itemize}

	\item En direkte forbindelse med kabel giver en local address.
\end{itemize}







\newpage
\section{Location based services}

\begin{itemize}
	\item Findes i mobiltelefoner
	\item "Valuable assets" --  biler, fly, cykler mm.
	\item Privacy er SUPER VIGTIGT!
	\item Forskellig præcision til forskellige apps
	\begin{itemize}
		\item Biler -- meget
		\item Generel placering -- ikke så meget
	\end{itemize}

	\item Langt de fleste apps bruger koordinater. Det gør den i lag -- L5S8.
	\item Hvis man kender noise, kan de filtreres væk.
\end{itemize}

Moving average filter:
\begin{itemize}
	\item Pludselige forandringer kan fjernes og holde den underlæggende trend.
	\item Det mest besværlige er at berenge gennemsnittet.
	\item Window Size -- hvor mange smaples skal der bruges?
	\item Hvis man forsøger sig med 2D-planlægning, skal man bruge et 2-dimentionelt data array. Man skal blot holde styr på flere dimensioner her.
	\item Trådløse signal modtages i dBm -- L5S13.
\end{itemize}

Moving median filtering position data:
\begin{itemize}
	\item Medianen (tal i midten efter sortering) vægter ikke tal. Derfor vil gennemsnit af $[1, 3, 100 ]$ og medianen være meget forskellig
\end{itemize}














\end{document}