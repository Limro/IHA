\documentclass[danish, english]{article}
%Preamble

% Følgende er til koder.
%----------------------------------------------------------
%\begin{lstlisting}[caption=Overskrift på boks, style=Code-C++, label=lst:referenceLabel]
%public void hello(){}
%\end{lstlisting}
%----------------------------------------------------------

%Exstra space
\usepackage{xspace}
%Navn på bokse efterfulgt af \xspace (hvis det skal være mellemrum
%gives det med denne udvidelse. Ellers ingen mellemrum.
\newcommand{\codeTitle}{Code snippet\xspace}

%Pakker der skal bruges til lstlisting
\usepackage{listings}
\usepackage{color}
\usepackage{textcomp}
\definecolor{listinggray}{gray}{0.9}
\definecolor{lbcolor}{rgb}{0.9,0.9,0.9}
\renewcommand{\lstlistingname}{\codeTitle}
\lstdefinestyle{Code}
{
	keywordstyle	= \bfseries\ttfamily\color[rgb]{0,0,1},
	identifierstyle	= \ttfamily,
	commentstyle	= \color[rgb]{0.133,0.545,0.133},
	stringstyle		= \ttfamily\color[rgb]{0.627,0.126,0.941},
	showstringspaces= false,
	basicstyle		= \small,
	numberstyle		= \footnotesize,
%	numbers			= left, % Tal? Udkommenter hvis ikke
	stepnumber		= 2,
	numbersep		= 6pt,
	tabsize			= 2,
	breaklines		= true,
	prebreak 		= \raisebox{0ex}[0ex][0ex]{\ensuremath{\hookleftarrow}},
	breakatwhitespace= false,
%	aboveskip		= {1.5\baselineskip},
  	columns			= fixed,
  	upquote			= true,
  	extendedchars	= true,
 	backgroundcolor = \color{lbcolor},
	lineskip		= 1pt,
%	xleftmargin		= 17pt,
%	framexleftmargin= 17pt,
	framexrightmargin	= 0pt, %6pt
%	framexbottommargin	= 4pt,
}

%Bredde der bruges til indryk
%Den skal være 6 pt mindre
\usepackage{calc}
\newlength{\mywidth}
\setlength{\mywidth}{1.435\textwidth} % Hvis bredden header ikke virker er dette hvad skal ændres!


% Forskellige styles for forskellige kodetyper
\usepackage{caption}
\DeclareCaptionFont{white}{\color{white}}
\DeclareCaptionFormat{listing}%
{\colorbox[cmyk]{0.43, 0.35, 0.35,0.35}{\parbox{\mywidth}{\hspace{5pt}#1#2#3}}}
\captionsetup[lstlisting]
{
	format			= listing,
	labelfont		= white,
	textfont		= white, 
	singlelinecheck	= false, 
	width			= \mywidth,
	margin			= 0pt, 
	font			= {bf,footnotesize}
}

\lstdefinestyle{Code-C} {language=C, style=Code}
\lstdefinestyle{Code-Java} {language=Java, style=Code}
\lstdefinestyle{Code-C++} {language=[Visual]C++, style=Code}
\lstdefinestyle{Code-VHDL} {language=VHDL, style=Code}
\lstdefinestyle{Code-Bash} {language=Bash, style=Code}
\lstdefinestyle{Code-Matlab} {language=Matlab, style=Code}
\lstdefinestyle{Code-Prolog} {language=Prolog, style=Code}
%Speciel skrift for enkelt linje kode
%--------------------------------------------------
%Udskriver med fonten 'Courier'
%Mere info her: http://tex.stackexchange.com/questions/25249/how-do-i-use-a-particular-font-for-a-small-section-of-text-in-my-document
%Eksempel: Funktionen \code{void Hello()} giver et output
%--------------------------------------------------
\newcommand{\code}[1]{{\fontfamily{pcr}\selectfont #1}}

%Seperated files
%--------------------------------------------------
%Opret filer således:
%\documentclass[Navn-på-hovedfil]{subfiles}
%\begin{document}
% Indmad
%\end{document}
%
% I hovedfil inkluderes således:
% \subfile{navn-på-subfil}
%--------------------------------------------------
\usepackage{subfiles}
%Text typesetting
%--------------------------------------------------------
\usepackage[T1]{fontenc} 	% Can use danish characters
\usepackage[utf8]{inputenc} % Input encoding. Can be used on Linux, Mac and Windows         
\usepackage[danish]{babel} 	% Split words accoding to English
\usepackage{lmodern} 		% Font

\setlength\parindent{0pt} 	% No indent
\setlength\parskip{12pt} 	% More than a single line break will give ONE linebreak.

%Margin
\usepackage[left=2cm,right=2cm,top=2.5cm,bottom=2cm]{geometry}

%Margin
\usepackage[left=3cm,right=2cm,top=2.5cm,bottom=2cm]{geometry}

%Mellemrum mellem linjerne    
\linespread{1.5}

\title{Assignment 2 for TEDI}
\author{Rasmus Bækgaard, 10893}
\date{April 28, 2014}
\title{Lektion 1}
\begin{document}
\maketitle
\section{Linear equation}
\begin{theo}
A system of linear equations has
\begin{enumerate}
\item no solution, or
\item exactly one solution, or
\item infinitely many solutions.
\end{enumerate}
A system of linear equations is said to be \textbf{consistent} if it has either one solution or infinitely many solutions; a system is \textbf{inconsistent} if it has no solution.
Inconsistent er også, hvis der ikke er \textit{n} pivot'er.
\end{theo}

Generel udtryk:\\
$a_1x_1+a_2x_2+\dots+a_nx_n=b$
\\
\\
	\begin{solu}
	x_1-2x_2+2x_3 = 10\\
	2x_1  - 2x_2 +6x_3 =24 \\
	-x_1+2x_2-x_3 = -7
	\end{solu} = coifficient
		\begin{ArgMat}
		1 & -2 & 2 \\
		2&-2&6 \\
		-1 & 2 & -1
		\end{ArgMat} = argument
		\begin{ArgMat}
		1 & -2 & 2 & 10\\
		2&-2&6 & 24\\
		-1 & 2 & -1 &-7
		\end{ArgMat}		
\\
\\
		
$A\text{x} = \lambda \text{x}$
\\
\\

	\begin{itemize}
	\item Matices: Captial letters
	\item Vectors: bold or arrow
	\item scalars: standard letters
	\item m $\times$ n matrix: \textit{m} rows, \textit{n columns}
	\end{itemize}

\subsection*{1 løsning}
	\begin{ArgMat}
	1 & 0&0&\alpha \\
	0 & 1&0&\beta\\
	0&0&1& \gamma
	\end{ArgMat} =
		\begin{solu}
		x_1 = \alpha\\
		x_2 = \beta\\
		x_3 = \gamma
		\end{solu}

\subsection*{2. løsning}
	\begin{ArgMat}
	1 & 0&0&\alpha \\
	0 & 1&0&\beta\\
	0&0&0& \gamma
	\end{ArgMat} =
		\begin{solu}
		x_1 = \alpha\\
		x_2 = \beta\\
		x_3 = \gamma
		\end{solu}
		
$0x_1 + 0x_2+0x_3 = \gamma$ (Inconsistent -- false -- no solution)


\section{Elementary row operations}


	\begin{itemize}
	\item Replacement: $r_i \rightarrow r_i +  \gamma \cdot r_j$
	\item Swap: $r_i \leftrightarrow r_j$
	\item Scale: $r_i \rightarrow \gamma \cdot r_j$
	\end{itemize}


\begin{theo} 
Hvis man kan gå fra en matrix til en anden med ovenstående regler, er de \textbf{row equevalent}
\end{theo}

Eksempel:

	\begin{ArgMat}
	1 & 1\\
	2 &2
	\end{ArgMat} $\sim$
	\begin{ArgMat}
	1 & 1\\
	0 & 0
	\end{ArgMat}



	\begin{ArgMat}
	1 & -2 & 2& 10\\
	2 & -2 & 6 & 24\\
	-1 & 2 & -1 & -7
	\end{ArgMat} $r_3 \rightarrow r_3 + r_1 $
	\begin{ArgMat}
	1 & -2 & 2 6 & 10\\
	2 & -2 & 6 & 24\\
	0 & 0 &1&3
	\end{ArgMat} $r_2 \rightarrow r_2 - 2r_1$
	\begin{ArgMat}
	1&-2&2&10\\
	0&2&2&4\\
	0&0&1&3
	\end{ArgMat}

	$r_1 \rightarrow r_1 + r_2$
	\begin{ArgMat}
	1&0&4&14\\
	0&2&2&4\\
	0&0&1&3
	\end{ArgMat} $r_2 \rightarrow \frac{1}{2}r_2$
	\begin{ArgMat}
	1&0&4&14\\
	0&1&1&2\\
	0&0&1&3
	\end{ArgMat} $r_2 \rightarrow r_2-r_3$
	\begin{ArgMat}
	1&0&4&14\\
	0&1&0&-1\\
	0&0&11&3
	\end{ArgMat}

	$r_1 \rightarrow r_1 - 4r_3$
	\begin{ArgMat}
	1&0&0&2\\
	0&1&0&-1\\
	0&0&1&3
	\end{ArgMat}
	\begin{solu}
	x_1 = &2\\
	x_2 = &-1\\
	x_3 = &3
	\end{solu}
	

\section{Echelon form / row echelon form}

\begin{theo}[Definition] 
A rectangular matrix is in \textbf{echelon form} (or \textbf{row echelon form}) if it has the following three properties:
	\begin{enumerate}
	\item All nonzero rows are above any rows of all zeros.
	\item Each leading entry of a row is in a column to the right of the leading entry of the row above it.
	\item All entries in a column below a leading entry are zeros.

If a matrix in echelon form satisfies the following additional conditions, then it is in \textbf{reduced echelon form} (or \textbf{reduced row echelon form}):
	\item The leading entry is each nonzero row is 1.
	\item Each leading 1 is the only nonzero entry in its column.
	\end{enumerate}

\end{theo}

\begin{theo}[Theorem 1: Uniqueness of the Reduced Echelon Form] 
Each matrix is row equivalent to one and only one reduced echelon matrix.
\end{theo}
	
Eg:\\

A=	\begin{ArgMat}
	11&2\\
	3&4
	\end{ArgMat} =
	\begin{ArgMat}
	1&2\\
	0&-2
	\end{ArgMat} $-\frac{1}{2} = $
	\begin{ArgMat}
	1&2\\
	0&1
	\end{ArgMat} =	
	\begin{ArgMat}
	1&0\\
	0&1
	\end{ArgMat}





\newpage

\section{Pivot position}
\begin{theo}[Pivots betydning] 
Pivot betyder "nøgle" og udgør en nøglebetydning for en matrix.
\begin{enumerate}
\item Hver pivot fortæller "her starter noget der ikke blot kan skaleres".
\item Der er kun det antal dimensioner der er pivot'er.
\item En pivot's tal kaldes og \textbf{basic variables}. 
\item Hver kolonne i en \textit{echelon form} er en variabel ($x_1, x_2 \dots$ -- Har en kolonne ingen pivot er det en \textbf{free variable}
\item Hver kolonne angiver en \textit{x}-parameter
\end{enumerate}
\end{theo}


\begin{theo}[Theorem 2: Existence and Uniqueness Theorem] 
A linear system is consistent if and only if the rightmost column of the augmented matrix is not a pivot column—that is, if and only if an echelon form of the augmented matrix has no row of the form

\centerline{
\begin{ArgMat}
0 & \dots & 0 & b
\end{ArgMat} with b nonzero }

If a linear system is consistent, then the solution set contains either (i) a unique solution, when there are no free variables, or (ii) infinitely many solutions, when there is at least one free variable.
\end{theo}

example:\\
\begin{align*}
x_1 + 2x_2 - x_3 = 5\\
x_2-x_3 = 1\\
2x_1+3x_2-x_3 = 9
\end{align*}

	
		\begin{ArgMat}
		1&2&-1&5\\
		0&1&-1&1\\
		2&3&-1&9
		\end{ArgMat} $r_3 \rightarrow r_3-2r_1$
		\begin{ArgMat}
		1&2&-1&5\\
		0&1&-1&1\\
		0&-1&1&-1
		\end{ArgMat} 
		$r_3 \rightarrow r_3 + r_2$
		\begin{ArgMat}
		1&2&-1&5\\
		0&1&-1&1\\
		0&0&0&0
		\end{ArgMat} 
		
		$r_1 \rightarrow r_1-2r_2$
		\begin{ArgMat}
		1&0&1&3\\
		0&1&-1&1\\
		0&0&0&0
		\end{ArgMat}
\\
\\
Pivot in row 1 and 2.
No pivot in row 3  $\Rightarrow x_3$ is a free variable
	
	
		\begin{solu}
		x_1 + x_3 = 3\\
		x_2 - x_3 = 1\\
		x_3 = x_3
		\end{solu} $\Rightarrow	$
		\begin{solu}
		x_1 = 3-x_3\\
		x_2 = 1+x_3\\
		x_3=x_3
		\end{solu} $\Rightarrow$
		\begin{ArgMat}
		x_1\\
		x_2\\
		x_3
		\end{ArgMat}=
		\begin{ArgMat}
		3-x_3\\
		1+x_3\\
		x_3
		\end{ArgMat}=
		\begin{ArgMat}
		3\\
		1\\
		0
		\end{ArgMat} + $x_3$
		\begin{ArgMat}
		-1\\
		1
		\end{ArgMat}
		
	
	
\section{Vector equations}
2 fundamentale regler:
\begin{itemize}
\item $\vec{U} \rightarrow c\vec{U}$
\item $\vec{U}, \vec{V} $
\end{itemize}

\subsection{Linear combinations}

\begin{theo} 
A vector equation\\
\centerline{$x_1a_1 + x_2a_2 + \dots + x_na_n = b$}
\\
has the same solution set as the linear system whose augmented matrix is \centerline{\begin{ArgMat}
a_1 & a_2 & \dots &a_n &b
\end{ArgMat}}
In particular, \textbf{b} can be generated by a linear combination of $\textbf{a}_1, \dots, \textbf{a}_n$ 
if and only if there exists a solution to the linear system corresponding to the matrix.
\end{theo}


Given vectors $\vec{V_1} \dots \vec{V_p}$ in $\mathbb{R}^2$ and scalars $C_1 \dots c_p$ then we can form the linear combination $\vec{y} = c_1\vec{V_1} + c_2\vec{V_2}+ \dots + c_p\vec{V_p}$
\\
\\
$\vec{a_1}$ = \begin{ArgMat}
1\\
5\\
4
\end{ArgMat}, 
$\vec{a_2}$ = \begin{ArgMat}
2\\
-1\\
3
\end{ArgMat}, 
$\vec{b}$ = \begin{ArgMat}
10\\
5\\
25
\end{ArgMat}
\\
Is $\vec{b}$ a lin. comb. og $\vec{a_1}$ and $ \vec{a_2}$?
\\
\\
$x_1\vec{a_1} +x_2\vec{a_2} = \vec{b}$
\\
\\
$x_1$\begin{ArgMat}
1\\
2\\
4
\end{ArgMat}+$x_2$\begin{ArgMat}
2\\
-1\\
3
\end{ArgMat}
\\
\\
Span for én vektor er en linje i rummet.\\
Span\{$C_1\vec{U} + C_2\vec{U}$\} er et område i rummet

\end{document}