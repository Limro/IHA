\documentclass[danish, english]{article}
%Preamble

% Følgende er til koder.
%----------------------------------------------------------
%\begin{lstlisting}[caption=Overskrift på boks, style=Code-C++, label=lst:referenceLabel]
%public void hello(){}
%\end{lstlisting}
%----------------------------------------------------------

%Exstra space
\usepackage{xspace}
%Navn på bokse efterfulgt af \xspace (hvis det skal være mellemrum
%gives det med denne udvidelse. Ellers ingen mellemrum.
\newcommand{\codeTitle}{Code snippet\xspace}

%Pakker der skal bruges til lstlisting
\usepackage{listings}
\usepackage{color}
\usepackage{textcomp}
\definecolor{listinggray}{gray}{0.9}
\definecolor{lbcolor}{rgb}{0.9,0.9,0.9}
\renewcommand{\lstlistingname}{\codeTitle}
\lstdefinestyle{Code}
{
	keywordstyle	= \bfseries\ttfamily\color[rgb]{0,0,1},
	identifierstyle	= \ttfamily,
	commentstyle	= \color[rgb]{0.133,0.545,0.133},
	stringstyle		= \ttfamily\color[rgb]{0.627,0.126,0.941},
	showstringspaces= false,
	basicstyle		= \small,
	numberstyle		= \footnotesize,
%	numbers			= left, % Tal? Udkommenter hvis ikke
	stepnumber		= 2,
	numbersep		= 6pt,
	tabsize			= 2,
	breaklines		= true,
	prebreak 		= \raisebox{0ex}[0ex][0ex]{\ensuremath{\hookleftarrow}},
	breakatwhitespace= false,
%	aboveskip		= {1.5\baselineskip},
  	columns			= fixed,
  	upquote			= true,
  	extendedchars	= true,
 	backgroundcolor = \color{lbcolor},
	lineskip		= 1pt,
%	xleftmargin		= 17pt,
%	framexleftmargin= 17pt,
	framexrightmargin	= 0pt, %6pt
%	framexbottommargin	= 4pt,
}

%Bredde der bruges til indryk
%Den skal være 6 pt mindre
\usepackage{calc}
\newlength{\mywidth}
\setlength{\mywidth}{1.435\textwidth} % Hvis bredden header ikke virker er dette hvad skal ændres!


% Forskellige styles for forskellige kodetyper
\usepackage{caption}
\DeclareCaptionFont{white}{\color{white}}
\DeclareCaptionFormat{listing}%
{\colorbox[cmyk]{0.43, 0.35, 0.35,0.35}{\parbox{\mywidth}{\hspace{5pt}#1#2#3}}}
\captionsetup[lstlisting]
{
	format			= listing,
	labelfont		= white,
	textfont		= white, 
	singlelinecheck	= false, 
	width			= \mywidth,
	margin			= 0pt, 
	font			= {bf,footnotesize}
}

\lstdefinestyle{Code-C} {language=C, style=Code}
\lstdefinestyle{Code-Java} {language=Java, style=Code}
\lstdefinestyle{Code-C++} {language=[Visual]C++, style=Code}
\lstdefinestyle{Code-VHDL} {language=VHDL, style=Code}
\lstdefinestyle{Code-Bash} {language=Bash, style=Code}
\lstdefinestyle{Code-Matlab} {language=Matlab, style=Code}
\lstdefinestyle{Code-Prolog} {language=Prolog, style=Code}
%Speciel skrift for enkelt linje kode
%--------------------------------------------------
%Udskriver med fonten 'Courier'
%Mere info her: http://tex.stackexchange.com/questions/25249/how-do-i-use-a-particular-font-for-a-small-section-of-text-in-my-document
%Eksempel: Funktionen \code{void Hello()} giver et output
%--------------------------------------------------
\newcommand{\code}[1]{{\fontfamily{pcr}\selectfont #1}}

%Seperated files
%--------------------------------------------------
%Opret filer således:
%\documentclass[Navn-på-hovedfil]{subfiles}
%\begin{document}
% Indmad
%\end{document}
%
% I hovedfil inkluderes således:
% \subfile{navn-på-subfil}
%--------------------------------------------------
\usepackage{subfiles}
%Text typesetting
%--------------------------------------------------------
\usepackage[T1]{fontenc} 	% Can use danish characters
\usepackage[utf8]{inputenc} % Input encoding. Can be used on Linux, Mac and Windows         
\usepackage[danish]{babel} 	% Split words accoding to English
\usepackage{lmodern} 		% Font

\setlength\parindent{0pt} 	% No indent
\setlength\parskip{12pt} 	% More than a single line break will give ONE linebreak.

%Margin
\usepackage[left=2cm,right=2cm,top=2.5cm,bottom=2cm]{geometry}

%Margin
\usepackage[left=3cm,right=2cm,top=2.5cm,bottom=2cm]{geometry}

%Mellemrum mellem linjerne    
\linespread{1.5}

\title{Assignment 2 for TEDI}
\author{Rasmus Bækgaard, 10893}
\date{April 28, 2014}
\title{Lektion 13}

\begin{document}
\maketitle


\section*{Symetric matrices}
\begin{theo} 
$A^T = A$\\
\\
\begin{ArgMat}
1 &3\\
3&0
\end{ArgMat} symetrisk\\
\\
\begin{ArgMat}
1 & 3\\
2 &0
\end{ArgMat} Ikke symetrisk\\
\\
$A=PDP^{-1}$ er ikke altid muligt -- men i dette kursus er det, hvor $P^{-1}=P^T$ så $A=PDP^T$
\\
P har eigenvektorerne for matrix A.
Disse er også basis værdier, når de skaleres op til '1'.
\end{theo}


\textbf{Example: }(chap 7.1)
\\
\begin{itemize}
\item Skriv argument matrix
\item Skriv eigenværdier ($\lambda$)
\item Find eigenvektorer (\textit{eig(A)})
\item Tag prikproduktet mellem eigenvektorerne (2 af gangen) betyder '0' ortogonal.
\item Såfrem de ikke er ortogonale alle sammen, skal der laves nogle der er.
\begin{itemize}
\item Lav en vektor $u_1$ og $u_2=v_2-\dfrac{v_2\cdot u_1}{u_1\cdot u_1}$.
\item Check om de nu er ortogonale alle sammen.
\item \textit{[V, D] = eig(A); dot(V(:,2),V(:,3))}
\end{itemize}
\item angiv svar som "$\left\{\vec{v_1},\vec{v_2},\vec{v_3}\right\}$ er et ortogonalt set" og "A=
\begin{ArgMat}
V (\text{som er P})
\end{ArgMat}
\begin{ArgMat}
\lambda_1&0&0\\
0&\lambda&0\\
0&0&\lambda
\end{ArgMat}
\begin{ArgMat}
P^T
\end{ArgMat}"

\end{itemize}



\newpage
\section*{Spectial decomposition}
\begin{theo} 
$A=PDP^T$\\
Hvad giver det, når $A\vec{x}$?
\\
\\
$A=PDP^T$
\begin{ArgMat}
\vec{u}_1 & \vec{u}_2 & \dots &\vec{u}_n
\end{ArgMat}
\begin{ArgMat}
\lambda_1 & & 0\\
& \dots & \\
0& & \lambda_n
\end{ArgMat}
\begin{ArgMat}
\vec{u}_1^T \\
\dots \\
\vec{u}_n^T\\
\end{ArgMat}
\\
$A=\lambda_1\vec{u}_1\vec{u}_1^T+\lambda_2\vec{u}_2\vec{u}_2^T+\dots+\lambda_n\vec{u}_n\vec{u}_n^T$
\\
\\
$A\vec{x}=\lambda_1\vec{u}_1\vec{u}_1^T\vec{x}
+\lambda_2\vec{u}_2\vec{u}_2^T\vec{x}
+\dots
+\lambda_n\vec{u}_n\vec{u}_n^T\vec{x}$
\\
\\
$\vec{u}_n\vec{u}_n^T\vec{x} \Leftrightarrow \dfrac{x\cdot u}{u \cdot u}u \Leftrightarrow \vec{u}(u^T \cdot \vec{x})$ (projektering på $u_1^T$)
\end{theo}

\begin{theo} 
$Q(\vec{x})=\vec{x}^TA\vec{x}$ = ét tal
\end{theo}
\textbf{Eksempel:}
\begin{itemize}
\item Skriv en matrix, A
\item Beregn Q(x) ved at indsætte x som row vector (på langs), gang med A og skriv x som column vector
\item Gang det sammen og udskriv det (husk at reducere).
\end{itemize}


\textbf{Eksempel:} (Baglænds)
\begin{itemize}
\item $Q(\vec{x}) = ax_1^2+ax_1x_2+ax_2$
\item Del ledet med samlet x'er ($x_1x_2$) med 2.
\item Skriv A som en symetrisk matrix med disse (se $x_ax_b$ som $row_a col_b$).
\end{itemize}

\textbf{Eksempel:}
\begin{itemize}
\item A= \begin{ArgMat}
2 &0\\
0 &2
\end{ArgMat} og Q(x)=8 
\item $\Rightarrow 2x_1^2+2x_2=8 \Leftrightarrow x_1^2+x_2^2 = 2^2$
\item Dette er en cirkel med radius 2 og centrum (0,0)
\end{itemize}




\textbf{Eksempel:}
\begin{itemize}
\item A= \begin{ArgMat}
2 &0\\
0 &3
\end{ArgMat} og Q(x)=8 
\item $\Rightarrow 2x_1^2+3x_2=8 \Leftrightarrow \dots$
\item Dette er en elipse med højde b, brede a og centrum (0,0)
\end{itemize}


\begin{theo} 
A er symetrisk, så $A=PDP^T$, hvilket giver $\vec{x}=P\vec{y} \Leftrightarrow \vec{y}=P^T\vec{x}$
\\
$Q(x)=\vec{y}^T D\vec{y}$
\\
\\
Eigenvektorer viser et "drejet koordinat system"
\\
\\
En andengradsligning vil flytte en elipse.
\end{theo}


\begin{theo}[Theorem ?] 
Ved en symetrisk mtriz gælder det ved quadrtisk form $x^T Ax$ at
\begin{itemize}
\item Den har positiv værdi, hvis alle eigenvalues for A er positive.
\item Den har negativ værdi, hvis alle eigenvalues for A er negative.
\item Den har ubestemt fortegn, hvis eigenvalues for A er forskellige.
\end{itemize}
\end{theo}



\begin{theo}[Theorem 6]
Hvor stor kan Q(x) blive, hvis $||\vec{x}||=1$ -- husk den kan være en elipse.
\end{theo}



















\end{document}