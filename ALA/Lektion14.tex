\documentclass[danish, english]{article}
\usepackage{tcolorbox}
\usepackage{ulem} %math
\usepackage{amsmath}
\usepackage{amsfonts}
\usepackage{amssymb}
\usepackage{graphicx}
\usepackage{enumerate}


%Create a box for theorems
%\begin{theo}[titel] %optional
%tekst
%\end{theo}
\newenvironment{theo}[1][Vigtigt]{%
\begin{tcolorbox}[colback=green!5,colframe=green!40!black,title=\textbf{#1}]
}{%
\end{tcolorbox}
}




%Create a square matrix
%\begin{ArgMat}{2}
%21 & 22 & 23 \\  
%a & b & c
%\end{ArgMat}
%
% Info: http://tex.stackexchange.com/questions/2233/whats-the-best-way-make-an-augmented-coefficient-matrix
%
\newenvironment{ArgMat}{%
$
  \left[\begin{array}{@{}*{100}{r}r@{}}
}{%
  \end{array}\right]
  $
}

\newenvironment{deter}{%
$
  \left|\begin{array}{@{}*{100}{r}r@{}}
}{%
  \end{array}\right|
  $
}


%Create multiple lines with holes
%\begin{SysEqu}
%x_1 && &- &5x_3 &+ &2x_4=& 1 \\
%x_1 &+ &x_2 &+ &x_3 && =& 4 \\
%&&&&&&0 =& 0
%\end{SysEqu}
\newenvironment{SysEqu}{%
$  \setlength\arraycolsep{0.1em}
  \begin{array}{@{}*{100}{r}r@{}}
}{%
  \end{array}$
}

%Create solution for x_1, x_n...
%\begin{solu}
%x_1 &= d \\
%x_2 &= e \\
%x_3 &= s
%\end{solu}
\newenvironment{solu}{%
$
  \setlength\arraycolsep{0.1em}
  \left\{\begin{array}{@{}*{100}{r}r@{}}
}{%
  \end{array}\right.
$
}

\usepackage{lastpage}


\newcommand{\HRule}{\rule{\linewidth}{0.8mm}}

%Tekst i fotter
\newcommand{\footerText}{\thepage\xspace /\pageref{LastPage}}
\newcommand{\ProjectName}{433 MHz styring af AeroQuad}


\chapterstyle{hangnum}




\nouppercaseheads
\makepagestyle{mystyle} 

\makeevenhead{mystyle}{}{\\ \leftmark}{} 
\makeoddhead{mystyle}{}{\\ \leftmark}{} 
\makeevenfoot{mystyle}{}{\footerText}{} 
\makeoddfoot{mystyle}{}{\footerText}{} 
\makeatletter
\makepsmarks{mystyle}{% Overskriften på sidehovedet
  \createmark{chapter}{left}{shownumber}{\@chapapp\ }{.\ }} 
\makeatother
\makefootrule{mystyle}{\textwidth}{\normalrulethickness}{0.4pt}
\makeheadrule{mystyle}{\textwidth}{\normalrulethickness}

\makepagestyle{plain}
\makeevenhead{plain}{}{}{}
\makeoddhead{plain}{}{}{}
\makeevenfoot{plain}{}{\footerText}{}
\makeoddfoot{plain}{}{\footerText}{}
\makefootrule{plain}{\textwidth}{\normalrulethickness}{0.4pt}

\pagestyle{mystyle}

%%----------------------------------------------------------------------
%
%%Redefining chapter style
%%\renewcommand\chapterheadstart{\vspace*{\beforechapskip}}
%\renewcommand\chapterheadstart{\vspace*{10pt}}
%\renewcommand\printchaptername{\chapnamefont }%\@chapapp}
%\renewcommand\chapternamenum{\space}
%\renewcommand\printchapternum{\chapnumfont \thechapter}
%\renewcommand\afterchapternum{\space: }%\par\nobreak\vskip \midchapskip}
%\renewcommand\printchapternonum{}
%\renewcommand\printchaptertitle[1]{\chaptitlefont #1}
\setlength{\beforechapskip}{0pt} 
\setlength{\afterchapskip}{0pt} 
%\setlength{\voffset}{0pt} 
\setlength{\headsep}{25pt}
%\setlength{\topmargin}{35pt}
%%\setlength{\headheight}{102pt}
%\setlength{\textheight}{302pt}
\renewcommand\afterchaptertitle{\par\nobreak\vskip \afterchapskip}
%%----------------------------------------------------------------------




%Sidehoved og -fod pakke
%Margin
\usepackage[left=2cm,right=2cm,top=2.5cm,bottom=2cm]{geometry}
\usepackage{lastpage}



%%URL kommandoer og sidetal farve
%%Kaldes med \url{www...}
%\usepackage{color} %Skal også bruges
\usepackage{hyperref}
\hypersetup{ 
	colorlinks	= true, 	% false: boxed links; true: colored links
    urlcolor	= blue,		% color of external links
    linkcolor	= black, 	% color of page numbers
    citecolor	= blue,
}



%Mellemrum mellem linjerne    
\linespread{1.5}


%Seperated files
%--------------------------------------------------
%Opret filer således:
%\documentclass[Navn-på-hovedfil]{subfiles}
%\begin{document}
% Indmad
%\end{document}
%
% I hovedfil inkluderes således:
% \subfile{navn-på-subfil}
%--------------------------------------------------
\usepackage{subfiles}

%Prevent wierd placement of figures
%\usepackage[section]{placeins}

%Standard sti at søge efter billeder
%--------------------------------------------------
%\begin{figure}[hbtp]
%\centering
%\includegraphics[scale=1]{filnavn-for-png}
%\caption{Titel}
%\label{fig:referenceNavn}
%\end{figure}
%--------------------------------------------------
\usepackage{graphicx}
\usepackage{subcaption}
\usepackage{float}
\graphicspath{{../Figures/}}

%Speciel skrift for enkelt linje kode
%--------------------------------------------------
%Udskriver med fonten 'Courier'
%Mere info her: http://tex.stackexchange.com/questions/25249/how-do-i-use-a-particular-font-for-a-small-section-of-text-in-my-document
%Eksempel: Funktionen \code{void Hello()} giver et output
%--------------------------------------------------
\newcommand{\code}[1]{{\fontfamily{pcr}\selectfont #1}}


% Følgende er til koder.
%----------------------------------------------------------
%\begin{lstlisting}[caption=Overskrift på boks, style=Code-C++, label=lst:referenceLabel]
%public void hello(){}
%\end{lstlisting}
%----------------------------------------------------------

%Exstra space
\usepackage{xspace}
%Navn på bokse efterfulgt af \xspace (hvis det skal være mellemrum
%gives det med denne udvidelse. Ellers ingen mellemrum.
\newcommand{\codeTitle}{Kodeudsnit\xspace}

%Pakker der skal bruges til lstlisting
\usepackage{listings}
\usepackage{color}
\usepackage{textcomp}
\definecolor{listinggray}{gray}{0.9}
\definecolor{lbcolor}{rgb}{0.9,0.9,0.9}
\renewcommand{\lstlistingname}{\codeTitle}
\lstdefinestyle{Code}
{
	keywordstyle	= \bfseries\ttfamily\color[rgb]{0,0,1},
	identifierstyle	= \ttfamily,
	commentstyle	= \color[rgb]{0.133,0.545,0.133},
	stringstyle		= \ttfamily\color[rgb]{0.627,0.126,0.941},
	showstringspaces= false,
	basicstyle		= \small,
	numberstyle		= \footnotesize,
%	numbers			= left, % Tal? Udkommenter hvis ikke
	stepnumber		= 2,
	numbersep		= 6pt,
	tabsize			= 2,
	breaklines		= true,
	prebreak 		= \raisebox{0ex}[0ex][0ex]{\ensuremath{\hookleftarrow}},
	breakatwhitespace= false,
%	aboveskip		= {1.5\baselineskip},
  	columns			= fixed,
  	upquote			= true,
  	extendedchars	= true,
 	backgroundcolor = \color{lbcolor},
	lineskip		= 1pt,
%	xleftmargin		= 17pt,
%	framexleftmargin= 17pt,
	framexrightmargin	= 0pt, %6pt
%	framexbottommargin	= 4pt,
}

%Bredde der bruges til indryk
%Den skal være 6 pt mindre
\usepackage{calc}
\newlength{\mywidth}
\setlength{\mywidth}{\textwidth-6pt}


% Forskellige styles for forskellige kodetyper
\usepackage{caption}
\DeclareCaptionFont{white}{\color{white}}
\DeclareCaptionFormat{listing}%
{\colorbox[cmyk]{0.43, 0.35, 0.35,0.35}{\parbox{\mywidth}{\hspace{5pt}#1#2#3}}}
\captionsetup[lstlisting]
{
	format			= listing,
	labelfont		= white,
	textfont		= white, 
	singlelinecheck	= false, 
	width			= \mywidth,
	margin			= 0pt, 
	font			= {bf,footnotesize}
}

\lstdefinestyle{Code-C} {language=C, style=Code}
\lstdefinestyle{Code-Java} {language=Java, style=Code}
\lstdefinestyle{Code-C++} {language=[Visual]C++, style=Code}
\lstdefinestyle{Code-VHDL} {language=VHDL, style=Code}
\lstdefinestyle{Code-Bash} {language=Bash, style=Code}

%Text typesetting
%--------------------------------------------------------
%\usepackage{baskervald}
\usepackage{lmodern}
\usepackage[T1]{fontenc}              
\usepackage[utf8]{inputenc}         
\usepackage[english]{babel}       

\setlength{\parindent}{0pt}
\nonzeroparskip

%\setaftersubsecskip{1sp}
%\setaftersubsubsecskip{1sp}
 


%Dybde på indholdsfortegnelse
%----------------------------------------------------------
%Chapter, section, subsection, subsubsection
%----------------------------------------------------------
\setcounter{secnumdepth}{3}
\setcounter{tocdepth}{3}


%Tables
%----------------------------------------------------------
\usepackage{tabularx}
\usepackage{array}
\usepackage{multirow} 
\usepackage{multicol} 
\usepackage{booktabs}
\usepackage{wrapfig}
\renewcommand{\arraystretch}{1.5}



%Misc
%----------------------------------------------------------
\usepackage{cite}
\usepackage{appendix}
\usepackage{amssymb}
\usepackage{url,ragged2e}
\usepackage{enumerate}
\usepackage{amsmath} %Math bibliotek


\usepackage{longtable}

\title{Lektion 14}

\begin{document}
\maketitle


\section*{The singular value decomposition, SVD}
\begin{theo} 
$A=PDP^{-1} \text{ for } n \times n\\
A=U\sum V^T \text{ for } m \times n\\
A^TA$ er symetrisk\\
$(A^TA)^T = A^TA$ -- dette er ortogonal diagonliserbart
\\
\\
$\left\{ \vec{v}_1, \vec{v}_2, \dots , \vec{v}_n\right\}$ orthonormal basis for $\mathbb{R}^n$ (eigenvectors of $A^TA$) or corresponding eigenvalues $\lambda_1 \dots \lambda_2$
\\
\\
$||A\vec{v}_i||=\lambda_i$
\\
\\
\textbf{The singular values of A are defined as $\sigma_i=\sqrt{\lambda_i}$}\\
Hvis $A^TA$ kun har $\lambda$-værdierne, sæt alle disse tal til positiv og du har singular values
\end{theo}

\begin{theo}[Orthogonal set] 
Er $\left\{A\vec{v}_1\dots A\vec{v}_n\right\}$ ortogonal?\\
\\
$\vec{v}_i$ og $\vec{v}_j$ er ortogonal\\
\\
${\vec{v}_i}^T\vec{v}_j = \delta_{ij}$
\\
\\
$\delta_{ij}=$
\begin{solu}
1 & \text{if }i=j\\
0 & \text{if }i\neq j
\end{solu}

$(Av_i)^TA\vec{v}_j = \lambda_i\delta_{ij}\\
||A\vec{v}_i||^2 = \sigma_i$ thus $||A\vec{v}_i||=0 \text{ for } i > r \text{ and }A\vec{v}_i=\vec{0} \text{ for } i>r$
\\
\\
Any vector in col A can be written as a linear combination of A's columns:
\\
$\vec{y}=A\vec{x}\\
 \vec{x} = c_1\vec{v}_1 +\dots+\vec{c}_n\vec{v}_n$
 \\
 \\
r = rank A
$\vec{y}=v_1A\vec{v}_1 + \dots + \vec{c}_rA\vec{v}_r$
Det betyder, at $\vec{y}$ er i $span\left\{A\vec{v}_1 \dots A\vec{v}_r\right\}$ dermed $\left\{A\vec{v}_1 \dots A\vec{v}_r\right\}$ er en ortogonal basis for col A.
\end{theo}

$A=U\Sigma V^T$\\
U = singular value vector\\
$\Sigma = $
\begin{ArgMat}
D &0\\
0 &0
\end{ArgMat}

$\left\{A\vec{v}_1,\dots, A\vec{v}_r\right\}$ orthogonal basis for colA.
Lets make it ortho\textbf{normal}:
\\
$\vec{U}_1=
\dfrac{A\vec{v}_1}{||A\vec{v}_1||}=
\dfrac{A\vec{v}_1}{\sigma_1}$ for 1 to \textit{r}.
\\
$\vec{U}_i$ is a $\mathbb{R}^m$ vector\\
$\left\{\vec{u}_1,\dots, \vec{u}_r\right\}$ kan blive udvidet til en basis for $\mathbb{R}^m$.
\\
\\
AV= \begin{ArgMat}
\sigma_1 \vec{u}_1 & \dots & \sigma_r A\vec{u}_r & \vec{0} \dots \vec{0}
\end{ArgMat}
\\
\\
$U\Sigma=$
\begin{ArgMat}
\sigma_1\vec{u}_1 &
\sigma_2\vec{u}_2 &
\dots &
\sigma_r\vec{u}_r &
\vec{0} & \vec{0}
\end{ArgMat}
\\
$AV = U\Sigma \Leftrightarrow A = U\Sigma V^T$
\\
\\
\textbf{Eksempel}: (Matlab: \textit{[U,S,V]=svd(A)})
\begin{itemize}
\item Opskriv A
\item Beregn $A^TA$
\item Find $\lambda$ og sikre at det$(A^TA-\lambda I)=0$
\item Lav \begin{ArgMat}
A^TA-\lambda I &|0
\end{ArgMat} for HVER $\lambda$
\item rref denne og find $\vec{v_n}$ (én pr.  vector equation)
\item Lav $A\vec{v}_n$ for hver \textit{n}
\item Beregn $\vec{u}_1=\dfrac{A\vec{v}_n}{\sigma_n}$ for hver \textit{n}
\item Beregn $A=U\Sigma V^T$
\end{itemize}



\section*{Pseudoinverse}
\begin{theo} 
$m\times n$ matrix A med rank \textit{r}\\
$\Sigma$ contains rows or columns\\
$\Sigma$=
\begin{ArgMat}
D &0\\
0&0
\end{ArgMat}
D=
\begin{ArgMat}
\sigma_1 &&\\
&\dots&\\
&&\sigma_r
\end{ArgMat}
\\
\\
$A=U\Sigma V^T = $
\begin{ArgMat}
U_r & U_{m-r}
\end{ArgMat}
\begin{ArgMat}
D &0\\
0&0
\end{ArgMat}
\begin{ArgMat}
V_r^T\\
V_{n-r}^T
\end{ArgMat} = $U_rDV_r^T$
\\
\\
$U_r^TU_r=I\\
V_r^TV_r=I'$
\\
\\
$U_r, V_r$ are orthogonal matrices\\
\textit{D} is invertible.
\\
\\
$A^T \equiv V_rD^{-1}U_r^T$ og $AA^T = I$
\end{theo}

\textbf{Eksempel:}
Hvis man har 2 vektorer med ukendt skalar på:
\begin{itemize}
\item Opskriv $A\vec{x}=\vec{b}$
\item Skriv A og $\vec{b}$
\item Lad $A\vec{x}=\vec{b} \Leftrightarrow \hat{x}=A^T\vec{b }$
\end{itemize}






























\end{document}