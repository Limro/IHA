\documentclass[danish, english]{article}
\usepackage{tcolorbox}
\usepackage{ulem} %math
\usepackage{amsmath}
\usepackage{amsfonts}
\usepackage{amssymb}
\usepackage{graphicx}
\usepackage{enumerate}


%Create a box for theorems
%\begin{theo}[titel] %optional
%tekst
%\end{theo}
\newenvironment{theo}[1][Vigtigt]{%
\begin{tcolorbox}[colback=green!5,colframe=green!40!black,title=\textbf{#1}]
}{%
\end{tcolorbox}
}




%Create a square matrix
%\begin{ArgMat}{2}
%21 & 22 & 23 \\  
%a & b & c
%\end{ArgMat}
%
% Info: http://tex.stackexchange.com/questions/2233/whats-the-best-way-make-an-augmented-coefficient-matrix
%
\newenvironment{ArgMat}{%
$
  \left[\begin{array}{@{}*{100}{r}r@{}}
}{%
  \end{array}\right]
  $
}

\newenvironment{deter}{%
$
  \left|\begin{array}{@{}*{100}{r}r@{}}
}{%
  \end{array}\right|
  $
}


%Create multiple lines with holes
%\begin{SysEqu}
%x_1 && &- &5x_3 &+ &2x_4=& 1 \\
%x_1 &+ &x_2 &+ &x_3 && =& 4 \\
%&&&&&&0 =& 0
%\end{SysEqu}
\newenvironment{SysEqu}{%
$  \setlength\arraycolsep{0.1em}
  \begin{array}{@{}*{100}{r}r@{}}
}{%
  \end{array}$
}

%Create solution for x_1, x_n...
%\begin{solu}
%x_1 &= d \\
%x_2 &= e \\
%x_3 &= s
%\end{solu}
\newenvironment{solu}{%
$
  \setlength\arraycolsep{0.1em}
  \left\{\begin{array}{@{}*{100}{r}r@{}}
}{%
  \end{array}\right.
$
}

\usepackage{lastpage}


\newcommand{\HRule}{\rule{\linewidth}{0.8mm}}

%Tekst i fotter
\newcommand{\footerText}{\thepage\xspace /\pageref{LastPage}}
\newcommand{\ProjectName}{433 MHz styring af AeroQuad}


\chapterstyle{hangnum}




\nouppercaseheads
\makepagestyle{mystyle} 

\makeevenhead{mystyle}{}{\\ \leftmark}{} 
\makeoddhead{mystyle}{}{\\ \leftmark}{} 
\makeevenfoot{mystyle}{}{\footerText}{} 
\makeoddfoot{mystyle}{}{\footerText}{} 
\makeatletter
\makepsmarks{mystyle}{% Overskriften på sidehovedet
  \createmark{chapter}{left}{shownumber}{\@chapapp\ }{.\ }} 
\makeatother
\makefootrule{mystyle}{\textwidth}{\normalrulethickness}{0.4pt}
\makeheadrule{mystyle}{\textwidth}{\normalrulethickness}

\makepagestyle{plain}
\makeevenhead{plain}{}{}{}
\makeoddhead{plain}{}{}{}
\makeevenfoot{plain}{}{\footerText}{}
\makeoddfoot{plain}{}{\footerText}{}
\makefootrule{plain}{\textwidth}{\normalrulethickness}{0.4pt}

\pagestyle{mystyle}

%%----------------------------------------------------------------------
%
%%Redefining chapter style
%%\renewcommand\chapterheadstart{\vspace*{\beforechapskip}}
%\renewcommand\chapterheadstart{\vspace*{10pt}}
%\renewcommand\printchaptername{\chapnamefont }%\@chapapp}
%\renewcommand\chapternamenum{\space}
%\renewcommand\printchapternum{\chapnumfont \thechapter}
%\renewcommand\afterchapternum{\space: }%\par\nobreak\vskip \midchapskip}
%\renewcommand\printchapternonum{}
%\renewcommand\printchaptertitle[1]{\chaptitlefont #1}
\setlength{\beforechapskip}{0pt} 
\setlength{\afterchapskip}{0pt} 
%\setlength{\voffset}{0pt} 
\setlength{\headsep}{25pt}
%\setlength{\topmargin}{35pt}
%%\setlength{\headheight}{102pt}
%\setlength{\textheight}{302pt}
\renewcommand\afterchaptertitle{\par\nobreak\vskip \afterchapskip}
%%----------------------------------------------------------------------




%Sidehoved og -fod pakke
%Margin
\usepackage[left=2cm,right=2cm,top=2.5cm,bottom=2cm]{geometry}
\usepackage{lastpage}



%%URL kommandoer og sidetal farve
%%Kaldes med \url{www...}
%\usepackage{color} %Skal også bruges
\usepackage{hyperref}
\hypersetup{ 
	colorlinks	= true, 	% false: boxed links; true: colored links
    urlcolor	= blue,		% color of external links
    linkcolor	= black, 	% color of page numbers
    citecolor	= blue,
}



%Mellemrum mellem linjerne    
\linespread{1.5}


%Seperated files
%--------------------------------------------------
%Opret filer således:
%\documentclass[Navn-på-hovedfil]{subfiles}
%\begin{document}
% Indmad
%\end{document}
%
% I hovedfil inkluderes således:
% \subfile{navn-på-subfil}
%--------------------------------------------------
\usepackage{subfiles}

%Prevent wierd placement of figures
%\usepackage[section]{placeins}

%Standard sti at søge efter billeder
%--------------------------------------------------
%\begin{figure}[hbtp]
%\centering
%\includegraphics[scale=1]{filnavn-for-png}
%\caption{Titel}
%\label{fig:referenceNavn}
%\end{figure}
%--------------------------------------------------
\usepackage{graphicx}
\usepackage{subcaption}
\usepackage{float}
\graphicspath{{../Figures/}}

%Speciel skrift for enkelt linje kode
%--------------------------------------------------
%Udskriver med fonten 'Courier'
%Mere info her: http://tex.stackexchange.com/questions/25249/how-do-i-use-a-particular-font-for-a-small-section-of-text-in-my-document
%Eksempel: Funktionen \code{void Hello()} giver et output
%--------------------------------------------------
\newcommand{\code}[1]{{\fontfamily{pcr}\selectfont #1}}


% Følgende er til koder.
%----------------------------------------------------------
%\begin{lstlisting}[caption=Overskrift på boks, style=Code-C++, label=lst:referenceLabel]
%public void hello(){}
%\end{lstlisting}
%----------------------------------------------------------

%Exstra space
\usepackage{xspace}
%Navn på bokse efterfulgt af \xspace (hvis det skal være mellemrum
%gives det med denne udvidelse. Ellers ingen mellemrum.
\newcommand{\codeTitle}{Kodeudsnit\xspace}

%Pakker der skal bruges til lstlisting
\usepackage{listings}
\usepackage{color}
\usepackage{textcomp}
\definecolor{listinggray}{gray}{0.9}
\definecolor{lbcolor}{rgb}{0.9,0.9,0.9}
\renewcommand{\lstlistingname}{\codeTitle}
\lstdefinestyle{Code}
{
	keywordstyle	= \bfseries\ttfamily\color[rgb]{0,0,1},
	identifierstyle	= \ttfamily,
	commentstyle	= \color[rgb]{0.133,0.545,0.133},
	stringstyle		= \ttfamily\color[rgb]{0.627,0.126,0.941},
	showstringspaces= false,
	basicstyle		= \small,
	numberstyle		= \footnotesize,
%	numbers			= left, % Tal? Udkommenter hvis ikke
	stepnumber		= 2,
	numbersep		= 6pt,
	tabsize			= 2,
	breaklines		= true,
	prebreak 		= \raisebox{0ex}[0ex][0ex]{\ensuremath{\hookleftarrow}},
	breakatwhitespace= false,
%	aboveskip		= {1.5\baselineskip},
  	columns			= fixed,
  	upquote			= true,
  	extendedchars	= true,
 	backgroundcolor = \color{lbcolor},
	lineskip		= 1pt,
%	xleftmargin		= 17pt,
%	framexleftmargin= 17pt,
	framexrightmargin	= 0pt, %6pt
%	framexbottommargin	= 4pt,
}

%Bredde der bruges til indryk
%Den skal være 6 pt mindre
\usepackage{calc}
\newlength{\mywidth}
\setlength{\mywidth}{\textwidth-6pt}


% Forskellige styles for forskellige kodetyper
\usepackage{caption}
\DeclareCaptionFont{white}{\color{white}}
\DeclareCaptionFormat{listing}%
{\colorbox[cmyk]{0.43, 0.35, 0.35,0.35}{\parbox{\mywidth}{\hspace{5pt}#1#2#3}}}
\captionsetup[lstlisting]
{
	format			= listing,
	labelfont		= white,
	textfont		= white, 
	singlelinecheck	= false, 
	width			= \mywidth,
	margin			= 0pt, 
	font			= {bf,footnotesize}
}

\lstdefinestyle{Code-C} {language=C, style=Code}
\lstdefinestyle{Code-Java} {language=Java, style=Code}
\lstdefinestyle{Code-C++} {language=[Visual]C++, style=Code}
\lstdefinestyle{Code-VHDL} {language=VHDL, style=Code}
\lstdefinestyle{Code-Bash} {language=Bash, style=Code}

%Text typesetting
%--------------------------------------------------------
%\usepackage{baskervald}
\usepackage{lmodern}
\usepackage[T1]{fontenc}              
\usepackage[utf8]{inputenc}         
\usepackage[english]{babel}       

\setlength{\parindent}{0pt}
\nonzeroparskip

%\setaftersubsecskip{1sp}
%\setaftersubsubsecskip{1sp}
 


%Dybde på indholdsfortegnelse
%----------------------------------------------------------
%Chapter, section, subsection, subsubsection
%----------------------------------------------------------
\setcounter{secnumdepth}{3}
\setcounter{tocdepth}{3}


%Tables
%----------------------------------------------------------
\usepackage{tabularx}
\usepackage{array}
\usepackage{multirow} 
\usepackage{multicol} 
\usepackage{booktabs}
\usepackage{wrapfig}
\renewcommand{\arraystretch}{1.5}



%Misc
%----------------------------------------------------------
\usepackage{cite}
\usepackage{appendix}
\usepackage{amssymb}
\usepackage{url,ragged2e}
\usepackage{enumerate}
\usepackage{amsmath} %Math bibliotek


\usepackage{longtable}


\title{Lektion 2}
\begin{document}
\maketitle

\section*{1.4 The Matrix Equation $A\text{x}=b$}


\begin{theo}[En matrix equation]
\begin{ArgMat}
a & b & c\\
d & e & f
\end{ArgMat}
\begin{ArgMat}
x_1\\
x_2\\
x_3
\end{ArgMat} =
\begin{ArgMat}
\alpha\\
\beta
\end{ArgMat}

\end{theo}


\begin{theo}[Theorem 7] 
If A is an $m \times n$ matrix, with columns $a_1, \dots , a_n$, and if \textbf{b} is in $\mathbb{R}^m$, the \textbf{matrix equation}
\[Ax = b \]
has the same solution set as the \textbf{vector equation}
\[x_1a_1 + x_2a_2 + \dots + x_na_n = b \]
which, in turn, has the same solution set as the system of linear equations whose \textbf{augmented matrix} is\\
\centerline{
\begin{ArgMat}
a_1 & a_2 & \dots & a_n &b
\end{ArgMat}}

\end{theo}

\begin{theo}
The equation $A\textbf{x} = \textbf{b}$ has a solution if and only if \textbf{b} is a linear combination of the columns of $A$.
\end{theo}

A=\begin{ArgMat}
a_{l1} &a_{l2}&\dots&a_{ln}\\
\dots\\
a_{m1} &a_{m2}&\dots&a_{mn}
\end{ArgMat}=
\begin{ArgMat}
\vec{a_1}, \vec{a_2}, \dots, \vec{a_n}
\end{ArgMat}
\\
\\
$\vec{X}$ is  a vector in $\mathbb{R}^n$ \begin{ArgMat}
x_1\\
\dots\\
x_n
\end{ArgMat}
\\
\\
$A\vec{x}=$\begin{ArgMat}
x_1\\
\dots\\
x_n
\end{ArgMat}= $x_1\vec{a_1}+x_2\vec{a_2}+\dots+x_n\vec{a_n}$


\begin{theo}[Theorem 4] 
Let $A$ be $a_{nm} \times n$ matrix. Then the following statements are logically equivalent.
That is, for a particular $A$, either they are all true statements or they are all false.

\begin{enumerate}
\item For each b in $\mathbb{R}^m$, the equation $Ax = b$ has a solution.
\item Each b in $\mathbb{R}^m$ is a linear combination of the columns of A.
\item The columns of A span $\mathbb{R}^m$.
\item A has a pivot position in every row.
\end{enumerate}
\end{theo}
Eksempel:\\
A=\begin{ArgMat}
\vec{a_1} & \vec{a_2} & \vec{a_3} & \vec{a_4}
\end{ArgMat}=
\begin{ArgMat}
1 & 4 & 2&1\\
1&1&3&3\\
0&0&0&0
\end{ArgMat}
\\
\begin{enumerate}
\item No, Choose $\vec{b}$ =
\begin{ArgMat}
\alpha\\
\beta\\
\neq 0
\end{ArgMat}

\item No, Choose $\vec{b}$ =
\begin{ArgMat}
\alpha\\
\beta\\
\neq 0
\end{ArgMat}

\item Can't get a $\vec{x_3}$ direction (det skal være i 3 dimensioner)

\item Pivot is not in every row. (Check med MatLab: \textit{rref(A)})
\end{enumerate}

 

\section*{1.5 Solution Sets of Linear Systems}

\begin{theo} 
$A\vec{x}=\vec{0}$ -- Homogen equation(zero) -- $\vec{x} = \vec{0}$ er altid en solution, også kaldt "trivial solution".\\
$A\vec{x}=\vec{b}$ -- Inhomogen equation
\end{theo}


Eksempel:\\
\begin{align*}
3x_1+5x_2-4x_3 = 0\\
-3x_1-2x_2+4x_3=0\\
6x_1+x_2-8x_3=0
\end{align*}
Skriv en \begin{ArgMat}
A &|&\vec{0}
\end{ArgMat} $\Rightarrow$
\begin{ArgMat}
3&5&-4&0\\
-3&-2&4&0\\
6&1&-8&0
\end{ArgMat} $\sim$
\begin{ArgMat}
1&0&-\dfrac{4}{3}&0\\
0&1&0&0\\
0&0&0&0
\end{ArgMat} $\Rightarrow$
\begin{ArgMat}
x_1\\
x_2\\
x_3
\end{ArgMat} = $x_3$
\begin{ArgMat}
\dfrac{4}{3}\\
0\\
1
\end{ArgMat}
\\
\\
Eksempel:\\
$5x_1+4_x2-x_3=0 \Rightarrow$
\begin{ArgMat}
A&|&0
\end{ArgMat}=
\begin{ArgMat}
5&3&-1&0
\end{ArgMat}$\sim$
\begin{ArgMat}
1&\dfrac{4}{5} & \dfrac{-1}{5}&|&0
\end{ArgMat}\\
Note -- ingen pivot i de to sidste (kun den første) $\Rightarrow$
\\
\begin{ArgMat}
x_1\\
x_2\\
x_3
\end{ArgMat}=
\begin{ArgMat}
\dfrac{-4}{5}x_2 + \dfrac{1}{5}x_3\\
x_2\\
x_3
\end{ArgMat}=$x_2$
\begin{ArgMat}
\dfrac{-4}{5}\\
1\\
0
\end{ArgMat}+$x_3$
\begin{ArgMat}
\dfrac{1}{5}\\
0\\
1
\end{ArgMat}





\section*{1.7 Linear Independence}
\begin{theo} 
\begin{itemize}
\item Hvis en span indeholder $\vec{0}$ er den \underline{altid} \textbf{dependant}.
\item Hvis du ikke kan skalere op og få samme en vektor er den \textbf{independant}
\item Hvis der er en free variable er den \textbf{dependant}
\end{itemize}

\end{theo}
\textbf{Indsæt definition}\\
\\

$\vec{v_1} = $\begin{ArgMat}
1\\
4\\
3
\end{ArgMat},
$\vec{v_2} =$
\begin{ArgMat}
2\\
3\\
5
\end{ArgMat},
$\vec{v_3} =$
\begin{ArgMat}
3\\
6\\
2
\end{ArgMat}
\\
Er \{$\vec{v_1} \vec{v_2} \vec{v_3}$\} linear independant?\\
\\
\begin{ArgMat}
\vec{v_1}&\vec{v_2}&\vec{v_3}
\end{ArgMat}
\begin{ArgMat}
x_1\\
x_2\\
x_3
\end{ArgMat}=$\vec{0} \rightarrow$
\begin{ArgMat}
1 &2&3&0\\
4&3&6&0\\
3&5&2&0
\end{ArgMat} $\sim \sim \sim$
\begin{ArgMat}
1 &0 &0&0\\
0&1&0&0\\
0&0&1&0
\end{ArgMat}
\begin{solu}
x_1 = 0 \\
x_2 = 0\\
x_3 = 0	\\
\end{solu}\\
Dette er den \underline{eneste} trivial solution. 
Den er derfor linear independant.




\end{document}