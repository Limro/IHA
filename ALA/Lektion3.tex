\documentclass[danish, english]{article}
%Preamble

% Følgende er til koder.
%----------------------------------------------------------
%\begin{lstlisting}[caption=Overskrift på boks, style=Code-C++, label=lst:referenceLabel]
%public void hello(){}
%\end{lstlisting}
%----------------------------------------------------------

%Exstra space
\usepackage{xspace}
%Navn på bokse efterfulgt af \xspace (hvis det skal være mellemrum
%gives det med denne udvidelse. Ellers ingen mellemrum.
\newcommand{\codeTitle}{Code snippet\xspace}

%Pakker der skal bruges til lstlisting
\usepackage{listings}
\usepackage{color}
\usepackage{textcomp}
\definecolor{listinggray}{gray}{0.9}
\definecolor{lbcolor}{rgb}{0.9,0.9,0.9}
\renewcommand{\lstlistingname}{\codeTitle}
\lstdefinestyle{Code}
{
	keywordstyle	= \bfseries\ttfamily\color[rgb]{0,0,1},
	identifierstyle	= \ttfamily,
	commentstyle	= \color[rgb]{0.133,0.545,0.133},
	stringstyle		= \ttfamily\color[rgb]{0.627,0.126,0.941},
	showstringspaces= false,
	basicstyle		= \small,
	numberstyle		= \footnotesize,
%	numbers			= left, % Tal? Udkommenter hvis ikke
	stepnumber		= 2,
	numbersep		= 6pt,
	tabsize			= 2,
	breaklines		= true,
	prebreak 		= \raisebox{0ex}[0ex][0ex]{\ensuremath{\hookleftarrow}},
	breakatwhitespace= false,
%	aboveskip		= {1.5\baselineskip},
  	columns			= fixed,
  	upquote			= true,
  	extendedchars	= true,
 	backgroundcolor = \color{lbcolor},
	lineskip		= 1pt,
%	xleftmargin		= 17pt,
%	framexleftmargin= 17pt,
	framexrightmargin	= 0pt, %6pt
%	framexbottommargin	= 4pt,
}

%Bredde der bruges til indryk
%Den skal være 6 pt mindre
\usepackage{calc}
\newlength{\mywidth}
\setlength{\mywidth}{1.435\textwidth} % Hvis bredden header ikke virker er dette hvad skal ændres!


% Forskellige styles for forskellige kodetyper
\usepackage{caption}
\DeclareCaptionFont{white}{\color{white}}
\DeclareCaptionFormat{listing}%
{\colorbox[cmyk]{0.43, 0.35, 0.35,0.35}{\parbox{\mywidth}{\hspace{5pt}#1#2#3}}}
\captionsetup[lstlisting]
{
	format			= listing,
	labelfont		= white,
	textfont		= white, 
	singlelinecheck	= false, 
	width			= \mywidth,
	margin			= 0pt, 
	font			= {bf,footnotesize}
}

\lstdefinestyle{Code-C} {language=C, style=Code}
\lstdefinestyle{Code-Java} {language=Java, style=Code}
\lstdefinestyle{Code-C++} {language=[Visual]C++, style=Code}
\lstdefinestyle{Code-VHDL} {language=VHDL, style=Code}
\lstdefinestyle{Code-Bash} {language=Bash, style=Code}
\lstdefinestyle{Code-Matlab} {language=Matlab, style=Code}
\lstdefinestyle{Code-Prolog} {language=Prolog, style=Code}
%Speciel skrift for enkelt linje kode
%--------------------------------------------------
%Udskriver med fonten 'Courier'
%Mere info her: http://tex.stackexchange.com/questions/25249/how-do-i-use-a-particular-font-for-a-small-section-of-text-in-my-document
%Eksempel: Funktionen \code{void Hello()} giver et output
%--------------------------------------------------
\newcommand{\code}[1]{{\fontfamily{pcr}\selectfont #1}}

%Seperated files
%--------------------------------------------------
%Opret filer således:
%\documentclass[Navn-på-hovedfil]{subfiles}
%\begin{document}
% Indmad
%\end{document}
%
% I hovedfil inkluderes således:
% \subfile{navn-på-subfil}
%--------------------------------------------------
\usepackage{subfiles}
%Text typesetting
%--------------------------------------------------------
\usepackage[T1]{fontenc} 	% Can use danish characters
\usepackage[utf8]{inputenc} % Input encoding. Can be used on Linux, Mac and Windows         
\usepackage[danish]{babel} 	% Split words accoding to English
\usepackage{lmodern} 		% Font

\setlength\parindent{0pt} 	% No indent
\setlength\parskip{12pt} 	% More than a single line break will give ONE linebreak.

%Margin
\usepackage[left=2cm,right=2cm,top=2.5cm,bottom=2cm]{geometry}

%Margin
\usepackage[left=3cm,right=2cm,top=2.5cm,bottom=2cm]{geometry}

%Mellemrum mellem linjerne    
\linespread{1.5}

\title{Assignment 2 for TEDI}
\author{Rasmus Bækgaard, 10893}
\date{April 28, 2014}

\title{Lektion 3}
\begin{document}
\maketitle

\section{Matrix algebra}
\begin{theo}[Multiplication] 
AB = A
\begin{ArgMat}
b_1 & b_2 & \dots & b_p
\end{ArgMat}=
\begin{ArgMat}
Ab_1 & Ab_2 & \dots & Ab_p
\end{ArgMat}
\\
\\
$A \cdot B =$
\begin{ArgMat}
a & b\\
c & d
\end{ArgMat} $\cdot$
\begin{ArgMat}
e & f\\
g & h
\end{ArgMat} =
\begin{ArgMat}
ae + bg & af+bh\\
ce+dg & cf+dh
\end{ArgMat}
\\
\\
$B \cdot A =$
\begin{ArgMat}
e & f\\
g & h
\end{ArgMat} $\cdot$
\begin{ArgMat}
a & b\\
c & d
\end{ArgMat} =
\begin{ArgMat}
ac + fc & eb+fd\\
ga+hc & gb+hd
\end{ArgMat}

Hvis man ganger to matrix'er med forskellig størrelse, skal de have størrelsen $m \times n$ og $n \times p$, hvilket giver resultatet $m \times p$
\end{theo}

\begin{theo} 
\begin{itemize}
\item Hvis kolonne 3 i matrix A er alle nuller bliver alle elementer i AB's række 3 nuller.
\item Hvis kolonne 2 i matrix B er alle nuller, bliver alle elementer i AB's kolonne 2 nuller.
\item Hvis kolonne 1 og 2 i matrix B er identiske, bliver AB's rækker (i de første to kolonner) ens
\end{itemize}
\end{theo}



\section*{The inverse matrix}
\begin{theo}[Regneregler] 
\textit{I} = identisk matrix\\
A is invertible if it is row equivanlent with \textit{I}\\
\begin{ArgMat}
A &|I
\end{ArgMat}$\sim$
\begin{ArgMat}
I&|A^{-1}
\end{ArgMat}
\begin{align*}
A\vec{x} &= \vec{b}\\
A^{-1}A\vec{x} &= A^{-1}\vec{b}\\
I\vec{x} &= A^{-1}\vec{b}\\
\vec{x}&=A^{-1}\vec{b}
\end{align*}
Huske at gange ind fra den "rigtige side"
\end{theo}

\newpage
\subsection*{Elementary row operations}
\begin{theo}[Apply to matrix multiplication] 
\textbf{Scaling:}\\
A=\begin{ArgMat}
a & b &c\\
d&e&f\\
g&h&i
\end{ArgMat}, $E_s=$
\begin{ArgMat}
1&0&0\\
0&2&0\\
0&0&1
\end{ArgMat}
\\
\\
$E_sA =$
\begin{ArgMat}
a&b&c\\
2d&2e&2f\\
g&h&i
\end{ArgMat}
$E_s^{-1}E_sA$ \dots\\
$E_s^{-1}=$
\begin{ArgMat}
1&0&0\\
0&\dfrac{1}{2}&0\\
0&0&1
\end{ArgMat}
\\
\\
\textbf{Swap:}\\
$E_s=$
\begin{ArgMat}
0&\textbf{1}&0\\
\textbf{1}&0&0\\
0&0&1
\end{ArgMat},
$E_{sw}A=$
\begin{ArgMat}
d&e&f\\
a&b&c\\
g&h&i
\end{ArgMat}
\\
$E_{sw}^{-1}E_{sw}A = A$ \dots
\\
\\
\textbf{Adding:}
\\
$E_A =$
\begin{ArgMat}
1&0&0\\
0&1&0\\
\textbf{2}&0&1
\end{ArgMat}, $E_AA=$
\begin{ArgMat}
a&b&c\\
d&e&f\\
2a+g&2b+h&ec+i
\end{ArgMat}




\end{theo}

\begin{theo}[Homogen] 
Hvis \textit{A} er homogen matrix til $A\textbf{x}=0$ og kun den trivielle løsning $x=0$, er A invertible, da trivielle løsninger har pivot elementer i alle rækker. 
Uden frie variabler er den invertible.
\\
\begin{ArgMat}
A &| \vec{0}
\end{ArgMat}$\sim$
\begin{ArgMat}
I &| \vec{0}
\end{ArgMat}
\end{theo}



\begin{theo}[Theorem 8: The Invertible Matrix Theorem] 
Let A be a square $n \times n$ matrix. Then the following statements are equivalent.
That is, for a given A, the statements are either all true or all false.
\begin{enumerate}[a)]
\item A is an invertible matrix.
\item  A is row equivalent to the $n \times n$ identity matrix.
\item  A has \textit{n} pivot positions.
\item  The equation Ax = 0 has only the trivial solution.
\item The columns of A form a linearly independent set.
\item  The linear transformation x -> Ax is one-to-one.
\item  The equation Ax = b has a unique solution for each \textit{b} in $\mathbb{R}^n$.
\item  The columns of A span $\mathbb{R}^n$.
\item  The linear transformation x -> Ax maps $\mathbb{R}^n$ onto $\mathbb{R}^n$.
\item  There is an $n \times n$ matrix C such that CA = I .
\item  There is an $n \times n$ matrix D such that AD = I .
\item  $A^T$ is an invertible matrix.
\end{enumerate}
\end{theo}
\end{document}