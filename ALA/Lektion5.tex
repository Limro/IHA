\documentclass[danish, english]{article}
\usepackage{tcolorbox}
\usepackage{ulem} %math
\usepackage{amsmath}
\usepackage{amsfonts}
\usepackage{amssymb}
\usepackage{graphicx}
\usepackage{enumerate}


%Create a box for theorems
%\begin{theo}[titel] %optional
%tekst
%\end{theo}
\newenvironment{theo}[1][Vigtigt]{%
\begin{tcolorbox}[colback=green!5,colframe=green!40!black,title=\textbf{#1}]
}{%
\end{tcolorbox}
}




%Create a square matrix
%\begin{ArgMat}{2}
%21 & 22 & 23 \\  
%a & b & c
%\end{ArgMat}
%
% Info: http://tex.stackexchange.com/questions/2233/whats-the-best-way-make-an-augmented-coefficient-matrix
%
\newenvironment{ArgMat}{%
$
  \left[\begin{array}{@{}*{100}{r}r@{}}
}{%
  \end{array}\right]
  $
}

\newenvironment{deter}{%
$
  \left|\begin{array}{@{}*{100}{r}r@{}}
}{%
  \end{array}\right|
  $
}


%Create multiple lines with holes
%\begin{SysEqu}
%x_1 && &- &5x_3 &+ &2x_4=& 1 \\
%x_1 &+ &x_2 &+ &x_3 && =& 4 \\
%&&&&&&0 =& 0
%\end{SysEqu}
\newenvironment{SysEqu}{%
$  \setlength\arraycolsep{0.1em}
  \begin{array}{@{}*{100}{r}r@{}}
}{%
  \end{array}$
}

%Create solution for x_1, x_n...
%\begin{solu}
%x_1 &= d \\
%x_2 &= e \\
%x_3 &= s
%\end{solu}
\newenvironment{solu}{%
$
  \setlength\arraycolsep{0.1em}
  \left\{\begin{array}{@{}*{100}{r}r@{}}
}{%
  \end{array}\right.
$
}

\usepackage{lastpage}


\newcommand{\HRule}{\rule{\linewidth}{0.8mm}}

%Tekst i fotter
\newcommand{\footerText}{\thepage\xspace /\pageref{LastPage}}
\newcommand{\ProjectName}{433 MHz styring af AeroQuad}


\chapterstyle{hangnum}




\nouppercaseheads
\makepagestyle{mystyle} 

\makeevenhead{mystyle}{}{\\ \leftmark}{} 
\makeoddhead{mystyle}{}{\\ \leftmark}{} 
\makeevenfoot{mystyle}{}{\footerText}{} 
\makeoddfoot{mystyle}{}{\footerText}{} 
\makeatletter
\makepsmarks{mystyle}{% Overskriften på sidehovedet
  \createmark{chapter}{left}{shownumber}{\@chapapp\ }{.\ }} 
\makeatother
\makefootrule{mystyle}{\textwidth}{\normalrulethickness}{0.4pt}
\makeheadrule{mystyle}{\textwidth}{\normalrulethickness}

\makepagestyle{plain}
\makeevenhead{plain}{}{}{}
\makeoddhead{plain}{}{}{}
\makeevenfoot{plain}{}{\footerText}{}
\makeoddfoot{plain}{}{\footerText}{}
\makefootrule{plain}{\textwidth}{\normalrulethickness}{0.4pt}

\pagestyle{mystyle}

%%----------------------------------------------------------------------
%
%%Redefining chapter style
%%\renewcommand\chapterheadstart{\vspace*{\beforechapskip}}
%\renewcommand\chapterheadstart{\vspace*{10pt}}
%\renewcommand\printchaptername{\chapnamefont }%\@chapapp}
%\renewcommand\chapternamenum{\space}
%\renewcommand\printchapternum{\chapnumfont \thechapter}
%\renewcommand\afterchapternum{\space: }%\par\nobreak\vskip \midchapskip}
%\renewcommand\printchapternonum{}
%\renewcommand\printchaptertitle[1]{\chaptitlefont #1}
\setlength{\beforechapskip}{0pt} 
\setlength{\afterchapskip}{0pt} 
%\setlength{\voffset}{0pt} 
\setlength{\headsep}{25pt}
%\setlength{\topmargin}{35pt}
%%\setlength{\headheight}{102pt}
%\setlength{\textheight}{302pt}
\renewcommand\afterchaptertitle{\par\nobreak\vskip \afterchapskip}
%%----------------------------------------------------------------------




%Sidehoved og -fod pakke
%Margin
\usepackage[left=2cm,right=2cm,top=2.5cm,bottom=2cm]{geometry}
\usepackage{lastpage}



%%URL kommandoer og sidetal farve
%%Kaldes med \url{www...}
%\usepackage{color} %Skal også bruges
\usepackage{hyperref}
\hypersetup{ 
	colorlinks	= true, 	% false: boxed links; true: colored links
    urlcolor	= blue,		% color of external links
    linkcolor	= black, 	% color of page numbers
    citecolor	= blue,
}



%Mellemrum mellem linjerne    
\linespread{1.5}


%Seperated files
%--------------------------------------------------
%Opret filer således:
%\documentclass[Navn-på-hovedfil]{subfiles}
%\begin{document}
% Indmad
%\end{document}
%
% I hovedfil inkluderes således:
% \subfile{navn-på-subfil}
%--------------------------------------------------
\usepackage{subfiles}

%Prevent wierd placement of figures
%\usepackage[section]{placeins}

%Standard sti at søge efter billeder
%--------------------------------------------------
%\begin{figure}[hbtp]
%\centering
%\includegraphics[scale=1]{filnavn-for-png}
%\caption{Titel}
%\label{fig:referenceNavn}
%\end{figure}
%--------------------------------------------------
\usepackage{graphicx}
\usepackage{subcaption}
\usepackage{float}
\graphicspath{{../Figures/}}

%Speciel skrift for enkelt linje kode
%--------------------------------------------------
%Udskriver med fonten 'Courier'
%Mere info her: http://tex.stackexchange.com/questions/25249/how-do-i-use-a-particular-font-for-a-small-section-of-text-in-my-document
%Eksempel: Funktionen \code{void Hello()} giver et output
%--------------------------------------------------
\newcommand{\code}[1]{{\fontfamily{pcr}\selectfont #1}}


% Følgende er til koder.
%----------------------------------------------------------
%\begin{lstlisting}[caption=Overskrift på boks, style=Code-C++, label=lst:referenceLabel]
%public void hello(){}
%\end{lstlisting}
%----------------------------------------------------------

%Exstra space
\usepackage{xspace}
%Navn på bokse efterfulgt af \xspace (hvis det skal være mellemrum
%gives det med denne udvidelse. Ellers ingen mellemrum.
\newcommand{\codeTitle}{Kodeudsnit\xspace}

%Pakker der skal bruges til lstlisting
\usepackage{listings}
\usepackage{color}
\usepackage{textcomp}
\definecolor{listinggray}{gray}{0.9}
\definecolor{lbcolor}{rgb}{0.9,0.9,0.9}
\renewcommand{\lstlistingname}{\codeTitle}
\lstdefinestyle{Code}
{
	keywordstyle	= \bfseries\ttfamily\color[rgb]{0,0,1},
	identifierstyle	= \ttfamily,
	commentstyle	= \color[rgb]{0.133,0.545,0.133},
	stringstyle		= \ttfamily\color[rgb]{0.627,0.126,0.941},
	showstringspaces= false,
	basicstyle		= \small,
	numberstyle		= \footnotesize,
%	numbers			= left, % Tal? Udkommenter hvis ikke
	stepnumber		= 2,
	numbersep		= 6pt,
	tabsize			= 2,
	breaklines		= true,
	prebreak 		= \raisebox{0ex}[0ex][0ex]{\ensuremath{\hookleftarrow}},
	breakatwhitespace= false,
%	aboveskip		= {1.5\baselineskip},
  	columns			= fixed,
  	upquote			= true,
  	extendedchars	= true,
 	backgroundcolor = \color{lbcolor},
	lineskip		= 1pt,
%	xleftmargin		= 17pt,
%	framexleftmargin= 17pt,
	framexrightmargin	= 0pt, %6pt
%	framexbottommargin	= 4pt,
}

%Bredde der bruges til indryk
%Den skal være 6 pt mindre
\usepackage{calc}
\newlength{\mywidth}
\setlength{\mywidth}{\textwidth-6pt}


% Forskellige styles for forskellige kodetyper
\usepackage{caption}
\DeclareCaptionFont{white}{\color{white}}
\DeclareCaptionFormat{listing}%
{\colorbox[cmyk]{0.43, 0.35, 0.35,0.35}{\parbox{\mywidth}{\hspace{5pt}#1#2#3}}}
\captionsetup[lstlisting]
{
	format			= listing,
	labelfont		= white,
	textfont		= white, 
	singlelinecheck	= false, 
	width			= \mywidth,
	margin			= 0pt, 
	font			= {bf,footnotesize}
}

\lstdefinestyle{Code-C} {language=C, style=Code}
\lstdefinestyle{Code-Java} {language=Java, style=Code}
\lstdefinestyle{Code-C++} {language=[Visual]C++, style=Code}
\lstdefinestyle{Code-VHDL} {language=VHDL, style=Code}
\lstdefinestyle{Code-Bash} {language=Bash, style=Code}

%Text typesetting
%--------------------------------------------------------
%\usepackage{baskervald}
\usepackage{lmodern}
\usepackage[T1]{fontenc}              
\usepackage[utf8]{inputenc}         
\usepackage[english]{babel}       

\setlength{\parindent}{0pt}
\nonzeroparskip

%\setaftersubsecskip{1sp}
%\setaftersubsubsecskip{1sp}
 


%Dybde på indholdsfortegnelse
%----------------------------------------------------------
%Chapter, section, subsection, subsubsection
%----------------------------------------------------------
\setcounter{secnumdepth}{3}
\setcounter{tocdepth}{3}


%Tables
%----------------------------------------------------------
\usepackage{tabularx}
\usepackage{array}
\usepackage{multirow} 
\usepackage{multicol} 
\usepackage{booktabs}
\usepackage{wrapfig}
\renewcommand{\arraystretch}{1.5}



%Misc
%----------------------------------------------------------
\usepackage{cite}
\usepackage{appendix}
\usepackage{amssymb}
\usepackage{url,ragged2e}
\usepackage{enumerate}
\usepackage{amsmath} %Math bibliotek


\usepackage{longtable}

\title{Lektion 5}

\begin{document}
\maketitle


\section*{Vector space}

\begin{theo}[De ti Axioms]
A vector space is a nonempty set V of objects, called vectors, on which are defined two operations, called addition and multiplication by scalars (real numbers), subject to the ten axioms (or rules) listed below.
The axioms must hold for all vectors u, v, and w in V and for all scalars c and d

\begin{enumerate}
\item The sum of u and v, denoted by u + v, is in V .
\item u + v = v + u.
\item (u + v) + w = u + (v + w).
\item \textbf{There is a zero vector 0 in V} such that u + 0 = u.
\item For each u in V , there is a vector -u in V such that u + (-u) = 0.
\item The scalar multiple of u by c, denoted by cu, is in V .
\item c(u + v) = cu + cv.
\item (c + d)u = cu + du.
\item c(d u) = (cd)u.
\item 1u = u.

\end{enumerate}
\end{theo}

\begin{itemize}
\item \textbf{"Closed under addition"} for et set betyder, at hvis 2 tilfældige lægges sammen er resultatet i listen.

\item \textbf{"Trace"} betyder en sum af de diagonale elementer

\item $y_k = \left\{ \dots, y_{-2}, y_{-1}, y_0, y_1, y_2, \dots\right\}$ og $z_k = \left\{ \dots, z_{-2}, z_{-1}, z_0, z_1, z_2, \dots\right\}$ udgør et vector space

\item Tænk på en vektor $\mathbb{R}^3$ som et diagram med kun \textit{x-} og \textit{y-}akse, hvor hvert x-går ud af aksen og fortæller sin værdi på y
\end{itemize}


\newpage
\subsection*{Subspace}
\begin{theo}[Defination] 
A subspace of a vector space V is a subset H of V that has three properties:
\begin{enumerate}[a)]
\item The zero vector of V is in H.
\item H is closed under vector addition. That is, for each u and v in H, the sum u + v is in H.
\item H is closed under multiplication by scalars. That is, for each u in H and each scalar c, the vector cu is in H.
\end{enumerate}

\begin{itemize}
\item A subspace forms a vector space by itself.
\item Tag den Argumenterede matrix af et span og en vector og se om det kan løses. Kan det ligger det i span
\end{itemize}

\end{theo}
Med formen 
\begin{ArgMat}
a\\
b\\
a+b
\end{ArgMat}:
\\
\textbf{Addition:}
\\
\begin{ArgMat}
a_1\\
b_1\\
a_1+b_1
\end{ArgMat} +
\begin{ArgMat}
a_2\\
b_2\\
a_2+b_2
\end{ArgMat} =
\begin{ArgMat}
a_1+a_2\\
b_1+b_2\\
a_1+b_1+a_2+b_2
\end{ArgMat}=
\begin{ArgMat}
a_1+a_2\\
b_1+b_2\\
a_1+a_2+b_1+b_2
\end{ArgMat}
\\
Dette giver  en nu vektor som er \textbf{closed under addition}.
\\
\\
\textbf{Multiplication:}
\\
c
\begin{ArgMat}
a\\
b\\
a+b
\end{ArgMat}=
\begin{ArgMat}
ca\\
cb\\
c(a+b)
\end{ArgMat}=
\begin{ArgMat}
ca\\
cb\\
ca+cb
\end{ArgMat}
\\
Da det sidste er en sum af de to øverste er \textbf{closed under multiplication}
\\
\\
\textbf{Zero vector}:
\\
\begin{ArgMat}
0\\
0\\
0
\end{ArgMat}
\\
\\
Med formen
\begin{ArgMat}
a\\
b\\
1
\end{ArgMat}:
\\
\\
\textbf{Addition:}
\\
\begin{ArgMat}
a_1\\
b_1\\
1
\end{ArgMat}+
\begin{ArgMat}
a_2\\
b_2\\
1
\end{ArgMat}=
\begin{ArgMat}
a_1+a_2\\
b_1+b_2\\
2
\end{ArgMat}\\
2 er ikke \textbf{closed under addition}.
\\
\\
\textbf{Multiplication: }
\\
c\begin{ArgMat}
a\\
b\\
1
\end{ArgMat}=
\begin{ArgMat}
ca\\
cb\\
c
\end{ArgMat}\\
\textit{c} er ikke \textbf{closed under multiplication}
\\
\\
\textbf{Zero vector}:
\\
\begin{ArgMat}
0\\
0\\
1
\end{ArgMat} $\neq$ 0
\\
No zero vector





\newpage
\subsection*{Nulspace}
\begin{theo}[Definition] 
The null space of an $m \times n$ matrix A, written as \textit{Nul A}, is the set of all solutions
of the homogeneous equation Ax = 0. 
\\
\centerline{$Nul A= \{x | x \in \mathbb{R}^n $ and $Ax = 0\}$}
\\
\\
\textbf{Hvis der er frie variabler er der uendelig mange løsninger.}
\end{theo}


\begin{theo}[Theorem 2] 
The nullspace of a $m \times x$ matrix is a subspace og $\mathbb{R}^n$
\end{theo}

\textbf{Zerovector}:\\
$A\cdot \vec{0}=\vec{0}$
\\
\\
\textbf{Addition}:\\
$A\vec{u}=\vec{0}$ and $A\vec{v}=\vec{0}$\\
$A(\vec{u}+\vec{v})=A\vec{u} + A\vec{v}=\vec{0}+\vec{0}=\vec{0}$
\\
\\
\textbf{Multiplication}:\\
$A(c\vec{u}) = c(A\vec{u})=c\cdot \vec{0}=\vec{0}$
\\
\\
\textit{Find the nullspace of A} (Find A for $\vec{0}$):\\
\begin{ArgMat}
1&2&1&0\\
-1&3&1&0
\end{ArgMat}$\sim$
\begin{ArgMat}
1&2&1\\
0&5&2
\end{ArgMat}$\sim\sim$
\begin{ArgMat}
1& 0& \frac{1}{5}&0\\
0&1&\frac{2}{5}&0
\end{ArgMat}\\
\begin{ArgMat}
x_1\\
x_2\\
x_3
\end{ArgMat}=\dots







\subsection*{Columnspace}
A=\begin{ArgMat}
1&2\\
3&3\\
1&4
\end{ArgMat} (Da dette er en 3x2 er der en fri variabel.)
\\
\\
\textit{Zero vector?}\\
Yes, choose $x_!=0, x_2 = 0$ and get \begin{ArgMat}
0\\
0\\
0
\end{ArgMat}
\\
\\
\textit{Addition?}\\
$\vec{b_1}=A\vec{x_1}, \vec{b_2}=A\vec{x_2}$\\
$\vec{b_1}+\vec{b_2}=A\vec{x_1}+A\vec{x_2}=A(\vec{x_1}+\vec{x_2})$
\\
\\
\textit{Multiplication?}\\
$\vec{b}=A\vec{x}, c\vec{b}=cA\vec{x}=A(c\vec{x})$









\newpage
\subsection*{Linear tranformation}

















\end{document}