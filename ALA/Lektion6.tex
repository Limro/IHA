\documentclass[danish, english]{article}
%Preamble

% Følgende er til koder.
%----------------------------------------------------------
%\begin{lstlisting}[caption=Overskrift på boks, style=Code-C++, label=lst:referenceLabel]
%public void hello(){}
%\end{lstlisting}
%----------------------------------------------------------

%Exstra space
\usepackage{xspace}
%Navn på bokse efterfulgt af \xspace (hvis det skal være mellemrum
%gives det med denne udvidelse. Ellers ingen mellemrum.
\newcommand{\codeTitle}{Code snippet\xspace}

%Pakker der skal bruges til lstlisting
\usepackage{listings}
\usepackage{color}
\usepackage{textcomp}
\definecolor{listinggray}{gray}{0.9}
\definecolor{lbcolor}{rgb}{0.9,0.9,0.9}
\renewcommand{\lstlistingname}{\codeTitle}
\lstdefinestyle{Code}
{
	keywordstyle	= \bfseries\ttfamily\color[rgb]{0,0,1},
	identifierstyle	= \ttfamily,
	commentstyle	= \color[rgb]{0.133,0.545,0.133},
	stringstyle		= \ttfamily\color[rgb]{0.627,0.126,0.941},
	showstringspaces= false,
	basicstyle		= \small,
	numberstyle		= \footnotesize,
%	numbers			= left, % Tal? Udkommenter hvis ikke
	stepnumber		= 2,
	numbersep		= 6pt,
	tabsize			= 2,
	breaklines		= true,
	prebreak 		= \raisebox{0ex}[0ex][0ex]{\ensuremath{\hookleftarrow}},
	breakatwhitespace= false,
%	aboveskip		= {1.5\baselineskip},
  	columns			= fixed,
  	upquote			= true,
  	extendedchars	= true,
 	backgroundcolor = \color{lbcolor},
	lineskip		= 1pt,
%	xleftmargin		= 17pt,
%	framexleftmargin= 17pt,
	framexrightmargin	= 0pt, %6pt
%	framexbottommargin	= 4pt,
}

%Bredde der bruges til indryk
%Den skal være 6 pt mindre
\usepackage{calc}
\newlength{\mywidth}
\setlength{\mywidth}{1.435\textwidth} % Hvis bredden header ikke virker er dette hvad skal ændres!


% Forskellige styles for forskellige kodetyper
\usepackage{caption}
\DeclareCaptionFont{white}{\color{white}}
\DeclareCaptionFormat{listing}%
{\colorbox[cmyk]{0.43, 0.35, 0.35,0.35}{\parbox{\mywidth}{\hspace{5pt}#1#2#3}}}
\captionsetup[lstlisting]
{
	format			= listing,
	labelfont		= white,
	textfont		= white, 
	singlelinecheck	= false, 
	width			= \mywidth,
	margin			= 0pt, 
	font			= {bf,footnotesize}
}

\lstdefinestyle{Code-C} {language=C, style=Code}
\lstdefinestyle{Code-Java} {language=Java, style=Code}
\lstdefinestyle{Code-C++} {language=[Visual]C++, style=Code}
\lstdefinestyle{Code-VHDL} {language=VHDL, style=Code}
\lstdefinestyle{Code-Bash} {language=Bash, style=Code}
\lstdefinestyle{Code-Matlab} {language=Matlab, style=Code}
\lstdefinestyle{Code-Prolog} {language=Prolog, style=Code}
%Speciel skrift for enkelt linje kode
%--------------------------------------------------
%Udskriver med fonten 'Courier'
%Mere info her: http://tex.stackexchange.com/questions/25249/how-do-i-use-a-particular-font-for-a-small-section-of-text-in-my-document
%Eksempel: Funktionen \code{void Hello()} giver et output
%--------------------------------------------------
\newcommand{\code}[1]{{\fontfamily{pcr}\selectfont #1}}

%Seperated files
%--------------------------------------------------
%Opret filer således:
%\documentclass[Navn-på-hovedfil]{subfiles}
%\begin{document}
% Indmad
%\end{document}
%
% I hovedfil inkluderes således:
% \subfile{navn-på-subfil}
%--------------------------------------------------
\usepackage{subfiles}
%Text typesetting
%--------------------------------------------------------
\usepackage[T1]{fontenc} 	% Can use danish characters
\usepackage[utf8]{inputenc} % Input encoding. Can be used on Linux, Mac and Windows         
\usepackage[danish]{babel} 	% Split words accoding to English
\usepackage{lmodern} 		% Font

\setlength\parindent{0pt} 	% No indent
\setlength\parskip{12pt} 	% More than a single line break will give ONE linebreak.

%Margin
\usepackage[left=2cm,right=2cm,top=2.5cm,bottom=2cm]{geometry}

%Margin
\usepackage[left=3cm,right=2cm,top=2.5cm,bottom=2cm]{geometry}

%Mellemrum mellem linjerne    
\linespread{1.5}

\title{Assignment 2 for TEDI}
\author{Rasmus Bækgaard, 10893}
\date{April 28, 2014}
\title{Lektion 6}

\begin{document}
\maketitle

\begin{theo}[Theorem 4] 
An indexed set $\{v_1, \dots, v_p\} $of two or more vectors, with $v_1 \neq 0$, is linearly dependent if and only if some $v_j$ (with j > 1) is a linear combination of the preceding vectors, $\{v_1, \dots, v_{j-1}\} $
\end{theo}

\begin{theo} 
\begin{itemize}
\item Fri variabel betyder \textbf{linear dependant}
\item Hvis man ikke kan $c_1  \vec{v_1}+c_2\vec{v_2}+\dots+c_p\vec{v_p}=\vec{0}$ er det \textbf{linear independant}
\end{itemize}
\end{theo}

\textbf{Example:}\\
$A_1$ =
\begin{ArgMat}
1 &-1\\
2&0
\end{ArgMat}, $A_2$=
\begin{ArgMat}
2&1\\
0&3
\end{ArgMat}, $A_3$=
\begin{ArgMat}
1&-2\\
2&1
\end{ArgMat}
\\
\\
$c_1A_1 + c_2A_2 + c_3A_3 =$ 
\begin{ArgMat}
0&0\\
0&0
\end{ArgMat} $\Rightarrow c_1$
\begin{ArgMat}
1 &-1\\
2&0
\end{ArgMat} $+ c_2$
\begin{ArgMat}
2&1\\
0&3
\end{ArgMat}, $+ c_3$
\begin{ArgMat}
1&-2\\
2&1
\end{ArgMat} = 
\begin{ArgMat}
0&0\\
0&0
\end{ArgMat}
\\
\\
\begin{align*}
c_1+2c_2+c_3 &=0\\
-c_1+c_2-c_3 &= 0\\
2c_1+2c_3 &= 0\\
3c_2+c_3&=0
\end{align*}
\begin{ArgMat}
1&2&1&0\\
-1&1&-1&0\\
2&0&2&0\\
0&3&1&0
\end{ArgMat}$\sim$
\begin{ArgMat}
1&0&0&0\\
0&1&0&0\\
0&0&1&0\\
0&0&0&0
\end{ArgMat}
$c_1=c_2=c_3=0 \Leftrightarrow$ Kun den trivielle løsning, aka \textbf{linear independant}




\newpage
\subsection*{Spanning set}
\begin{ArgMat}
1&4&0&2&0\\
0&0&1&-1&0\\
0&0&0&0&1\\
0&0&0&0&0
\end{ArgMat} 
Find pivot elementer.
Ingen pivot $\Rightarrow$ fjern kolonne.
Husk dog, at hvis en kolonne ikke har pivot kan den bruges hvis den er independant.
\\
\\
Basis for col B = \{\begin{ArgMat}
1\\
0\\
0\\
0
\end{ArgMat},
\begin{ArgMat}
0\\
1\\
0\\
0
\end{ArgMat},
\begin{ArgMat}
0\\
0\\
1\\
0
\end{ArgMat} \} 
\\
\\
A=
\begin{ArgMat}
1&2&3&1\\
4&5&6&1\\
1&1&2&1
\end{ArgMat}$\sim$
\begin{ArgMat}
1&0&0&0\\
0&1&0&-1\\
0&0&1&1
\end{ArgMat}Dette er pivot elementerne. Disse kan bruges.
\\
Basis for col A = \{ \begin{ArgMat}
1\\
4\\
2
\end{ArgMat}, 
\begin{ArgMat}
2\\
5\\
1
\end{ArgMat},
\begin{ArgMat}
3\\
6\\
2
\end{ArgMat} \}



\newpage
\section*{Coordinate systems}
\begin{theo}[Indsæt definition] 
•
\end{theo}
\textbf{Example:}\\
$p=\{$\begin{ArgMat}
2\\
1\\
0
\end{ArgMat},
\begin{ArgMat}
3\\
1\\
0
\end{ArgMat},
\begin{ArgMat}
2\\
1\\
2
\end{ArgMat} \}
\\
\begin{ArgMat}
x
\end{ArgMat}$_\beta=$
\begin{ArgMat}
-1\\
2\\
1
\end{ArgMat} -- What is $\vec{x}$?
\\
\\
$\vec{x} = (-1)$\begin{ArgMat}
2\\
1\\
0
\end{ArgMat}+2
\begin{ArgMat}
3\\
1\\
0
\end{ArgMat}+1
\begin{ArgMat}
2\\
1\\
2
\end{ArgMat}=
\begin{ArgMat}
6\\
2\\
2
\end{ArgMat}
\\
\\
$\vec{x}=$
\begin{ArgMat}
-1\\
2\\
1
\end{ArgMat} What is \begin{ArgMat}
\vec{x}
\end{ArgMat}$_\beta$
\\
\\
\begin{ArgMat}
-1\\
2\\
1
\end{ArgMat}=  $c_1$
\begin{ArgMat}

\end{ArgMat}=  $+ c_2$
\begin{ArgMat}
3\\
1\\
0
\end{ArgMat}=  $+ c_3$
\begin{ArgMat}
-1\\
2\\
1
\end{ArgMat} $\Rightarrow$




\subsection*{Transistion matrix}
$P_\beta=\left[\vec{1}_1 \dots \vec{b}_n\right]$


\begin{theo}[Isomorphism] 
One-to-one linear transformation from a vectorspace \textit{V} to a vectorspace \textit{W}
\end{theo}

$\mathbb{P}_3$\\
$\beta = \{1, t, t^2, t^3\}$\\
$p(t)=a_0+a_1t+a_2t^2+a_3t^3$

\begin{align*}
p_1(t)=1+2t^2\\
p_2(t)=4+t+5t^2\\
p_3(t)=3+2t
\end{align*}
$\left[P\right]_\beta = $
\begin{ArgMat}
a_0\\
a_1\\
a_2\\
a_3
\end{ArgMat}
\\
$\left[P_1\right]_\beta = $
\begin{ArgMat}
1\\
0\\
2
\end{ArgMat}
,
$[P_2]_\beta = $
\begin{ArgMat}
4\\
1\\
5
\end{ArgMat}
,
$[P_3]_\beta = $
\begin{ArgMat}
3\\
2\\
0
\end{ArgMat}
\\
\\
\begin{ArgMat}
1&4&3&0\\
0&1&2&0\\
2&5&0&0
\end{ArgMat}$\sim$
\begin{ArgMat}
1&4&3&0\\
0&1&2&0\\
0&0&0&0
\end{ArgMat}. Her er free variables = $\infty$ solutions = lin. dep.
\\
\\
$M_{2 \times 2} = $\begin{ArgMat}
a &b\\
c&d
\end{ArgMat} -- samme som 
\begin{ArgMat}
1\\
b\\
c\\
d
\end{ArgMat}
\\
$M_{2 \times 2}$ og $\mathbb{R}^4$ er isomorphism.

















\end{document}