\documentclass[danish,english]{article}
%Preamble

% Følgende er til koder.
%----------------------------------------------------------
%\begin{lstlisting}[caption=Overskrift på boks, style=Code-C++, label=lst:referenceLabel]
%public void hello(){}
%\end{lstlisting}
%----------------------------------------------------------

%Exstra space
\usepackage{xspace}
%Navn på bokse efterfulgt af \xspace (hvis det skal være mellemrum
%gives det med denne udvidelse. Ellers ingen mellemrum.
\newcommand{\codeTitle}{Code snippet\xspace}

%Pakker der skal bruges til lstlisting
\usepackage{listings}
\usepackage{color}
\usepackage{textcomp}
\definecolor{listinggray}{gray}{0.9}
\definecolor{lbcolor}{rgb}{0.9,0.9,0.9}
\renewcommand{\lstlistingname}{\codeTitle}
\lstdefinestyle{Code}
{
	keywordstyle	= \bfseries\ttfamily\color[rgb]{0,0,1},
	identifierstyle	= \ttfamily,
	commentstyle	= \color[rgb]{0.133,0.545,0.133},
	stringstyle		= \ttfamily\color[rgb]{0.627,0.126,0.941},
	showstringspaces= false,
	basicstyle		= \small,
	numberstyle		= \footnotesize,
%	numbers			= left, % Tal? Udkommenter hvis ikke
	stepnumber		= 2,
	numbersep		= 6pt,
	tabsize			= 2,
	breaklines		= true,
	prebreak 		= \raisebox{0ex}[0ex][0ex]{\ensuremath{\hookleftarrow}},
	breakatwhitespace= false,
%	aboveskip		= {1.5\baselineskip},
  	columns			= fixed,
  	upquote			= true,
  	extendedchars	= true,
 	backgroundcolor = \color{lbcolor},
	lineskip		= 1pt,
%	xleftmargin		= 17pt,
%	framexleftmargin= 17pt,
	framexrightmargin	= 0pt, %6pt
%	framexbottommargin	= 4pt,
}

%Bredde der bruges til indryk
%Den skal være 6 pt mindre
\usepackage{calc}
\newlength{\mywidth}
\setlength{\mywidth}{1.435\textwidth} % Hvis bredden header ikke virker er dette hvad skal ændres!


% Forskellige styles for forskellige kodetyper
\usepackage{caption}
\DeclareCaptionFont{white}{\color{white}}
\DeclareCaptionFormat{listing}%
{\colorbox[cmyk]{0.43, 0.35, 0.35,0.35}{\parbox{\mywidth}{\hspace{5pt}#1#2#3}}}
\captionsetup[lstlisting]
{
	format			= listing,
	labelfont		= white,
	textfont		= white, 
	singlelinecheck	= false, 
	width			= \mywidth,
	margin			= 0pt, 
	font			= {bf,footnotesize}
}

\lstdefinestyle{Code-C} {language=C, style=Code}
\lstdefinestyle{Code-Java} {language=Java, style=Code}
\lstdefinestyle{Code-C++} {language=[Visual]C++, style=Code}
\lstdefinestyle{Code-VHDL} {language=VHDL, style=Code}
\lstdefinestyle{Code-Bash} {language=Bash, style=Code}
\lstdefinestyle{Code-Matlab} {language=Matlab, style=Code}
\lstdefinestyle{Code-Prolog} {language=Prolog, style=Code}
%Speciel skrift for enkelt linje kode
%--------------------------------------------------
%Udskriver med fonten 'Courier'
%Mere info her: http://tex.stackexchange.com/questions/25249/how-do-i-use-a-particular-font-for-a-small-section-of-text-in-my-document
%Eksempel: Funktionen \code{void Hello()} giver et output
%--------------------------------------------------
\newcommand{\code}[1]{{\fontfamily{pcr}\selectfont #1}}

%Seperated files
%--------------------------------------------------
%Opret filer således:
%\documentclass[Navn-på-hovedfil]{subfiles}
%\begin{document}
% Indmad
%\end{document}
%
% I hovedfil inkluderes således:
% \subfile{navn-på-subfil}
%--------------------------------------------------
\usepackage{subfiles}
%Text typesetting
%--------------------------------------------------------
\usepackage[T1]{fontenc} 	% Can use danish characters
\usepackage[utf8]{inputenc} % Input encoding. Can be used on Linux, Mac and Windows         
\usepackage[danish]{babel} 	% Split words accoding to English
\usepackage{lmodern} 		% Font

\setlength\parindent{0pt} 	% No indent
\setlength\parskip{12pt} 	% More than a single line break will give ONE linebreak.

%Margin
\usepackage[left=2cm,right=2cm,top=2.5cm,bottom=2cm]{geometry}

%Margin
\usepackage[left=3cm,right=2cm,top=2.5cm,bottom=2cm]{geometry}

%Mellemrum mellem linjerne    
\linespread{1.5}

\title{Assignment 2 for TEDI}
\author{Rasmus Bækgaard, 10893}
\date{April 28, 2014}

\title{Lektion 8}

\begin{document}
\maketitle

\section*{Eigenvectors}
$A\vec{x} = \vec{b}$ og $A\vec{x} = \vec{o}$\\
$A\vec{x} = \lambda\vec{x}$ og $A=PDP^{-1}$
\\
\\
\textbf{Example:}\\
A=
\begin{ArgMat}
5&-6\\
2&-2
\end{ArgMat}
\\
\\
A\begin{ArgMat}
3\\
2
\end{ArgMat} = 1
\begin{ArgMat}
3\\
2
\end{ArgMat}, hvor 1 = $\lambda$
\\
\\
A\begin{ArgMat}
6\\
4
\end{ArgMat} = 1
\begin{ArgMat}
6\\
4
\end{ArgMat}, hvor 1 = $\lambda$
\\
\\
A\begin{ArgMat}
2\\
1
\end{ArgMat} = 1
\begin{ArgMat}
4\\
2
\end{ArgMat} = 2
\begin{ArgMat}
2\\
1
\end{ArgMat}, hvor 2 = $\lambda$


\begin{theo}[Definition] 
If $\vec{v}$ is an eigevector $c\vec{v}$ is also an eigenvector.
\end{theo}

\begin{theo}[Definition] 
An eigenvector of an $n \times n$ matrix \textit{A} is a nonzero vector \textit{x} such that $Ax = \lambda x$ for some scalar $\lambda$. 
A scalar $\lambda$ is called an eigenvalue of \textit{A} if there is a nontrivial solution \textit{x} of $Ax = \lambda x$ such an \textit{x} is called an \textbf{eigenvector corresponding} to $\lambda$
\end{theo}


\begin{theo}[Regneregler] 
\begin{align*}
A\vec{x}&=\lambda I \vec{x}\\
A\vec{x}-\lambda I \vec{x} &= \vec{0}\\
(A-\lambda I)\vec{x}&=\vec{0}
\end{align*}
\begin{itemize}
\item Find $A-\lambda I$
\item Lav
\begin{ArgMat}
A-\lambda I&|\vec{0}
\end{ArgMat}
\item Isoler $x_1$ og $x_2$
\item Opskriv
\begin{ArgMat}
x_1\\
x_2
\end{ArgMat}= $x_1$
\begin{ArgMat}
a_1\\
b_1
\end{ArgMat}+$x_2$
\begin{ArgMat}
a_2\\
b_2
\end{ArgMat} (Null space af $A-\lambda I$)
\item Find $\vec{v}_2$
\end{itemize}

Find \textbf{Subspace}:\\
\begin{enumerate}
\item Zerovector: $A\vec{0}=\lambda\vec{0} \leftrightarrow \vec{0} = 0\vec{v}$
\item Multiplication: 
\item Addition: 
\end{enumerate}
\textbf{Opløftet:}\\
$A\vec{x} =\lambda \vec{x}\\
A^2\vec{x}=A(A\vec{x}) = A(\lambda \vec{x}) = \lambda(A\vec{x})= \lambda^2\vec{x}\\
A^n\vec{x} = \lambda^n\vec{x}$
\end{theo}

A with $\lambda_1$ and $\lambda_2$ with corresponding eigenvectos $\vec{v}_1$ and $\vec{v}_2$\\
$A\vec{v}_1=\lambda_1\vec{v}_1$\\
$A\vec{v}_2=\lambda_2\vec{v}_2$
\\
\\
$A(c\vec{v}_1+d\vec{v}_2 = c\lambda_1 \vec{v}_1+d\lambda_2 \vec{v}_2$\\
$A^2(c\vec{v}_1+d\vec{v}_2 = c\lambda_1^2\vec{v}_1+d\lambda_2^2\vec{v}_2$



\begin{theo}[Properbility vector] 
$\vec{x}_{k+1} = A\vec{x}_k $ (Markov chain)\\
$\vec{x}=$
\begin{ArgMat}
x_1\\
\dots\\
x_n
\end{ArgMat},
$\sum\limits_{i=1}^n x_u = 1$
\end{theo}

$A\vec{x} = \lambda\vec{x}\\
(A-\lambda I)\vec{x}=\vec{0}$
\\
Find nontrivial $\vec{x}$'s therefore, choose $\lambda$ so $det(A-\lambda I)=0$
\\
\\
A=
\begin{ArgMat}
3&2\\
3&8
\end{ArgMat}\\
Form $A-\lambda I=$
\begin{ArgMat}
3&2\\
3&8
\end{ArgMat}$-\lambda$
\begin{ArgMat}
1&0\\
0&1
\end{ArgMat} =
\begin{ArgMat}
3-\lambda&2\\
3&8-\lambda
\end{ArgMat}\\
\begin{align*}
det(A-\lambda I)&=(3-\lambda)(8-\lambda)-2\cdot 3 &=0\\	
&=\lambda^2-11\lambda+25-6 &= 0\\
&=\lambda^2-11\lambda+18 &= 0\\
\lambda &= \dfrac{11 +- \sqrt{-(11)^2-4\cdot 1\cdot 18}}{2\cdot 1} &= \left\{2 under(1)\right\}\\
\lambda &= 2
\end{align*}
\begin{ArgMat}
A-\lambda I&|\vec{0}
\end{ArgMat}=
\begin{ArgMat}
1&2&0\\
3&6&0
\end{ArgMat}$\sim$
\begin{ArgMat}
1&2&0\\
0&0&0
\end{ArgMat} (fri variabel)
\\
$x_1+2x_2=0$ and $x_2=x_2$\\
\begin{ArgMat}
x_1\\
x_2
\end{ArgMat}=$x_2$
\begin{ArgMat}
-2\\
1
\end{ArgMat}
\\
\\
\\
$A^k$\\
Nemt, hvis \textit{A} er diagonal:\\
A=
\begin{ArgMat}
x&0&0\\
0&y&0\\
0&0&z
\end{ArgMat}, $A^2 =$
\begin{ArgMat}
x^2 & 0&0\\
0&y^2&0\\
0&0&z^2
\end{ArgMat}, $A^k =$
\begin{ArgMat}
x^k & 0&0\\
0&y^k&0\\
0&0&z^k
\end{ArgMat}
\\
\\
\\
If $A=PDP^{-1}$, $A^k$ is easy:\\
$A^k=A\cdot A \dots \cdot A= (PDP^{-1})(PDP^{-1}) \dots(PDP^{-1}) = PD^k P^{-1}$

\begin{theo}[Regneregel] 
\begin{itemize}
\item $A=PDP^{-1}  \Leftrightarrow AP=PD$
\item $AP=[\lambda_1 \vec{v}_1 \dots \lambda_n \vec{v}_n]$
\item $PD=[\lambda_1 \vec{v}_1 \dots \lambda_n \vec{v}_n]$
\item $A\vec{x}=PDP^{-1}\vec{x} = PD[\vec{x}]_{tV}$
\end{itemize}
\end{theo}



























\end{document}