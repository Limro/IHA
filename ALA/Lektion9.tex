\documentclass[danish, english]{article}
%Preamble

% Følgende er til koder.
%----------------------------------------------------------
%\begin{lstlisting}[caption=Overskrift på boks, style=Code-C++, label=lst:referenceLabel]
%public void hello(){}
%\end{lstlisting}
%----------------------------------------------------------

%Exstra space
\usepackage{xspace}
%Navn på bokse efterfulgt af \xspace (hvis det skal være mellemrum
%gives det med denne udvidelse. Ellers ingen mellemrum.
\newcommand{\codeTitle}{Code snippet\xspace}

%Pakker der skal bruges til lstlisting
\usepackage{listings}
\usepackage{color}
\usepackage{textcomp}
\definecolor{listinggray}{gray}{0.9}
\definecolor{lbcolor}{rgb}{0.9,0.9,0.9}
\renewcommand{\lstlistingname}{\codeTitle}
\lstdefinestyle{Code}
{
	keywordstyle	= \bfseries\ttfamily\color[rgb]{0,0,1},
	identifierstyle	= \ttfamily,
	commentstyle	= \color[rgb]{0.133,0.545,0.133},
	stringstyle		= \ttfamily\color[rgb]{0.627,0.126,0.941},
	showstringspaces= false,
	basicstyle		= \small,
	numberstyle		= \footnotesize,
%	numbers			= left, % Tal? Udkommenter hvis ikke
	stepnumber		= 2,
	numbersep		= 6pt,
	tabsize			= 2,
	breaklines		= true,
	prebreak 		= \raisebox{0ex}[0ex][0ex]{\ensuremath{\hookleftarrow}},
	breakatwhitespace= false,
%	aboveskip		= {1.5\baselineskip},
  	columns			= fixed,
  	upquote			= true,
  	extendedchars	= true,
 	backgroundcolor = \color{lbcolor},
	lineskip		= 1pt,
%	xleftmargin		= 17pt,
%	framexleftmargin= 17pt,
	framexrightmargin	= 0pt, %6pt
%	framexbottommargin	= 4pt,
}

%Bredde der bruges til indryk
%Den skal være 6 pt mindre
\usepackage{calc}
\newlength{\mywidth}
\setlength{\mywidth}{1.435\textwidth} % Hvis bredden header ikke virker er dette hvad skal ændres!


% Forskellige styles for forskellige kodetyper
\usepackage{caption}
\DeclareCaptionFont{white}{\color{white}}
\DeclareCaptionFormat{listing}%
{\colorbox[cmyk]{0.43, 0.35, 0.35,0.35}{\parbox{\mywidth}{\hspace{5pt}#1#2#3}}}
\captionsetup[lstlisting]
{
	format			= listing,
	labelfont		= white,
	textfont		= white, 
	singlelinecheck	= false, 
	width			= \mywidth,
	margin			= 0pt, 
	font			= {bf,footnotesize}
}

\lstdefinestyle{Code-C} {language=C, style=Code}
\lstdefinestyle{Code-Java} {language=Java, style=Code}
\lstdefinestyle{Code-C++} {language=[Visual]C++, style=Code}
\lstdefinestyle{Code-VHDL} {language=VHDL, style=Code}
\lstdefinestyle{Code-Bash} {language=Bash, style=Code}
\lstdefinestyle{Code-Matlab} {language=Matlab, style=Code}
\lstdefinestyle{Code-Prolog} {language=Prolog, style=Code}
%Speciel skrift for enkelt linje kode
%--------------------------------------------------
%Udskriver med fonten 'Courier'
%Mere info her: http://tex.stackexchange.com/questions/25249/how-do-i-use-a-particular-font-for-a-small-section-of-text-in-my-document
%Eksempel: Funktionen \code{void Hello()} giver et output
%--------------------------------------------------
\newcommand{\code}[1]{{\fontfamily{pcr}\selectfont #1}}

%Seperated files
%--------------------------------------------------
%Opret filer således:
%\documentclass[Navn-på-hovedfil]{subfiles}
%\begin{document}
% Indmad
%\end{document}
%
% I hovedfil inkluderes således:
% \subfile{navn-på-subfil}
%--------------------------------------------------
\usepackage{subfiles}
%Text typesetting
%--------------------------------------------------------
\usepackage[T1]{fontenc} 	% Can use danish characters
\usepackage[utf8]{inputenc} % Input encoding. Can be used on Linux, Mac and Windows         
\usepackage[danish]{babel} 	% Split words accoding to English
\usepackage{lmodern} 		% Font

\setlength\parindent{0pt} 	% No indent
\setlength\parskip{12pt} 	% More than a single line break will give ONE linebreak.

%Margin
\usepackage[left=2cm,right=2cm,top=2.5cm,bottom=2cm]{geometry}

%Margin
\usepackage[left=3cm,right=2cm,top=2.5cm,bottom=2cm]{geometry}

%Mellemrum mellem linjerne    
\linespread{1.5}

\title{Assignment 2 for TEDI}
\author{Rasmus Bækgaard, 10893}
\date{April 28, 2014}
\title{Lektion9}


\begin{document}
\maketitle
\section*{Complex eigenvalues}
Disse optræder normalt som "uheld".
F.eks, hvis man tager determinanten af og kvardratroden er negativ; 
$\dfrac{4 \pm \sqrt{-36}}{2}$
\begin{theo} 
\begin{itemize}
\item Hvis man laver en 
\begin{ArgMat}
A-\lambda I&|0
\end{ArgMat}
 hvor $\lambda = 2-3i$ ved vi, at de rækkerne er \textbf{linear dependant}
 
\item Hvis man har en matrix med kun reelle tal og én eigenvalue i komplex \textbf{kan det ikke gå op}.
\end{itemize}
\end{theo}

\textbf{Eksemple:}
\\
\begin{ArgMat}
A-\lambda I&|0
\end{ArgMat} =
\begin{ArgMat}
-1+3i&5&0\\
-2&1+3i&0
\end{ArgMat}\\
$x_2$ = free variabel:\\
$-2x_1+(1+3i)x_2=0, x_2=x_2 \Leftrightarrow x_1 \dfrac{1+3i}{2}x_2, x_2=x_2 \Rightarrow$
\begin{ArgMat}
x_1\\
x_2
\end{ArgMat}$=x_2$
\begin{ArgMat}
\dfrac{1+3i}{2}\\
1
\end{ArgMat}=
\begin{ArgMat}
1+3i\\
2
\end{ArgMat}=$\vec{v_2}$


\begin{theo}[Complex eigenvector] 
$A\vec{x}^* = \lambda^*\vec{x}^*$ er en ny eigenvalue equation.\\
$\lambda^*$ er en eigenvalue med eigenvector $\vec{x}^*$.
\end{theo}


$A=PDP^{-1}\\
C=$
\begin{ArgMat}
a&-b\\
b&a
\end{ArgMat}, a, b $\in \mathbb{R}$ og $a \neq 0, b\neq 0$\\
$r=|\lambda|=\sqrt{a^2+b^2}$\\
$det(C-\lambda I) =(a-\lambda)^2+b^2=0 \Rightarrow \lambda=a \pm ib$

$C=r\cdot I$
\begin{ArgMat}
\frac{a}{r} & \frac{-b}{r}\\
\frac{b}{r} & \frac{a}{r}
\end{ArgMat}=
\begin{ArgMat}
r&0\\
0&r
\end{ArgMat}
\begin{ArgMat}
cos \phi &-sin\phi\\
sin\phi &cos\phi
\end{ArgMat}\\
Første del er scaling, anden del er rotation


\begin{theo}[•] 
$\dfrac{dz}{dt}=xa, x(t)c\cdot e^{a\cdot t}\\
x_n' = a_{n1}x_1+\dots+a_{nn}x_n\\
\vec{x}'=A\vec{x}\\
\vec{x}'=\vec{x}'(t)=$
\begin{ArgMat}
x_1'(t)\\
\dots\\
x_n'(t)
\end{ArgMat}
\end{theo}




























\end{document}