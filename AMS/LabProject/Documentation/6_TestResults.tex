\documentclass[Aflevering]{subfiles}
\begin{document}

\section{Testresultater}

Da systemet blev udviklet, var det vigtigt at teste koden og komponenter, som tingene blev kodet og udviklet.
Til at starte med, blev de enkelte dele af systemet testet hver for sig, s� man var sikker p�, den enkelte del virkede, f�r den blev koblet sammen med noget andet. 
F.eks. blev det v�rdier fra LM75 testet via seriel forbindelse gennem RS232 til en computer. 
Med programmet Tera Term, kan man se, om der kommer nogle v�rdier ud, og om de passer nogenlunde overens med realiteten. 
De andre dele af systemet(GSM, Lcd162 og buzzer), blev ogs� testet individuelt inden nogle tasks blev sat sammen.
\\
\\
N�r alle tasks virkede individuelt, var det n�ste skridt at koble nogle sammen. 
For at kunne holde �je med, hvor i koden man befandt sig n�r der opstod fejl, blev der gjort brug af LED driveren. 
Denne LED driver indeholder en simpelt toggle funktion, som f�r en bestemt LED til at skifte tilstand n�r den kaldes. 
Et eksempel ses i listing \ref{lst:modem}


\begin{lstlisting}[style=code-C, caption=ModemOutput med LED-test, label=lst:modem]
void ModemOutput(void *pvParameters)
{
	char buffer = 0;
	char flag = 1;
	
	while(1)
	{
		while(!CharReady())
		{
			vTaskDelay(100);
			toggleLED(1);
		}
		LCDGotoXY(0,1);
	
		while(flag)
		{
			toggleLED(2);
			buffer = ReadChar();
			if(buffer == '\r')
			{
				flag = 0;
			}
			LCDDispChar(buffer);
		}
		
		flag = 1;	
		toggleLED(3);
		vTaskDelay(50);
	}	
}
\end{lstlisting}


\end{document} 