\documentclass[Aflevering]{subfiles}
\begin{document}

\section{Testresultater}

Da systemet blev udviklet, var det vigtigt at teste koden og komponenter, som tingene blev kodet og udviklet. 
Til at starte med blev de enkelte dele af systemet testet hver for sig, s� man var sikker p�, den enkelte del virkede, f�r den blev koblet sammen med andre dele. 
F.eks. blev v�rdierne fra LM75 testet via seriel forbindelse gennem RS232 til en computer. 
Med programmet \textit{Tera Term}, kan man se, om der kommer nogle v�rdier ud, og om de passer nogenlunde overens med realiteten. 
De andre dele af systemet(GSM, LCD162 og Buzzer), blev ogs� testet individuelt inden nogle tasks blev sat sammen.
\\
\\
N�r alle tasks virkede individuelt, var det n�ste skridt at teste nogle samlet. 
For at kunne holde �je med, hvor i koden man befandt sig, n�r der opstod fejl, var det brugbart at bruge LED-driveren. 
Denne indeholder en simpelt toggle funktion, som f�r en bestemt LED til at skifte n�r denne kaldes. En test kan ses for funktionen \code{ModemOutput} p� Listing \ref{lst:modem}.
Hvis programmet skulle g� i st� et sted, i f.eks. en deadlock, er det ogs� muligt at se, hvor i programmet man er kommet til via LEDs.
\\
\\
Alle individuelle tests, samt enkelte sammensatte tests virkede efter hensigten. 
Problemet kom f�rst, da alt blev sat sammen. 
Det var dog ikke rigtigt muligt at teste her, pga. hukommelses plads. 
Selvom nogle �ndringer blev lavet i \code{FreeRTOSconfig.h} filen, s� det kunne compile, ville programmet slet ikke starte p� Mega32-processoren.
\\
\\
For at sikre, at den indbyggede timing i FreeRTOS ikke blev �ndret af tilf�jet kode\footnote{FreeRTOS bruger Timer1 til at schedulere med.}, blev der oprettet en simpel task. 
Den task havde kun �n funktion, nemlig at toggle LED 0 hvert 100 millisekund. 
Herved var det hurtigt at opdage, hvis programmet fra start ikke k�rte, hvis programmet pludseligt �ndrede timing eller programmet gik i st�. 
Denne task blev slettet, da hukommelsespladsen l�b op.


\begin{lstlisting}[style=code-C, caption=ModemOutput med LED-test, label=lst:modem]
void ModemOutput(void *pvParameters)
{
	char buffer = 0;
	char flag = 1;
	
	while(1)
	{
		while(!CharReady())
		{
			vTaskDelay(100);
			toggleLED(1);
		}
		LCDGotoXY(0,1);
	
		while(flag)
		{
			toggleLED(2);
			buffer = ReadChar();
			if(buffer == '\r')
			{
				flag = 0;
			}
			LCDDispChar(buffer);
		}
		
		flag = 1;	
		toggleLED(3);
		vTaskDelay(50);
	}	
}
\end{lstlisting}


\end{document} 