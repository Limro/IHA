\documentclass[Notes]{subfiles}
\begin{document}
\section{I2C}

Generelt:
	\begin{itemize}
	\item "Two-wire interface - �n for clock - �n for data
	\item Forskellige versioner fra 100 kHz til 5 MHz
	\item Short-range
	\item Bruger ACK /NACK protokol
	\item 
	\end{itemize}
Opbygningen:
	\begin{itemize}
	\item Hver slave har et ID
	\item Hver forbundet med en clock og en data
	\item Man bruger pull-ups
	\item Signalet bliver forsinket ved l�ngere afstand
	\end{itemize}
Software:

	\begin{itemize}
	\item Pull-low for datapin og s� datapin betyder "nu sker der noget"
	\item Herefter sendes addressen til modtageren (typisk 7-bit og det sidste er RW)
	\item N�r der er sendt 8 bit ventes p� ACK
	\item Stop signaleres ved at s�tte clock og data h�j
	\end{itemize}
ACK:

	\begin{itemize}
	\item Sender ACK for hver byte (8 bit)
	\item Hvis en slave ser en addresse der ikke er dens egen bliver den ved med at lytte.
	\item Hvis det er dens adresse, sender den et ACK retur. Herefter sendes master data / lytter p� data
	\end{itemize}
Registers:

	\begin{itemize}
	\item TWBR
	\item TWCR
	
		\begin{itemize}
		\item TWINT - interrupt pin. S�ttes til 1 n�r en bus operation ends
		\item TWEA - S�t til 1 hvis der skal genereres ACK ved n�ste modtagesle
		\item TWSTA - 1 for at starte
		\item TWSTO - 1 generer stop
		\item TWWC s�t til 1 for
		\item TWEN - SKAL V�RE 1
		\end{itemize}
	
	
	\item TWSR
	
		\begin{itemize}
		\item Bit 7-3: Status p� I2C bus
		\item Bit 1-0: Clock Prescaler
		\item 1, 4, 16, 64 (00, 01, 10, 11)
		\end{itemize}
	
	
	\item TWDR
	\item TWAR
	
		\begin{itemize}
		\item Kan automatisk registrere adressen p� slaves
		\end{itemize}
	
	
	\end{itemize}












\end{document} 