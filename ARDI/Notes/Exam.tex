\documentclass{article}

%Preamble

% Følgende er til koder.
%----------------------------------------------------------
%\begin{lstlisting}[caption=Overskrift på boks, style=Code-C++, label=lst:referenceLabel]
%public void hello(){}
%\end{lstlisting}
%----------------------------------------------------------

%Exstra space
\usepackage{xspace}
%Navn på bokse efterfulgt af \xspace (hvis det skal være mellemrum
%gives det med denne udvidelse. Ellers ingen mellemrum.
\newcommand{\codeTitle}{Code snippet\xspace}

%Pakker der skal bruges til lstlisting
\usepackage{listings}
\usepackage{color}
\usepackage{textcomp}
\definecolor{listinggray}{gray}{0.9}
\definecolor{lbcolor}{rgb}{0.9,0.9,0.9}
\renewcommand{\lstlistingname}{\codeTitle}
\lstdefinestyle{Code}
{
	keywordstyle	= \bfseries\ttfamily\color[rgb]{0,0,1},
	identifierstyle	= \ttfamily,
	commentstyle	= \color[rgb]{0.133,0.545,0.133},
	stringstyle		= \ttfamily\color[rgb]{0.627,0.126,0.941},
	showstringspaces= false,
	basicstyle		= \small,
	numberstyle		= \footnotesize,
%	numbers			= left, % Tal? Udkommenter hvis ikke
	stepnumber		= 2,
	numbersep		= 6pt,
	tabsize			= 2,
	breaklines		= true,
	prebreak 		= \raisebox{0ex}[0ex][0ex]{\ensuremath{\hookleftarrow}},
	breakatwhitespace= false,
%	aboveskip		= {1.5\baselineskip},
  	columns			= fixed,
  	upquote			= true,
  	extendedchars	= true,
 	backgroundcolor = \color{lbcolor},
	lineskip		= 1pt,
%	xleftmargin		= 17pt,
%	framexleftmargin= 17pt,
	framexrightmargin	= 0pt, %6pt
%	framexbottommargin	= 4pt,
}

%Bredde der bruges til indryk
%Den skal være 6 pt mindre
\usepackage{calc}
\newlength{\mywidth}
\setlength{\mywidth}{1.435\textwidth} % Hvis bredden header ikke virker er dette hvad skal ændres!


% Forskellige styles for forskellige kodetyper
\usepackage{caption}
\DeclareCaptionFont{white}{\color{white}}
\DeclareCaptionFormat{listing}%
{\colorbox[cmyk]{0.43, 0.35, 0.35,0.35}{\parbox{\mywidth}{\hspace{5pt}#1#2#3}}}
\captionsetup[lstlisting]
{
	format			= listing,
	labelfont		= white,
	textfont		= white, 
	singlelinecheck	= false, 
	width			= \mywidth,
	margin			= 0pt, 
	font			= {bf,footnotesize}
}

\lstdefinestyle{Code-C} {language=C, style=Code}
\lstdefinestyle{Code-Java} {language=Java, style=Code}
\lstdefinestyle{Code-C++} {language=[Visual]C++, style=Code}
\lstdefinestyle{Code-VHDL} {language=VHDL, style=Code}
\lstdefinestyle{Code-Bash} {language=Bash, style=Code}
\lstdefinestyle{Code-Matlab} {language=Matlab, style=Code}
\lstdefinestyle{Code-Prolog} {language=Prolog, style=Code}
%Speciel skrift for enkelt linje kode
%--------------------------------------------------
%Udskriver med fonten 'Courier'
%Mere info her: http://tex.stackexchange.com/questions/25249/how-do-i-use-a-particular-font-for-a-small-section-of-text-in-my-document
%Eksempel: Funktionen \code{void Hello()} giver et output
%--------------------------------------------------
\newcommand{\code}[1]{{\fontfamily{pcr}\selectfont #1}}

%Seperated files
%--------------------------------------------------
%Opret filer således:
%\documentclass[Navn-på-hovedfil]{subfiles}
%\begin{document}
% Indmad
%\end{document}
%
% I hovedfil inkluderes således:
% \subfile{navn-på-subfil}
%--------------------------------------------------
\usepackage{subfiles}
%Text typesetting
%--------------------------------------------------------
\usepackage[T1]{fontenc} 	% Can use danish characters
\usepackage[utf8]{inputenc} % Input encoding. Can be used on Linux, Mac and Windows         
\usepackage[danish]{babel} 	% Split words accoding to English
\usepackage{lmodern} 		% Font

\setlength\parindent{0pt} 	% No indent
\setlength\parskip{12pt} 	% More than a single line break will give ONE linebreak.

%Margin
\usepackage[left=2cm,right=2cm,top=2.5cm,bottom=2cm]{geometry}

%Margin
\usepackage[left=3cm,right=2cm,top=2.5cm,bottom=2cm]{geometry}

%Mellemrum mellem linjerne    
\linespread{1.5}

\title{Assignment 2 for TEDI}
\author{Rasmus Bækgaard, 10893}
\date{April 28, 2014}

\begin{document}
\section{Reactor Pattern}

The Reactor architectural pattern allows event-driven applications to demultiplex \& dispatch service requests that are delivered to an application from one or more clients.

Generelt:
\begin{itemize}
	\item Bruges primært til servere, der \textit{venter} på events.
	Den checker dog hele tiden på, om der er kommet noget nyt.
	\item \code{Reactor} håndtere events til forskellige processer
	\item Modtages synkront og serielt
	\item Hvert event sorteres af en \code{Multiplex}er (baseret på indhold) og uddeles af en \code{Dispatcher} til en process. 
	Alle applikationsspecifike funtioner udføres af specifikke \code{EventHandler}s.
	\item Inversion of Control bruges (dependency (interface) injection) og gøre koden generisk. Se \codeTitle \ref{lst:DI}.
\end{itemize}

Implementering:
\begin{itemize}
	\item Klassen \code{Reactor} bruger GoF Bridge pattern (pointer til implementering -- også kaldet \emph{pimpl idiom}, da der kun er én implementering). 
	\item Klassen \code{Reactor} bruges som en Singleton.
	\item Klassen \code{Demultiplexter}'s \code{select()} har "file descripter set"    parametre. 
	Disse checkes for \code{READ}, \code{WRITE}, exceptions og time outs.
\end{itemize}

Fordele:
\begin{itemize}
	\item Separation of concerns
	\item[] Afkobler \code{Demux} og \code{Dispatcher} fra applikationsspecifik kode
	
	\item Modularity and reuseability
	\item[] Hvert event er sit eget komponent.

	\item Portability
	\item[] \code{Reactor} er frakoblet low-level kode og kan bruges af højre levels.

	\item Har et super godt samspil med \code{Wrapper Facade} pattern til low-level.
\end{itemize}

Ulemper:
\begin{itemize}
	\item Semi-svær at debugge.
	\item Single-threaded applikationer standser \code{Reactor} til de er færdige.
\end{itemize}

Andet:
\begin{itemize}
	\item \code{Reactor} kan bruges på clienten, hvis denne har nogle faste mønstre der kan sendes og modtages data i.
	I modsætning til serveren vil den kun kalde \code{HandleEvent()} én gang og exite.
	\item Sammensæt med Acceptor pattern, således serveren kalder accepter, og den registrere et gyldigt event.
	\item \code{Proactor}\fxnote{Skriv om dette}
\end{itemize}

\begin{lstlisting}[caption=Bridge / interface injection, style=Code-C++, label=lst:DI]
Reactor(IClass *impl)
{ impl->getData(); }
\end{lstlisting}

\begin{figure}[hbtp]
\centering
\includegraphics[width=0.9 \textwidth]{Reactor}
\end{figure}





\newpage
\section{Paradigms for Distributed Communication}

Space, time and flow:
\begin{itemize}
	\item Space decoupling betyder, at en publisher ikke kender til sine subscribers.
	Der er en service der klarer dette for den.

	\item Time decoupling betyder, at beskeder fra en publisher ikke skal komme per omgående til subscriber.
	Når først pubslisher udgiver, kan der gå tid til subscriber modtager.

	\item Flow decoupling betyder, at den forstyrrer når den vil?\fxnote{Check om dette er rigtigt}
\end{itemize}


RMI og publish/subscriber:
\begin{itemize}
	\item RMI -- Remote Method Invocation betyder, at du over et netværk kan aktivere en kommando, der bliver eksikveret (og muligvis sendt tilbage).

	\begin{itemize}
	 	\item Understøttes af Java, CORBA og Mircrosoft's DCOM
	 	\item Dur kun mellem 2 maskiner.
		\item Time og space er coupled :(
		\item Flow couple er meget stærk på consumer siden.
		\begin{itemize}
		 	\item Er ikke stærkt, hvis producer ikke forventer et svar
		 	\item Altså asynk
		 \end{itemize} 
	\end{itemize} 

	\item Publisher/Subscriber
	\begin{itemize}
		\item Decoupler space, time og flow ved at lade alt køre i en kanal (Socket/kø)
		\item The channel decouples
	\end{itemize}
\end{itemize}

\begin{figure}[ht]
\begin{minipage}[b]{0.45\linewidth}
\centering
\includegraphics[width=\textwidth]{RMI}
\caption{RMI}
\label{fig:figure1}
\end{minipage} 						% No linebreak from here
\hspace{0.5cm}						% ...
\begin{minipage}[b]{0.45\linewidth}	% to here. They will over-under with linebreak
\centering
\includegraphics[width=\textwidth]{PS}
\caption{Publish/Subscribe}
\label{fig:figure2}
\end{minipage}
\end{figure}


Skal vi fortælle om de 3 versioner af p/s eller de andre, ringe løsninger?









\newpage
\section{Acceptor/Connector Pattern}

















































\end{document}