\documentclass[10pt, a4]{Memoir}

\usepackage{tcolorbox}
\usepackage{ulem} %math
\usepackage{amsmath}
\usepackage{amsfonts}
\usepackage{amssymb}
\usepackage{graphicx}
\usepackage{enumerate}


%Create a box for theorems
%\begin{theo}[titel] %optional
%tekst
%\end{theo}
\newenvironment{theo}[1][Vigtigt]{%
\begin{tcolorbox}[colback=green!5,colframe=green!40!black,title=\textbf{#1}]
}{%
\end{tcolorbox}
}




%Create a square matrix
%\begin{ArgMat}{2}
%21 & 22 & 23 \\  
%a & b & c
%\end{ArgMat}
%
% Info: http://tex.stackexchange.com/questions/2233/whats-the-best-way-make-an-augmented-coefficient-matrix
%
\newenvironment{ArgMat}{%
$
  \left[\begin{array}{@{}*{100}{r}r@{}}
}{%
  \end{array}\right]
  $
}

\newenvironment{deter}{%
$
  \left|\begin{array}{@{}*{100}{r}r@{}}
}{%
  \end{array}\right|
  $
}


%Create multiple lines with holes
%\begin{SysEqu}
%x_1 && &- &5x_3 &+ &2x_4=& 1 \\
%x_1 &+ &x_2 &+ &x_3 && =& 4 \\
%&&&&&&0 =& 0
%\end{SysEqu}
\newenvironment{SysEqu}{%
$  \setlength\arraycolsep{0.1em}
  \begin{array}{@{}*{100}{r}r@{}}
}{%
  \end{array}$
}

%Create solution for x_1, x_n...
%\begin{solu}
%x_1 &= d \\
%x_2 &= e \\
%x_3 &= s
%\end{solu}
\newenvironment{solu}{%
$
  \setlength\arraycolsep{0.1em}
  \left\{\begin{array}{@{}*{100}{r}r@{}}
}{%
  \end{array}\right.
$
}

\usepackage{lastpage}


\newcommand{\HRule}{\rule{\linewidth}{0.8mm}}

%Tekst i fotter
\newcommand{\footerText}{\thepage\xspace /\pageref{LastPage}}
\newcommand{\ProjectName}{433 MHz styring af AeroQuad}


\chapterstyle{hangnum}




\nouppercaseheads
\makepagestyle{mystyle} 

\makeevenhead{mystyle}{}{\\ \leftmark}{} 
\makeoddhead{mystyle}{}{\\ \leftmark}{} 
\makeevenfoot{mystyle}{}{\footerText}{} 
\makeoddfoot{mystyle}{}{\footerText}{} 
\makeatletter
\makepsmarks{mystyle}{% Overskriften på sidehovedet
  \createmark{chapter}{left}{shownumber}{\@chapapp\ }{.\ }} 
\makeatother
\makefootrule{mystyle}{\textwidth}{\normalrulethickness}{0.4pt}
\makeheadrule{mystyle}{\textwidth}{\normalrulethickness}

\makepagestyle{plain}
\makeevenhead{plain}{}{}{}
\makeoddhead{plain}{}{}{}
\makeevenfoot{plain}{}{\footerText}{}
\makeoddfoot{plain}{}{\footerText}{}
\makefootrule{plain}{\textwidth}{\normalrulethickness}{0.4pt}

\pagestyle{mystyle}

%%----------------------------------------------------------------------
%
%%Redefining chapter style
%%\renewcommand\chapterheadstart{\vspace*{\beforechapskip}}
%\renewcommand\chapterheadstart{\vspace*{10pt}}
%\renewcommand\printchaptername{\chapnamefont }%\@chapapp}
%\renewcommand\chapternamenum{\space}
%\renewcommand\printchapternum{\chapnumfont \thechapter}
%\renewcommand\afterchapternum{\space: }%\par\nobreak\vskip \midchapskip}
%\renewcommand\printchapternonum{}
%\renewcommand\printchaptertitle[1]{\chaptitlefont #1}
\setlength{\beforechapskip}{0pt} 
\setlength{\afterchapskip}{0pt} 
%\setlength{\voffset}{0pt} 
\setlength{\headsep}{25pt}
%\setlength{\topmargin}{35pt}
%%\setlength{\headheight}{102pt}
%\setlength{\textheight}{302pt}
\renewcommand\afterchaptertitle{\par\nobreak\vskip \afterchapskip}
%%----------------------------------------------------------------------




%Sidehoved og -fod pakke
%Margin
\usepackage[left=2cm,right=2cm,top=2.5cm,bottom=2cm]{geometry}
\usepackage{lastpage}



%%URL kommandoer og sidetal farve
%%Kaldes med \url{www...}
%\usepackage{color} %Skal også bruges
\usepackage{hyperref}
\hypersetup{ 
	colorlinks	= true, 	% false: boxed links; true: colored links
    urlcolor	= blue,		% color of external links
    linkcolor	= black, 	% color of page numbers
    citecolor	= blue,
}



%Mellemrum mellem linjerne    
\linespread{1.5}


%Seperated files
%--------------------------------------------------
%Opret filer således:
%\documentclass[Navn-på-hovedfil]{subfiles}
%\begin{document}
% Indmad
%\end{document}
%
% I hovedfil inkluderes således:
% \subfile{navn-på-subfil}
%--------------------------------------------------
\usepackage{subfiles}

%Prevent wierd placement of figures
%\usepackage[section]{placeins}

%Standard sti at søge efter billeder
%--------------------------------------------------
%\begin{figure}[hbtp]
%\centering
%\includegraphics[scale=1]{filnavn-for-png}
%\caption{Titel}
%\label{fig:referenceNavn}
%\end{figure}
%--------------------------------------------------
\usepackage{graphicx}
\usepackage{subcaption}
\usepackage{float}
\graphicspath{{../Figures/}}

%Speciel skrift for enkelt linje kode
%--------------------------------------------------
%Udskriver med fonten 'Courier'
%Mere info her: http://tex.stackexchange.com/questions/25249/how-do-i-use-a-particular-font-for-a-small-section-of-text-in-my-document
%Eksempel: Funktionen \code{void Hello()} giver et output
%--------------------------------------------------
\newcommand{\code}[1]{{\fontfamily{pcr}\selectfont #1}}


% Følgende er til koder.
%----------------------------------------------------------
%\begin{lstlisting}[caption=Overskrift på boks, style=Code-C++, label=lst:referenceLabel]
%public void hello(){}
%\end{lstlisting}
%----------------------------------------------------------

%Exstra space
\usepackage{xspace}
%Navn på bokse efterfulgt af \xspace (hvis det skal være mellemrum
%gives det med denne udvidelse. Ellers ingen mellemrum.
\newcommand{\codeTitle}{Kodeudsnit\xspace}

%Pakker der skal bruges til lstlisting
\usepackage{listings}
\usepackage{color}
\usepackage{textcomp}
\definecolor{listinggray}{gray}{0.9}
\definecolor{lbcolor}{rgb}{0.9,0.9,0.9}
\renewcommand{\lstlistingname}{\codeTitle}
\lstdefinestyle{Code}
{
	keywordstyle	= \bfseries\ttfamily\color[rgb]{0,0,1},
	identifierstyle	= \ttfamily,
	commentstyle	= \color[rgb]{0.133,0.545,0.133},
	stringstyle		= \ttfamily\color[rgb]{0.627,0.126,0.941},
	showstringspaces= false,
	basicstyle		= \small,
	numberstyle		= \footnotesize,
%	numbers			= left, % Tal? Udkommenter hvis ikke
	stepnumber		= 2,
	numbersep		= 6pt,
	tabsize			= 2,
	breaklines		= true,
	prebreak 		= \raisebox{0ex}[0ex][0ex]{\ensuremath{\hookleftarrow}},
	breakatwhitespace= false,
%	aboveskip		= {1.5\baselineskip},
  	columns			= fixed,
  	upquote			= true,
  	extendedchars	= true,
 	backgroundcolor = \color{lbcolor},
	lineskip		= 1pt,
%	xleftmargin		= 17pt,
%	framexleftmargin= 17pt,
	framexrightmargin	= 0pt, %6pt
%	framexbottommargin	= 4pt,
}

%Bredde der bruges til indryk
%Den skal være 6 pt mindre
\usepackage{calc}
\newlength{\mywidth}
\setlength{\mywidth}{\textwidth-6pt}


% Forskellige styles for forskellige kodetyper
\usepackage{caption}
\DeclareCaptionFont{white}{\color{white}}
\DeclareCaptionFormat{listing}%
{\colorbox[cmyk]{0.43, 0.35, 0.35,0.35}{\parbox{\mywidth}{\hspace{5pt}#1#2#3}}}
\captionsetup[lstlisting]
{
	format			= listing,
	labelfont		= white,
	textfont		= white, 
	singlelinecheck	= false, 
	width			= \mywidth,
	margin			= 0pt, 
	font			= {bf,footnotesize}
}

\lstdefinestyle{Code-C} {language=C, style=Code}
\lstdefinestyle{Code-Java} {language=Java, style=Code}
\lstdefinestyle{Code-C++} {language=[Visual]C++, style=Code}
\lstdefinestyle{Code-VHDL} {language=VHDL, style=Code}
\lstdefinestyle{Code-Bash} {language=Bash, style=Code}

%Text typesetting
%--------------------------------------------------------
%\usepackage{baskervald}
\usepackage{lmodern}
\usepackage[T1]{fontenc}              
\usepackage[utf8]{inputenc}         
\usepackage[english]{babel}       

\setlength{\parindent}{0pt}
\nonzeroparskip

%\setaftersubsecskip{1sp}
%\setaftersubsubsecskip{1sp}
 


%Dybde på indholdsfortegnelse
%----------------------------------------------------------
%Chapter, section, subsection, subsubsection
%----------------------------------------------------------
\setcounter{secnumdepth}{3}
\setcounter{tocdepth}{3}


%Tables
%----------------------------------------------------------
\usepackage{tabularx}
\usepackage{array}
\usepackage{multirow} 
\usepackage{multicol} 
\usepackage{booktabs}
\usepackage{wrapfig}
\renewcommand{\arraystretch}{1.5}



%Misc
%----------------------------------------------------------
\usepackage{cite}
\usepackage{appendix}
\usepackage{amssymb}
\usepackage{url,ragged2e}
\usepackage{enumerate}
\usepackage{amsmath} %Math bibliotek


\usepackage{longtable}


\begin{document}

\section{Lesson 2}

\subsection{Changing environments}
\begin{itemize}
\item With relational database
\subitem Could be windmills and collecting data from many mills

\item With external information providers
	\begin{itemize}
	\item RMI: kalder et objekt på en anden maskine.
	\item RPC: kalder en funktion på en anden maskine.
	\end{itemize}

\item with Internet application
	\begin{itemize}
	\item[] Slide 4
	\item Server can place a cookie on the computer, due to bad connection
	\item QoS (latency, performance) not guarenteed. 
	\end{itemize}

\item QoS
	\begin{itemize}
	\item slide 5
	\item CORBA tries to implement it in version 3.0
	\end{itemize}

\item Nomadic mobility (telefoner)
	\begin{itemize}
	\item slide 6
	\item Lysmaster med wifi hotspots
	\item Det mobile object kan skifte driver afhængig af hvor den er, eller hvad der om den.
	\end{itemize}

\item Qbiquitos/pervasive computing
	\begin{itemize}
	\item slide 7
	\item Adhoc network (spontant netværk)
	\subitem En masse sensorer laver et netværk og opsamler data 	
	\subitem Et netværk i alle biler
	\item At give en lile enhed en IPv6 er spild af plads, hvis den blot smides ud senere.
	\end{itemize}
\end{itemize}

\subsection{Programming Models}
\begin{itemize}
\item Client-server
	\begin{itemize}
	\item Du puller data
	\end{itemize}

\item Publish/subscribe
\item Peer-to-peer
\item Async interaction
	\begin{itemize}
	\item slide 9/10
	\item RMI and RPC
	\end{itemize}

\item Shared memory
	\begin{itemize}
	\item slide 11
	\item Will look like shared memory for multiple computers (Dropbox - data forsvinder og rettes med det samme / Google Docs).
	\item JavaSpace does the same trick
	\end{itemize}

\item Mobile code and mobile agents
	\begin{itemize}
	\item slide 12
	\item Travels nodes to find out where to go next
	\item Will seem autonomous.
	\end{itemize}

\end{itemize}


\subsection{Arkitektur}

\begin{itemize}
\item Distributed transparency
	\begin{itemize}
	\item slide 13
	\item I stedet for 2 funktioner på én maskine er der to funktioner på 2 maskiner (én hver). Disse kan snakke sammen på samme måde
	\subitem Nogle af disse har exceptions implementeret
	\end{itemize}

\item Layering
	\begin{itemize}
	\item Man tilgår kun laget over eller under (kan også have adgang til alle lag)
	\end{itemize}

\item Monolithic 
	\begin{itemize}
	\item Small devices der skal bruge det - Corba har det, men er alt for stort!
	\end{itemize}

\end{itemize}

\subsection{Dynamic configurations}

\begin{itemize}
\item Disconnect operations
	\begin{itemize}
	\item slide 16
	\item Bluetooth's nyeste version (BLE -- low energy) disconnecter på meget kort forbindelsen før opretelse igen for at spare strøm.
	\end{itemize}

\item Adaptive applications
	\begin{itemize}
	\item Lav internethastighed kan give lidt net
	\end{itemize}

\end{itemize}






\section{SAP 1}
Der er 3 actors i systemet:
\begin{itemize}
\item Der er en infrastructure, som bruges af en app (og eges af infrastruktur owner). Denne har derfor en høj abstraction.
\item Applikations owner ønsker så noget lavet i infrastrukturen. 
Dette gør Component owner sammen med infrastruktur owner.
\item Ved at gøre det multiparadigm kan mange forskellige elementer hentes ind til infrastrukturen.
\end{itemize}


\section{Lesson 3 -- Reactor pattern}

Kan bruges ved Embedded system og netværk.
\\
Det fungerer som følger:
\begin{itemize}
\item En til flere clienter forsøger at forbinde til en server
\item Reactor ligger på server (men kan også ligge på clienten)
\item Serveren venter på noget sker og gør ikke noget indtil et event (connect / read $dots$) kommer ind.
\subitem Den bruger main thread
\item 
\end{itemize}


\end{document}