\documentclass[oneside, 10pt]{memoir}

%Preamble

% Følgende er til koder.
%----------------------------------------------------------
%\begin{lstlisting}[caption=Overskrift på boks, style=Code-C++, label=lst:referenceLabel]
%public void hello(){}
%\end{lstlisting}
%----------------------------------------------------------

%Exstra space
\usepackage{xspace}
%Navn på bokse efterfulgt af \xspace (hvis det skal være mellemrum
%gives det med denne udvidelse. Ellers ingen mellemrum.
\newcommand{\codeTitle}{Code snippet\xspace}

%Pakker der skal bruges til lstlisting
\usepackage{listings}
\usepackage{color}
\usepackage{textcomp}
\definecolor{listinggray}{gray}{0.9}
\definecolor{lbcolor}{rgb}{0.9,0.9,0.9}
\renewcommand{\lstlistingname}{\codeTitle}
\lstdefinestyle{Code}
{
	keywordstyle	= \bfseries\ttfamily\color[rgb]{0,0,1},
	identifierstyle	= \ttfamily,
	commentstyle	= \color[rgb]{0.133,0.545,0.133},
	stringstyle		= \ttfamily\color[rgb]{0.627,0.126,0.941},
	showstringspaces= false,
	basicstyle		= \small,
	numberstyle		= \footnotesize,
%	numbers			= left, % Tal? Udkommenter hvis ikke
	stepnumber		= 2,
	numbersep		= 6pt,
	tabsize			= 2,
	breaklines		= true,
	prebreak 		= \raisebox{0ex}[0ex][0ex]{\ensuremath{\hookleftarrow}},
	breakatwhitespace= false,
%	aboveskip		= {1.5\baselineskip},
  	columns			= fixed,
  	upquote			= true,
  	extendedchars	= true,
 	backgroundcolor = \color{lbcolor},
	lineskip		= 1pt,
%	xleftmargin		= 17pt,
%	framexleftmargin= 17pt,
	framexrightmargin	= 0pt, %6pt
%	framexbottommargin	= 4pt,
}

%Bredde der bruges til indryk
%Den skal være 6 pt mindre
\usepackage{calc}
\newlength{\mywidth}
\setlength{\mywidth}{1.435\textwidth} % Hvis bredden header ikke virker er dette hvad skal ændres!


% Forskellige styles for forskellige kodetyper
\usepackage{caption}
\DeclareCaptionFont{white}{\color{white}}
\DeclareCaptionFormat{listing}%
{\colorbox[cmyk]{0.43, 0.35, 0.35,0.35}{\parbox{\mywidth}{\hspace{5pt}#1#2#3}}}
\captionsetup[lstlisting]
{
	format			= listing,
	labelfont		= white,
	textfont		= white, 
	singlelinecheck	= false, 
	width			= \mywidth,
	margin			= 0pt, 
	font			= {bf,footnotesize}
}

\lstdefinestyle{Code-C} {language=C, style=Code}
\lstdefinestyle{Code-Java} {language=Java, style=Code}
\lstdefinestyle{Code-C++} {language=[Visual]C++, style=Code}
\lstdefinestyle{Code-VHDL} {language=VHDL, style=Code}
\lstdefinestyle{Code-Bash} {language=Bash, style=Code}
\lstdefinestyle{Code-Matlab} {language=Matlab, style=Code}
\lstdefinestyle{Code-Prolog} {language=Prolog, style=Code}
%Speciel skrift for enkelt linje kode
%--------------------------------------------------
%Udskriver med fonten 'Courier'
%Mere info her: http://tex.stackexchange.com/questions/25249/how-do-i-use-a-particular-font-for-a-small-section-of-text-in-my-document
%Eksempel: Funktionen \code{void Hello()} giver et output
%--------------------------------------------------
\newcommand{\code}[1]{{\fontfamily{pcr}\selectfont #1}}

%Seperated files
%--------------------------------------------------
%Opret filer således:
%\documentclass[Navn-på-hovedfil]{subfiles}
%\begin{document}
% Indmad
%\end{document}
%
% I hovedfil inkluderes således:
% \subfile{navn-på-subfil}
%--------------------------------------------------
\usepackage{subfiles}
%Text typesetting
%--------------------------------------------------------
\usepackage[T1]{fontenc} 	% Can use danish characters
\usepackage[utf8]{inputenc} % Input encoding. Can be used on Linux, Mac and Windows         
\usepackage[danish]{babel} 	% Split words accoding to English
\usepackage{lmodern} 		% Font

\setlength\parindent{0pt} 	% No indent
\setlength\parskip{12pt} 	% More than a single line break will give ONE linebreak.

%Margin
\usepackage[left=2cm,right=2cm,top=2.5cm,bottom=2cm]{geometry}

%Margin
\usepackage[left=3cm,right=2cm,top=2.5cm,bottom=2cm]{geometry}

%Mellemrum mellem linjerne    
\linespread{1.5}

\title{Assignment 2 for TEDI}
\author{Rasmus Bækgaard, 10893}
\date{April 28, 2014}

\begin{document}


\chapter{ARDI}

\section{Lesson 2}

\subsection{Changing environments}
\begin{itemize}
\item With relational database
\subitem Could be windmills and collecting data from many mills

\item With external information providers
	\begin{itemize}
	\item RMI: kalder et objekt på en anden maskine.
	\item RPC: kalder en funktion på en anden maskine.
	\end{itemize}

\item with Internet application
	\begin{itemize}
	\item[] Slide 4
	\item Server can place a cookie on the computer, due to bad connection
	\item QoS (latency, performance) not guarenteed. 
	\end{itemize}

\item QoS
	\begin{itemize}
	\item slide 5
	\item CORBA tries to implement it in version 3.0
	\end{itemize}

\item Nomadic mobility (telefoner)
	\begin{itemize}
	\item slide 6
	\item Lysmaster med wifi hotspots
	\item Det mobile object kan skifte driver afhængig af hvor den er, eller hvad der om den.
	\end{itemize}

\item Qbiquitos/pervasive computing
	\begin{itemize}
	\item slide 7
	\item Adhoc network (spontant netværk)
	\subitem En masse sensorer laver et netværk og opsamler data 	
	\subitem Et netværk i alle biler
	\item At give en lile enhed en IPv6 er spild af plads, hvis den blot smides ud senere.
	\end{itemize}
\end{itemize}

\subsection{Programming Models}
\begin{itemize}
\item Client-server
	\begin{itemize}
	\item Du puller data
	\end{itemize}

\item Publish/subscribe
\item Peer-to-peer
\item Async interaction
	\begin{itemize}
	\item slide 9/10
	\item RMI and RPC
	\end{itemize}

\item Shared memory
	\begin{itemize}
	\item slide 11
	\item Will look like shared memory for multiple computers (Dropbox - data forsvinder og rettes med det samme / Google Docs).
	\item JavaSpace does the same trick
	\end{itemize}

\item Mobile code and mobile agents
	\begin{itemize}
	\item slide 12
	\item Travels nodes to find out where to go next
	\item Will seem autonomous.
	\end{itemize}

\end{itemize}


\subsection{Arkitektur}

\begin{itemize}
\item Distributed transparency
	\begin{itemize}
	\item slide 13
	\item I stedet for 2 funktioner på én maskine er der to funktioner på 2 maskiner (én hver). Disse kan snakke sammen på samme måde
	\subitem Nogle af disse har exceptions implementeret
	\end{itemize}

\item Layering
	\begin{itemize}
	\item Man tilgår kun laget over eller under (kan også have adgang til alle lag)
	\end{itemize}

\item Monolithic 
	\begin{itemize}
	\item Small devices der skal bruge det - Corba har det, men er alt for stort!
	\end{itemize}

\end{itemize}

\subsection{Dynamic configurations}

\begin{itemize}
\item Disconnect operations
	\begin{itemize}
	\item slide 16
	\item Bluetooth's nyeste version (BLE -- low energy) disconnecter på meget kort forbindelsen før opretelse igen for at spare strøm.
	\end{itemize}

\item Adaptive applications
	\begin{itemize}
	\item Lav internethastighed kan give lidt net
	\end{itemize}

\end{itemize}






\section{SAP 1}
Der er 3 actors i systemet:
\begin{itemize}
\item Der er en infrastructure, som bruges af en app (og eges af infrastruktur owner). Denne har derfor en høj abstraction.
\item Applikations owner ønsker så noget lavet i infrastrukturen. 
Dette gør Component owner sammen med infrastruktur owner.
\item Ved at gøre det multiparadigm kan mange forskellige elementer hentes ind til infrastrukturen.
\end{itemize}

\newpage
\section{Lesson 3 -- Reactor pattern}

Kan bruges ved Embedded system og netværk.
\\
Det fungerer som følger:

\begin{itemize}
\item En til flere clienter forsøger at forbinde til en server
\item Reactor ligger på server (men kan også ligge på clienten)

\item Serveren venter på noget sker og gør ikke noget indtil et event (connect / read $dots$) kommer ind.
\subitem Den bruger main thread

\item Sockets er inkapslet i et \code{handle}
\item Hvert request er processeret synkron og i seriel
\subitem Ved modtagelse laves et demultiplex og eventet sendes til den service der processerer dem.
\item Bruger IoC
\end{itemize}
Der er følgende klasser:

\begin{itemize}
\item EventHandler
	\begin{itemize}
	\item handle\_event()
	\item get\_handle()
	\end{itemize}
\item[] Denne del er et interface for konkrete eventhandlers fra clienter.
\item[] Den har et 1-til-1 ejerskab med \code{Handle}-klassen.

\item Reacter
	\begin{itemize}
	\item Handle\_events()
	\item register\_handler()
	\item remove\_handler()
	\end{itemize}
\item[] De sidste to tilføjer og fjerner den enkelte EventHandler's handle
\item[] \code{Reactor} dispatcher flere EventHandler
\item[] \code{Reactor} er en singleton

\item Demultiplexer
	\begin{itemize}
	\item select()
	\end{itemize}
\item[] Bruges af \code{Reactor}
\end{itemize}
Implementation
\begin{enumerate}
\item Definer \code{EventHandler} interface
\subitem Slide 12-13
\item Definer \code{reactor} interface
\subitem Slide 14
\subitem \code{register\_handler}'s overloading kunne klares med en default parameter for \code{Handle}
\item Implementer \code{reactor} interface
\item Beslut antal af \code{reactors} der skal bruges
\subitem Slide 19
\item Implementer konkrete \code{EventHandler}s
\end{enumerate}
Fordele:

\begin{itemize}
\item Slide 25
\item Kan bruges på flere OSs
\item Kan bruges som et modul
\item Kan serilizeres
\end{itemize}
Ulemper:
\begin{itemize}
\item Slide 26
\item Kan være svært at teste
\item Single thread kan stoppe andre i at køre (duhh)
\end{itemize}






\newpage
\section*{Lesson 5 -- Aceeptor/Connector pattern}

En client, $C$, en server, $S$, og to peers, $p1, p2$.
\begin{itemize}
	\item $C$ skal blot kommunikere med $S$ eller omvendt.
	\item $S$ kan have en $reactor$ og dispatche $event handlers$ (kan også på $C$, men det er lidt underligt, hvis den ikke skal andet end forbinde til serveren. Til $p1$ er det fint at have den).
	\item $Acceptor$ og $Connector$ placeres over Wrapper'en.
	\item $C$ har connector, $S$ har acceptor, $p1$ og $p2$ har begge
	\item Når forbindelsen mellem $p1$ og $p2$ er sat op har de en $logical channel$ / \code{sock\_stream} 
	\item Over \code{Acceptor} pg \code{Connector} laves selve applikationen.
	\item \code{Reactor} og \code{Proactor}'en kaldes tilsammen \code{dispatcher}.
	\item En \code{Service handler} er et interface der bruges sammen med \code{Acceptor} og \code{Connector}'en.
	\item Alle 3 arver fra \code{Event Handler} sammen med \code{Reactor}'en.
\end{itemize}
Applikationer for \code{Service Handler}
\begin{itemize}
	\item Slide 45
	\item S
\end{itemize}


\newpage
\section*{Lesson 6 -- Requirements og arkitektur}

\begin{itemize}
	\item Funktionelle krav beskrives med Use Cases
	
	\item Non-functionelle krav \dots
	
	\begin{itemize}
		\item Distribution
		\item performance
		\item Svalability
		\item Availability
		\item Reliability
		\item Portability
		\item Dynamic configuration
		\item Heterogeneity and lagacy systems
		\item Security
		\item \dots
	\end{itemize}
	
	\item Når UC'erne er lavet til en model, kan der laves forskellige løsninger (med eller uden distribuerede systemer).
	
	\item Validation technique: ATAM (Architecture Tradeoff Analysis Method).
	
	\begin{itemize}
		\item Processen beskrives på slide 8
		\item Phase 4 er design-delen, da man må undvære noget for noget andet (flere udviklere for nok design koster penge).
		\item Phase 1 er linear med at finde krav og lasve en liste over dem. Hvad er vigtist?
		\item Phase 2 er en cycle
	\end{itemize}


\end{itemize}




\newpage
\section*{Lesson 7 -- Proactor}

\begin{itemize}
	\item Proactor bruges mest på serveren (og ikke så meget på clienten)
	
	\item Bruges som "proactive" 
	\begin{itemize}
		\item Beder andre processor (trådet/asynkrone) om at gøre noget
		\item Håndterer de forskellige processors output
		\item Note: Dette kan kun bruges på større (32 bit) systemer -- ikke på RTOS
	\end{itemize}

	\item Når de asynkorne processor er færdige, bruges en \code{CompletionHandler} til at færdigegøre og afslutte processen.

	\item \code{Event Demuxer} er asynkron

	\item En \code{initiator} bruger Asynkrone Operation, \code{Completion Event Queue}, \code{Proactor} og \code{Completion Handler}

	\item 
\end{itemize}





\newpage
\section*{Lesson 8}

 \code{ServiceHandlers} klasser kan decoples med flere patterns:
\begin{itemize}
	
	\item Proxy Pattern
	\begin{itemize}
		\item Man kalder sit subject igennem en proxy, der forwarder til det rigtige objekt
		\item Dette kan være på den lokale maskine, men også over en netværksforbindelse
		\item Det kan også være en testklasse
	\end{itemize}

	\item Broker Structure
	\begin{itemize}
		\item Broker'en sættes mellem en server proxy og en client proxy.
		\item Den er altså et mellemled, der blot forwarder et kald.
	\end{itemize}

	\item Half Object Plus Protocol (HOPP)
	\begin{itemize}
		\item Hvordan kan et objekt optræde i flere netværk
		\item Man kan lave et Network Proxy Object (NPO) (ikke bedst)
		\item Lav 2 objekter, der kommunikerer med hinanden (Synchronization Protocol)
	\end{itemize}

	\item Data Trandfer Object (DTO)
	\begin{itemize}
		\item Kræver en \code{get} og en \code{set} for hver variabel
		\item Alt kan dog skrives i ét hug.
	\end{itemize}
\end{itemize}


\newpage
\section*{Lesson 9}

\begin{itemize}
	\item Half Sync / Half Async
	
	\begin{itemize}
		\item Der findes to lag: synkron og asynkron.
		
		\item Synk kan have flere tråde.
		
		\item Asynk tager imod events og kan betegnes som \texttt{ISR} (Interrupt Service Routine).
		
		\item Synk er for App. developers.

		\item Asynk er for System developers (low level niveau).
		\begin{itemize}
			\item Her skal performance improves.
			\item Skriv så korte interrupts som muligt.
		\end{itemize}
		
		\item Beskeder imellemsendes igennem en queue.
		
		\item Implementering:
		\begin{itemize}
			\item Én queue til flere services. Denne må supporte flere waits.
			\item Flere queues til flere services. Dette kræver en demultiplexer strategi (hvilke beskeder skal i hvor?).
		\end{itemize}

		\item Bennefits:
		\begin{itemize}
			\item Bedre performance of simplificering, da enkelte tråde ikke skal håndtere andet end det de skal lave.
			\item Seperation of concerns
		\end{itemize}

		\item Liabiliites
		\begin{itemize}
			\item Boundary-crossing i form af data copying, synkronisering.
			\item Højerelevel applikationer får ikke så meget ud af det
			\item Debugging og testing bliver sværre. 
		\end{itemize}
	\end{itemize}


	\item Leader / Follower
	\begin{itemize}
		\item Man har én elader og flere followers.
		\item Når leaderen skal gøre noget, overdrages leader til den næste follower med \code{PromoteNewLeader()}. (Hvis en tråd kommer tilbage uden en leader bliver denne automatisk det.)
	\end{itemize}

\end{itemize}



\newpage
\section*{Interceptor}

\begin{itemize}
	\item Bruges med frameworks og til komponenter.
	
	\item Man har 2 dele:
	\begin{itemize}
		\item App, udviklet af app. devs.
		\item Framework, udviklet af FW. devs.
	\end{itemize}

	\item Man skal vælge hvor funktionaliteten skal ligge (ikke ligegyldigt).
	\subitem Det skal kunne genbruges!
	
	\item I App'en er der lavet handlers som intercepter events fra frameworket. (Det minder om callbacks), som gerne må kalde flere interceptors.
	
	\item App'en må gerne ændre hvordan frameworket virker, men det skal man være påpasselig med.

	\item Mange frameworks har en Facade, eller blot et interface, men kan også være helt åbne.

	\item Det giver en god opdeling af app'en og FW'et.

	\item Golden rule: Udvikel forskellige apps før frameworket laves.
\end{itemize}

\subsection*{Intercepter pattern}
\begin{itemize}
	\item FW'et kan ikke forudsige hvem den skal servicere

	\item Skal tillade integration af flere services (ellers er det ikke et framework)

	\item Disse må ikke påvirke det nuværende system.

	\item FW'et skal muligvis monitoreres og kontrolleres.

	\item Applikation skal registreres med 'out-of-band' interfaces.

	\item Dispatcher (i FW'et) kan dynamisk tilføje og fjerne apps (interceptors).

	\item FW'et har et Concrete fw som laver en Context. Denne bruges af interceptoren.

	\item 
\end{itemize}


\section*{Extension pattern}
\begin{itemize}
	\item Hver rolle har sit eget interface

	\item Disse bør aldrig ændres, da der ikke ønskes breaking changes

	\item Brug i stedet en extension
\end{itemize}






\newpage
\section*{JAWS}
Adaptive web Server
\begin{itemize}
	\item Created with ACE fw
	\item It must use a concurrency model (threads)
	\item Dispatcher (synk eller asynk)
	\item File caching
	\item Protocol HTTP/1.0 og/eller /1.1
	\item Skal skrædersyes til brugernes behov (små filer / store billeder)
	\item Bygges normalt med selvskrevet kode og nogle patterns - når man skriver fw bruges der kun de komponenter stillet til rådighed til at skrive det.
\end{itemize}
JAWS:
\begin{itemize}
	\item Event Dispatcher m. Acceptor 
	\item Concurrencu med State
	\item I/O med Proactor og asynk
	\item \dots se slides
\end{itemize}
Protocol Pipeline
\begin{itemize}
	\item CGI er langsom, da det starter en ny tråd
	\item Hvert filter data kommer igennem er en process der starter.
	\item 
\end{itemize}





\newpage
\section*{Component configuration}
\begin{itemize}
	\item Du skal altid have din server til at køre 24/7.
	\item Serveren bruger \code{.dll}-filer, som skal kunne ændres (importes) on-the-fly.
	\item Det kan \emph{ikke} bruges på 8-bit systemer.
	\item Årsag kan være til at ændre funktionalitet, eller rette en bug.
\end{itemize}
Christian algorithm

\begin{itemize}
	\item Kører godt til www-modtagere.
	\item Kaldes "the time server" (passive).
	\item Andre spørger efter tiden.
\end{itemize}
Berkeley

\begin{itemize}
	\item Er aktiv.
	\item Pusher tiden ud.
\end{itemize}
Implementering (Se slides)

\begin{itemize}
	\item Interfaces på komponenterne
	\item Lav et \code{Component Repository}, der kan behandle \code{.dll}-filerne (insert, remove, find, suspend, resume)
	\item Det enkelte komponent har metoderne init, fini(sh), suspend, resume, info
	\item De konkrete komponenter arver fra ovenstående.
	\item Selve applikationen vil bruger \code{Component Configurator} til at få de konkrete implementeringer.

	\item Først initierer man komponentet, derefter indsættes det.
	\item Noter, at applicationen har et interface til det enkelte komponent (ikke det med init, fini(sh)), men et andet, specifikt interface.
	\item Se life cycle på slide 11

	\item I komponentet har \code{init} samme parametre som \code{Main} har.

	\item Importer med et lille script (slide 17)
	\item Det kan have sikkerhedsproblemer
	\item Kan gøre det hele lidt mere kompliseret
	\item Der kan kun bruges et interface (men man skriver selv sine filer...)
\end{itemize}







\section*{Eksamen}
\begin{itemize}
	\item 20 pr mand inkl votering
	\item Husk at lave agenda (og skriv den op)
	\item Undlad at læse op
	\item Hvert spørgsmål har 3 underspørgsmål
	\item Der er muligvis uddybende spørgsmål fra dem
	\item 
\end{itemize}
Spørgsmål:
\begin{itemize}
	\item 2: Hvad betyder det
	\item Måske noget artikel

	\item 3: Forskellen på c er et reaktor pattern på den ene af dem

	\item 4: Proaktor - den kan invoke asyn operation, men en reaktor kan have leader/follow og gøre det sammen

	\item 5: ATEM skulle lave den nye arkitektur ?

	\item 6: Exercise 5 her
	\item[] Husk at fordele tiden ud over alle spørgsmål

	\item 8: 
\end{itemize}




\end{document}