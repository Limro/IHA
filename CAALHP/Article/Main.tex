\documentclass[jou]{apa6}

%Preamble

% Følgende er til koder.
%----------------------------------------------------------
%\begin{lstlisting}[caption=Overskrift på boks, style=Code-C++, label=lst:referenceLabel]
%public void hello(){}
%\end{lstlisting}
%----------------------------------------------------------

%Exstra space
\usepackage{xspace}
%Navn på bokse efterfulgt af \xspace (hvis det skal være mellemrum
%gives det med denne udvidelse. Ellers ingen mellemrum.
\newcommand{\codeTitle}{Code snippet\xspace}

%Pakker der skal bruges til lstlisting
\usepackage{listings}
\usepackage{color}
\usepackage{textcomp}
\definecolor{listinggray}{gray}{0.9}
\definecolor{lbcolor}{rgb}{0.9,0.9,0.9}
\renewcommand{\lstlistingname}{\codeTitle}
\lstdefinestyle{Code}
{
	keywordstyle	= \bfseries\ttfamily\color[rgb]{0,0,1},
	identifierstyle	= \ttfamily,
	commentstyle	= \color[rgb]{0.133,0.545,0.133},
	stringstyle		= \ttfamily\color[rgb]{0.627,0.126,0.941},
	showstringspaces= false,
	basicstyle		= \small,
	numberstyle		= \footnotesize,
%	numbers			= left, % Tal? Udkommenter hvis ikke
	stepnumber		= 2,
	numbersep		= 6pt,
	tabsize			= 2,
	breaklines		= true,
	prebreak 		= \raisebox{0ex}[0ex][0ex]{\ensuremath{\hookleftarrow}},
	breakatwhitespace= false,
%	aboveskip		= {1.5\baselineskip},
  	columns			= fixed,
  	upquote			= true,
  	extendedchars	= true,
 	backgroundcolor = \color{lbcolor},
	lineskip		= 1pt,
%	xleftmargin		= 17pt,
%	framexleftmargin= 17pt,
	framexrightmargin	= 0pt, %6pt
%	framexbottommargin	= 4pt,
}

%Bredde der bruges til indryk
%Den skal være 6 pt mindre
\usepackage{calc}
\newlength{\mywidth}
\setlength{\mywidth}{1.435\textwidth} % Hvis bredden header ikke virker er dette hvad skal ændres!


% Forskellige styles for forskellige kodetyper
\usepackage{caption}
\DeclareCaptionFont{white}{\color{white}}
\DeclareCaptionFormat{listing}%
{\colorbox[cmyk]{0.43, 0.35, 0.35,0.35}{\parbox{\mywidth}{\hspace{5pt}#1#2#3}}}
\captionsetup[lstlisting]
{
	format			= listing,
	labelfont		= white,
	textfont		= white, 
	singlelinecheck	= false, 
	width			= \mywidth,
	margin			= 0pt, 
	font			= {bf,footnotesize}
}

\lstdefinestyle{Code-C} {language=C, style=Code}
\lstdefinestyle{Code-Java} {language=Java, style=Code}
\lstdefinestyle{Code-C++} {language=[Visual]C++, style=Code}
\lstdefinestyle{Code-VHDL} {language=VHDL, style=Code}
\lstdefinestyle{Code-Bash} {language=Bash, style=Code}
\lstdefinestyle{Code-Matlab} {language=Matlab, style=Code}
\lstdefinestyle{Code-Prolog} {language=Prolog, style=Code}
%Speciel skrift for enkelt linje kode
%--------------------------------------------------
%Udskriver med fonten 'Courier'
%Mere info her: http://tex.stackexchange.com/questions/25249/how-do-i-use-a-particular-font-for-a-small-section-of-text-in-my-document
%Eksempel: Funktionen \code{void Hello()} giver et output
%--------------------------------------------------
\newcommand{\code}[1]{{\fontfamily{pcr}\selectfont #1}}

%Seperated files
%--------------------------------------------------
%Opret filer således:
%\documentclass[Navn-på-hovedfil]{subfiles}
%\begin{document}
% Indmad
%\end{document}
%
% I hovedfil inkluderes således:
% \subfile{navn-på-subfil}
%--------------------------------------------------
\usepackage{subfiles}
%Text typesetting
%--------------------------------------------------------
\usepackage[T1]{fontenc} 	% Can use danish characters
\usepackage[utf8]{inputenc} % Input encoding. Can be used on Linux, Mac and Windows         
\usepackage[danish]{babel} 	% Split words accoding to English
\usepackage{lmodern} 		% Font

\setlength\parindent{0pt} 	% No indent
\setlength\parskip{12pt} 	% More than a single line break will give ONE linebreak.

%Margin
\usepackage[left=2cm,right=2cm,top=2.5cm,bottom=2cm]{geometry}

%Margin
\usepackage[left=3cm,right=2cm,top=2.5cm,bottom=2cm]{geometry}

%Mellemrum mellem linjerne    
\linespread{1.5}

\title{Assignment 2 for TEDI}
\author{Rasmus Bækgaard, 10893}
\date{April 28, 2014}

%BibLatex
\usepackage[style=ieee, sortcites=true, sorting=nyt, backend=biber]{biblatex} 
\DeclareLanguageMapping{american}{american-ieee} %BibLatex mapping
\addbibresource{References.bib}

\title{False-positive filter in the CAAALHP}
\shorttitle{}

\author{Rasmus Bækgaard}

\affiliation{Aarhus University}

\leftheader{Bækgaard}

% Abstract:
% - Background (brug for CAS)
% - Hvad skete der
% - Der blev brugt en method
% - Results

\abstract{This paper introduces the reader to considerations and guidelines of how a false-positive filter can created and implemented in the Common Ambient Assisted Living Homecare Platform, CAALHP.
The paper will present its findings of an actual implementation for this system, and suggest ways of improving the implementation for others, who wishes make a similar filter.}

\keywords{CAALHP, False-positive filter} %shown after abstract

\date{\today}

\begin{document}

\maketitle

\section{Introduction}
This paper is written based on a Research and Development project for Aarhus University's Engineer Department.
The project is handling the problem of detecting a false-positive of sensors, when they obviously does not happen, and the given sensor will trigger a critical alarm.
The case taken in the project is based on a patient wearing a fall detector, and a bed with a 'in-bed' detector.
Should said patient lie down on the bed and somehow trigger the fall detector, both the fall detector and the 'in-bed' detector will send events of occurrence. 
But lying down in bed is not a fall, so the system should be able to recognize this as a false-positive from the fall detector.

To ensure that this can be implemented to an existing system, it's important not to rewrite code, but add to it.
One technique to use is a Context-Awareness system, like e.g. the Java Context-Awareness Framework\fxnote{reference til JCAF}, JCAF.
The JCAF is built upon layers to ensure functionality and responsibility is divided to those who should handle it, allowing devices monitoring service by to subscribe to them, letting services subscribe to devices and publish them to monitors.
Another context framework implementing this technique, is the Context Toolkit\fxnote{reference til Toolkit}, which consist of several types of class, including \texttt{interpreters} to convert context to higher level information, and \texttt{discoverers} to register capabilities in the framework.

By using an object oriented model, a service for CAALHP (written in C\#) will be able to use inheritance, objects, encapsulations, and more to represent the data retrieved from the sensor and present them to monitors or other services, much as show on Figure \fxnote{Insert picture of JCAF, slide 35.}.
The service will be automatically added to the CAALHP with context discoverers.





Artikel:
- Lagdeling
- Måden at sende beskeder (Carestore's beskedmekanisme -> afkobloing)
- Context, heirarki, gemt i service
- Hvordan deles op, hvem beslutter


\section{Purpose}
- At skabe leightweight system

\section{method}
- Proof of concept
- Udvikles en evalueringsprototype
	- CareBed og Shimmer tester, at det virker

\section{Results}
- Arkitektur
- Fesibility study
	- Lavet og testet (hvordan) - det giver kvalitet

\section{Diskussion}
- Hvad er godt
- Hvad kan JCAF
- Hvad mangler
- Hvad skal gøres bedre


% \begin{figure}[ht!]
% \centering
% \includegraphics[width=0.5\textwidth]{cameraman}
% \end{figure}

% \newpage
% \documentclass[Preamble]{subfiles}
\begin{document}

\chapter*{Abstract}
\addcontentsline{toc}{chapter}{Abstract} 

About your reports and this template
\begin{itemize}
\item Use this template for your project work reports
\item Substitute the template's place-holder dates and titles with appropriate ones
\item Place-holders are marked with square brackets, i.e. [place-holder]
\item For each report, you hand in both a tex and a pdf file
\item The report should be 10-15 pages in total and it must be written in English
\end{itemize}

About the abstract, i.e. the current section
\begin{itemize}
\item An abstract is a brief summary of the report that helps the
  reader quickly ascertain the report's purpose. The abstract should
  be approximately half a page.
\end{itemize}


\setcounter{tocdepth}{1}
\tableofcontents



\end{document} 

% \tableofcontents 	% Table of Contents
% \listoffixmes 	%See LatexModule 

% \newpage

% Main contribution & main challenges & lessons

% %!TEX root = Main.tex
\section{Introduction}

For elder people, one fear of being alone is what will happen if they fall and cannot get up by them self\cite{bell2000characteristics}.
Some nursing homes and sheltered housings have installed sensors of various forms\cite{Tun-Loft}, to detect e.g. a fall in the building and alert personal should a fall occur. 
But sensors that detects falls can be error-prone\cite{bardram2008context} and provide false-positives, which will trigger events that calls for personal to help.
This is most unwanted and should be possible to avoid using data from other sensors, to help determine whether the signal is a false-positive or not.
\\
This paper presents the work from a study for a service that will assist in doing just that.
The aim is to evaluate a feasibility prototype of such a false-positive filter, that will analyze incoming data from various sensors, and thereby determine a probability of false-positive events, based on custom user credentials of what should influent the decision.

% This paper is written based on a Research and Development project for Aarhus University's Engineering Department.
% The project is handling the problem of detecting a false-positive from sensors, when they obviously does not happen, and the given sensor will trigger a critical alarm.
% The case taken in the project is based on a patient wearing a fall detector, and a bed with a 'in-bed' detector.
% Should said patient lie down on the bed and somehow trigger the fall detector, both the fall detector and the 'in-bed' detector will send events of occurrence. 
% But lying down in bed is not a fall, so the system should be able to recognize this as a false-positive from the fall detector.

% \include{Theory}
% \include{Implementation}
% \include{Test_Results}
% \documentclass[Preamble]{subfiles}
\begin{document}

\chapter{Conclusion}
Approximately 1-2 pages covering conclusion, discussion, and perspectives.
\section{Conclusion}
Conlude on your investigations.
\section{Discussion}
Discuss your project work.
\section{Perspectives}
What are the perspectives on the technology and your prototype? 


\end{document} 

% \newpage	
% \appendix

% \bibliographystyle{ieeetr}
% \bibliography{Referencer}
% \printbibliography

% \input{AppendixList}


\end{document}