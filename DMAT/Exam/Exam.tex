\documentclass[english,10pt,a4paper]{article}
%Preamble

% Følgende er til koder.
%----------------------------------------------------------
%\begin{lstlisting}[caption=Overskrift på boks, style=Code-C++, label=lst:referenceLabel]
%public void hello(){}
%\end{lstlisting}
%----------------------------------------------------------

%Exstra space
\usepackage{xspace}
%Navn på bokse efterfulgt af \xspace (hvis det skal være mellemrum
%gives det med denne udvidelse. Ellers ingen mellemrum.
\newcommand{\codeTitle}{Code snippet\xspace}

%Pakker der skal bruges til lstlisting
\usepackage{listings}
\usepackage{color}
\usepackage{textcomp}
\definecolor{listinggray}{gray}{0.9}
\definecolor{lbcolor}{rgb}{0.9,0.9,0.9}
\renewcommand{\lstlistingname}{\codeTitle}
\lstdefinestyle{Code}
{
	keywordstyle	= \bfseries\ttfamily\color[rgb]{0,0,1},
	identifierstyle	= \ttfamily,
	commentstyle	= \color[rgb]{0.133,0.545,0.133},
	stringstyle		= \ttfamily\color[rgb]{0.627,0.126,0.941},
	showstringspaces= false,
	basicstyle		= \small,
	numberstyle		= \footnotesize,
%	numbers			= left, % Tal? Udkommenter hvis ikke
	stepnumber		= 2,
	numbersep		= 6pt,
	tabsize			= 2,
	breaklines		= true,
	prebreak 		= \raisebox{0ex}[0ex][0ex]{\ensuremath{\hookleftarrow}},
	breakatwhitespace= false,
%	aboveskip		= {1.5\baselineskip},
  	columns			= fixed,
  	upquote			= true,
  	extendedchars	= true,
 	backgroundcolor = \color{lbcolor},
	lineskip		= 1pt,
%	xleftmargin		= 17pt,
%	framexleftmargin= 17pt,
	framexrightmargin	= 0pt, %6pt
%	framexbottommargin	= 4pt,
}

%Bredde der bruges til indryk
%Den skal være 6 pt mindre
\usepackage{calc}
\newlength{\mywidth}
\setlength{\mywidth}{1.435\textwidth} % Hvis bredden header ikke virker er dette hvad skal ændres!


% Forskellige styles for forskellige kodetyper
\usepackage{caption}
\DeclareCaptionFont{white}{\color{white}}
\DeclareCaptionFormat{listing}%
{\colorbox[cmyk]{0.43, 0.35, 0.35,0.35}{\parbox{\mywidth}{\hspace{5pt}#1#2#3}}}
\captionsetup[lstlisting]
{
	format			= listing,
	labelfont		= white,
	textfont		= white, 
	singlelinecheck	= false, 
	width			= \mywidth,
	margin			= 0pt, 
	font			= {bf,footnotesize}
}

\lstdefinestyle{Code-C} {language=C, style=Code}
\lstdefinestyle{Code-Java} {language=Java, style=Code}
\lstdefinestyle{Code-C++} {language=[Visual]C++, style=Code}
\lstdefinestyle{Code-VHDL} {language=VHDL, style=Code}
\lstdefinestyle{Code-Bash} {language=Bash, style=Code}
\lstdefinestyle{Code-Matlab} {language=Matlab, style=Code}
\lstdefinestyle{Code-Prolog} {language=Prolog, style=Code}
%Speciel skrift for enkelt linje kode
%--------------------------------------------------
%Udskriver med fonten 'Courier'
%Mere info her: http://tex.stackexchange.com/questions/25249/how-do-i-use-a-particular-font-for-a-small-section-of-text-in-my-document
%Eksempel: Funktionen \code{void Hello()} giver et output
%--------------------------------------------------
\newcommand{\code}[1]{{\fontfamily{pcr}\selectfont #1}}

%Seperated files
%--------------------------------------------------
%Opret filer således:
%\documentclass[Navn-på-hovedfil]{subfiles}
%\begin{document}
% Indmad
%\end{document}
%
% I hovedfil inkluderes således:
% \subfile{navn-på-subfil}
%--------------------------------------------------
\usepackage{subfiles}
%Text typesetting
%--------------------------------------------------------
\usepackage[T1]{fontenc} 	% Can use danish characters
\usepackage[utf8]{inputenc} % Input encoding. Can be used on Linux, Mac and Windows         
\usepackage[danish]{babel} 	% Split words accoding to English
\usepackage{lmodern} 		% Font

\setlength\parindent{0pt} 	% No indent
\setlength\parskip{12pt} 	% More than a single line break will give ONE linebreak.

%Margin
\usepackage[left=2cm,right=2cm,top=2.5cm,bottom=2cm]{geometry}

%Margin
\usepackage[left=3cm,right=2cm,top=2.5cm,bottom=2cm]{geometry}

%Mellemrum mellem linjerne    
\linespread{1.5}

\title{Assignment 2 for TEDI}
\author{Rasmus Bækgaard, 10893}
\date{April 28, 2014}

\title{Exam quistions}
\author{10893, Rasmus Bækgaard}
\date{August 30th, 2013}
\begin{document}
\maketitle

\section{Set}

\begin{theo}[Symbols] 

\begin{itemize}
\item \textbf{N} -- natural number (positive integers)
\item \textbf{Z} -- integers -- hele tal.
\item \textbf{Q} -- rational numbers -- $\dfrac{m}{n}$ where $m, n \in \mathbb{Z}$
\item \textbf{I} -- irrational numbers -- like $\sqrt{2}, \pi$
\item \textbf{R} -- real numbers -- Everything with and without comma.
\item \textbf{C} -- complex numbers -- $a+b\cdot i$
\item $\emptyset$ -- Empty set -- nothing
\end{itemize}

\end{theo}

\begin{theo}[Cardinal number / cardinality] 
The amount of elements in a set:\\
$|S| = \{ 2, -3, \emptyset\} = 3$

\end{theo}




\subsection{Subset}
\begin{itemize}
\item A set within a set: $S=\{a, b, c\}, T = \{a, b \}, U= \{ a, b, c\}, V=\{c\}$
\item Can be written as $T \subseteq S$
	\begin{itemize}
	\item Pronounced \textit{T is a \textbf{proper} subset of S}
	\end{itemize}
\item Can be written as $S \subseteq U$
	\begin{itemize}
	\item Pronounced \textit{S is a subset of U}
	\end{itemize}
	
\item If a set is not in another set it's written as $T \not \subseteq V$
\end{itemize}


\begin{theo}[Intervals] 
\begin{itemize}
\item Open, $(a,b) = \{ x \in \mathbb{R}: a < x < b \}$
\item Closed, $[a,b] = \{ x \in \mathbb{R}: a \leq x \leq b \}$
\item Half open, $[a,b) = \{ x \in \mathbb{R}: a \leq x < b \}$
\item Half closed, $(a,b] = \{ x \in \mathbb{R}: a < x \leq b \}$
\end{itemize}
\end{theo}


\begin{theo}[Power set] 
\begin{itemize}
\item A combination of all elements as \textbf{subsets}:
\item[] $A=\emptyset, B= \{a,b\}, C=\{1,2,3\}$
\item $\mathcal{P}(A) = \{\emptyset\} $
\item $\mathcal{P}(B) = \{\emptyset, \{a\}, \{b\}, \{a,b\} \}$
\item $\mathcal{P}(C) = \{\emptyset, \{1\}, \{2\}, \{3\}, \{1,2\}, \{1,3\}, \{2, 3\}, \{1,2,3\} \}$
\item Cardinality: $|\mathcal{P}(A)| = 2^{|A|}$
\item $\mathcal{P}(set) = \{subset : subset \subseteq set\}$
\end{itemize}
\end{theo}



\subsection{Set operations}
\begin{theo}[Union] 
\begin{itemize}
\item Means "in total"
\item Written as $A \cup B$
\item[] \includegraphics[scale=1]{union} 
\item SQL: \code{SELECT A, B IN Sets}
\end{itemize}

\end{theo}



\begin{theo}[Intersection] 
\begin{itemize}
\item Means "in common"
\item Written as $A \cap B$
\item[] \includegraphics[scale=1]{Intersection} 
\item If nothing is in common it's called \textbf{disjoint} and written $A \cap B = \emptyset$
\item Can be written as $A \cap B = \{ x \in A \vee x \in B : x \in A \wedge x \in B \}$
\item SQL: \\\code{SELECT A, B \\ FROM SetA \\INNER JOIN SetB \\ON A.a = b.a}
\end{itemize}

\end{theo}



\begin{theo}[Difference] 
\begin{itemize}
\item Means "What does A have which B does not"
\item Written as $A - B$
\item[] \includegraphics[scale=1]{Difference} 
\item Can also be written as $A \setminus B $
\item SQL: \\\code{SELECT A, B \\ FROM SetA \\INNER JOIN SetB \\ON A.a = b.a}
\end{itemize}

\end{theo}





\section{Proofs}


\begin{theo}[Trivial proof] 
\begin{itemize}
\item Something that is true -- no need to prove it.
\item Let $n \in \mathbb{Z}$. If $n^3 > 0$ then 3 is odd
\end{itemize}
\end{theo}



\begin{theo}[Vacouos proof] 
\begin{itemize}
\item If something is always proven wrong
\item Let $n \in \mathbb{Z}$. If \textbf{3 is even}, then $n^3 > 0$
\item[] Clearly wrong!
\end{itemize}
\end{theo}



\begin{theo}[Direct proof] 
\begin{itemize}
\item Show only what needs to be shown
\item $\forall x \in S, P(x) \Rightarrow Q(x)$
\item[] Show only that this is true and where it is false, e.g. show truth table.
\item Is shown from lemmas and other proves.
\end{itemize}
\end{theo}



\begin{theo}[Indirect proof / proof by contrapositive] 
\begin{itemize}
\item Reverse the result an means.
\item Let $x \in S$. If $Q(x)$, then $P(x) \Rightarrow$
\item[] Let $x \in S$. If $\neg Q(x)$, then $\neg P(x)$
\end{itemize}
\end{theo}



\begin{theo}[Proof by cases] 
\begin{itemize}
\item Do subcases and show they span the result.
\item Case 1: $n$ is even (U). Case 2: $n$ is odd (L). $\mathbb{Z} = U \cup L $
\item Case 1: $n>0$ (U). Case 2: $n<0$ (L). $\mathbb{Z} = U \cup L$
\item \textbf{Show $P \Leftrightarrow Q$ example}
\end{itemize}
\end{theo}






\end{document}