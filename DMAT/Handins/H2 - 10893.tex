\documentclass[english,10pt,a4paper]{article}
%Preamble

% Følgende er til koder.
%----------------------------------------------------------
%\begin{lstlisting}[caption=Overskrift på boks, style=Code-C++, label=lst:referenceLabel]
%public void hello(){}
%\end{lstlisting}
%----------------------------------------------------------

%Exstra space
\usepackage{xspace}
%Navn på bokse efterfulgt af \xspace (hvis det skal være mellemrum
%gives det med denne udvidelse. Ellers ingen mellemrum.
\newcommand{\codeTitle}{Code snippet\xspace}

%Pakker der skal bruges til lstlisting
\usepackage{listings}
\usepackage{color}
\usepackage{textcomp}
\definecolor{listinggray}{gray}{0.9}
\definecolor{lbcolor}{rgb}{0.9,0.9,0.9}
\renewcommand{\lstlistingname}{\codeTitle}
\lstdefinestyle{Code}
{
	keywordstyle	= \bfseries\ttfamily\color[rgb]{0,0,1},
	identifierstyle	= \ttfamily,
	commentstyle	= \color[rgb]{0.133,0.545,0.133},
	stringstyle		= \ttfamily\color[rgb]{0.627,0.126,0.941},
	showstringspaces= false,
	basicstyle		= \small,
	numberstyle		= \footnotesize,
%	numbers			= left, % Tal? Udkommenter hvis ikke
	stepnumber		= 2,
	numbersep		= 6pt,
	tabsize			= 2,
	breaklines		= true,
	prebreak 		= \raisebox{0ex}[0ex][0ex]{\ensuremath{\hookleftarrow}},
	breakatwhitespace= false,
%	aboveskip		= {1.5\baselineskip},
  	columns			= fixed,
  	upquote			= true,
  	extendedchars	= true,
 	backgroundcolor = \color{lbcolor},
	lineskip		= 1pt,
%	xleftmargin		= 17pt,
%	framexleftmargin= 17pt,
	framexrightmargin	= 0pt, %6pt
%	framexbottommargin	= 4pt,
}

%Bredde der bruges til indryk
%Den skal være 6 pt mindre
\usepackage{calc}
\newlength{\mywidth}
\setlength{\mywidth}{1.435\textwidth} % Hvis bredden header ikke virker er dette hvad skal ændres!


% Forskellige styles for forskellige kodetyper
\usepackage{caption}
\DeclareCaptionFont{white}{\color{white}}
\DeclareCaptionFormat{listing}%
{\colorbox[cmyk]{0.43, 0.35, 0.35,0.35}{\parbox{\mywidth}{\hspace{5pt}#1#2#3}}}
\captionsetup[lstlisting]
{
	format			= listing,
	labelfont		= white,
	textfont		= white, 
	singlelinecheck	= false, 
	width			= \mywidth,
	margin			= 0pt, 
	font			= {bf,footnotesize}
}

\lstdefinestyle{Code-C} {language=C, style=Code}
\lstdefinestyle{Code-Java} {language=Java, style=Code}
\lstdefinestyle{Code-C++} {language=[Visual]C++, style=Code}
\lstdefinestyle{Code-VHDL} {language=VHDL, style=Code}
\lstdefinestyle{Code-Bash} {language=Bash, style=Code}
\lstdefinestyle{Code-Matlab} {language=Matlab, style=Code}
\lstdefinestyle{Code-Prolog} {language=Prolog, style=Code}
%Speciel skrift for enkelt linje kode
%--------------------------------------------------
%Udskriver med fonten 'Courier'
%Mere info her: http://tex.stackexchange.com/questions/25249/how-do-i-use-a-particular-font-for-a-small-section-of-text-in-my-document
%Eksempel: Funktionen \code{void Hello()} giver et output
%--------------------------------------------------
\newcommand{\code}[1]{{\fontfamily{pcr}\selectfont #1}}

%Seperated files
%--------------------------------------------------
%Opret filer således:
%\documentclass[Navn-på-hovedfil]{subfiles}
%\begin{document}
% Indmad
%\end{document}
%
% I hovedfil inkluderes således:
% \subfile{navn-på-subfil}
%--------------------------------------------------
\usepackage{subfiles}
%Text typesetting
%--------------------------------------------------------
\usepackage[T1]{fontenc} 	% Can use danish characters
\usepackage[utf8]{inputenc} % Input encoding. Can be used on Linux, Mac and Windows         
\usepackage[danish]{babel} 	% Split words accoding to English
\usepackage{lmodern} 		% Font

\setlength\parindent{0pt} 	% No indent
\setlength\parskip{12pt} 	% More than a single line break will give ONE linebreak.

%Margin
\usepackage[left=2cm,right=2cm,top=2.5cm,bottom=2cm]{geometry}

%Margin
\usepackage[left=3cm,right=2cm,top=2.5cm,bottom=2cm]{geometry}

%Mellemrum mellem linjerne    
\linespread{1.5}

\title{Assignment 2 for TEDI}
\author{Rasmus Bækgaard, 10893}
\date{April 28, 2014}

\title{Handin 2}
\author{10893, Rasmus Bækgaard}
\date{September 6th, 2013}
\begin{document}
\maketitle

\section*{Problem 1}
$P(x) = x^2 + 1 = \text{even}, Q(x)=x^2 = \text{even}$ -- All values are odd.
\\
\\
$P(x)$: For all elements in \textit{S}, $P(x) = x^2 +1$. 
This is true, since all odd, square values are odd and plus one give an even number.
\\
$Q(x)$: For some elements in \textit{S}, $Q(x)=x^2$. This is false, since odd, square values are odd.


\section*{Problem 2}
Let P(n): $\dfrac{5n-6}{3}$ is an integer be an open sentence over the domain $\mathbb{Z}$. Determine, with explanations, whether the following statements are true:
\begin{enumerate}[a]
\item $\forall z \in \mathbb{Z}, P(z)$
\item[] This statement is false, since not all integer values will give an integer. 
Example: $\dfrac{5\cdot 1-6}{3}=\dfrac{-1}{3} \not = \mathbb{Z}$
\item $\exists z \in \mathbb{Z}, P(z)$
\item[] This statement is true, since some integer values will give an integer. 
Example: $\dfrac{5\cdot 6 - 6}{3}=\dfrac{24}{3} = 8  = \mathbb{Z}$
\end{enumerate}



\section*{Problem 3}
It is tempting to think that, if a statement involving the natural number \textit{n} is true for many consecutive values of \textit{n}, it must be true for all \textit{n}. 
In that connection the example from Euler is illustrative.
 Let $f(n) = n^2 + n + 41$ 
\begin{enumerate}[a]

\item Convince yourself (perhaps with the use of Mable or another tool) that f(1), f(2), … f(39) is prime, but f(40) is not. 
\item[] Write a script in \textbf{MatLab} as follow:
\end{enumerate}

\begin{lstlisting}[caption=title, style=Code-Matlab, label=lst:ref]
syms n;
f(n) =n^2+n+41;
ArrSize = 40;
Array=[zeros(ArrSize,3)];

for i=1: ArrSize
    Array(i,1) = i;
    Array(i,2) =f(i);
    Array(i,3) = isprime(Array(i,2));
end
Array
\end{lstlisting}
\begin{enumerate}[b]
\item[] Looking at the array list will show the values and if it's a prime number or not ( mark with '1' for \textit{true} and '0' for \textit{false}).
Since the array list does not contain a '0' they are all primes but the last element.

\item Show that, for any \textit{n} of the form $n = k^2 + 40$, \textit{f(n)} is not prime.
\item[] If $f(n)$ is not a prime, it means atleast 2 values can be added to each other and it will give $f(n)$.
\begin{align}
\text{If } n &= k^2 + 40 \\
f(n) &=\Big(k^2+40\Big)^2+\Big(k^2+40\Big)+41  & \text{Insert \textit{n} in } f(n)\\
f(k) &= k^4 + 40^2 + 2k^2+k^2+81 & \text{Dissolve the paratheses}\\
	&= k^4+3k^2+1681 & \text{Sum up}\\
	&= \Big(k^2+3k+41 \Big)^2 & \text{Square root it}\\
	&= \Big( k^2+3k+41 \Big) \cdot \Big( k^2+3k+41 \Big) & \text{rewrite it} \\
	&= x \cdot x & \text{Notice it is a product} \label{equ:product}
\end{align}
\item[] From (\ref{equ:product}) we can see, that $f(n)$ is not a prime, since any number, \textit{k}, inserted in $f(k) =\Big(k^2+3k+41 \Big)^2$ gives a non-prime which means for any value we assign \textit{k} in $n=k^2+40$ will give us a non-prime.
\end{enumerate}



\section*{Problem 4}
Prove that between any two different rational numbers there is another rational number.
\\
\\
Since a rational number is a number expressed by two numbers, like $\dfrac{m}{n}$ you can always find a number between two other rational numbers.
This is due to the amount of decimal will increase with higher numbers and if the rational numbers goes to rational you can always find a number between two rational numbers.


\section*{Problem 5}
Let $n \in N$. Prove that if $n^3 - 5n - 8 > 0$ then $n \geq 3$. What proof technique did you use? Why?
\\
\\
An easy illustration, Figure \ref{fig:prob5}, created in \textbf{Matlab} will show why only the natural numbers from 3 and above is positive:

\begin{lstlisting}[caption=title, style=Code-Matlab, label=lst:ref]
n=0.1:1:10;
f=power(n,3)-5*n-8;
figure(1)
subplot(1,1,1)
plot(n, f, 's', n, f)
axis([0 5 -25 100])
title('f(n)')
set(gca, 'XTick', [0:1:6])
set(gca, 'YTick', [0])
grid on
\end{lstlisting}
\begin{figure}[hbtp]
\centering
\includegraphics[scale=0.9]{H2-5}
\caption{Diagram of equation}
\label{fig:prob5}
\end{figure}


\section*{Problem 6}






\end{document}