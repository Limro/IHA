\documentclass[english,10pt,a4paper]{article}
%Preamble

% Følgende er til koder.
%----------------------------------------------------------
%\begin{lstlisting}[caption=Overskrift på boks, style=Code-C++, label=lst:referenceLabel]
%public void hello(){}
%\end{lstlisting}
%----------------------------------------------------------

%Exstra space
\usepackage{xspace}
%Navn på bokse efterfulgt af \xspace (hvis det skal være mellemrum
%gives det med denne udvidelse. Ellers ingen mellemrum.
\newcommand{\codeTitle}{Code snippet\xspace}

%Pakker der skal bruges til lstlisting
\usepackage{listings}
\usepackage{color}
\usepackage{textcomp}
\definecolor{listinggray}{gray}{0.9}
\definecolor{lbcolor}{rgb}{0.9,0.9,0.9}
\renewcommand{\lstlistingname}{\codeTitle}
\lstdefinestyle{Code}
{
	keywordstyle	= \bfseries\ttfamily\color[rgb]{0,0,1},
	identifierstyle	= \ttfamily,
	commentstyle	= \color[rgb]{0.133,0.545,0.133},
	stringstyle		= \ttfamily\color[rgb]{0.627,0.126,0.941},
	showstringspaces= false,
	basicstyle		= \small,
	numberstyle		= \footnotesize,
%	numbers			= left, % Tal? Udkommenter hvis ikke
	stepnumber		= 2,
	numbersep		= 6pt,
	tabsize			= 2,
	breaklines		= true,
	prebreak 		= \raisebox{0ex}[0ex][0ex]{\ensuremath{\hookleftarrow}},
	breakatwhitespace= false,
%	aboveskip		= {1.5\baselineskip},
  	columns			= fixed,
  	upquote			= true,
  	extendedchars	= true,
 	backgroundcolor = \color{lbcolor},
	lineskip		= 1pt,
%	xleftmargin		= 17pt,
%	framexleftmargin= 17pt,
	framexrightmargin	= 0pt, %6pt
%	framexbottommargin	= 4pt,
}

%Bredde der bruges til indryk
%Den skal være 6 pt mindre
\usepackage{calc}
\newlength{\mywidth}
\setlength{\mywidth}{1.435\textwidth} % Hvis bredden header ikke virker er dette hvad skal ændres!


% Forskellige styles for forskellige kodetyper
\usepackage{caption}
\DeclareCaptionFont{white}{\color{white}}
\DeclareCaptionFormat{listing}%
{\colorbox[cmyk]{0.43, 0.35, 0.35,0.35}{\parbox{\mywidth}{\hspace{5pt}#1#2#3}}}
\captionsetup[lstlisting]
{
	format			= listing,
	labelfont		= white,
	textfont		= white, 
	singlelinecheck	= false, 
	width			= \mywidth,
	margin			= 0pt, 
	font			= {bf,footnotesize}
}

\lstdefinestyle{Code-C} {language=C, style=Code}
\lstdefinestyle{Code-Java} {language=Java, style=Code}
\lstdefinestyle{Code-C++} {language=[Visual]C++, style=Code}
\lstdefinestyle{Code-VHDL} {language=VHDL, style=Code}
\lstdefinestyle{Code-Bash} {language=Bash, style=Code}
\lstdefinestyle{Code-Matlab} {language=Matlab, style=Code}
\lstdefinestyle{Code-Prolog} {language=Prolog, style=Code}
%Speciel skrift for enkelt linje kode
%--------------------------------------------------
%Udskriver med fonten 'Courier'
%Mere info her: http://tex.stackexchange.com/questions/25249/how-do-i-use-a-particular-font-for-a-small-section-of-text-in-my-document
%Eksempel: Funktionen \code{void Hello()} giver et output
%--------------------------------------------------
\newcommand{\code}[1]{{\fontfamily{pcr}\selectfont #1}}

%Seperated files
%--------------------------------------------------
%Opret filer således:
%\documentclass[Navn-på-hovedfil]{subfiles}
%\begin{document}
% Indmad
%\end{document}
%
% I hovedfil inkluderes således:
% \subfile{navn-på-subfil}
%--------------------------------------------------
\usepackage{subfiles}
%Text typesetting
%--------------------------------------------------------
\usepackage[T1]{fontenc} 	% Can use danish characters
\usepackage[utf8]{inputenc} % Input encoding. Can be used on Linux, Mac and Windows         
\usepackage[danish]{babel} 	% Split words accoding to English
\usepackage{lmodern} 		% Font

\setlength\parindent{0pt} 	% No indent
\setlength\parskip{12pt} 	% More than a single line break will give ONE linebreak.

%Margin
\usepackage[left=2cm,right=2cm,top=2.5cm,bottom=2cm]{geometry}

%Margin
\usepackage[left=3cm,right=2cm,top=2.5cm,bottom=2cm]{geometry}

%Mellemrum mellem linjerne    
\linespread{1.5}

\title{Assignment 2 for TEDI}
\author{Rasmus Bækgaard, 10893}
\date{April 28, 2014}

\title{Handin 3}
\author{10893, Rasmus Bækgaard}
\date{September 6th, 2013}
\begin{document}
\maketitle

\section*{Problem 1}
\textbf{Disprove the statement: If $n \in \{0, 1, 2, 3, 4\}$, then $2^n + 3^n + n(n-1)(n-2)$ is prime.}


To disprove this a contraexample is needed. 
Type in the following in Matlab:

\begin{lstlisting}[caption=Problem 1, style=Code-Matlab, label=lst:ref]
syms n;
f(n)=2^n + 3^n + n*(n-1)*(n-2);

Arr = zeros(5, 3);
for i = 1:5
    Arr(i,1)=i-1;
    Arr(i,2)=f(i-1);
    Arr(i,3)=isprime(Arr(i,2));
end

Arr
\end{lstlisting}
The result is listed in Figure \ref{fig:prob1}, where the first column is \textit{n}, second is the result and the third is a \code{bool} for if it is a prime or not.
\begin{figure}[hbtp]
\centering
\includegraphics[scale=0.9]{H3-1}
\caption{Result of Matlab script}
\label{fig:prob1}
\end{figure}
This shows, that $f(4) \not=$ prime.


\section*{Problem 2}
\textbf{Let $a,b \in \mathbb{Z}$. Disprove the statement; if $a\cdot b$ then $(a+b)^2$ are of opposite parity, then $a^2b^2$ and $a+ab+b$ are of opposite parity.}


To disprove this we have to make a counterexample where you can start by saying $a=1, b=1$ and then $a=1, b=2$.
To minimize time spend doing this and making a general solution for this, we make the following Matlab script:

\begin{lstlisting}[caption=Problem 2, style=Code-Java, label=lst:prob2]
arrSize=5;
Arr = zeros(arrSize, 5);
a=1;
for i = 1:arrSize
    b=i;
   Arr(i,1)=a*b;
   Arr(i,2)=(a+b)^2;
   Arr(i,3)= a^2*b^2;
   Arr(i,4)=a+a*b+b;
   c = mod(Arr(i,1),2) ~= mod(Arr(i,2),2);
   d = mod(Arr(i,3),2) ~= mod(Arr(i,4),2);
   if c == d
       Arr(i,5)=0;
   else
       Arr(i,5)=1;
   end
end

Arr
\end{lstlisting}
Which gives us the result in Figure \ref{fig:prob2}, where column 1 is $a\cdot b$, column 2 is $(a+b)^2$, column 3 is $a^2b^2$, column 4 is $a+ab+b$ and column 5 is whether the first two columns are not same parity and the next two columns are not same parity (1 indeicate 'no').
\begin{figure}[hbtp]
\centering
\includegraphics[scale=0.9]{H3-2}
\caption{Result of Matlab script}
\label{fig:prob2}
\end{figure}
This is only the first 5 combinations, where $a=1$ and \textit{b} runs from 1 to 5.
If $a=1, b=1$ the counterexample is shown.



\section*{Problem 3}
\textbf{Let $x, y \in \mathbb{R}^{+}$. Use a proof by contradiction to prove that if $x<y$ then $\sqrt{x} < \sqrt{y}$}
\\
\\
Lets rewrite the statement:
\begin{align}
\forall x, y \in \mathbb{R}^+, &P \Rightarrow Q & P = x < y \text{ and } Q = \sqrt{x}<\sqrt{y} \\
\exists x, y \in \mathbb{R}^+, &P \Rightarrow \neg Q \\
\neg Q &= \neg(\sqrt{x} < \sqrt{y})\\
\neg\sqrt{x} &< \neg \sqrt{y} \\
\sqrt{x} &\geq \sqrt{y} & \text{Remove negation}\\
x &\geq y & \text{Square both sides}
\end{align}

Since $x<y$ and $x\geq y$ is clearly not the same, the statement is proven.


\section*{Problem 4}
\textbf{Prove that there is no largest negative rational number. (Note: -1 is larger than -2.)}
\\
\\
A rational number is defined by $\mathbb{Q}: \{ \frac{a}{b}:a, b \in (\mathbb{Q} \cup \mathbb{Z})\}$.
By increasing \textit{b} and letting \textit{a} be a negativ number the number will be higher and higher but you can always find a higher number.
\[ \dfrac{-a}{b} < \dfrac{-a}{b+1}\]



\section*{Problem 5}
\textbf{Prove that there exits no positive integer \textit{x} such that $2x < x^2 < 3x$}
\\
\\
To prove there exist no positive integer, we can reduce the statement:
\begin{align}
2x < x^2 < 3x & \quad \text{Divide by } x\\
2 < x < 3 \label{equ:p5}
\end{align}
Since there exist no integer in Equation (\ref{equ:p5}) between 2 and 3, the statement is proven.



\section*{Problem 6}
\textbf{Prove that if \textit{n} is an odd integer, then $7n-5$ is even by}
\begin{enumerate}[a]
\item \textbf{direct proof}
\begin{itemize}
\item  Since an odd number, \textit{n}, times another odd numner, 7, is odd, substracting a thrid odd number, 5, will give an even number.
\item It can also be shown as follow:
\begin{align}
n&=odd\\
P(n)&=7(2x+1)-5\\
	&=14x+7-5\\
	&=2(x+2)\\
	&=2k
\end{align}
\end{itemize}
\item \textbf{proof by contrapositive}
\item[] This will give "\textit{n} is an odd integer, and $7n-5$ is odd"
\begin{align}
n&=even\\
P(n)&=7(2x)-5\\
	&=14x-5\\
	&=2x-5\\
	&=2(x-2)-1\\
	&=2k-1
\end{align}
\item \textbf{proof by contradiction}
\item []
\end{enumerate}




\end{document}