\documentclass[english,11pt,a4paper]{article}
%Preamble

% Følgende er til koder.
%----------------------------------------------------------
%\begin{lstlisting}[caption=Overskrift på boks, style=Code-C++, label=lst:referenceLabel]
%public void hello(){}
%\end{lstlisting}
%----------------------------------------------------------

%Exstra space
\usepackage{xspace}
%Navn på bokse efterfulgt af \xspace (hvis det skal være mellemrum
%gives det med denne udvidelse. Ellers ingen mellemrum.
\newcommand{\codeTitle}{Code snippet\xspace}

%Pakker der skal bruges til lstlisting
\usepackage{listings}
\usepackage{color}
\usepackage{textcomp}
\definecolor{listinggray}{gray}{0.9}
\definecolor{lbcolor}{rgb}{0.9,0.9,0.9}
\renewcommand{\lstlistingname}{\codeTitle}
\lstdefinestyle{Code}
{
	keywordstyle	= \bfseries\ttfamily\color[rgb]{0,0,1},
	identifierstyle	= \ttfamily,
	commentstyle	= \color[rgb]{0.133,0.545,0.133},
	stringstyle		= \ttfamily\color[rgb]{0.627,0.126,0.941},
	showstringspaces= false,
	basicstyle		= \small,
	numberstyle		= \footnotesize,
%	numbers			= left, % Tal? Udkommenter hvis ikke
	stepnumber		= 2,
	numbersep		= 6pt,
	tabsize			= 2,
	breaklines		= true,
	prebreak 		= \raisebox{0ex}[0ex][0ex]{\ensuremath{\hookleftarrow}},
	breakatwhitespace= false,
%	aboveskip		= {1.5\baselineskip},
  	columns			= fixed,
  	upquote			= true,
  	extendedchars	= true,
 	backgroundcolor = \color{lbcolor},
	lineskip		= 1pt,
%	xleftmargin		= 17pt,
%	framexleftmargin= 17pt,
	framexrightmargin	= 0pt, %6pt
%	framexbottommargin	= 4pt,
}

%Bredde der bruges til indryk
%Den skal være 6 pt mindre
\usepackage{calc}
\newlength{\mywidth}
\setlength{\mywidth}{1.435\textwidth} % Hvis bredden header ikke virker er dette hvad skal ændres!


% Forskellige styles for forskellige kodetyper
\usepackage{caption}
\DeclareCaptionFont{white}{\color{white}}
\DeclareCaptionFormat{listing}%
{\colorbox[cmyk]{0.43, 0.35, 0.35,0.35}{\parbox{\mywidth}{\hspace{5pt}#1#2#3}}}
\captionsetup[lstlisting]
{
	format			= listing,
	labelfont		= white,
	textfont		= white, 
	singlelinecheck	= false, 
	width			= \mywidth,
	margin			= 0pt, 
	font			= {bf,footnotesize}
}

\lstdefinestyle{Code-C} {language=C, style=Code}
\lstdefinestyle{Code-Java} {language=Java, style=Code}
\lstdefinestyle{Code-C++} {language=[Visual]C++, style=Code}
\lstdefinestyle{Code-VHDL} {language=VHDL, style=Code}
\lstdefinestyle{Code-Bash} {language=Bash, style=Code}
\lstdefinestyle{Code-Matlab} {language=Matlab, style=Code}
\lstdefinestyle{Code-Prolog} {language=Prolog, style=Code}
%Speciel skrift for enkelt linje kode
%--------------------------------------------------
%Udskriver med fonten 'Courier'
%Mere info her: http://tex.stackexchange.com/questions/25249/how-do-i-use-a-particular-font-for-a-small-section-of-text-in-my-document
%Eksempel: Funktionen \code{void Hello()} giver et output
%--------------------------------------------------
\newcommand{\code}[1]{{\fontfamily{pcr}\selectfont #1}}

%Seperated files
%--------------------------------------------------
%Opret filer således:
%\documentclass[Navn-på-hovedfil]{subfiles}
%\begin{document}
% Indmad
%\end{document}
%
% I hovedfil inkluderes således:
% \subfile{navn-på-subfil}
%--------------------------------------------------
\usepackage{subfiles}
%Text typesetting
%--------------------------------------------------------
\usepackage[T1]{fontenc} 	% Can use danish characters
\usepackage[utf8]{inputenc} % Input encoding. Can be used on Linux, Mac and Windows         
\usepackage[danish]{babel} 	% Split words accoding to English
\usepackage{lmodern} 		% Font

\setlength\parindent{0pt} 	% No indent
\setlength\parskip{12pt} 	% More than a single line break will give ONE linebreak.

%Margin
\usepackage[left=2cm,right=2cm,top=2.5cm,bottom=2cm]{geometry}

%Margin
\usepackage[left=3cm,right=2cm,top=2.5cm,bottom=2cm]{geometry}

%Mellemrum mellem linjerne    
\linespread{1.5}

\title{Assignment 2 for TEDI}
\author{Rasmus Bækgaard, 10893}
\date{April 28, 2014}

\title{Handin 4}
\author{10893, Rasmus Bækgaard}
\date{September 20th, 2013}
\begin{document}
\maketitle

\section*{Problem 1}
\textbf{Which of the  following sets are well-ordered? (Why/why not?)}
\begin{enumerate}[a]
\item $S=\{x \in \mathbb{Q}: x\geq -10\}$
\item[] $S$ is not well-ordered, since there is a subset which has no least number. A subset could be written as  $x=]10:\infty)$ 

\item $S=\{-2 ,-1, 0, 1, 2\}$
\item[] $S$ is well-ordered, since it is finite and and only with natural numbers which can be written as $S=[-2:2]$.


\item $S=\{x \in \mathbb{Q}: -1\leq x \leq 1\}$
\item[] $S$ is not well-ordered, since it is in $\mathbb{Q}$ and a subset can be written as $U=]-1,1[$ which means we have no least value when we can divide by a larger number to get an even smaller number.


\item $S=\{p :p \text{ is prime}\} = \{2, 3,5,7,9,11,13,\dots\}$
\item[] $S$ is well-ordered, since there is a lower limit for any subset.
\end{enumerate}



\section*{Problem 2}
\textbf{Use mathematical induction to prove that $1+5+9+\ldots+(4n-3)=2n^2-n$ for every positive integer $n$}.
\\
\\
\textbf{Base case} 
\\
$P(1)$ should be true:
\begin{align}
4n-3 &= 2n^2-n \\
4\cdot 1-3 &= 2\cdot 1^2-1 \\
1 &= 1 \label{equ:p2-1}
\end{align}
From (\ref{equ:p2-1}) we have shown the base case is in fact true.
\\
\\
\textbf{Induction step}
\\
The following should apply: $\forall k \in \mathbb{N}, P(k) \Rightarrow P(k+1)$
\begin{align}
1+5+9+\ldots+(4k-3) &= 2k^2-k & P(k)\\
1+5+9+\ldots+(4(k+1)-3) &= 2(k+1)^2-k+1 & P(k+1)\\
\overbrace{1+5+9+\ldots+(4k-3)}^{\text{This can be rewritten}}+(4(k+1)-3) &= \overbrace{k(4k-3)}^\text{to this}+(4(k+1)-3)\\
	&= 4k^2-3k+4k+4-3 & \text{Now what?} \\
	&= 4k^2+k+1 \label{equ:p2-2}\\
%	&= (4k^2+4k)-3k-3\\
%	&= 4(k^2+k)-3k-3 \\
%	&= (2k+1)^2-1-3k-3\\
	&= 2k^2+2k^2+k+1\\
 2(k+1)^2 &= 2(k^2+1+2k)\\
	&= 2k^2+2+4k\\
%	&= 4k^2+k+1 & \text{Insert in (\ref{equ:p2-2})}\\
%	&= 2k^2 + 2k^2+k+1\\
%	&=	
\end{align}



\section*{Problem 3}
\textbf{Prove that $2^n \geq n^3$ for every integer $n \geq 10$}.
\\
\\
\textbf{Base case:}
\\
$P(10): 2^{10}\geq 10^3 \Leftrightarrow 1024 \geq 1000$ which is true.
\\
\\
\textbf{Induction case:}
\\
\begin{align}
2^k &\geq 10^k & P(k)\\
2^{k+1} &\geq 10^{k+1} & P(k+1)
\end{align}


\end{document}