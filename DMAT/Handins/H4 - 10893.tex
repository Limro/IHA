\documentclass[english,11pt,a4paper]{article}
\usepackage{tcolorbox}
\usepackage{ulem} %math
\usepackage{amsmath}
\usepackage{amsfonts}
\usepackage{amssymb}
\usepackage{graphicx}
\usepackage{enumerate}


%Create a box for theorems
%\begin{theo}[titel] %optional
%tekst
%\end{theo}
\newenvironment{theo}[1][Vigtigt]{%
\begin{tcolorbox}[colback=green!5,colframe=green!40!black,title=\textbf{#1}]
}{%
\end{tcolorbox}
}




%Create a square matrix
%\begin{ArgMat}{2}
%21 & 22 & 23 \\  
%a & b & c
%\end{ArgMat}
%
% Info: http://tex.stackexchange.com/questions/2233/whats-the-best-way-make-an-augmented-coefficient-matrix
%
\newenvironment{ArgMat}{%
$
  \left[\begin{array}{@{}*{100}{r}r@{}}
}{%
  \end{array}\right]
  $
}

\newenvironment{deter}{%
$
  \left|\begin{array}{@{}*{100}{r}r@{}}
}{%
  \end{array}\right|
  $
}


%Create multiple lines with holes
%\begin{SysEqu}
%x_1 && &- &5x_3 &+ &2x_4=& 1 \\
%x_1 &+ &x_2 &+ &x_3 && =& 4 \\
%&&&&&&0 =& 0
%\end{SysEqu}
\newenvironment{SysEqu}{%
$  \setlength\arraycolsep{0.1em}
  \begin{array}{@{}*{100}{r}r@{}}
}{%
  \end{array}$
}

%Create solution for x_1, x_n...
%\begin{solu}
%x_1 &= d \\
%x_2 &= e \\
%x_3 &= s
%\end{solu}
\newenvironment{solu}{%
$
  \setlength\arraycolsep{0.1em}
  \left\{\begin{array}{@{}*{100}{r}r@{}}
}{%
  \end{array}\right.
$
}

\usepackage{lastpage}


\newcommand{\HRule}{\rule{\linewidth}{0.8mm}}

%Tekst i fotter
\newcommand{\footerText}{\thepage\xspace /\pageref{LastPage}}
\newcommand{\ProjectName}{433 MHz styring af AeroQuad}


\chapterstyle{hangnum}




\nouppercaseheads
\makepagestyle{mystyle} 

\makeevenhead{mystyle}{}{\\ \leftmark}{} 
\makeoddhead{mystyle}{}{\\ \leftmark}{} 
\makeevenfoot{mystyle}{}{\footerText}{} 
\makeoddfoot{mystyle}{}{\footerText}{} 
\makeatletter
\makepsmarks{mystyle}{% Overskriften på sidehovedet
  \createmark{chapter}{left}{shownumber}{\@chapapp\ }{.\ }} 
\makeatother
\makefootrule{mystyle}{\textwidth}{\normalrulethickness}{0.4pt}
\makeheadrule{mystyle}{\textwidth}{\normalrulethickness}

\makepagestyle{plain}
\makeevenhead{plain}{}{}{}
\makeoddhead{plain}{}{}{}
\makeevenfoot{plain}{}{\footerText}{}
\makeoddfoot{plain}{}{\footerText}{}
\makefootrule{plain}{\textwidth}{\normalrulethickness}{0.4pt}

\pagestyle{mystyle}

%%----------------------------------------------------------------------
%
%%Redefining chapter style
%%\renewcommand\chapterheadstart{\vspace*{\beforechapskip}}
%\renewcommand\chapterheadstart{\vspace*{10pt}}
%\renewcommand\printchaptername{\chapnamefont }%\@chapapp}
%\renewcommand\chapternamenum{\space}
%\renewcommand\printchapternum{\chapnumfont \thechapter}
%\renewcommand\afterchapternum{\space: }%\par\nobreak\vskip \midchapskip}
%\renewcommand\printchapternonum{}
%\renewcommand\printchaptertitle[1]{\chaptitlefont #1}
\setlength{\beforechapskip}{0pt} 
\setlength{\afterchapskip}{0pt} 
%\setlength{\voffset}{0pt} 
\setlength{\headsep}{25pt}
%\setlength{\topmargin}{35pt}
%%\setlength{\headheight}{102pt}
%\setlength{\textheight}{302pt}
\renewcommand\afterchaptertitle{\par\nobreak\vskip \afterchapskip}
%%----------------------------------------------------------------------




%Sidehoved og -fod pakke
%Margin
\usepackage[left=2cm,right=2cm,top=2.5cm,bottom=2cm]{geometry}
\usepackage{lastpage}



%%URL kommandoer og sidetal farve
%%Kaldes med \url{www...}
%\usepackage{color} %Skal også bruges
\usepackage{hyperref}
\hypersetup{ 
	colorlinks	= true, 	% false: boxed links; true: colored links
    urlcolor	= blue,		% color of external links
    linkcolor	= black, 	% color of page numbers
    citecolor	= blue,
}



%Mellemrum mellem linjerne    
\linespread{1.5}


%Seperated files
%--------------------------------------------------
%Opret filer således:
%\documentclass[Navn-på-hovedfil]{subfiles}
%\begin{document}
% Indmad
%\end{document}
%
% I hovedfil inkluderes således:
% \subfile{navn-på-subfil}
%--------------------------------------------------
\usepackage{subfiles}

%Prevent wierd placement of figures
%\usepackage[section]{placeins}

%Standard sti at søge efter billeder
%--------------------------------------------------
%\begin{figure}[hbtp]
%\centering
%\includegraphics[scale=1]{filnavn-for-png}
%\caption{Titel}
%\label{fig:referenceNavn}
%\end{figure}
%--------------------------------------------------
\usepackage{graphicx}
\usepackage{subcaption}
\usepackage{float}
\graphicspath{{../Figures/}}

%Speciel skrift for enkelt linje kode
%--------------------------------------------------
%Udskriver med fonten 'Courier'
%Mere info her: http://tex.stackexchange.com/questions/25249/how-do-i-use-a-particular-font-for-a-small-section-of-text-in-my-document
%Eksempel: Funktionen \code{void Hello()} giver et output
%--------------------------------------------------
\newcommand{\code}[1]{{\fontfamily{pcr}\selectfont #1}}


% Følgende er til koder.
%----------------------------------------------------------
%\begin{lstlisting}[caption=Overskrift på boks, style=Code-C++, label=lst:referenceLabel]
%public void hello(){}
%\end{lstlisting}
%----------------------------------------------------------

%Exstra space
\usepackage{xspace}
%Navn på bokse efterfulgt af \xspace (hvis det skal være mellemrum
%gives det med denne udvidelse. Ellers ingen mellemrum.
\newcommand{\codeTitle}{Kodeudsnit\xspace}

%Pakker der skal bruges til lstlisting
\usepackage{listings}
\usepackage{color}
\usepackage{textcomp}
\definecolor{listinggray}{gray}{0.9}
\definecolor{lbcolor}{rgb}{0.9,0.9,0.9}
\renewcommand{\lstlistingname}{\codeTitle}
\lstdefinestyle{Code}
{
	keywordstyle	= \bfseries\ttfamily\color[rgb]{0,0,1},
	identifierstyle	= \ttfamily,
	commentstyle	= \color[rgb]{0.133,0.545,0.133},
	stringstyle		= \ttfamily\color[rgb]{0.627,0.126,0.941},
	showstringspaces= false,
	basicstyle		= \small,
	numberstyle		= \footnotesize,
%	numbers			= left, % Tal? Udkommenter hvis ikke
	stepnumber		= 2,
	numbersep		= 6pt,
	tabsize			= 2,
	breaklines		= true,
	prebreak 		= \raisebox{0ex}[0ex][0ex]{\ensuremath{\hookleftarrow}},
	breakatwhitespace= false,
%	aboveskip		= {1.5\baselineskip},
  	columns			= fixed,
  	upquote			= true,
  	extendedchars	= true,
 	backgroundcolor = \color{lbcolor},
	lineskip		= 1pt,
%	xleftmargin		= 17pt,
%	framexleftmargin= 17pt,
	framexrightmargin	= 0pt, %6pt
%	framexbottommargin	= 4pt,
}

%Bredde der bruges til indryk
%Den skal være 6 pt mindre
\usepackage{calc}
\newlength{\mywidth}
\setlength{\mywidth}{\textwidth-6pt}


% Forskellige styles for forskellige kodetyper
\usepackage{caption}
\DeclareCaptionFont{white}{\color{white}}
\DeclareCaptionFormat{listing}%
{\colorbox[cmyk]{0.43, 0.35, 0.35,0.35}{\parbox{\mywidth}{\hspace{5pt}#1#2#3}}}
\captionsetup[lstlisting]
{
	format			= listing,
	labelfont		= white,
	textfont		= white, 
	singlelinecheck	= false, 
	width			= \mywidth,
	margin			= 0pt, 
	font			= {bf,footnotesize}
}

\lstdefinestyle{Code-C} {language=C, style=Code}
\lstdefinestyle{Code-Java} {language=Java, style=Code}
\lstdefinestyle{Code-C++} {language=[Visual]C++, style=Code}
\lstdefinestyle{Code-VHDL} {language=VHDL, style=Code}
\lstdefinestyle{Code-Bash} {language=Bash, style=Code}

%Text typesetting
%--------------------------------------------------------
%\usepackage{baskervald}
\usepackage{lmodern}
\usepackage[T1]{fontenc}              
\usepackage[utf8]{inputenc}         
\usepackage[english]{babel}       

\setlength{\parindent}{0pt}
\nonzeroparskip

%\setaftersubsecskip{1sp}
%\setaftersubsubsecskip{1sp}
 


%Dybde på indholdsfortegnelse
%----------------------------------------------------------
%Chapter, section, subsection, subsubsection
%----------------------------------------------------------
\setcounter{secnumdepth}{3}
\setcounter{tocdepth}{3}


%Tables
%----------------------------------------------------------
\usepackage{tabularx}
\usepackage{array}
\usepackage{multirow} 
\usepackage{multicol} 
\usepackage{booktabs}
\usepackage{wrapfig}
\renewcommand{\arraystretch}{1.5}



%Misc
%----------------------------------------------------------
\usepackage{cite}
\usepackage{appendix}
\usepackage{amssymb}
\usepackage{url,ragged2e}
\usepackage{enumerate}
\usepackage{amsmath} %Math bibliotek


\usepackage{longtable}


\title{Handin 4}
\author{10893, Rasmus Bækgaard}
\date{September 20th, 2013}
\begin{document}
\maketitle

\section*{Problem 1}
\textbf{Which of the  following sets are well-ordered? (Why/why not?)}
\begin{enumerate}[a]
\item $S=\{x \in \mathbb{Q}: x\geq -10\}$
\item[] $S$ is not well-ordered, since there is a subset which has no least number. A subset could be written as  $x=(-10:\infty)$ 

\item $S=\{-2 ,-1, 0, 1, 2\}$
\item[] $S$ is well-ordered, since it is finite and and only with natural numbers which can be written as $S=[-2:2]$.


\item $S=\{x \in \mathbb{Q}: -1\leq x \leq 1\}$
\item[] $S$ is not well-ordered, since it is in $\mathbb{Q}$ and a subset can be written as $U=(-1,1]$ which means we have no least value when we can divide by a larger number to get an even smaller number.


\item $S=\{p :p \text{ is prime}\} = \{2, 3,5,7,9,11,13,\dots\}$
\item[] $S$ is well-ordered, since there is a lower limit for any subset.
\end{enumerate}



\section*{Problem 2}
\textbf{Use mathematical induction to prove that $1+5+9+\ldots+(4n-3)=2n^2-n$ for every positive integer $n$}.
\\
\\
\textbf{Base case} 
\\
$P(1)$ should be true:
\begin{align}
4n-3 &= 2n^2-n \\
4\cdot 1-3 &= 2\cdot 1^2-1 \\
1 &= 1 \label{equ:p2-1}
\end{align}
From (\ref{equ:p2-1}) we have shown the base case is in fact true.
\\
\\
\textbf{Induction step}
\\
The following should apply: $\forall k \in \mathbb{N}, P(k) \Rightarrow P(k+1)$
\begin{proof}
\begin{align}
1+5+9+\ldots+(4k-3) &= 2k^2-k & P(k)\\
1+5+9+\ldots+(4(k+1)-3) &= 2(k+1)^2-k+1 & P(k+1) \label{equ:2-2}\\
	&= \underbrace{1+5+9+\ldots+(4k-3)}_{\mathclap{\text{Lefthand side of P(k)}}}+(4(k+1)-3)\\
	&=2k^2-k+4k+1\\
	&=2k^2+3k+1\\
	&=2(k+1)^2-(k+1) 
\end{align}
The result the same as (\ref{equ:2-2}) true and therefor follows the Principle of Mathematical Induction.\end{proof}



\section*{Problem 3}
\textbf{Prove that $2^n \geq n^3$ for every integer $n \geq 10$}.
\begin{proof}
The inequality holds for $n=10$ since $2^{10} \geq 10^3 $. 
Assume that $2^k \geq k^3$, where $k$ is a nonnegative integer. We show that $2^{k+1} \geq (k+1)^3$. When $k=10$ , we have $2^{10} = 1024 \geq 1000 = 10^3$. We therefor assume that $k \geq 10$. Then
\begin{align}
2^{k+1} = 2\cdot 2^k &\geq k^3+3k^2+3k+1 \label{equ:3-1}\\
2\cdot 2^k &\geq 2\cdot k^3 & 2\cdot 10^2= 2048, 2\cdot 10^3 = 2000\\
	&\geq k^3+k^3 & \text{Isolate the first } k^3\\
	&\geq k^3+ 10k^2 & k^3 \geq 10k^2\\
	&= k^3+ 3k^2+7k^2\\
	&\geq k^3+3k^2+3k+1 & 7k^2 \geq 3k+1 \label{equ:3-2}
\end{align}
By the Principle of Mathematical Induction, $2^n \geq n^3$ for every integer $n \geq 10$, since (\ref{equ:3-1}) is equal to (\ref{equ:3-2}).
\end{proof}





\section*{Problem 4}
\textbf{Use the method of minimum counterexample to prove that $3|\Big(2^{2n}-1\Big)$ for every positive integer \textit{n}.}
\begin{proof}
Assume, to the contraty, that there are positive integers $n$ such, that $6\not |(2^{2n}-1$. Then there is a a smallest positive integer $n$ such that $6\not |(2^{2n}-1$.
Let $m$ be this integer.
If $n=1$, then $2^{2n}-1 =3$; while if $n=2$, then $2^{2n}-1=15$.
Sine $3|3$ and $3|15$, it follows that $3|(2^{2n}-1)$ for $n=1$ and $n=2$. Therefor, $m\geq 3$. So we can write $m=k+2$ where $1\leq k< m$. Observe that
\begin{align}
m^{2m}-1 &= m^2\cdot m^m -1 \\
	&= (k+2)^2 \cdot (k+2)^{k+2} -1 \\
	&= (k+2)^2 \cdot \Big((k+2)^k+(k+2)^2\Big)-1 \\
	&= (k^2+2k+4)\cdot (m^{m+2}+2^{m+2}
\end{align}
\end{proof}


\section*{Problem 5}
\textbf{Use the Strong Principle of Mathematical to prove the following.\\
Let $S= \{ i \in \mathbb{Z} : i \geq 2\}$ \\
and the $P$ be a subset of $S$ with the properties that $2,3 \in P$ \\
and if $n \in S$, then either $n \in P$ or $n=ab$, where $a,b\in S$. \\
Then every element of $S$ either belongs to $P$ or it can be expressed as a product of elements of $P$.}
\begin{proof}
The set, $S$, consist of only integers greater than 2 and $P$ is a subset of $S$.
Should $n$ be in the set $S$, it can be expressed as a product of two numbers in the subset, or it is in the subset it self.
\\
Case 1: $n$ is not a prime -- let's say 6. Here $n$ can be expressed as $2 \cdot 3 = 6$ and $n$ is therefor not in the subset, $P$.
\\
Case 2: $n$ is not a prime -- let's say 12. Here $n$ can be expressed as $3 \cdot 4 = 12$ and $n$ is not in the subset, but 4 is.
\\
Case 3: $n$ is a prime -- let's say 5. Here $n$ can not be expressed by the product of the elements in $P$ and is only the the set, $S$.
\\
With the cases we can see, that all elements of $S$ can be in the subset, $P$. 
Some elements of $S$ cannot be the product of elements in $P$ and must therefor be in the subset.
\end{proof}

\textbf{Mellem 2 og k betyder det, at \dots rigtigt}



\section*{Problem 6}
\textbf{Use the Strong Principle of Mathematical Induction to prove, that for each integer $n\geq 12$, there are non-negative integers $a$ and $b$ such the $n=3a+7b$.}


\end{document}