\documentclass[english,11pt,a4paper]{article}
\usepackage{tcolorbox}
\usepackage{ulem} %math
\usepackage{amsmath}
\usepackage{amsfonts}
\usepackage{amssymb}
\usepackage{graphicx}
\usepackage{enumerate}


%Create a box for theorems
%\begin{theo}[titel] %optional
%tekst
%\end{theo}
\newenvironment{theo}[1][Vigtigt]{%
\begin{tcolorbox}[colback=green!5,colframe=green!40!black,title=\textbf{#1}]
}{%
\end{tcolorbox}
}




%Create a square matrix
%\begin{ArgMat}{2}
%21 & 22 & 23 \\  
%a & b & c
%\end{ArgMat}
%
% Info: http://tex.stackexchange.com/questions/2233/whats-the-best-way-make-an-augmented-coefficient-matrix
%
\newenvironment{ArgMat}{%
$
  \left[\begin{array}{@{}*{100}{r}r@{}}
}{%
  \end{array}\right]
  $
}

\newenvironment{deter}{%
$
  \left|\begin{array}{@{}*{100}{r}r@{}}
}{%
  \end{array}\right|
  $
}


%Create multiple lines with holes
%\begin{SysEqu}
%x_1 && &- &5x_3 &+ &2x_4=& 1 \\
%x_1 &+ &x_2 &+ &x_3 && =& 4 \\
%&&&&&&0 =& 0
%\end{SysEqu}
\newenvironment{SysEqu}{%
$  \setlength\arraycolsep{0.1em}
  \begin{array}{@{}*{100}{r}r@{}}
}{%
  \end{array}$
}

%Create solution for x_1, x_n...
%\begin{solu}
%x_1 &= d \\
%x_2 &= e \\
%x_3 &= s
%\end{solu}
\newenvironment{solu}{%
$
  \setlength\arraycolsep{0.1em}
  \left\{\begin{array}{@{}*{100}{r}r@{}}
}{%
  \end{array}\right.
$
}

\usepackage{lastpage}


\newcommand{\HRule}{\rule{\linewidth}{0.8mm}}

%Tekst i fotter
\newcommand{\footerText}{\thepage\xspace /\pageref{LastPage}}
\newcommand{\ProjectName}{433 MHz styring af AeroQuad}


\chapterstyle{hangnum}




\nouppercaseheads
\makepagestyle{mystyle} 

\makeevenhead{mystyle}{}{\\ \leftmark}{} 
\makeoddhead{mystyle}{}{\\ \leftmark}{} 
\makeevenfoot{mystyle}{}{\footerText}{} 
\makeoddfoot{mystyle}{}{\footerText}{} 
\makeatletter
\makepsmarks{mystyle}{% Overskriften på sidehovedet
  \createmark{chapter}{left}{shownumber}{\@chapapp\ }{.\ }} 
\makeatother
\makefootrule{mystyle}{\textwidth}{\normalrulethickness}{0.4pt}
\makeheadrule{mystyle}{\textwidth}{\normalrulethickness}

\makepagestyle{plain}
\makeevenhead{plain}{}{}{}
\makeoddhead{plain}{}{}{}
\makeevenfoot{plain}{}{\footerText}{}
\makeoddfoot{plain}{}{\footerText}{}
\makefootrule{plain}{\textwidth}{\normalrulethickness}{0.4pt}

\pagestyle{mystyle}

%%----------------------------------------------------------------------
%
%%Redefining chapter style
%%\renewcommand\chapterheadstart{\vspace*{\beforechapskip}}
%\renewcommand\chapterheadstart{\vspace*{10pt}}
%\renewcommand\printchaptername{\chapnamefont }%\@chapapp}
%\renewcommand\chapternamenum{\space}
%\renewcommand\printchapternum{\chapnumfont \thechapter}
%\renewcommand\afterchapternum{\space: }%\par\nobreak\vskip \midchapskip}
%\renewcommand\printchapternonum{}
%\renewcommand\printchaptertitle[1]{\chaptitlefont #1}
\setlength{\beforechapskip}{0pt} 
\setlength{\afterchapskip}{0pt} 
%\setlength{\voffset}{0pt} 
\setlength{\headsep}{25pt}
%\setlength{\topmargin}{35pt}
%%\setlength{\headheight}{102pt}
%\setlength{\textheight}{302pt}
\renewcommand\afterchaptertitle{\par\nobreak\vskip \afterchapskip}
%%----------------------------------------------------------------------




%Sidehoved og -fod pakke
%Margin
\usepackage[left=2cm,right=2cm,top=2.5cm,bottom=2cm]{geometry}
\usepackage{lastpage}



%%URL kommandoer og sidetal farve
%%Kaldes med \url{www...}
%\usepackage{color} %Skal også bruges
\usepackage{hyperref}
\hypersetup{ 
	colorlinks	= true, 	% false: boxed links; true: colored links
    urlcolor	= blue,		% color of external links
    linkcolor	= black, 	% color of page numbers
    citecolor	= blue,
}



%Mellemrum mellem linjerne    
\linespread{1.5}


%Seperated files
%--------------------------------------------------
%Opret filer således:
%\documentclass[Navn-på-hovedfil]{subfiles}
%\begin{document}
% Indmad
%\end{document}
%
% I hovedfil inkluderes således:
% \subfile{navn-på-subfil}
%--------------------------------------------------
\usepackage{subfiles}

%Prevent wierd placement of figures
%\usepackage[section]{placeins}

%Standard sti at søge efter billeder
%--------------------------------------------------
%\begin{figure}[hbtp]
%\centering
%\includegraphics[scale=1]{filnavn-for-png}
%\caption{Titel}
%\label{fig:referenceNavn}
%\end{figure}
%--------------------------------------------------
\usepackage{graphicx}
\usepackage{subcaption}
\usepackage{float}
\graphicspath{{../Figures/}}

%Speciel skrift for enkelt linje kode
%--------------------------------------------------
%Udskriver med fonten 'Courier'
%Mere info her: http://tex.stackexchange.com/questions/25249/how-do-i-use-a-particular-font-for-a-small-section-of-text-in-my-document
%Eksempel: Funktionen \code{void Hello()} giver et output
%--------------------------------------------------
\newcommand{\code}[1]{{\fontfamily{pcr}\selectfont #1}}


% Følgende er til koder.
%----------------------------------------------------------
%\begin{lstlisting}[caption=Overskrift på boks, style=Code-C++, label=lst:referenceLabel]
%public void hello(){}
%\end{lstlisting}
%----------------------------------------------------------

%Exstra space
\usepackage{xspace}
%Navn på bokse efterfulgt af \xspace (hvis det skal være mellemrum
%gives det med denne udvidelse. Ellers ingen mellemrum.
\newcommand{\codeTitle}{Kodeudsnit\xspace}

%Pakker der skal bruges til lstlisting
\usepackage{listings}
\usepackage{color}
\usepackage{textcomp}
\definecolor{listinggray}{gray}{0.9}
\definecolor{lbcolor}{rgb}{0.9,0.9,0.9}
\renewcommand{\lstlistingname}{\codeTitle}
\lstdefinestyle{Code}
{
	keywordstyle	= \bfseries\ttfamily\color[rgb]{0,0,1},
	identifierstyle	= \ttfamily,
	commentstyle	= \color[rgb]{0.133,0.545,0.133},
	stringstyle		= \ttfamily\color[rgb]{0.627,0.126,0.941},
	showstringspaces= false,
	basicstyle		= \small,
	numberstyle		= \footnotesize,
%	numbers			= left, % Tal? Udkommenter hvis ikke
	stepnumber		= 2,
	numbersep		= 6pt,
	tabsize			= 2,
	breaklines		= true,
	prebreak 		= \raisebox{0ex}[0ex][0ex]{\ensuremath{\hookleftarrow}},
	breakatwhitespace= false,
%	aboveskip		= {1.5\baselineskip},
  	columns			= fixed,
  	upquote			= true,
  	extendedchars	= true,
 	backgroundcolor = \color{lbcolor},
	lineskip		= 1pt,
%	xleftmargin		= 17pt,
%	framexleftmargin= 17pt,
	framexrightmargin	= 0pt, %6pt
%	framexbottommargin	= 4pt,
}

%Bredde der bruges til indryk
%Den skal være 6 pt mindre
\usepackage{calc}
\newlength{\mywidth}
\setlength{\mywidth}{\textwidth-6pt}


% Forskellige styles for forskellige kodetyper
\usepackage{caption}
\DeclareCaptionFont{white}{\color{white}}
\DeclareCaptionFormat{listing}%
{\colorbox[cmyk]{0.43, 0.35, 0.35,0.35}{\parbox{\mywidth}{\hspace{5pt}#1#2#3}}}
\captionsetup[lstlisting]
{
	format			= listing,
	labelfont		= white,
	textfont		= white, 
	singlelinecheck	= false, 
	width			= \mywidth,
	margin			= 0pt, 
	font			= {bf,footnotesize}
}

\lstdefinestyle{Code-C} {language=C, style=Code}
\lstdefinestyle{Code-Java} {language=Java, style=Code}
\lstdefinestyle{Code-C++} {language=[Visual]C++, style=Code}
\lstdefinestyle{Code-VHDL} {language=VHDL, style=Code}
\lstdefinestyle{Code-Bash} {language=Bash, style=Code}

%Text typesetting
%--------------------------------------------------------
%\usepackage{baskervald}
\usepackage{lmodern}
\usepackage[T1]{fontenc}              
\usepackage[utf8]{inputenc}         
\usepackage[english]{babel}       

\setlength{\parindent}{0pt}
\nonzeroparskip

%\setaftersubsecskip{1sp}
%\setaftersubsubsecskip{1sp}
 


%Dybde på indholdsfortegnelse
%----------------------------------------------------------
%Chapter, section, subsection, subsubsection
%----------------------------------------------------------
\setcounter{secnumdepth}{3}
\setcounter{tocdepth}{3}


%Tables
%----------------------------------------------------------
\usepackage{tabularx}
\usepackage{array}
\usepackage{multirow} 
\usepackage{multicol} 
\usepackage{booktabs}
\usepackage{wrapfig}
\renewcommand{\arraystretch}{1.5}



%Misc
%----------------------------------------------------------
\usepackage{cite}
\usepackage{appendix}
\usepackage{amssymb}
\usepackage{url,ragged2e}
\usepackage{enumerate}
\usepackage{amsmath} %Math bibliotek


\usepackage{longtable}


\title{Handin 5}
\author{10893, Rasmus Bækgaard}
\date{September 27th, 2013}
\begin{document}
\maketitle

\section*{Problem 1}
\textbf{Consider the following statements:}
\\
$(1+2)^2-1^2=2^3\\
(1+2+3)^2-(1+2)^2=3^3\\
(1+2+3+4)^2-(1+2+3)^2=4^3$
\begin{enumerate}[a]
\item Based on the three statement given above, what is the next statement suggested by these?
\item[] $(1+2+3+4+5)^2-(1+2+3+4)^2=5^3$
\item \label{enu:1}What conjecture is suggested by these statements?
\item[] $(1+\ldots+n)^2-(1+\ldots+(n-1))^2=n^3$
\item Verify the conjecture in (\ref{enu:1})
\item[] Use the induction method:
\item[] 
\begin{align}
1+2+3+\ldots+n&=\dfrac{n(n+1)}{2}\\
\bigg(\dfrac{n(n+1)}{2}\bigg)^2-\Bigg(\dfrac{(n-1)\Big((n-1)+1\Big)}{2}\Bigg)^2
	&=\dfrac{1}{4}\Big(n^4+2n^3+n^2\Big)-\dfrac{1}{4}\Big(n^4-2n^3+n^2\Big)\\
	&=\dfrac{n^4+2n^3+n^2-n^4+2n^3-n^2}{4}\\
	&=\dfrac{4n^3}{4}\\
	&=n^3
\end{align}
\end{enumerate}



\section*{Problem 2}
\textbf{By an ordered partition of an integer $n \geq 2$ is meant a sequence of positive integers whose sum is $n$. For example, the ordered partition of 3 are $3,1+2,2+1,1+1+1$}
\begin{enumerate}[a]
\item Determine the ordered partition of 4
\item[] This gives 8 partitions:\\
$4, \\
1+3, 3+1, \\
2+2\\
1+1+2, 1+2+1, 2+1+1, \\
1+1+1+1$
\item Determine the ordered partition of 5
\item[] This gives 16 partitions: \\
$5, \\
1+4,4+1,\\
3+2,2+3,\\
3+1+1,1+3+1,3+1+1\\
1+2+2,2+1+2,2+2+1,\\
1+1+1+2,1+1+2+1,1+2+1+1,2+1+1+1,\\
1+1+1+1+1$
\item Make a conjecture converting the number of ordered partition of an integer $n\geq 2$.
\item[] For 3 it's 4 partitions, for 4 it's 8 partitions, for 5 it's 16 partitions.\\
This will be $n=2^{n-1}$
\end{enumerate}




\section*{Problem 3}
\textbf{Express the following quantified statement in symbols:}
\begin{enumerate}[a]
\item \label{enu:3}For every odd integer $n$, the integer $3n+1$ is even:
\item[]$\forall n, m \in \mathbb{Z}, n\%2 = 1: m = 3n+1 \Rightarrow m\%2=0$
\item Prove that the statement in (\ref{enu:3}) is true.
\item[] 
\begin{align}
n &= 2k+1\\
m &= 3n+1\\
	&=3(2k+1)+1\\
	&=6k+1+1\\
	&=2(3k+1)\\
	&=2a\label{equ:3}
\end{align}
From (\ref{equ:3}) we can see, that all odd integers gives an even number.
\end{enumerate}



\section*{Problem 4}
\textbf{Express the following quantified statement in symbols}
\begin{enumerate}[a]
\item \label{enu:4}There exists a positive even integer $n$, such that $3n+2^{n-2}$ is odd.
\item[] This can be written as follow:
\begin{align}
\exists n, m \in \mathbb{Z}^+, n\%2=0 : m= 3n+2^{n-2} \Rightarrow m\%2=1
\end{align}
\item Prove that the statement in (\ref{enu:4}) is true.
\item[] To prove the statement in (\ref{enu:4}) is true, we write $n=2$:
\begin{align}
m &= 3\cdot 2+2^{2-2}\\
	&=6+2^0\\
	&=6+1\\
	&=7
\end{align}
\end{enumerate}



\section*{Problem 5}
\textbf{Prove or disprove: The sum of every five consecutive integers is divisible by 5 and the sum of no six consecutive integers is divisible by 6}
\\
\\
It can be written as follow:
\begin{align}
\dfrac{n+(n+1)+(n+2)+(n+3)+(n+4)}{5} 
	&=\dfrac{5n+10}{5}\\
	&=n+5 \\
\dfrac{n+(n+1)+(n+2)+(n+3)+(n+4)+(n+5)}{6}
	&= \dfrac{6n+15}{6} \\
	&= n\cdot \dfrac{15}{6} \label{equ:5}
\end{align}
From (\ref{equ:5}) we can see that it is not an integer.
Therefor the statement is false.




\end{document}