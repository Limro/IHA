\documentclass[english,11pt,a4paper]{article}
\usepackage{tcolorbox}
\usepackage{ulem} %math
\usepackage{amsmath}
\usepackage{amsfonts}
\usepackage{amssymb}
\usepackage{graphicx}
\usepackage{enumerate}


%Create a box for theorems
%\begin{theo}[titel] %optional
%tekst
%\end{theo}
\newenvironment{theo}[1][Vigtigt]{%
\begin{tcolorbox}[colback=green!5,colframe=green!40!black,title=\textbf{#1}]
}{%
\end{tcolorbox}
}




%Create a square matrix
%\begin{ArgMat}{2}
%21 & 22 & 23 \\  
%a & b & c
%\end{ArgMat}
%
% Info: http://tex.stackexchange.com/questions/2233/whats-the-best-way-make-an-augmented-coefficient-matrix
%
\newenvironment{ArgMat}{%
$
  \left[\begin{array}{@{}*{100}{r}r@{}}
}{%
  \end{array}\right]
  $
}

\newenvironment{deter}{%
$
  \left|\begin{array}{@{}*{100}{r}r@{}}
}{%
  \end{array}\right|
  $
}


%Create multiple lines with holes
%\begin{SysEqu}
%x_1 && &- &5x_3 &+ &2x_4=& 1 \\
%x_1 &+ &x_2 &+ &x_3 && =& 4 \\
%&&&&&&0 =& 0
%\end{SysEqu}
\newenvironment{SysEqu}{%
$  \setlength\arraycolsep{0.1em}
  \begin{array}{@{}*{100}{r}r@{}}
}{%
  \end{array}$
}

%Create solution for x_1, x_n...
%\begin{solu}
%x_1 &= d \\
%x_2 &= e \\
%x_3 &= s
%\end{solu}
\newenvironment{solu}{%
$
  \setlength\arraycolsep{0.1em}
  \left\{\begin{array}{@{}*{100}{r}r@{}}
}{%
  \end{array}\right.
$
}

\usepackage{lastpage}


\newcommand{\HRule}{\rule{\linewidth}{0.8mm}}

%Tekst i fotter
\newcommand{\footerText}{\thepage\xspace /\pageref{LastPage}}
\newcommand{\ProjectName}{433 MHz styring af AeroQuad}


\chapterstyle{hangnum}




\nouppercaseheads
\makepagestyle{mystyle} 

\makeevenhead{mystyle}{}{\\ \leftmark}{} 
\makeoddhead{mystyle}{}{\\ \leftmark}{} 
\makeevenfoot{mystyle}{}{\footerText}{} 
\makeoddfoot{mystyle}{}{\footerText}{} 
\makeatletter
\makepsmarks{mystyle}{% Overskriften på sidehovedet
  \createmark{chapter}{left}{shownumber}{\@chapapp\ }{.\ }} 
\makeatother
\makefootrule{mystyle}{\textwidth}{\normalrulethickness}{0.4pt}
\makeheadrule{mystyle}{\textwidth}{\normalrulethickness}

\makepagestyle{plain}
\makeevenhead{plain}{}{}{}
\makeoddhead{plain}{}{}{}
\makeevenfoot{plain}{}{\footerText}{}
\makeoddfoot{plain}{}{\footerText}{}
\makefootrule{plain}{\textwidth}{\normalrulethickness}{0.4pt}

\pagestyle{mystyle}

%%----------------------------------------------------------------------
%
%%Redefining chapter style
%%\renewcommand\chapterheadstart{\vspace*{\beforechapskip}}
%\renewcommand\chapterheadstart{\vspace*{10pt}}
%\renewcommand\printchaptername{\chapnamefont }%\@chapapp}
%\renewcommand\chapternamenum{\space}
%\renewcommand\printchapternum{\chapnumfont \thechapter}
%\renewcommand\afterchapternum{\space: }%\par\nobreak\vskip \midchapskip}
%\renewcommand\printchapternonum{}
%\renewcommand\printchaptertitle[1]{\chaptitlefont #1}
\setlength{\beforechapskip}{0pt} 
\setlength{\afterchapskip}{0pt} 
%\setlength{\voffset}{0pt} 
\setlength{\headsep}{25pt}
%\setlength{\topmargin}{35pt}
%%\setlength{\headheight}{102pt}
%\setlength{\textheight}{302pt}
\renewcommand\afterchaptertitle{\par\nobreak\vskip \afterchapskip}
%%----------------------------------------------------------------------




%Sidehoved og -fod pakke
%Margin
\usepackage[left=2cm,right=2cm,top=2.5cm,bottom=2cm]{geometry}
\usepackage{lastpage}



%%URL kommandoer og sidetal farve
%%Kaldes med \url{www...}
%\usepackage{color} %Skal også bruges
\usepackage{hyperref}
\hypersetup{ 
	colorlinks	= true, 	% false: boxed links; true: colored links
    urlcolor	= blue,		% color of external links
    linkcolor	= black, 	% color of page numbers
    citecolor	= blue,
}



%Mellemrum mellem linjerne    
\linespread{1.5}


%Seperated files
%--------------------------------------------------
%Opret filer således:
%\documentclass[Navn-på-hovedfil]{subfiles}
%\begin{document}
% Indmad
%\end{document}
%
% I hovedfil inkluderes således:
% \subfile{navn-på-subfil}
%--------------------------------------------------
\usepackage{subfiles}

%Prevent wierd placement of figures
%\usepackage[section]{placeins}

%Standard sti at søge efter billeder
%--------------------------------------------------
%\begin{figure}[hbtp]
%\centering
%\includegraphics[scale=1]{filnavn-for-png}
%\caption{Titel}
%\label{fig:referenceNavn}
%\end{figure}
%--------------------------------------------------
\usepackage{graphicx}
\usepackage{subcaption}
\usepackage{float}
\graphicspath{{../Figures/}}

%Speciel skrift for enkelt linje kode
%--------------------------------------------------
%Udskriver med fonten 'Courier'
%Mere info her: http://tex.stackexchange.com/questions/25249/how-do-i-use-a-particular-font-for-a-small-section-of-text-in-my-document
%Eksempel: Funktionen \code{void Hello()} giver et output
%--------------------------------------------------
\newcommand{\code}[1]{{\fontfamily{pcr}\selectfont #1}}


% Følgende er til koder.
%----------------------------------------------------------
%\begin{lstlisting}[caption=Overskrift på boks, style=Code-C++, label=lst:referenceLabel]
%public void hello(){}
%\end{lstlisting}
%----------------------------------------------------------

%Exstra space
\usepackage{xspace}
%Navn på bokse efterfulgt af \xspace (hvis det skal være mellemrum
%gives det med denne udvidelse. Ellers ingen mellemrum.
\newcommand{\codeTitle}{Kodeudsnit\xspace}

%Pakker der skal bruges til lstlisting
\usepackage{listings}
\usepackage{color}
\usepackage{textcomp}
\definecolor{listinggray}{gray}{0.9}
\definecolor{lbcolor}{rgb}{0.9,0.9,0.9}
\renewcommand{\lstlistingname}{\codeTitle}
\lstdefinestyle{Code}
{
	keywordstyle	= \bfseries\ttfamily\color[rgb]{0,0,1},
	identifierstyle	= \ttfamily,
	commentstyle	= \color[rgb]{0.133,0.545,0.133},
	stringstyle		= \ttfamily\color[rgb]{0.627,0.126,0.941},
	showstringspaces= false,
	basicstyle		= \small,
	numberstyle		= \footnotesize,
%	numbers			= left, % Tal? Udkommenter hvis ikke
	stepnumber		= 2,
	numbersep		= 6pt,
	tabsize			= 2,
	breaklines		= true,
	prebreak 		= \raisebox{0ex}[0ex][0ex]{\ensuremath{\hookleftarrow}},
	breakatwhitespace= false,
%	aboveskip		= {1.5\baselineskip},
  	columns			= fixed,
  	upquote			= true,
  	extendedchars	= true,
 	backgroundcolor = \color{lbcolor},
	lineskip		= 1pt,
%	xleftmargin		= 17pt,
%	framexleftmargin= 17pt,
	framexrightmargin	= 0pt, %6pt
%	framexbottommargin	= 4pt,
}

%Bredde der bruges til indryk
%Den skal være 6 pt mindre
\usepackage{calc}
\newlength{\mywidth}
\setlength{\mywidth}{\textwidth-6pt}


% Forskellige styles for forskellige kodetyper
\usepackage{caption}
\DeclareCaptionFont{white}{\color{white}}
\DeclareCaptionFormat{listing}%
{\colorbox[cmyk]{0.43, 0.35, 0.35,0.35}{\parbox{\mywidth}{\hspace{5pt}#1#2#3}}}
\captionsetup[lstlisting]
{
	format			= listing,
	labelfont		= white,
	textfont		= white, 
	singlelinecheck	= false, 
	width			= \mywidth,
	margin			= 0pt, 
	font			= {bf,footnotesize}
}

\lstdefinestyle{Code-C} {language=C, style=Code}
\lstdefinestyle{Code-Java} {language=Java, style=Code}
\lstdefinestyle{Code-C++} {language=[Visual]C++, style=Code}
\lstdefinestyle{Code-VHDL} {language=VHDL, style=Code}
\lstdefinestyle{Code-Bash} {language=Bash, style=Code}

%Text typesetting
%--------------------------------------------------------
%\usepackage{baskervald}
\usepackage{lmodern}
\usepackage[T1]{fontenc}              
\usepackage[utf8]{inputenc}         
\usepackage[english]{babel}       

\setlength{\parindent}{0pt}
\nonzeroparskip

%\setaftersubsecskip{1sp}
%\setaftersubsubsecskip{1sp}
 


%Dybde på indholdsfortegnelse
%----------------------------------------------------------
%Chapter, section, subsection, subsubsection
%----------------------------------------------------------
\setcounter{secnumdepth}{3}
\setcounter{tocdepth}{3}


%Tables
%----------------------------------------------------------
\usepackage{tabularx}
\usepackage{array}
\usepackage{multirow} 
\usepackage{multicol} 
\usepackage{booktabs}
\usepackage{wrapfig}
\renewcommand{\arraystretch}{1.5}



%Misc
%----------------------------------------------------------
\usepackage{cite}
\usepackage{appendix}
\usepackage{amssymb}
\usepackage{url,ragged2e}
\usepackage{enumerate}
\usepackage{amsmath} %Math bibliotek


\usepackage{longtable}


\title{Handin 6}
\author{10893, Rasmus Bækgaard}
\date{October 4th, 2013}
\begin{document}
\maketitle

\section*{Problem 1}
\textbf{Let $A= \{ a, b, c, d\}$. Give an example (with justification) of a relation $R$ on $A$ that has none of the following properties: reflexive, symmetric, transitive.}
\\
\\
Let have the following pairs: $(a,b), (b,c), (c,d)$.
\begin{itemize}
\item It is not reflective, since it does not contain any $a\textbf{R}a$.
\item It is not symmetric, since there is no $a\textbf{R}b \Rightarrow b\textbf{R}a$.
\item It is not transitive, since it is not $a\textbf{R}b \wedge b\textbf{R}c \Rightarrow a\textbf{R}c$

\end{itemize}


\section*{Problem 2}
\textbf{A relation $R$ is defined on $\mathbb{Z}$ by $a\textbf{R}b$ if $|a-b| \leq 2$. Which of the properties reflexive, symmetric and transitive does the relation $R$ possess? Justify your answers.}
\\
\\
The following holds:
\begin{itemize}
\item It is reflective, since two equal number will give 0, which is less than 2.
\item It is symmetric, since the absolute value of two numbers less or equal than 2 will be the same -- also when they are flipped.
\item It is NOT transitive: The pair (0,2) holds the criteria, (2,4) holds the criteria, but (0,4) does not. 
\end{itemize}




\section*{Problem 3}
Let $A$ and $B$ be sets with $|A|=|B|=4$.
\begin{enumerate}[a]
\item \label{enu:3} \textbf{Prove or disprove: If $R$ is a relation from $A$ to $B$ where $|R|=9$ and $R=R^{-1}$, then $A=B$}
\item[] Assume we have a the set $A=\{1,2,3,4\}$ and $B=\{1,2,3,5\}$. Bot sets has the cardinality of 4. 
\item[] A counter example is as follow: $R=\{ (1,1), (2,2), (3,3), (1,2), (2,1), (1,3), (3,1), (2,3), (3,2) \}$ where $|R| = 9$ and $R^{-1} = R$. $R$ is a subset of both $A$ and $B$.
However the criteria is furfilled but $A$ and $B$ are not equal.
Therefor it is false.


\item \textbf{Show that by making a small change in the statement in (\ref{enu:3}), a different response to the resulting statement can be obtained.}
\item[] Change $|R|=10$. 
Because the set $R$ is a subset of both $A$ and $B$, they will have the same elements, since another pair is needed. 
Because $R=R^{-1}$ the last pair must be the symmetric.
That leaves us with the same last element in both $A$ and $B$.
\end{enumerate}



\section*{Problem 4}
\textbf{Let $R$ be a relation defined on $\mathbb{Z}$ by $aRb$ if $a^3=b^3$. Show that $R$ is an equivalence relation on $\mathbb{Z}$ and determine the distinct equivalence classes.}
\begin{proof}
\begin{itemize}
\item[]
\item Since $a^3=a^3$ for each $a \in \mathbb{Z}$, it follows that $ a R a$ and $R$ is reflexive. 
\item Let $a,b \in \mathbb{Z}$ such that $a R b$. 
Then $a^3=b^3$ and therefore $b^3 = a^3$. Thus $b R a$ and $R$ is symmetric.
\item Let $a,b,c \in \mathbb{Z}$ such that $aRb$ and $bRc$. Thus $a^3=b^3$ and $b^3=c^3$. 
Hence $a^3=c^3$ and so $aRc$ and $R$ is transitive.
\end{itemize}
\end{proof}


\section*{Problem 5}
\textbf{Let $H=\{ 2^m : m \in \mathbb{Z}\}$. A relation $R$ is defined on the set $\mathbb{Q}^+$ of positive rationals by $aRb$ if $\dfrac{a}{b} \in H$}
\begin{enumerate}[a]
\item \textbf{Show that $R$ is an equivalence relation}
\begin{itemize}
\item Reflective: $\dfrac{a}{a} \Leftrightarrow 2^0$
\item[] Thereby $aRa$ and $R$ is reflective.

\item Symmetric: If $\dfrac{a}{b} =2^m$, then $\dfrac{b}{a}= 2^{-m}$ and therefor is symmetric.

\item Transitive: 
\begin{align}
\dfrac{a}{b} &= 2^m \\
\dfrac{b}{c} &= 2^n \\
\dfrac{a}{b} \cdot \dfrac{b}{c} &= 2^m \cdot 2^n \\ \dfrac{a}{c} &= 2^{n + m} \\
	&= 2^k, &k=m + n \\
	 &k \in \mathbb{Z}
\end{align} 
\end{itemize}
Here we can see, that $k$ is an integer and $\dfrac{a}{c} = 2^k$

\item \textbf{Describe the elements in the equivalence class [3]}
\item[] The elements must meet the requirements of  $\dfrac{x}{3} = 2^k \Leftrightarrow \dfrac{x}{2^k}=3$.
We can therefor state that $[3] = \{\forall x,m \in \mathbb{Z} : \dfrac{x}{2^k} = 3, 2^k\cdot x = 3\} = \{ \ldots, 1.5, 3,6,12, \ldots\}$
\end{enumerate}


\newpage
\section*{Problem 6}
\textbf{Two parts:}
\begin{enumerate}[a]
\item \label{enu:6}\textbf{Prove that the intersection of two equivalence relations on a nonempty set is an equivalence relation}
\item[]Suppose that $R_1$ and $R_2$ are two equivalence ralations defined on a set $S$.
Let $R= R_1 \cap R_2$.
First, we show that $R$ is reflexive.
Let $a \in S$. Since $R_1$ and $R_2$ are equivalence relations on $S$, it follows that $(a,a) \in R_1$ and $(a,a) \in R_2$. 
Thus $(a,a) \in R$ and so $R$ is reflexive. 
Assume that $aRb$, where $a, b \in S$.
Then $(a,b) \in R = r_a \cap R_2$. 
Thus $(a,b) \in R_1$ and $(a,b)\in R_2$. 
Since $R_1$ and $R_2$ are symmetric, $(b,a)\in R_1$ and $(b,a) \in R_2$. 
Thus $(b,a) \in R$ and so $bRa$.
Hence $R$ is symmetric.
Finally, show that R is transitive.

\item \textbf{Consider the equivalence relations $R_2$ and $R_3$ defined on $\mathbb{Z}$ by $aR_2b$ if $a \equiv b$ (mod 2) and $aR_3b$ if $a \equiv b$ (mod 3). By (\ref{enu:6}), $R_1=R_2 \cap R_3$ is an equivalence relation on $\mathbb{Z}$. Determine the distinct equivalence classes in $R_1$}
\item[] Let $a \in \mathbb{Z}$.
For $x \in \mathbb{Z}$, it follows that $xR_1a$ if and only if $xR_2a$ and $xR_3 a$. 
That is, $x R_1a$ if and only if $x \equiv a$(mod 2) and $x\equiv a$(mod 3).

First, suppose that $x \equiv a$ (mod 2) and $x\equiv a$(mod 3).
Hence $x=a+2k$ and $x=a+3\alpha$ for some integers $k$ and $\alpha$.
Therefor, $2k=3 \alpha$ and so $\alpha$ is even.
This $\alpha = 2m$ for some integer $m$, implying that $x=a+3\alpha=a+3(2m)=a+6m$ and so $x-a=6m$.
Hence $x \equiv a$ (mod 6).

If $x \equiv a$(mod 6), then $x \equiv a$(mod 2) and $x \equiv a$ (mod 3).
Thus $[a]=\{x \in \mathbb{Z}: x \equiv a$ (mod 6)$\}$.

\end{enumerate}


\section*{Problem 7}
\textbf{Let $A=\{1,2,3\}$ and $B=\{a, b, c, d\}$. Give an example of a relation $R$ from $A$ to $B$ containing exactly three elements such that $R$ is not a function from $A$ to $B$. Explain why $R$ is not a function.}
\\
\\
Let $R=\{(1,a),(2,b),(2,c)\}$.
This is not a function, because there is splitting in the relation.

\section*{Problem 8}
\textbf{For a function $f: A \rightarrow B$ and subsets $C$ and $D$ of $A$ and $E$ and $F$ of $B$, prove the following:}
\begin{enumerate}[a]
\item $f(C \cup D) = f(C)\cup f(D)$
\begin{align}
f(C \cup D) &= f(C)\cup f(D)\\
	&= \ldots \text{See Figure \ref{fig:venn}}
\end{align}
\item $f(C \cap D) \subseteq f(C)\cap f(D)$
\item $f(C)-F(D) \subseteq f(C-D)$
\item $f^{-1}(E\cup F)=f^{-1}(E) \cup f^{-1}(F)$
\item $f^{-1}(E\cap F)=f^{-1}(E) \cap f^{-1}(F)$
\item $f^{-1}(E - F)=f^{-1}(E) - f^{-1}(F)$
\end{enumerate}
\begin{figure}[hbtp]
\centering
\includegraphics[scale=0.75]{no}
\caption{My reaction for lacking material in the text book}
\label{fig:venn}
\end{figure}


\section*{Problem 9}
\textbf{Prove that the function $f: R \rightarrow R$ defined by $f(x)=7x-2$ is bijective.}
\\
\\
To prove the function is bijective if it is well-defined; that is, if $[a] = [b]$, then $f([a])=f([b])$.
Assume then that $[a]=[b]$.
Thus $a \equiv b$ (mod 1) and so $1|(a-b)$. 
Hence $a-b=k$ for some integer $k$. Therefor,
\begin{align}
(7a-2)-(7b-2) &= 7(a-b)\\
	&= 7k
\end{align}
Since $7k$ is an integer, $1|[(7a-2)-(7b-2)]$. Thus $(7a-2)\equiv (7b-2)$(mod 1) and $[7a-2]=[7b-2]$; so $f([a])=f([b])$. Hence $f$ is well-defined and thereby proven.

\section*{Problem 10}
chap. 9.5, p. 227-228\\

\textbf{Prove or disprove the following:}

\begin{enumerate}[a]
\item \textbf{If two functions $f: A \rightarrow B$ and $g: B \rightarrow C$ are both bijective, then $g \cirdot f: A \rightarrow C$ is bijective.}
\item[] Let $a \in A$ and suppose that $f(a)=b, g(b)=c$. Then
\begin{align}
(g\circ f)(a) &= g \circ f(a)\\
	&= g\circ (b)\\
	&= g(b)\\
	&= c
\end{align}
Thus $g \circ f = A \rightarrow C$ and thereby proven.

\item \label{enu:7-1}\textbf{Let $f: A \rightarrow B$ and $g: B \rightarrow C$ be two functions. If $g$ is onto, then $g \cirdot f: A \rightarrow C$ is onto.}
\item[] Let $f$ and $g$ be surjective functions and let $c \in C$. Since $g$ is surjective, there exists $b\in B$ such that $g(b)=c$. On the other hand, since $f$ is surjective, it follows that there exists $a \in A$ such that $f(a)=b$. Hence $(g \circ f)(a)=g(f(a))=g(b)=c$. Thereby proven.

\item \label{enu:7-2}\textbf{Let $f: A \rightarrow B$ and $g: B \rightarrow C$ be two functions. if $g$ is one-to-one, then $g \cirdot f:A \rightarrow C$ is one-to-one.}
\item[]Let $f$ and $g$ be injective functions. Assume that $(g\circ f)(a_1)=(f \circ g)(a_2)$, where $a_1, a_2 \in A$. By definition, $g(f(a_1))=g(f(a_2))$. Since $g$ is injective, it follows that $f(a_1)=f(a_2)$. However, since $f$ is injective, it follows that $a_1=a_2$. This implies that $g \circ f$ is injective.

\item \textbf{There exist a function $f: A \rightarrow B$ and $g: B \rightarrow C$ such that $f$ is not onto and $g\cirdot f: A \rightarrow C$ is onto.}
\item[] Disproven, due to (\ref{enu:7-1}) counts for all functions.

\item \textbf{There exist a function $f: A \rightarrow B$ and $g: B \rightarrow C$ such that $f$ is not one-to-one and $g\cirdot f: A \rightarrow C$ is one-to-one.}
\item[] Disproven, due to (\ref{enu:7-2}) counts for all functions.
\end{enumerate}





\end{document}