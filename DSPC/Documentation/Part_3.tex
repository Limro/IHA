\documentclass[Aflevering]{subfiles}
\begin{document}

\section{Konklusion}
Som to IKT-ingeni�rer har det p� nogle punkter v�ret en k�mpe udfordring at skulle udf�re dette projekt, fordi vi ikke er van til at v�re s� t�t p� hardware og signaler. Omvendt f�ler vi ogs� vi har f�et rigtig meget ud af det, netop fordi det har v�ret en udfordring.

Den viden vi har f�et gennem undervisningen har vi s�vidt styrket som brugt rigtig meget i projektet til at opn� netop den l�sning vi er kommet frem til. Nogle af de elementer vi har brugt i forbindelse med undervisningen havde vi ikke p� det tidspunkt helt styr p�, men des mere man arbejdede med det, des mere fik man en forst�else herfor. Herunder kan fx n�vnes det at designe, opbygge og teste komponenter, samt at samle dem til sidst.

Helt overordnet er vi ret tilfredse med vores endelige resultat og har opn�et det vi gerne ville med projektet. Vi har l�bende fundet ud af at ikke alt det vi lavede ville blive 100\% perfekte, men vi har ogs� af den grund fors�gt at l�gge v�gt p� det som ville give v�rdi for os og projektet. Vi har simpelt valgt at lave dette som et \textit{Proof-of-Concept} frem for et produkt der ville v�re klar til hylderne.
















\end{document}