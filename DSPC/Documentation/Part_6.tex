\documentclass[Aflevering]{subfiles}
\begin{document}


\subsection{Sinus kode}
\label{App:SinusKode}
\begin{lstlisting}[style=code-C++, caption=Sinus generation for data output, label=lst:sinusG]
// Write number of samples to RAM-controller
sendNumberOfSamples(full_samples);

// FIRST QUARTER
for(sample_i = 0; sample_i < quarter_samples; sample_i++)
{
	// Calculating first quarter. 
	// This array will be used for the remaining quarters
	quarter_sound_samples[sample_i] 
		= (1+sin((sample_i*2*PI)/(full_samples)))*HALF_MAX_CODEC_SIZE;
	sendNextSample(quarter_sound_samples[sample_i]);
}

// SECOND QUARTER
for(sample_i = quarter_samples; sample_i > 0; sample_i--)
{
	current_sample = quarter_sound_samples[sample_i-1];
	sendNextSample(current_sample);
}

// THIRD QUARTER
for(sample_i = 0; sample_i < quarter_samples; sample_i++)
{
	current_sample = MAX_CODEC_SIZE-quarter_sound_samples[sample_i];
	sendNextSample(current_sample);
}

// FOURTH QUARTER
for(sample_i = quarter_samples; sample_i > 0; sample_i--)
{
	current_sample 
		= MAX_CODEC_SIZE-quarter_sound_samples[sample_i-1];
	sendNextSample(current_sample);
	
}

\end{lstlisting}

\subsection{TransferProtocol}
\label{app:TP}
\begin{lstlisting}[style=code-VHDL, caption=TransferProtol valg af samples eller data, label=app:TP]
if address = "00000000" then			
	ramSamples_to_write := writedata(7 downto 0);
	index <= 0;
	
	if ramSamples_to_write = X"00" then
		ramSamples_to_read <= ramSamples_to_write;
		ram_to_play <= not ram_to_play;	
	end if;
	
--What to write on ram module
elsif address = "00000001" then	 	-- binary, address = 2
	ram_Addr <= index;				-- write addr to ram
	ram_Data <= writedata; 			-- write data to ram
	index <= index + 1;				-- increment index				    
end if;
\end{lstlisting}

\subsection{RamAccess-lagring}
\label{app:RA}
\begin{lstlisting}[style=code-VHDL, caption=Process for lagring i ram-modulerne]
process (clk, reset_n)
begin
	if reset_n = '0' then
		ram_block <= (others => (others => '0'));
		readData <= (others => '0');
		
	elsif rising_edge(clk) then
		if CS = '1' then
			ram_block(writeAddr) <= writedata;
		end if;
		
		readData <= ram_block(readAddr);
	end if;
end process;
\end{lstlisting}


\end{document}