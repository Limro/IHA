\documentclass[Aflevering]{subfiles}
\begin{document}

\subsection{test}
For at sikre, at VHDL-koden virkede som �nsket, er koden testet med programmet ModelSim 10.1b, hvor hvert komponent er testet individuelt samt en integreringstest for 2 af komponenterne.

Koden er testet s�ledes, at alle linjer er blevet afpr�vet, s� der ikke er en en linje der vil sende underlig data ud.
Ligeledes er der testet flere scenarier, hvor forskellige v�rdier sendes til og fra de forskellige komponenter.

Grundet den m�ngde af data der sendes til ram-modulerne for hver frekvens der �nskes afspillet, at der ikke testet n�r det antal der sendes over, men blot 

F�rste test er af komponentet \textit{TransportProtocol}, der st�r for overf�relsen af data fra microcontrolleren til ram-modulerne.
Herunder ses et wave-udsnit af \textit{TransportProtocol}'s testbench:

\testbench{tb-wave-TP, Waveforms for \textit{TransportProtocol}, fig:TP}.
\\
\\
P� figur\fxnote{F� fat i referencen} foretages der f�rst et reset af komponentet (0-50 ns), hvorefter der skrives hvor mange samples der kommer (60-70 ns), v�rdierne skrives med 10 ns mellemrum (70-100 ns).
Ved 95 ns bliver den sidste v�rdi overf�rt og det angives hvor mange samples der er overf�rt.
Alt dette skrives til \textit{ram\_cs\_module0}.


Efter 160 ns aktiveres \textit{CS} igen og der skrives �n ny v�rdi til ram-modulet. 
Denne skrives til \textit{ram\_cs\_module1}, hvor \textit{ram\_cs\_module0} er deaktiveret.



Der testes ligeledes ogs� n�r der ikke sendes v�rdier over -- alts� hvor der ikke skal afspilles noget.







\end{document}