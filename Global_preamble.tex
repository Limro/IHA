\usepackage[T1]{fontenc} 
\usepackage[english, danish]{babel}				% Dansk ordbog
\usepackage{amsmath, amsfonts, amssymb} % Matematiske pakker

%Grafiksk pakke
\usepackage{graphicx} 
%Sidejustering (margin)
\usepackage[left=3cm,right=2cm,top=2cm,bottom=2cm]{geometry} 

\usepackage{enumerate}					% Forskellige indryk

%\usepackage{fancyhdr}					% Header-style
%\pagestyle{fancy}

% Allow a single subfile to be built as its own (without master document)
\usepackage{subfiles} 


%Skrifttype til dansk, s� det ser p�nere ud
\usepackage{lmodern}



%----------------------------------------------------------
% F�lgende er til koder.
% Inds�ttes mellem \begin{lstlisting} og \end{lstlisting}
% Se eksempel under pakken
%----------------------------------------------------------
\usepackage{listings}
\usepackage{color}
\usepackage{textcomp}
\definecolor{listinggray}{gray}{0.9}
\definecolor{lbcolor}{rgb}{0.9,0.9,0.9}
\lstdefinestyle{Code}
{
	title 			= Kodeudsnit,
	keywordstyle	= \bfseries\ttfamily\color[rgb]{0,0,1},
	identifierstyle	= \ttfamily,
	commentstyle	= \color[rgb]{0.133,0.545,0.133},
	stringstyle		= \ttfamily\color[rgb]{0.627,0.126,0.941},
	showstringspaces= false,
	basicstyle		= \small,
	numberstyle		= \footnotesize,
	numbers			= left, % Tal? Udkommenter hvis ikke
	stepnumber		= 2,
	numbersep		= 6pt,
	tabsize			= 2,
	breaklines		= true,
	prebreak 		= \raisebox{0ex}[0ex][0ex]{\ensuremath{\hookleftarrow}},
	breakatwhitespace= false,
%	aboveskip		= {1.5\baselineskip},
  	columns			= fixed,
  	upquote			= true,
  	extendedchars	= true,
 	backgroundcolor = \color{lbcolor},
	lineskip		= 1pt,
%	xleftmargin		= 17pt,
%	framexleftmargin= 17pt,
	framexrightmargin	= 0pt, %6pt
%	framexbottommargin	= 4pt,
}
\renewcommand{\lstlistingname}{Kodeudsnit}

%Custom length for caption over lstlisting
%The caption must be 6 pt shorter
\usepackage{calc}
\newlength{\mywidth}
\setlength{\mywidth}{\textwidth-6pt}


% Customized style for the caption
\usepackage{caption}
\DeclareCaptionFont{white}{\color{white}}
\DeclareCaptionFormat{listing}%
{\colorbox[cmyk]{0.43, 0.35, 0.35,0.35}{\parbox{\mywidth}{\hspace{5pt}#1#2#3}}}
\captionsetup[lstlisting]
{
	format			= listing,
	labelfont		= white,
	textfont		= white, 
	singlelinecheck	= false, 
	width			= \mywidth,
	margin			= 0pt, 
	font			= {bf,footnotesize}
}

\lstdefinestyle{Code-C} {language=C, style=Code}
\lstdefinestyle{Code-C++} {language=[Visual]C++, style=Code}
\lstdefinestyle{Code-VHDL} {language=VHDL, style=Code}
\lstdefinestyle{Code-Bash {language=Bash, style=Code}

%Use these style as followed
%\begin{lstlisting}[caption=title, style=Code-C++]
%var code = new stuff();
%\end{lstlisting}




%Grader tegn
\newcommand{\degree}{\ensuremath{^\circ}}

%Backslash tegnet
\newcommand{\bs}{\textbackslash}



%\makepagestyle{plain}
%\makeevenfoot{plain}{}{\thepage}{}
%\makeoddfoot{plain}{}{\thepage}{}

%\chapterstyle{article}


%URL kommandoer og sidetal farve
%Kaldes med \url{www...}
%\usepackage{color} %Skal ogs� bruges
\usepackage{hyperref}
\hypersetup{ 
	colorlinks	= true, 	% false: boxed links; true: colored links
    urlcolor	= blue,		% color of external links
    linkcolor	= black, 	% color of page numbers
}



%Speciel skrift for enkelt linje kode
%Udskriver med fonten 'Courier'
%Mere info her: http://tex.stackexchange.com/questions/25249/how-do-i-use-a-particular-font-for-a-small-section-of-text-in-my-document
\newcommand{\code}[1]{{\fontfamily{pcr}\selectfont #1}}


%Citerings pakke.
%Brug \cite{...} til angivelse
\usepackage{cite}

%FixMe pakken viser sm� kommentarer, hvor der skal rettes
%Brug med f�lgende: \fxnote{det her skal uddybes!}
%Se liste over alle sixMe's: \listoffixmes
\usepackage[footnote,draft,danish,silent,nomargin]{fixme}