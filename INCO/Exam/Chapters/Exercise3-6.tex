\documentclass[Main]{subfiles}

\begin{document}
\section*{Exercise 3.6}

\paragraph{Determine the table of code vectors of the binary linear cyclic block code $C_{cyc}(6, 2)$ generated by the polynomial $g(X) = 1 + X + X^3 + X^4$.}

\begin{lstlisting}[caption=Matlab script for exercise 3.6a, style=Code-Matlab, label=lst:36a]
g = [1 1 0 1 1];
n = 6;
k = 2;

[parmat,genmat,h] = cyclgen(n,g,'system');
pattern = [ 0 0; 0 1; 1 0; 1 1];
ct = codeTable(genmat)
table = [ pattern ct]
\end{lstlisting}
The function \code{codeTable} is implemented in \codeTitle \ref{lst:codeTable}.

\begin{lstlisting}[caption=Matlab script for codeTable, style=Code-Matlab, label=lst:codeTable]
function c = codeTable(G)
% Calculate the code table given the generator matrix
% codeTable(G) returns the code table
[k,n] = size(G);
m = zeros(2^k,k);
c = zeros(k,1);

for i = 0:2^k-1
    m(i+1,:) = rot90(dec2binvec(i,k),2);
end

c = mod(m*G,2);

end
\end{lstlisting}

The coding table looks like this:
\\
\\
\code{Table = }
\begin{ArgMat}
0 & 0 & | & 0 & 0 & 0 & 0 & 0 & 0\\
0 & 1 & | & 1 & 0 & 1 & 1 & 0 & 1\\
1 & 0 & | & 1 & 1 & 0 & 1 & 1 & 0\\
1 & 1 & | & 0 & 1 & 1 & 0 & 1 & 1
\end{ArgMat}

\newpage
\paragraph{Calculate the minimum Hamming distance of the code, and its error correction capability.}

To determine the minimum Hamming distance \codeTitle \ref{lst:Hamming} is used:


\begin{lstlisting}[caption=Matlab script for exercise 3.6b, style=Code-Matlab, label=lst:36b]
dmin = HammingDistance(ct)
l = dmin - 1
t = floor((dmin - 1)/2)
\end{lstlisting}
The result is listed below:
\\
\\
\code{$d_{min}$ = 4}\\
\code{$l$ = 3}\\
\code{$t$ = 1}

\end{document}