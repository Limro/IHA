\documentclass[Main]{subfiles}
\begin{document}

\section{Exercise 3.7}

A binary linear cyclic block code with a code length of $n = 14$ has the generator polynomial $g(X) = 1 + X^2 + X^6$.

\paragraph{Determine the number of information and parity check bits in each code vector}
The information bits are as follow; $n =14$ and $g$ is of the 6th order, making 6 redundant bits, $r$.
\begin{align}
r &= 6\\
k = n-r &= 14-6 = 8
\end{align}



\paragraph{Determine the number of code vectors in the code}
The amount of code vectors is defined as $2^k = 2^8 = 256$


\paragraph{Determine the generator and parity check matrices of the code}
\begin{lstlisting}[caption=Matlab script for exercise 3.7c, style=Code-Matlab, label=lst:37c]
n = 14;
g = [1 0 1 0 0 0 1]; %g(x)=1+x^2+x^6

[parmat,genmat,h] = cyclgen(n,g,'system')
\end{lstlisting}
The result is listed below:
\\
\\
\code{Parity Matrix = }
\begin{ArgMat}
1 & 0 & 0 & 0 & 0 & 0 & 1 & 0 & 0 & 0 & 1 & 0 & 1 & 0\\
0 & 1 & 0 & 0 & 0 & 0 & 0 & 1 & 0 & 0 & 0 & 1 & 0 & 1\\
0 & 0 & 1 & 0 & 0 & 0 & 1 & 0 & 1 & 0 & 1 & 0 & 0 & 0\\
0 & 0 & 0 & 1 & 0 & 0 & 0 & 1 & 0 & 1 & 0 & 1 & 0 & 0\\
0 & 0 & 0 & 0 & 1 & 0 & 0 & 0 & 1 & 0 & 1 & 0 & 1 & 0\\
0 & 0 & 0 & 0 & 0 & 1 & 0 & 0 & 0 & 1 & 0 & 1 & 0 & 1
\end{ArgMat}
\\
\\
\\
\code{Generator matrix = }
\begin{ArgMat}
1 &0 &1 &0 &0 &0 &1 &0 &0 &0 &0 &0 &0 &0\\
0 &1 &0 &1 &0 &0 &0 &1 &0 &0 &0 &0 &0 &0\\
0 &0 &1 &0 &1 &0 &0 &0 &1 &0 &0 &0 &0 &0\\
0 &0 &0 &1 &0 &1 &0 &0 &0 &1 &0 &0 &0 &0\\
1 &0 &1 &0 &1 &0 &0 &0 &0 &0 &1 &0 &0 &0\\
0 &1 &0 &1 &0 &1 &0 &0 &0 &0 &0 &1 &0 &0\\
1 &0 &0 &0 &1 &0 &0 &0 &0 &0 &0 &0 &1 &0\\
0 &1 &0 &0 &0 &1 &0 &0 &0 &0 &0 &0 &0 &1
\end{ArgMat}

\paragraph{Determine the minimum Hamming distance of the code}
The minimum Hamming distance is found like in \codeTitle \ref{lst:Hamming}

\begin{lstlisting}[caption=Matlab script for exercise 3.7d, style=Code-Matlab, label=lst:37d]
dmin = HammingDistance(genmat)
\end{lstlisting}
\begin{align*}
d_{min} = 3
\end{align*}


\paragraph{Determine the burst error-detection capability of the code}


\paragraph{Describe briefly how to encode and decode this code}

\end{document}