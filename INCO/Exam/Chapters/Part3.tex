\documentclass[Main]{subfiles}

\begin{document}


\section*{Problem 3}

\paragraph{1. How many valid codewords of this code?}
The number of valid codewords are found witht he equation $2^k$.
Since the generator matrix is of size $G(n,k) = G(15,5)$ we have valid codewords for $2^5 = 32$.


\paragraph{2. Is this code cyclic? Explain the reasons.}
It is.
A theorem says a cyclic code has no remainder of a division of $X^n+1$ with the generator vector, as shown in \codeTitle \ref{cyc}

\begin{lstlisting}[caption=Remainder of cyclic division, style=Code-Matlab, label=lst:cyc]
G = [ ...
    1 1 1 0 1 1 0 0 1 0 1 0 0 0 0;
    0 1 1 1 0 1 1 0 0 1 0 1 0 0 0;
    0 0 1 1 1 0 1 1 0 0 1 0 1 0 0;
    0 0 0 1 1 1 0 1 1 0 0 1 0 1 0;
    0 0 0 0 1 1 1 0 1 1 0 0 1 0 1]

cyc = [ 1 zeros(1,size(G,2)-1) 1];
[~, p] = gfdeconv(cyc, G(1,:), 2)
\end{lstlisting}
\code{p = 0}.
\\
This means, that the division gives us the remainder of 0, which proves this is a cyclic code.


\paragraph{3. Convert the generator matrix in systematic form.}
The matrix can be converted by reducing the rows as in \codeTitle \ref{lst:syste}


\begin{lstlisting}[caption=Generator matrix in systematic form, style=Code-Matlab, label=lst:syste]
G = [ ...
    1 1 1 0 1 1 0 0 1 0 1 0 0 0 0;
    0 1 1 1 0 1 1 0 0 1 0 1 0 0 0;
    0 0 1 1 1 0 1 1 0 0 1 0 1 0 0;
    0 0 0 1 1 1 0 1 1 0 0 1 0 1 0;
    0 0 0 0 1 1 1 0 1 1 0 0 1 0 1];

Systematic = mod(rref(G),2)
[r,c] = size(Systematic);
P = Systematic(:,r+1:c);
Gsys = [P eye(r,r)]
\end{lstlisting}
\code{Gsys = }
\begin{ArgMat}
 1 & 1 & 1 & 0 & 1 & 1 & 0 & 0 & 1 & 0 & 1 & 0 & 0 & 0 & 0\\
 0 & 1 & 1 & 1 & 0 & 1 & 1 & 0 & 0 & 1 & 0 & 1 & 0 & 0 & 0\\
 1 & 1 & 0 & 1 & 0 & 1 & 1 & 1 & 1 & 0 & 0 & 0 & 1 & 0 & 0\\
 0 & 1 & 1 & 0 & 1 & 0 & 1 & 1 & 1 & 1 & 0 & 0 & 0 & 1 & 0\\
 1 & 1 & 0 & 1 & 1 & 0 & 0 & 1 & 0 & 1 & 0 & 0 & 0 & 0 & 1
\end{ArgMat}




\paragraph{4. Encode the source message m = (10110) in a systematic code, find the code vector c.}

To find the code vector $c$, multiply $m$ with the new generator matrix as in \codeTitle \ref{lst:vector}

\begin{lstlisting}[caption=Code vector, style=Code-Matlab, label=lst:vector]
m = [1 0 1 1 0];
c = mod(m * Gsys,2)
\end{lstlisting}
\code{c = [ 0 1 0 1 0 0 0 0 1 1 1 0 1 1 0]}



\end{document}