\documentclass[Main]{subfiles}

\begin{document}


\section*{Problem 4}
Problem 4: A double-error correction binary BCH code $C_{BCH}(15,7)$, the generator polynomial has roots in $GF(2^4)$ which is generated  y primitive polynomial $p_i(X) = 1 + X + X^4$. 

\paragraph{1. Find the generator polynomial of the BCH code.}
The BCH code has $t=2$, which means we must find $g(x)= LCM \floor{ \phi_1 \cdot \phi_3 \cdot \ldots \cdot \phi_{2t-1}}$ and since $2\cdot 2-1 = 3$ then $g(x) = \phi_1 \cdot \phi_3$.


\begin{lstlisting}[caption=g(x), style=Code-Matlab, label=lst:main]
g = [1 1 0 0 1]; %g(x)=1+x+x^4

[ E, V, P ] = gfPol2Table(g);
[E, P, V]

m = log2(length(E));
syms a x;

b1 = conjugateRoots2(V, a) %Find roots for alpha
b2 = conjugateRoots2(V, a^3) %Find roots for alpha^3

phi(1,1) = minimumPoly(E,P,V,a^1); %Find minimum polynomial for alpha
phi(2,1) = minimumPoly(E,P,V,a^3); %Find minimum polynomial for alpha^3

pretty(phi) %List the polynomials nice

pol = phi(1,1) * phi(2,1) %Multiply the polynomials
generator = mod(expand(pol),2) %Expand them and modulo 2
\end{lstlisting}
\code{[E, P, V] = }
\begin{ArgMat}
    0 &                 0 & 0 & 0 & 0 & 0\\
   1 &                 1 & 1 & 0 & 0 & 0\\
   a &                 a & 0 & 1 & 0 & 0\\
 a^2 &               a^2 & 0 & 0 & 1 & 0\\
  a^3 &               a^3 & 0 & 0 & 0 & 1\\
  a^4 &             a + 1 & 1 & 1 & 0 & 0\\
  a^5 &           a^2 + a & 0 & 1 & 1 & 0\\
  a^6 &         a^3 + a^2 & 0 & 0 & 1 & 1\\
  a^7 &       a^3 + a + 1 & 1 & 1 & 0 & 1\\
  a^8 &           a^2 + 1 & 1 & 0 & 1 & 0\\
  a^9 &           a^3 + a & 0 & 1 & 0 & 1\\
 a^{10} &       a^2 + a + 1 & 1 & 1 & 1 & 0\\
 a^{11} &     a^3 + a^2 + a & 0 & 1 & 1 & 1\\
 a^{12} & a^3 + a^2 + a + 1 & 1 & 1 & 1 & 1\\
 a^{13} &     a^3 + a^2 + 1 & 1 & 0 & 1 & 1\\
 a^{14} &           a^3 + 1 & 1 & 0 & 0 & 1
\end{ArgMat}
\\
\\
\code{b1 = [$a, a^2, a^4, a^8$]}
\\
\code{b2 = [$a^3, a^6, a^{12}, a^9$]}
\\
\code{pretty(phi) =}
\begin{ArgMat}
x^4+x+1\\
x^4+x^3+x^2+x+1
\end{ArgMat}
\\
\\
\code{pol = $(x^4+x+1)\cdot (X^4+x^3+x^2+x+1)$}
\\
\code{generator = $x^8+x^7+x^6+x^4+1$}


\paragraph{2. The received vector r = (000111111100011), determine the syndrome vector.}
First we'll find the parity check matrix, transpose this and multiply it with the received vector, as in \codeTitle \ref{lst:rec}

\begin{lstlisting}[caption=Syndrome, style=Code-Matlab, label=lst:rec]
g = [1 0 0 0 1 0 1 1 1];
r = [0 0 0 1 1 1 1 1 1 1 0 0 0 1 1];

t = 2;
SyndromeVectors(E, P, r, t)
\end{lstlisting}
\code{SyndromeVectors = [1 1 0 1]}



\paragraph{3. For simple error correction cases, there is a direct solution to find the error locations. 
The syndrome vector components satisfy the equation below:}
\begin{align*}
(s_1^3 + s_3)X^2+s_1^2X+s_1=0
\end{align*}
\paragraph{The roots of the equation represent the error locations, i.e., $X_1 = \alpha^{-j \cdot l}$. Find the errors locations using the direct solution.}





\paragraph{4. Using Euclidean algorithm to decode the received vector. Determine the error polynomial and the code polynomial. (Hint: the results of direct solution and Euclidean algorithm should be same.)}

\begin{figure}[H]
\centering
\includegraphics[width=0.5\textwidth]{Nothing}
\caption{My solution}
\end{figure}


\end{document}