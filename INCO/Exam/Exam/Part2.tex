\documentclass[Main]{subfiles}

\begin{document}
\section*{Problem 2}



\paragraph{1. Find the generator matrix of C, i.e., G, in the systematic form.}
First we find the generator in systematic form:
\\\\
\code{G = }
\begin{ArgMat}
0 & 1 & 1 & 1 & 0 & 0\\
1 & 0 & 1 & 0 & 1 & 0\\
1 & 1 & 0 & 0 & 0 & 1
\end{ArgMat}


\paragraph{2. Find the generator matrix of the dual code $C_\perp$ i.e., $G_\perp$ in the systematic form.}
Now we find the dual code with with the parity check matrix in systematic form:
\\
\\
\code{H = }
\begin{ArgMat}
1 & 0 & 0 & 0 & 1 & 1\\
0 & 1 & 0 & 1 & 0 & 1\\
0 & 0 & 1 & 1 & 1 & 0
\end{ArgMat}

\paragraph{3. What is the minimum distance of the code $C_\perp$?}
The minimum distance is calculated by comparing each vector to the others and finding the difference for each bit.
The two vectors with the lowest difference is the minimum Hamming distance.
\\
In this case it is 4, between the first and second row.


\paragraph{4. Determine the error detection and error correction capabilities of $C_\perp$}
Since $2t+1 \leq d_{min}$ (calculated above) the equation is:
\begin{align*}
2t+1 &\leq d_{min}\\
2t+1 &\leq 4\\
2\cdot1+1 & \leq 4\\
t = 1
\end{align*}
The error correction capability is therefor 1.
\\
The error detection is 2, obviously.


\paragraph{5. If the channel error rate is $10^{-3}$, please estimate the decoding failure probability at the decoder of the $C_\perp$}
To calculate the decoding failure probability we use the following formula:
\begin{align*}
P_{be} &= \left(\begin{matrix}
n-1\\
t
\end{matrix}\right) p^{t+1} \\
	&= \left(\begin{matrix}
	5\\
	1
	\end{matrix}\right)\cdot \big(10^{-3}\big)^2
\end{align*}











\end{document}