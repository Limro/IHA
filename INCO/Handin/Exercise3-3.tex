\documentclass[Main]{subfiles}

\begin{document}
\section*{Exercise 3.3}

A binary linear cyclic code $C_{cyc}(n, k)$ has code length $n = 7$ and generator polynomial $g(X) = 1 + X^2 + X^3 + X^4$.

\paragraph{(a) Find the code rate, the generator and parity check matrices of the code in
systematic form, and its minimum Hamming distance.}

To find the code rate, we must know the length of the message length.
The message length is defined $k$.
The polynomial, $g(x)$, is in general defined as $g(x) = g_0 + g_1\cdot x^1 + \ldots + x^r$.
The exercise gives us $g(x)$ where $r = 4$.

\begin{align}
\text{Coding rate} = \dfrac{k}{n} = \dfrac{4}{7} = 0.428
\end{align}
The generator matrix is found with Matlab, shown in \codeTitle \ref{lst:3.3a}.

\begin{lstlisting}[caption=Matlab script for exercise 3.3 a, style=Code-Matlab, label=lst:3.3a]
n = 7;
r = 4;
k = n-r;

pol = cyclpoly(n,k);
[parmat, genmat, length] = cyclgen(n, pol, 'system');

codingrate = k/n
H = ParityMatrix(parmat)
dmin = HammingDistance(H)
\end{lstlisting}
The function \code{ParityMatrix(parmat)} is defined in \codeTitle \ref{lst:Parity}.

\begin{lstlisting}[caption=Matlab script a parity matrix, style=Code-Matlab, label=lst:Parity]
function [res] = ParityMatrix(G)
%Parity check matrix.

[r, c] = size(G);
nk = c - log2(r);
P = G(:,1:nk);
EyeSize = length(P(1,:));
[res] = [eye(EyeSize) P'];

end
\end{lstlisting}
The function \code{HammingDistance(H)} is defined in \codeTitle \ref{lst:Hamming}.
It looks at all vectors in a matrix, finds the weight of all combinations and return the lowest.

\begin{lstlisting}[caption=Matlab script for minimum Hamming distance, style=Code-Matlab, label=lst:Hamming]
function dmin = HammingDistance(A)
% Calculate the Hamming distance of every two vectors of a vector.

s = length(A(:,1));
dmin = 0;
temp = 0;
for i = 1:s
    for j = i+1 : s
        temp = HammingWeight(A(i,:), A(j,:));
    end
    
    if temp < dmin
        dmin = temp;
    elseif dmin == 0 %First run goes here
        dmin = temp;
    end
end

end
\end{lstlisting}
The function \code{HammingWeight(A, B)} is defined in \codeTitle \ref{lst:HammingW}.
It looks at each bit in two vectors and compares them, returning the weight of the two vectors.

\begin{lstlisting}[caption=Matlab script for Hamming Weight, style=Code-Matlab, label=lst:HammingW]
function dist = HammingWeight(A, B)
% Calculate the Hamming weight of a vector.

dist = 0;
temp = 0;
for i = 1:length(A)
    if A(1,i) ~=  B(1,i)
        temp = temp + 1;
    end
    
    if temp == 0
        %The zero vector
    elseif temp > dist
        dist = temp;
    elseif dist == 0 %First run goes here
        dist = temp;
    end
end

end
\end{lstlisting}
The result is seen below:
\\
\\
\code{H = }
\begin{ArgMat}
1 & 0 & 0 & 0 & 0 & 1 & 0 & 0 & 0\\
0 & 1 & 0 & 0 & 0 & 0 & 1 & 0 & 0\\
0 & 0 & 1 & 0 & 0 & 0 & 0 & 1 & 0\\
0 & 0 & 0 & 1 & 0 & 0 & 0 & 0 & 1\\
0 & 0 & 0 & 0 & 1 & 1 & 0 & 1 & 1
\end{ArgMat}
\\
\\
$d_{min} = 4$


\paragraph{(b) If all the information symbols are '1's, what is the corresponding code vector in its systematic form?}

\end{document}