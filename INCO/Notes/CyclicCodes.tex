\documentclass[Main]{subfiles}
 
\begin{document}
 
\chapter{Cyclic Codes}

\section{Polynomial Representation of codewords}
For a given vector of n components, $c = (c_0 , c_1, \ldots, c_{n-1}$, a right-shift of its components generates a different vector.

Right shift the original vector for i times, the new vector is:
\\
$c^{(i)} = (c_{n-i}, c_{n-i+1}, \ldots c_{n-1}, C_0, c_1 \ldots)$

\textbf{Cyclic code}, $C_{cyc}(n,k)$ can be represented as a polynomials:
\\
$c=(c_0,c_1,\ldots, c_{c_n-1}$
\\
$c(X) = c_0 +c_1X + \ldots + c_{n-1}X^{n-1} c_i \in GF(2^m)$

\textbf{Example:}
$1 0 1 1 \Rightarrow 1+X^2+X^3$

In a $C(n, k)$ there will be $n$ polynoms.

\subsection{Addition and multiplication}
\textbf{Addition}\cite[p. 6-7]{Slide7}\\
$c_1(x) \bigoplus c_2(x) = (c_01 \bigoplus c_02) + (c_11 \bigoplus c_12)X + \ldots + (c_{n-1,1} \bigoplus c_{n-1,2})X^{n-1}$


\textbf{Multiplication:}\cite[p. 6-7]{Slide7}


\textbf{Division:}\cite[p. 9]{Slide7} 
"Polynomial division." \url{http://en.wikipedia.org/wiki/Polynomial_long_division}


\section{Generator polynomial of cyclic code}
\cite[p. 10-11]{Slide7}
$C^{(i)}(x) = C_{n-i} + C_{n-i+1} + \ldots + C_{n-i-1}c^{n-1}$
\\
$C^{(i)} = x^1C(x) mod (x^n+1)$
\\
$x^1C(c) = \ldots$

\begin{align*}
C &= (0 1 1 0 1 0 0) &\Rightarrow C^{(3)} &= (1 0 0 0 1 1 0)\\
C(x) &= x+ x^2 +x^4 & &C^{(3)}(x) &= 1+x^4+x^5\\
\end{align*}
$x^3 \dot C(x) mod (x^7+1)$
\\
$x^4+x^5+x^7 \Rightarrow division\ldots = x^5+x^4+1$

\cite[p. 14]{Slide7}
\begin{align*}
g(x) 			&= 1+ g_1x + \ldots + g_{r-1}x^{r-1}+x^r &\\
x \dot g(x) 	&= x\dot g(x) mod (x^n+1) &= g^{(1)(x})\\
x^2\dot g(x) 	&= x^2\dot g(x) mod (x^n+1) &=  g^{(2)+(x)}\\
				& \ldots 	&\\
x^{n-r-1}g(x) 	& =g^{(n-r-1)}(x)
\end{align*}


$c(x) = m(x)\dot g(x)$ \cite[p.15]{Slide7}. 
It means a code polynomial $c(x)$ is a multiple of the non-zero minimum-degree polynomial $g(x)$.
\\
\\
In a cyclic code $C_{cyc}(n, k)$ there is a unique non-zero minimum-degree code polynomial, and any other polynomial is a multiple of this polynomial.
\\
The non-zero minimum-degree code polynomial completely determines and generates the cyclic code. 
It is called generator polynomial.

\section{Cyclic codes in systematic form}
$c(x) = m(x)g(x)$ -- This is non-systematic (which means the messages is inside the code vector \cite[p. 20-21]{Slide7}.

\begin{align*}
x^{n-k} \cdot m(x) &\\
p(x) &= x^{n-k} \cdot m(x) mod g(x)\\
x^{n-k}\cdot m(x) &= q(x) \cdot g(x) + p(x)\\
q(x)\cdot g(x) &= x^{n-k}m(x)+p(x)
\end{align*}

Code vector can be expressed based on code polynomial as:
$ c=(p_0, p_1, \ldots, p_{n-k-1},m_0, m_1, \ldots, m_{k-1}) $\cite[p. 23]{Slide7}
\\
\\
\textbf{Summary on \cite[p. 24]{Slide7}}


\subsection{Generator matrix of a cyclic code}

\begin{ArgMat}
g_0 & g_1 & g_2 &\ldots & g_{n-k} & 0 & 0 & \ldots &0\\
0 & g_0 & g_1 & \ldots & g_{n-k-1} & g_{n-k} & 0 &\ldots & 0\\
\ldots & \ldots &&&&&&\\
0 & 0 & & \ldots & g_0 & g_1 & g_2 & \ldots & g_{n-k}
\end{ArgMat}
Where $g_0 = g_{n-k} = 1$\cite[p. 26]{Slide7}


$g(x) = 1+x \Rightarrow r(rank) = 1 \text{(power of x)}$\cite[p. 28]{Slide7}

\end{document}