\documentclass[Notes]{subfiles}
\begin{document}
\section{Lektion 1}

	\begin{itemize}
	\item WIN RT stammer fra b�de COM og .NET teknologien
	\end{itemize}
	
\subsection{Komponenter}

	\begin{itemize}
	\item Komponenter er udskiftelige.

	\item Kan interagere med hinanden.

	\item Kommer af "komponere" alts� s�tte sammen med andre komponenter.
	
	\item Forskellige producenter kan lave en enhed, s�l�nge de har det samme interface.
	
	\item P� denne m�de er det nemmere at lave nye ting.
	\end{itemize}
	
De var oprindeligt t�nkt som hardwarekomponenter, da man havde lavet hardware der kunne yde meget. 
Man manglede desv�rre software, der kunne udnytte hardwaren.
\\
Man lavede derfor nogle sm� dele (komponenter), der kunne bruges til noget af hardwaren.
\\
\\
Eftersom komponenter kan genbruges, kan de s�lges til andre.
Dette kan ogs� reducere udviklingstiden.
\\
En grov regel siger, at den skal kunne genbruges 3 gange, f�r det er tidsbesparende - men dette er m�ske lidt outdatet.
\\
\\
Bruges gerne p� store projekter, da det er nemmere for andre folk ikke at se p� source kode, men p� et interface og magi.
Det kr�ver derfor, at man f�r lavet nogle ordenlige interfaces.



\subsection{Load-time Dynamic Linking}

Udviklingen
	\begin{itemize}
	\item Man skriver et program (gerne i flere sourcefiler)
	\item En compiler laver objekt-filer
	\item En linker laver bin�re .exe-filer og .dll-filer
	\item Ved at k�re .exe-filen siger den "jeg skal bruge en .dll-fil"
	\item Den leder s� efter en .dll med et bestemt navn - her kan man inds�tte sin egen.
	\end{itemize}
Fordele:
	\begin{itemize}
	\item Dynamic linking sparrer plads
	\item Skal ikke rekompileres
	\end{itemize}

\subsection{Run-time Dynamic Linking}
	\begin{itemize}
	\item Her fort�lles hvilken dll, der �nskes loaded.
	\item N�r loaded har man en klub kode
	\item S� kaldes "GetProcAdrdress", hvilket giver os en funktions pointer"
	\end{itemize}



\subsection{DLL typer}
Traditionelle C-Style Win32 DLL'er:
	\begin{itemize}
	\item Standard som er en del af Windows.
	\item Bruger lavet
	\end{itemize}
COM DLL'er:
	\begin{itemize}
	\item Indeholder kun 4 funktioner 
	\end{itemize}
.NET DLL:


\subsection{DLL'er og Memory Management}
En dll kan bruges fra flere programmer og loades i hver sit program, med hver sin ram.



\end{document} 