\documentclass[Notes]{subfiles}
\begin{document}

\section{COM}
Distribuering af .dll-filer
	\begin{itemize}
	\item F�rhen brugte man .obj-filer. Disse fylder dog meget, og man kan ikke �ndre i dem (kun udviklerne kan). Disse kan dog linkes til en .exe-fil.
	
	\item I stedet kan man tage en .dll-fil, som kun tager det n�dvendige fra en .obj-fil.
	
	\item En .dll bliver kun oprettet �n gang og alle applikationer der bruger den, bliver opdateret.
	
	\item C++ er dog kun standardiseret p� Source-niveau og ikke p� Binary-niveau - dette kan medf�re problemer.
	
	\item Laver man en opdatering p� en .dll-fil, skal de programmer der bruger dem recompileres
	\end{itemize}
M�den man omg�r dette p�:

	\begin{itemize}
	\item Man laver en handler, der blot indeholder en pointer til det konkrete objekts funktioner. P� den m�de udvider .dll-filen sig ikke.
	
	\item Man laver en abstrakt base-class (pure virtual). Clienten har kun brug for interfacet. Her skal man dog have et .dll, der har en 'create new concrete object' samt en 'delte concrete object'.
	\end{itemize}
	
\subsection{Resource Management}

	\begin{itemize}
	\item Hold styr p�, hvor mange der bruger et object. N�r der ikke er flere, kan det nedl�gges.
	\end{itemize}


\subsection{Goals of COM}

	\begin{itemize}
	\item Binary encapsulation: Clienter skal ikke recompilere kode
	\item Binary compatibility: Filer kan bruges p�tv�r at platforme
	\end{itemize}

\subsection{Priciples}

	\begin{itemize}
	\item Alt info g�r igennem et interface
	\item Ret aldrig i et udgivet interface
	\end{itemize}
	
\subsection{Arkitektur}
Bruges lokalt, men er s� godt som helt d�dt p� remote.
	\begin{itemize}
	\item Client sp�rger COM Runtime (CR) efter et objekt.
	\item CR sp�rger server efter pointer, der returneres
	\item Client bruger pointer til server.
	\end{itemize}
COM bruger Registreringsdatabasen.
Her ligger alle informationerne om COM-elementerne.









	
\end{document} 