\section{This is the first section}		% file "sect1.tex"

This is the first section.
Below are examples of the three kinds of lists.
(Inspired users can also define their own
``generalized list'' style.)

\subsection{Unnumbered list}

\begin{itemize}
\item These entries are ``itemized''.
\item They are not numbered, just indicated with a ``bullet''
	(a black dot).
\item Entries are separated with a bit of vertical space.
\end{itemize}


\subsection{Numbered list}

The second common kind of list is {\em enumerated}.

\begin{enumerate}
\item These entries are numbered.
\item You would use this kind of list when the order is important,
	or to emphasize the total number of items.
\item Entries are again separated with extra vertical space.
\end{enumerate}


\subsection{Descriptive list}

This kind of list is great for defining
a number of words or expressions,
i.e., for glossaries.
Here are some definitions of the kinds of
mathematical environments
explained in Goldstein \cite{Gold3}
and in \S2.

\begin{description}
\item[Text Formulae]  are usually short formulae (sometimes a
	single variable) which occur within a line of
	text.\footnote{Notice how $x$ looks better than x}

\item[Displayed Formulae] are formulae which are set off from
	the text; they get entire lines of their
	own.\footnote{Naturally, displayed formulae tend
	to be larger than equivalent formulae in text}

\item[Multi-line formulae] are those that are too big to fit
	into one line, or multiple formulae which need to
	be aligned with each other in some
	way.\footnote{Very long equations must be
	broken into separate lines manually}
\end{description}


