\documentclass[oneside, 10pt]{article}

\usepackage{tcolorbox}
\usepackage{ulem} %math
\usepackage{amsmath}
\usepackage{amsfonts}
\usepackage{amssymb}
\usepackage{graphicx}
\usepackage{enumerate}


%Create a box for theorems
%\begin{theo}[titel] %optional
%tekst
%\end{theo}
\newenvironment{theo}[1][Vigtigt]{%
\begin{tcolorbox}[colback=green!5,colframe=green!40!black,title=\textbf{#1}]
}{%
\end{tcolorbox}
}




%Create a square matrix
%\begin{ArgMat}{2}
%21 & 22 & 23 \\  
%a & b & c
%\end{ArgMat}
%
% Info: http://tex.stackexchange.com/questions/2233/whats-the-best-way-make-an-augmented-coefficient-matrix
%
\newenvironment{ArgMat}{%
$
  \left[\begin{array}{@{}*{100}{r}r@{}}
}{%
  \end{array}\right]
  $
}

\newenvironment{deter}{%
$
  \left|\begin{array}{@{}*{100}{r}r@{}}
}{%
  \end{array}\right|
  $
}


%Create multiple lines with holes
%\begin{SysEqu}
%x_1 && &- &5x_3 &+ &2x_4=& 1 \\
%x_1 &+ &x_2 &+ &x_3 && =& 4 \\
%&&&&&&0 =& 0
%\end{SysEqu}
\newenvironment{SysEqu}{%
$  \setlength\arraycolsep{0.1em}
  \begin{array}{@{}*{100}{r}r@{}}
}{%
  \end{array}$
}

%Create solution for x_1, x_n...
%\begin{solu}
%x_1 &= d \\
%x_2 &= e \\
%x_3 &= s
%\end{solu}
\newenvironment{solu}{%
$
  \setlength\arraycolsep{0.1em}
  \left\{\begin{array}{@{}*{100}{r}r@{}}
}{%
  \end{array}\right.
$
}

\usepackage{lastpage}


\newcommand{\HRule}{\rule{\linewidth}{0.8mm}}

%Tekst i fotter
\newcommand{\footerText}{\thepage\xspace /\pageref{LastPage}}
\newcommand{\ProjectName}{433 MHz styring af AeroQuad}


\chapterstyle{hangnum}




\nouppercaseheads
\makepagestyle{mystyle} 

\makeevenhead{mystyle}{}{\\ \leftmark}{} 
\makeoddhead{mystyle}{}{\\ \leftmark}{} 
\makeevenfoot{mystyle}{}{\footerText}{} 
\makeoddfoot{mystyle}{}{\footerText}{} 
\makeatletter
\makepsmarks{mystyle}{% Overskriften på sidehovedet
  \createmark{chapter}{left}{shownumber}{\@chapapp\ }{.\ }} 
\makeatother
\makefootrule{mystyle}{\textwidth}{\normalrulethickness}{0.4pt}
\makeheadrule{mystyle}{\textwidth}{\normalrulethickness}

\makepagestyle{plain}
\makeevenhead{plain}{}{}{}
\makeoddhead{plain}{}{}{}
\makeevenfoot{plain}{}{\footerText}{}
\makeoddfoot{plain}{}{\footerText}{}
\makefootrule{plain}{\textwidth}{\normalrulethickness}{0.4pt}

\pagestyle{mystyle}

%%----------------------------------------------------------------------
%
%%Redefining chapter style
%%\renewcommand\chapterheadstart{\vspace*{\beforechapskip}}
%\renewcommand\chapterheadstart{\vspace*{10pt}}
%\renewcommand\printchaptername{\chapnamefont }%\@chapapp}
%\renewcommand\chapternamenum{\space}
%\renewcommand\printchapternum{\chapnumfont \thechapter}
%\renewcommand\afterchapternum{\space: }%\par\nobreak\vskip \midchapskip}
%\renewcommand\printchapternonum{}
%\renewcommand\printchaptertitle[1]{\chaptitlefont #1}
\setlength{\beforechapskip}{0pt} 
\setlength{\afterchapskip}{0pt} 
%\setlength{\voffset}{0pt} 
\setlength{\headsep}{25pt}
%\setlength{\topmargin}{35pt}
%%\setlength{\headheight}{102pt}
%\setlength{\textheight}{302pt}
\renewcommand\afterchaptertitle{\par\nobreak\vskip \afterchapskip}
%%----------------------------------------------------------------------




%Sidehoved og -fod pakke
%Margin
\usepackage[left=2cm,right=2cm,top=2.5cm,bottom=2cm]{geometry}
\usepackage{lastpage}



%%URL kommandoer og sidetal farve
%%Kaldes med \url{www...}
%\usepackage{color} %Skal også bruges
\usepackage{hyperref}
\hypersetup{ 
	colorlinks	= true, 	% false: boxed links; true: colored links
    urlcolor	= blue,		% color of external links
    linkcolor	= black, 	% color of page numbers
    citecolor	= blue,
}



%Mellemrum mellem linjerne    
\linespread{1.5}


%Seperated files
%--------------------------------------------------
%Opret filer således:
%\documentclass[Navn-på-hovedfil]{subfiles}
%\begin{document}
% Indmad
%\end{document}
%
% I hovedfil inkluderes således:
% \subfile{navn-på-subfil}
%--------------------------------------------------
\usepackage{subfiles}

%Prevent wierd placement of figures
%\usepackage[section]{placeins}

%Standard sti at søge efter billeder
%--------------------------------------------------
%\begin{figure}[hbtp]
%\centering
%\includegraphics[scale=1]{filnavn-for-png}
%\caption{Titel}
%\label{fig:referenceNavn}
%\end{figure}
%--------------------------------------------------
\usepackage{graphicx}
\usepackage{subcaption}
\usepackage{float}
\graphicspath{{../Figures/}}

%Speciel skrift for enkelt linje kode
%--------------------------------------------------
%Udskriver med fonten 'Courier'
%Mere info her: http://tex.stackexchange.com/questions/25249/how-do-i-use-a-particular-font-for-a-small-section-of-text-in-my-document
%Eksempel: Funktionen \code{void Hello()} giver et output
%--------------------------------------------------
\newcommand{\code}[1]{{\fontfamily{pcr}\selectfont #1}}


% Følgende er til koder.
%----------------------------------------------------------
%\begin{lstlisting}[caption=Overskrift på boks, style=Code-C++, label=lst:referenceLabel]
%public void hello(){}
%\end{lstlisting}
%----------------------------------------------------------

%Exstra space
\usepackage{xspace}
%Navn på bokse efterfulgt af \xspace (hvis det skal være mellemrum
%gives det med denne udvidelse. Ellers ingen mellemrum.
\newcommand{\codeTitle}{Kodeudsnit\xspace}

%Pakker der skal bruges til lstlisting
\usepackage{listings}
\usepackage{color}
\usepackage{textcomp}
\definecolor{listinggray}{gray}{0.9}
\definecolor{lbcolor}{rgb}{0.9,0.9,0.9}
\renewcommand{\lstlistingname}{\codeTitle}
\lstdefinestyle{Code}
{
	keywordstyle	= \bfseries\ttfamily\color[rgb]{0,0,1},
	identifierstyle	= \ttfamily,
	commentstyle	= \color[rgb]{0.133,0.545,0.133},
	stringstyle		= \ttfamily\color[rgb]{0.627,0.126,0.941},
	showstringspaces= false,
	basicstyle		= \small,
	numberstyle		= \footnotesize,
%	numbers			= left, % Tal? Udkommenter hvis ikke
	stepnumber		= 2,
	numbersep		= 6pt,
	tabsize			= 2,
	breaklines		= true,
	prebreak 		= \raisebox{0ex}[0ex][0ex]{\ensuremath{\hookleftarrow}},
	breakatwhitespace= false,
%	aboveskip		= {1.5\baselineskip},
  	columns			= fixed,
  	upquote			= true,
  	extendedchars	= true,
 	backgroundcolor = \color{lbcolor},
	lineskip		= 1pt,
%	xleftmargin		= 17pt,
%	framexleftmargin= 17pt,
	framexrightmargin	= 0pt, %6pt
%	framexbottommargin	= 4pt,
}

%Bredde der bruges til indryk
%Den skal være 6 pt mindre
\usepackage{calc}
\newlength{\mywidth}
\setlength{\mywidth}{\textwidth-6pt}


% Forskellige styles for forskellige kodetyper
\usepackage{caption}
\DeclareCaptionFont{white}{\color{white}}
\DeclareCaptionFormat{listing}%
{\colorbox[cmyk]{0.43, 0.35, 0.35,0.35}{\parbox{\mywidth}{\hspace{5pt}#1#2#3}}}
\captionsetup[lstlisting]
{
	format			= listing,
	labelfont		= white,
	textfont		= white, 
	singlelinecheck	= false, 
	width			= \mywidth,
	margin			= 0pt, 
	font			= {bf,footnotesize}
}

\lstdefinestyle{Code-C} {language=C, style=Code}
\lstdefinestyle{Code-Java} {language=Java, style=Code}
\lstdefinestyle{Code-C++} {language=[Visual]C++, style=Code}
\lstdefinestyle{Code-VHDL} {language=VHDL, style=Code}
\lstdefinestyle{Code-Bash} {language=Bash, style=Code}

%Text typesetting
%--------------------------------------------------------
%\usepackage{baskervald}
\usepackage{lmodern}
\usepackage[T1]{fontenc}              
\usepackage[utf8]{inputenc}         
\usepackage[english]{babel}       

\setlength{\parindent}{0pt}
\nonzeroparskip

%\setaftersubsecskip{1sp}
%\setaftersubsubsecskip{1sp}
 


%Dybde på indholdsfortegnelse
%----------------------------------------------------------
%Chapter, section, subsection, subsubsection
%----------------------------------------------------------
\setcounter{secnumdepth}{3}
\setcounter{tocdepth}{3}


%Tables
%----------------------------------------------------------
\usepackage{tabularx}
\usepackage{array}
\usepackage{multirow} 
\usepackage{multicol} 
\usepackage{booktabs}
\usepackage{wrapfig}
\renewcommand{\arraystretch}{1.5}



%Misc
%----------------------------------------------------------
\usepackage{cite}
\usepackage{appendix}
\usepackage{amssymb}
\usepackage{url,ragged2e}
\usepackage{enumerate}
\usepackage{amsmath} %Math bibliotek


\usepackage{longtable}


\title{Middleware and Communication}
\author{Rasmus Bækgaard}
\date{\today}

\begin{document}

\maketitle

\section{Precision Time Protocol}

\subsection{Determinism}:
\begin{itemize}
	\item Synkronisere clocks mellem forskellige enheder.
	\item Standard protokol, IEEE 1588.
	\item Alt er om deadlines (ikke hurtigt -- men før en deadline)
	\item Deadline beregning på relativ event eller absolut synkroniseret clock på L1S4.
	\item Timings, L1S5.
	\item Forskel på real-time (Hard, form, soft), L1S6-7
	\begin{itemize}
		\item Hard: bilen gør ikke det den skal.
		\item Firm: Ikke helt så meget (skade til materiel, men ikke noget slemt).
		\item Soft: Skype mister lidt forbindelse, men du snakker stadig.
	\end{itemize}

	\item Man møder deadline, eller man møder ikke (binært). Det kan ikke nedbrydes (selvom man gerne vil, som ovenover)
	\item Determinism: Det næste stadie vil være unikt og vi ved hvad det vil være.
	\item Non-determinism: man kan gå i flere stadier (og vi ved ikke hvilke).

	\item Ethernet er ikke deterministic, pga. en protocol med et element, som ikke er deterministic.
\end{itemize}

\subsection{Automa}
\begin{itemize}
	\item Hvis man laver et system med en formel bruger man en model, som checker formlen igenne en model-checker.
	\item Deterministics finite automaton, DFA, er en en Turing-machine. L1S11
	\begin{itemize}
		\item $\delta$: Én fra $Q$ og én fra $\Sigma$ giver én ny $Q$
		\item Dobbelt cirkel: slut/accept punkt
		\item L1S12: der er et lige antal '0'er -- derfor tager den imod en \code{string}
		\item $*$ betyder antal gange fra 0 til mange gange.
	\end{itemize}

	\item Timed Finite Automata, TFA:
	\begin{itemize}
		\item Der findes forskellige definationer af dette på nettet, men brug dette.
		\item Tager tid med i beregningerne
		\item $I$ er en initialiser for clock.
		\item $\sim \in \{ =, <, >, \leq, geq\}$
		\item $\neg$ betyder "ikke"
	\end{itemize}

	\item For hvert nyt stadie er clock'en en anden.
	\item S1L16: tegnene på linjerne betyder "given that"

\end{itemize}

\subsection{Precision Time Protocol, PTP}:
Generelt:
\begin{itemize}
	\item En teknologi
	\item Udgivet af IEEE
	\item Synkroniserer clocks på et netværk
	\item Serverne kan have forskellige tider, kaldet også "local time".
	\item Jo tættere på hardware et timestamp er, jo mindre er fejlen
	\subitem Der findes dedikeret hardware til dette.
	\item Synkronisering skal ske continuereligt
	\item PTP kan på mindre end 1 minut nå $\pm$ 50ns præcision. (Udstyr kan synkronisere så godt som hvert sekund!)
	\item Bruges i LTE/5G netværk
	\item Der findes en algoritme der afgører, hvilken clock er bedst egnet som master
	\begin{itemize}
	 	\item "Stratum" er en værdi for den bedste master clock.
	 	\item Hvis master clock fejler, vil den næste blive valgt.
	 \end{itemize} 
	 \item Der findes et PHY niveau under interrupt niveau, som modtager RX og sender TX.
\end{itemize}

Synkronisering:
\begin{itemize}
	\item L1S32-33
	\item Man kender Line delay
	\item Boundary clock (special switch), der vil agere som kopi af (Grand) Master clock (L1S35-37)
	\item Hver boundary switch har et delay, som den forsøger at eliminere selv.
\end{itemize}







\newpage
\section{Real-time Ethernet}

\subsection{Generelt}
\begin{itemize}
	\item Vanilla ethernet er \emph{ikke} deterministisk.
	\item Det kan dog gøres, med de rigtige værktøjer.
	\item Motivationen var, at få det rykket ind til firmaer (til ALT).
	\item Datamængden der sendes er meget lille, responstiden skal være meget lille, men det sker meget ofte, L2S5.
	\item OSI er en måde som en arkitektur skal/burde være.
	\item Ethernet kaldes IEEE 802.3 L2S8.
	\item Pakkerne i en ethernet pakke kaldes et "frame". L2S10
	\begin{itemize}
		\item Det man sender ligger i \code{Payload}
		\item 802.1Q tag \emph{kan} bruges også.
		\item Resten er noget, som blot skal være der
	\end{itemize}

	\item Fordelen ved optic: Ingen EMC forstyrelser
\end{itemize}

\subsection{Single segment}
\begin{itemize}
	\item CSMA/CD, Carrier Send Multiple Access / Collision detection
	\item Deadline gurantees må ikke have collisions!
	\item Man lytter, til der har været en hvis pause på linjen, L2S18
	\item Overførelsel sker, som på L2S20-21
	\begin{itemize}
		\item Når collision sker, har de 2 linjer en back-off algoritme (Many Faces of Ethernet)
	\end{itemize}
\end{itemize}


\subsection{Multi segment}
\begin{itemize}
	\item Bruger switches, L2S25.
	\item Store and foward (L2S29) er bedre ved industriel, mens Cut through er bedre ved ingen støj (et hjem).
	\item Variation i latency er det største problem, L2S30.
	\item Nagle's algoritme er dårlig for Real-time systemer, da en algoritme gør noget (lettere) ukontrolleret.
\end{itemize}

\subsection{Single segment real-time}

\begin{itemize}
	\item Shared medium (Christian gennemgik nogle grafer)
	\item Når man sender data, fyldes pakken ikke helt, men efterlader lidt i begge ender.
\end{itemize}

\subsection{Multi segment real-time}
\begin{itemize}
	\item EtherNet/IP $\rightarrow$ (Internet Protocol)
\end{itemize}







\newpage
\section{DDS middleware}

Middleware:
\begin{itemize}
	\item Abstraher netværk, transport, HW arkitektur, programming language mm. væk.
	\item Bruges ved netværks opgaver.
	\item Kaldes også "plumbing", fordi det forbinder ting -- eg. legacy systemer.
	\item Der findes forskelige interfaces (sprog) som mna kan bruges til det.
	\item MW \textbf{skulle} decouple space, time og flow (Om man bruger Pub/Sub eller andet).
	\item DDS er lavet til Realtime-systems.
\end{itemize}

Publish/subscriber:
\begin{itemize}
	\item Har Content/property, som kan filtrere beskeder.
	\item Der kan også angives en type som filter.
	\item De kan tilskrives til mere en ét \texttt{topic}.
\end{itemize}

DDS intro:
\begin{itemize}
	\item QoS parametre siger alt om den data der sendes
	\begin{itemize}
	 	\item Hvor meget
	 	\item Hvor når
	 	\item Hvor tit
	 	\item Hvor længe gemmes det
	 	\item mm.
	 \end{itemize} 

	 \item Meget skalerbar (men discovery protokollerne er ikke så effiktive).
	 \item Kan være det, der enabler Internet of Things og Industrial automation
	 \begin{itemize}
	 	\item Bluefinn har lavet en ubåd, som bruger dette.
	 \end{itemize}
\end{itemize}

DDS mekanismen:
\begin{itemize}
	\item DCPS: Data-centric pub-sub er det der vises på L3S36
	\item 3 metoder at modtage data, L3S40
	\item Topics: Et navn og en type, hvor navnet er unikt i domainet og typen er data definitionen i et Topic (specificeret med IDL).
	\item I stedet for flere, ens, topics kan man bruge et key L3S43-44
	\item Discovery er det, der finde ud af hvem der er med i klubben -- L3S49-52
	\begin{itemize}
		\item SDP -- ikke den bedste protokol (her kunne man lave mater thesis, da opskallering giver meget data trafik.)
	\end{itemize}
\end{itemize}

QoS:
\begin{itemize}
	\item Durability, L3S54
\end{itemize}

IDL:
\begin{itemize}
	\item Fælles sproget (som det hele er skrevet over)
\end{itemize}






\newpage
\section{Time-Triggered Systems}

X-by-Wire betyder, at man bruger elektriske ledning istedet for mekaniske/hydraliske. 
$X$, da der er flere forskellige typer.

Time-triggered:
\begin{itemize}
	\item Design time: Noget sker på et bestemt tidspunkt - e.g. 2 værdier læses med et givent mellemrum $T$. L4S4
	\item Transmission of message er clock'en.
	\item Task execution: Et handle, der bliver triggered ved læsning af værdien.
	\item Monitoring of external states: sampling af sensores.
	\item Kaldes TDMA (Time Division Multiple Access):
	\begin{itemize}
		\item Man må kun skrive i en bestemt timeslot
		\item Hvert slot kan have forskellig timeslots
		\item L4S5
	\end{itemize}
	\item Fault tolerance betyder flere buser at skrive på. Derfor må man ikke have et single point of failure.
\end{itemize}

Event-triggered:
\begin{itemize}
	\item Noget kan sende et event, som systemet reagere på.
	\item Normalt med event handlers.
\end{itemize}

SAE Comminication Classes:
\begin{itemize}
	\item har ca. 100k ingeniør medlemmer
	\item L4S11
\end{itemize}

X-by-Wire:
\begin{itemize}
	\item brake-, stee-,  mm.
	\item Det er meget nemmere at føre en printbane/ledning, end at have hydraliske lednninger til at bremse.
	\item Hvis man er for tæt på den forankørende bil, kan bilen selv tage over.
	\item Man kan flytte komponenter - eg. rattet i bilen.
	\item FTU - fault tolerant unit
\end{itemize}


CAN:
\begin{itemize}
	\item Oftes langsommere en 1 Mbit/s.
\end{itemize}

TTP:
\begin{itemize}
	\item A: en nedskallering.
	\item C: Fuld version -- det der fokuseres på.
	\item Composability: Udvikelt noder uafhængigt og sæt dem sammen senere.
	\begin{itemize}
		\item L4,2S7
		\item Hvis man har $x$-antal FTU'er forbundet til en bus, er arkitekturen.
		\item Node er det der designes i én enhed. Disse kan laves af sub-contractors.
	\end{itemize}
	\item Man skal altid vide hvad er aktivt og passiv (FT membership service) L4,2S6
	\item Kan skalere op til 25 Mbit/s
	\item En Round er en rotation af alle kritiske komponenter.
	\item Ikke-kritiske komponenter kan tilskrives til eks. hver 3. round. Dette kaldes Cluster round (når der er forskellige rounds)
	\item Bus guardian: Guard access to the network (intelligent switch), kender det schedule der skal bruge komponentet. Beskytter mod "the babling idiot" -- hvis komponentet bliver ved med at snakke, så snakker den ikke over sig.
	\item Bedre en en fuardian per node vil være et bus/star, L4,2S14

	\item Dual port ram: Det mellem host og protokol af de enkelte nodes. Den kan, eks., genstarte nodes der ikke svarer længere (watchdog). Den kan også se hvilke notes der virker, hvilke beskeder der sendes frem og tilbage.
	\item TTP Frame: De beskeder der sendes i et slot.
\end{itemize}

\end{document}