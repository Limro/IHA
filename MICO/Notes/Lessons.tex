\documentclass[oneside, 10pt]{article}

%Preamble

% Følgende er til koder.
%----------------------------------------------------------
%\begin{lstlisting}[caption=Overskrift på boks, style=Code-C++, label=lst:referenceLabel]
%public void hello(){}
%\end{lstlisting}
%----------------------------------------------------------

%Exstra space
\usepackage{xspace}
%Navn på bokse efterfulgt af \xspace (hvis det skal være mellemrum
%gives det med denne udvidelse. Ellers ingen mellemrum.
\newcommand{\codeTitle}{Code snippet\xspace}

%Pakker der skal bruges til lstlisting
\usepackage{listings}
\usepackage{color}
\usepackage{textcomp}
\definecolor{listinggray}{gray}{0.9}
\definecolor{lbcolor}{rgb}{0.9,0.9,0.9}
\renewcommand{\lstlistingname}{\codeTitle}
\lstdefinestyle{Code}
{
	keywordstyle	= \bfseries\ttfamily\color[rgb]{0,0,1},
	identifierstyle	= \ttfamily,
	commentstyle	= \color[rgb]{0.133,0.545,0.133},
	stringstyle		= \ttfamily\color[rgb]{0.627,0.126,0.941},
	showstringspaces= false,
	basicstyle		= \small,
	numberstyle		= \footnotesize,
%	numbers			= left, % Tal? Udkommenter hvis ikke
	stepnumber		= 2,
	numbersep		= 6pt,
	tabsize			= 2,
	breaklines		= true,
	prebreak 		= \raisebox{0ex}[0ex][0ex]{\ensuremath{\hookleftarrow}},
	breakatwhitespace= false,
%	aboveskip		= {1.5\baselineskip},
  	columns			= fixed,
  	upquote			= true,
  	extendedchars	= true,
 	backgroundcolor = \color{lbcolor},
	lineskip		= 1pt,
%	xleftmargin		= 17pt,
%	framexleftmargin= 17pt,
	framexrightmargin	= 0pt, %6pt
%	framexbottommargin	= 4pt,
}

%Bredde der bruges til indryk
%Den skal være 6 pt mindre
\usepackage{calc}
\newlength{\mywidth}
\setlength{\mywidth}{1.435\textwidth} % Hvis bredden header ikke virker er dette hvad skal ændres!


% Forskellige styles for forskellige kodetyper
\usepackage{caption}
\DeclareCaptionFont{white}{\color{white}}
\DeclareCaptionFormat{listing}%
{\colorbox[cmyk]{0.43, 0.35, 0.35,0.35}{\parbox{\mywidth}{\hspace{5pt}#1#2#3}}}
\captionsetup[lstlisting]
{
	format			= listing,
	labelfont		= white,
	textfont		= white, 
	singlelinecheck	= false, 
	width			= \mywidth,
	margin			= 0pt, 
	font			= {bf,footnotesize}
}

\lstdefinestyle{Code-C} {language=C, style=Code}
\lstdefinestyle{Code-Java} {language=Java, style=Code}
\lstdefinestyle{Code-C++} {language=[Visual]C++, style=Code}
\lstdefinestyle{Code-VHDL} {language=VHDL, style=Code}
\lstdefinestyle{Code-Bash} {language=Bash, style=Code}
\lstdefinestyle{Code-Matlab} {language=Matlab, style=Code}
\lstdefinestyle{Code-Prolog} {language=Prolog, style=Code}
%Speciel skrift for enkelt linje kode
%--------------------------------------------------
%Udskriver med fonten 'Courier'
%Mere info her: http://tex.stackexchange.com/questions/25249/how-do-i-use-a-particular-font-for-a-small-section-of-text-in-my-document
%Eksempel: Funktionen \code{void Hello()} giver et output
%--------------------------------------------------
\newcommand{\code}[1]{{\fontfamily{pcr}\selectfont #1}}

%Seperated files
%--------------------------------------------------
%Opret filer således:
%\documentclass[Navn-på-hovedfil]{subfiles}
%\begin{document}
% Indmad
%\end{document}
%
% I hovedfil inkluderes således:
% \subfile{navn-på-subfil}
%--------------------------------------------------
\usepackage{subfiles}
%Text typesetting
%--------------------------------------------------------
\usepackage[T1]{fontenc} 	% Can use danish characters
\usepackage[utf8]{inputenc} % Input encoding. Can be used on Linux, Mac and Windows         
\usepackage[danish]{babel} 	% Split words accoding to English
\usepackage{lmodern} 		% Font

\setlength\parindent{0pt} 	% No indent
\setlength\parskip{12pt} 	% More than a single line break will give ONE linebreak.

%Margin
\usepackage[left=2cm,right=2cm,top=2.5cm,bottom=2cm]{geometry}

%Margin
\usepackage[left=3cm,right=2cm,top=2.5cm,bottom=2cm]{geometry}

%Mellemrum mellem linjerne    
\linespread{1.5}

\title{Assignment 2 for TEDI}
\author{Rasmus Bækgaard, 10893}
\date{April 28, 2014}

\title{Middleware and Communication}
\author{Rasmus Bækgaard}
\date{\today}

\begin{document}

\maketitle

\section{Precision Time Protocol}

\subsection{Determinism}:
\begin{itemize}
	\item Synkronisere clocks mellem forskellige enheder.
	\item Standard protokol, IEEE 1588.
	\item Alt er om deadlines (ikke hurtigt -- men før en deadline)
	\item Deadline beregning på relativ event eller absolut synkroniseret clock på L1S4.
	\item Timings, L1S5.
	\item Forskel på real-time (Hard, form, soft), L1S6-7
	\begin{itemize}
		\item Hard: bilen gør ikke det den skal.
		\item Firm: Ikke helt så meget (skade til materiel, men ikke noget slemt).
		\item Soft: Skype mister lidt forbindelse, men du snakker stadig.
	\end{itemize}

	\item Man møder deadline, eller man møder ikke (binært). Det kan ikke nedbrydes (selvom man gerne vil, som ovenover)
	\item Determinism: Det næste stadie vil være unikt og vi ved hvad det vil være.
	\item Non-determinism: man kan gå i flere stadier (og vi ved ikke hvilke).

	\item Ethernet er ikke deterministic, pga. en protocol med et element, som ikke er deterministic.
\end{itemize}

\subsection{Automa}
\begin{itemize}
	\item Hvis man laver et system med en formel bruger man en model, som checker formlen igenne en model-checker.
	\item Deterministics finite automaton, DFA, er en en Turing-machine. L1S11
	\begin{itemize}
		\item $\delta$: Én fra $Q$ og én fra $\Sigma$ giver én ny $Q$
		\item Dobbelt cirkel: slut/accept punkt
		\item L1S12: der er et lige antal '0'er -- derfor tager den imod en \code{string}
		\item $*$ betyder antal gange fra 0 til mange gange.
	\end{itemize}

	\item Timed Finite Automata, TFA:
	\begin{itemize}
		\item Der findes forskellige definationer af dette på nettet, men brug dette.
		\item Tager tid med i beregningerne
		\item $I$ er en initialiser for clock.
		\item $\sim \in \{ =, <, >, \leq, geq\}$
		\item $\neg$ betyder "ikke"
	\end{itemize}

	\item For hvert nyt stadie er clock'en en anden.
	\item S1L16: tegnene på linjerne betyder "given that"

\end{itemize}

\subsection{Precision Time Protocol, PTP}:
Generelt:
\begin{itemize}
	\item En teknologi
	\item Udgivet af IEEE
	\item Synkroniserer clocks på et netværk
	\item Serverne kan have forskellige tider, kaldet også "local time".
	\item Jo tættere på hardware et timestamp er, jo mindre er fejlen
	\subitem Der findes dedikeret hardware til dette.
	\item Synkronisering skal ske continuereligt
	\item PTP kan på mindre end 1 minut nå $\pm$ 50ns præcision. (Udstyr kan synkronisere så godt som hvert sekund!)
	\item Bruges i LTE/5G netværk
	\item Der findes en algoritme der afgører, hvilken clock er bedst egnet som master
	\begin{itemize}
	 	\item "Stratum" er en værdi for den bedste master clock.
	 	\item Hvis master clock fejler, vil den næste blive valgt.
	 \end{itemize} 
	 \item Der findes et PHY niveau under interrupt niveau, som modtager RX og sender TX.
\end{itemize}

Synkronisering:
\begin{itemize}
	\item L1S32-33
	\item Man kender Line delay
	\item Boundary clock (special switch), der vil agere som kopi af (Grand) Master clock (L1S35-37)
	\item Hver boundary switch har et delay, som den forsøger at eliminere selv.
\end{itemize}







\newpage
\section{Real-time Ethernet}

\subsection{Generelt}
\begin{itemize}
	\item Vanilla ethernet er \emph{ikke} deterministisk.
	\item Det kan dog gøres, med de rigtige værktøjer.
	\item Motivationen var, at få det rykket ind til firmaer (til ALT).
	\item Datamængden der sendes er meget lille, responstiden skal være meget lille, men det sker meget ofte, L2S5.
	\item OSI er en måde som en arkitektur skal/burde være.
	\item Ethernet kaldes IEEE 802.3 L2S8.
	\item Pakkerne i en ethernet pakke kaldes et "frame". L2S10
	\begin{itemize}
		\item Det man sender ligger i \code{Payload}
		\item 802.1Q tag \emph{kan} bruges også.
		\item Resten er noget, som blot skal være der
	\end{itemize}

	\item Fordelen ved optic: Ingen EMC forstyrelser
\end{itemize}

\subsection{Single segment}
\begin{itemize}
	\item CSMA/CD, Carrier Send Multiple Access / Collision detection
	\item Deadline gurantees må ikke have collisions!
	\item Man lytter, til der har været en hvis pause på linjen, L2S18
	\item Overførelsel sker, som på L2S20-21
	\begin{itemize}
		\item Når collision sker, har de 2 linjer en back-off algoritme (Many Faces of Ethernet)
	\end{itemize}
\end{itemize}


\subsection{Multi segment}
\begin{itemize}
	\item Bruger switches, L2S25.
	\item Store and foward (L2S29) er bedre ved industriel, mens Cut through er bedre ved ingen støj (et hjem).
	\item Variation i latency er det største problem, L2S30.
	\item Nagle's algoritme er dårlig for Real-time systemer, da en algoritme gør noget (lettere) ukontrolleret.
\end{itemize}

\subsection{Single segment real-time}

\begin{itemize}
	\item Shared medium (Christian gennemgik nogle grafer)
	\item Når man sender data, fyldes pakken ikke helt, men efterlader lidt i begge ender.
\end{itemize}

\subsection{Multi segment real-time}
\begin{itemize}
	\item EtherNet/IP $\rightarrow$ (Internet Protocol)
\end{itemize}







\newpage
\section{DDS middleware}

Middleware:
\begin{itemize}
	\item Abstraher netværk, transport, HW arkitektur, programming language mm. væk.
	\item Bruges ved netværks opgaver.
	\item Kaldes også "plumbing", fordi det forbinder ting -- eg. legacy systemer.
	\item Der findes forskelige interfaces (sprog) som mna kan bruges til det.
	\item MW \textbf{skulle} decouple space, time og flow (Om man bruger Pub/Sub eller andet).
	\item DDS er lavet til Realtime-systems.
\end{itemize}

Publish/subscriber:
\begin{itemize}
	\item Har Content/property, som kan filtrere beskeder.
	\item Der kan også angives en type som filter.
	\item De kan tilskrives til mere en ét \texttt{topic}.
\end{itemize}

DDS intro:
\begin{itemize}
	\item QoS parametre siger alt om den data der sendes
	\begin{itemize}
	 	\item Hvor meget
	 	\item Hvor når
	 	\item Hvor tit
	 	\item Hvor længe gemmes det
	 	\item mm.
	 \end{itemize} 

	 \item Meget skalerbar (men discovery protokollerne er ikke så effiktive).
	 \item Kan være det, der enabler Internet of Things og Industrial automation
	 \begin{itemize}
	 	\item Bluefinn har lavet en ubåd, som bruger dette.
	 \end{itemize}
\end{itemize}

DDS mekanismen:
\begin{itemize}
	\item DCPS: Data-centric pub-sub er det der vises på L3S36
	\item 3 metoder at modtage data, L3S40
	\item Topics: Et navn og en type, hvor navnet er unikt i domainet og typen er data definitionen i et Topic (specificeret med IDL).
	\item I stedet for flere, ens, topics kan man bruge et key L3S43-44
	\item Discovery er det, der finde ud af hvem der er med i klubben -- L3S49-52
	\begin{itemize}
		\item SDP -- ikke den bedste protokol (her kunne man lave mater thesis, da opskallering giver meget data trafik.)
	\end{itemize}
\end{itemize}

QoS:
\begin{itemize}
	\item Durability, L3S54
\end{itemize}

IDL:
\begin{itemize}
	\item Fælles sproget (som det hele er skrevet over)
\end{itemize}






\newpage
\section{Time-Triggered Systems}

X-by-Wire betyder, at man bruger elektriske ledning istedet for mekaniske/hydraliske. 
$X$, da der er flere forskellige typer.

Time-triggered:
\begin{itemize}
	\item Design time: Noget sker på et bestemt tidspunkt - e.g. 2 værdier læses med et givent mellemrum $T$. L4S4
	\item Transmission of message er clock'en.
	\item Task execution: Et handle, der bliver triggered ved læsning af værdien.
	\item Monitoring of external states: sampling af sensores.
	\item Kaldes TDMA (Time Division Multiple Access):
	\begin{itemize}
		\item Man må kun skrive i en bestemt timeslot
		\item Hvert slot kan have forskellig timeslots
		\item L4S5
	\end{itemize}
	\item Fault tolerance betyder flere buser at skrive på. Derfor må man ikke have et single point of failure.
\end{itemize}

Event-triggered:
\begin{itemize}
	\item Noget kan sende et event, som systemet reagere på.
	\item Normalt med event handlers.
\end{itemize}

SAE Comminication Classes:
\begin{itemize}
	\item har ca. 100k ingeniør medlemmer
	\item L4S11
\end{itemize}

X-by-Wire:
\begin{itemize}
	\item brake-, stee-,  mm.
	\item Det er meget nemmere at føre en printbane/ledning, end at have hydraliske lednninger til at bremse.
	\item Hvis man er for tæt på den forankørende bil, kan bilen selv tage over.
	\item Man kan flytte komponenter - eg. rattet i bilen.
	\item FTU - fault tolerant unit
\end{itemize}


CAN:
\begin{itemize}
	\item Oftes langsommere en 1 Mbit/s.
\end{itemize}

TTP:
\begin{itemize}
	\item A: en nedskallering.
	\item C: Fuld version -- det der fokuseres på.
	\item Composability: Udvikelt noder uafhængigt og sæt dem sammen senere.
	\begin{itemize}
		\item L4,2S7
		\item Hvis man har $x$-antal FTU'er forbundet til en bus, er arkitekturen.
		\item Node er det der designes i én enhed. Disse kan laves af sub-contractors.
	\end{itemize}
	\item Man skal altid vide hvad er aktivt og passiv (FT membership service) L4,2S6
	\item Kan skalere op til 25 Mbit/s
	\item En Round er en rotation af alle kritiske komponenter.
	\item Ikke-kritiske komponenter kan tilskrives til eks. hver 3. round. Dette kaldes Cluster round (når der er forskellige rounds)
	\item Bus guardian: Guard access to the network (intelligent switch), kender det schedule der skal bruge komponentet. Beskytter mod "the babling idiot" -- hvis komponentet bliver ved med at snakke, så snakker den ikke over sig.
	\item Bedre en en fuardian per node vil være et bus/star, L4,2S14

	\item Dual port ram: Det mellem host og protokol af de enkelte nodes. Den kan, eks., genstarte nodes der ikke svarer længere (watchdog). Den kan også se hvilke notes der virker, hvilke beskeder der sendes frem og tilbage.
	\item TTP Frame: De beskeder der sendes i et slot.
\end{itemize}


\section{FlexRay}

\begin{itemize}
	\item Er blevet foreslået til biler, da man kan kombinere Timer Trigger (Static) og Event Trigger (dynamic).
	\item Det bruger X-by-wire.
\end{itemize}

Context:
\begin{itemize}
	\item Mest brugte bus er CAN-bus, 1 Mbit/s
	\item For FlexRay kan man få 10 Mbit/s
	\item Noter L4,3S5 -- FlexRay udgør backbone for systemet.
\end{itemize}

Generelt:
\begin{itemize}
	\item 1 channel giver ikke redundancy.
	\item 2 channels skipper security, men giver dobbelt hastighed.
\end{itemize}

Communication:
\begin{itemize}
	\item Der er clock synkronisation, der laves i intervaller (ala PTP).

	\item BF = buffer (L4S18)
	\item With/without protocol knowledge = Bus guard
\end{itemize}






\newpage
\section{CAN-BUS}

Generelt:
\begin{itemize}
	\item Over 50 m bliver hastigheden under 1 Mbit/s
	\item  Bruges til at sende beskeder på 8 bytes
	\item Bruges primært i biler og andre industri-setup
	\item FlexRay er 10 Mbit/s
	\item TTP er 25 Mbit/s \fxnote{Hvad er dette?}
	\item Man bruger CAN, fordi det kan distribueres.
	\item Man skal her starte med at finde ud af, hvilken funktionalitet/hardware skal man bruge, og så lave softwaren dertil.
\end{itemize}

Udviklings scenarier:
\begin{itemize}
	\item Køb noget fra et katalog, der har CAN interface
	\item Lav noget med et bestemt interface (ud over CAN) som subcontractor
	\item Lav noget hvor CAN blot bruges internt (proprietary product).
\end{itemize}

Pros:
\begin{itemize}
	\item Billige
	\item Kan købes seperat
	\item Meget udbredt
	\item Nem at forbinde
	\item Der findes rigtig mange produkter der blot kan købes
	\item Der er (C)OTS tools
\end{itemize}

More stuff:
\begin{itemize}
	\item Bruger CSMA/CA
	\begin{itemize}
		\item Er der trafik? Hvis nej, så kobler man sig på
		\item Hvis 2 joiner samtidig vil den éne vinde. Her bruges et Arbitration field
	\end{itemize}
	\item Det er event driven protokol
	\item Der findes både CAN 2.0A (11 bit) og CAN 2.0B (29 bit) og begge bruges.
\end{itemize}

Hvordan det virker:
\begin{itemize}
	\item Hver node har et ID, der programmeres ind i noden fra start.
	\item Jo lavere nummer, jo højere prioritet. Assign evntuelt et span til hver node, så der kan skubbes noget ind imellem.
	\item CAN 2.0A
	\begin{itemize}
		\item L5S17-18
		\item 
	\end{itemize}
\end{itemize}

Error detection:
\begin{itemize}
	\item EOF skal være rene 0'er - ellers er der fejl.
	\item Bit stuffing -- 6 ens bit i træk
\end{itemize}

Implementering:
\begin{itemize}
	\item Ens CAN Controller skal se efter, om beskeder fra det fysiske lag er til denne node
\end{itemize}


\subsection{HigherLayerProtocol}
L5,2


Man kan lave CAN på 3 måder:
\begin{itemize}
	\item Uden CAL
	\item Med CAN Kingdom, hvor man kan lave små grupper
	\item Med CANopen (i Europa) / DeviceNet (I USA og Asien)
\end{itemize}

Forskellen:
\begin{itemize}
	\item Beskeder over 8 bytes skal flyttes til HLP
	\item HLP har ikke et API udaf til.
\end{itemize}






\newpage
\section{Time Triggered CAN}
Slides: TT-CAN
\begin{itemize}
	\item Der kan laves mere deterministisk netværk med dette
	\item Kan bruges med andet triggered ethernet
	\item Alt styres af en Global Clock
	\begin{itemize}
		\item Dette gøres med en referencebesked (L5S7)
		\item Man kan bruge 8 forskellige, med forskellige ID
		\item Når en sådan node modtages sættes tiden fra referencen
		\item De har prioritet over alle andre.
		\item Dette er Level 1 extension for CAN
		\begin{itemize}
			\item 1 byte i et normalt frame af 8 -- én har kontrolinformation.
			\item Den, der sender skal have en TT controller, så den ved hvornår den skal sende fra sin node.
		\end{itemize}


		\item Level 2 extension
		\begin{itemize}
			\item Her bliver urene synkroniseret bedre.
			\item Der bruges 4 bytes på reference beskeden.
			\item Master sender sin egen tid \textbf{med} i beskeden.
		\end{itemize}
	\end{itemize}

	\item Hvert time window kan have forskellig størrelse. (L5S10)
	\item Hvis noget er korruptet, vil det gå indover det næste frame. Dette skal der gøres noget ved (men hvad, ved jeg ikke -- sende igen senere).
	\item TDMA = Basic Cycle
	\item Cluster Cycle = System Matrix
	\item L5S12 viser, hvornår man kan modtage og sende nodes. Den ved dog, at de kan være forsinket, så de er større end de bør være.

	\item Man kan kombinere hvornår man skal sende data. Den cycle man er i, L5S13, beskrives af reference beskeden i starten.

	\item Det skal selv implementeres eller købes.
\end{itemize}






\newpage
\section{TT-Ethernet}

Sammen med andre systemer:
\begin{itemize}
	\item CAN er Event triggered (1Mbit/s).
	\item TTCAN -- doh, Time triggered. 
	\item TTCAN hører sammen med TTCAN.
	\item TT-Ethernet (100 Mbit/s / 1 Gbit/s) er baseret på normalt (switched)Ethernet, FlexRay (10 Mbits/s) og TTP(25 Mbit/s).
	\item TT-E er det eneste, der kan udvikles til "safety critical" systems.
	\item TT-E har ikke kabler, kun switches\dots forskel?
\end{itemize}

Generelt:
\begin{itemize}
	\item Man \emph{skal} ikke bygge et safety critical system, men det er godt til det.
	\item Deterministisk på L6S4 betyder, at man kan fortælle hvornår signalet kommer (inden for en hvis tid -- definer selv).
	\item Control functions -- enheder der skal synkroniseres
	\item Diagnostic info -- fejlmeddeleser der skal flyttes rundt
	\item Multimedia -- music streaming, VOIP
	\item High availability -- Man skal kunne gengive systemet
	\item L6S10 -- TT Ethnet Controller sørger for, at man ikke behøver en TTP stack. Det klarer den selv
	\item Fungerer også med normale computere, tilsluttet gennem ethernet kabel
	\item Switches er der, der kender tiden (Master clock)
	\item L6S14 -- De blå er tidspunkter, hvor man kan sende sine egne pakker
	\item Corrupted pakker skal gensendes i standard ethernet. TT-Ethernet bruger cut-through.
	\item Unprotected betyder, der ikke er en guardian på noden. L6S17-18
\end{itemize}

Teknisk:
\begin{itemize}
	\item CAN: 8 bytes
	\item[] TTP: 240
	\item[] FlexRay: 254
	\item[] TT-Ethernet: 1500-1514.
	\item L6S22 -- man kan have flere bits, som ikke har membership.
	\item L6S25 -- Alle beskeder kan identificeres på afsendelse.
\end{itemize}

Safety Critical L6S27+:
\begin{itemize}
	\item L6S28 -- du kan købe switches uden guardians også. Alt andet skal også supportere 2 porte.
	\item For at synkronisere L6S33 hosts i venstre side, kan man IEEE 1588 dem (PTP)
\end{itemize}











\end{document}