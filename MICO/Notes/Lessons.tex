\documentclass[oneside, 10pt]{article}

%Preamble

% Følgende er til koder.
%----------------------------------------------------------
%\begin{lstlisting}[caption=Overskrift på boks, style=Code-C++, label=lst:referenceLabel]
%public void hello(){}
%\end{lstlisting}
%----------------------------------------------------------

%Exstra space
\usepackage{xspace}
%Navn på bokse efterfulgt af \xspace (hvis det skal være mellemrum
%gives det med denne udvidelse. Ellers ingen mellemrum.
\newcommand{\codeTitle}{Code snippet\xspace}

%Pakker der skal bruges til lstlisting
\usepackage{listings}
\usepackage{color}
\usepackage{textcomp}
\definecolor{listinggray}{gray}{0.9}
\definecolor{lbcolor}{rgb}{0.9,0.9,0.9}
\renewcommand{\lstlistingname}{\codeTitle}
\lstdefinestyle{Code}
{
	keywordstyle	= \bfseries\ttfamily\color[rgb]{0,0,1},
	identifierstyle	= \ttfamily,
	commentstyle	= \color[rgb]{0.133,0.545,0.133},
	stringstyle		= \ttfamily\color[rgb]{0.627,0.126,0.941},
	showstringspaces= false,
	basicstyle		= \small,
	numberstyle		= \footnotesize,
%	numbers			= left, % Tal? Udkommenter hvis ikke
	stepnumber		= 2,
	numbersep		= 6pt,
	tabsize			= 2,
	breaklines		= true,
	prebreak 		= \raisebox{0ex}[0ex][0ex]{\ensuremath{\hookleftarrow}},
	breakatwhitespace= false,
%	aboveskip		= {1.5\baselineskip},
  	columns			= fixed,
  	upquote			= true,
  	extendedchars	= true,
 	backgroundcolor = \color{lbcolor},
	lineskip		= 1pt,
%	xleftmargin		= 17pt,
%	framexleftmargin= 17pt,
	framexrightmargin	= 0pt, %6pt
%	framexbottommargin	= 4pt,
}

%Bredde der bruges til indryk
%Den skal være 6 pt mindre
\usepackage{calc}
\newlength{\mywidth}
\setlength{\mywidth}{1.435\textwidth} % Hvis bredden header ikke virker er dette hvad skal ændres!


% Forskellige styles for forskellige kodetyper
\usepackage{caption}
\DeclareCaptionFont{white}{\color{white}}
\DeclareCaptionFormat{listing}%
{\colorbox[cmyk]{0.43, 0.35, 0.35,0.35}{\parbox{\mywidth}{\hspace{5pt}#1#2#3}}}
\captionsetup[lstlisting]
{
	format			= listing,
	labelfont		= white,
	textfont		= white, 
	singlelinecheck	= false, 
	width			= \mywidth,
	margin			= 0pt, 
	font			= {bf,footnotesize}
}

\lstdefinestyle{Code-C} {language=C, style=Code}
\lstdefinestyle{Code-Java} {language=Java, style=Code}
\lstdefinestyle{Code-C++} {language=[Visual]C++, style=Code}
\lstdefinestyle{Code-VHDL} {language=VHDL, style=Code}
\lstdefinestyle{Code-Bash} {language=Bash, style=Code}
\lstdefinestyle{Code-Matlab} {language=Matlab, style=Code}
\lstdefinestyle{Code-Prolog} {language=Prolog, style=Code}
%Speciel skrift for enkelt linje kode
%--------------------------------------------------
%Udskriver med fonten 'Courier'
%Mere info her: http://tex.stackexchange.com/questions/25249/how-do-i-use-a-particular-font-for-a-small-section-of-text-in-my-document
%Eksempel: Funktionen \code{void Hello()} giver et output
%--------------------------------------------------
\newcommand{\code}[1]{{\fontfamily{pcr}\selectfont #1}}

%Seperated files
%--------------------------------------------------
%Opret filer således:
%\documentclass[Navn-på-hovedfil]{subfiles}
%\begin{document}
% Indmad
%\end{document}
%
% I hovedfil inkluderes således:
% \subfile{navn-på-subfil}
%--------------------------------------------------
\usepackage{subfiles}
%Text typesetting
%--------------------------------------------------------
\usepackage[T1]{fontenc} 	% Can use danish characters
\usepackage[utf8]{inputenc} % Input encoding. Can be used on Linux, Mac and Windows         
\usepackage[danish]{babel} 	% Split words accoding to English
\usepackage{lmodern} 		% Font

\setlength\parindent{0pt} 	% No indent
\setlength\parskip{12pt} 	% More than a single line break will give ONE linebreak.

%Margin
\usepackage[left=2cm,right=2cm,top=2.5cm,bottom=2cm]{geometry}

%Margin
\usepackage[left=3cm,right=2cm,top=2.5cm,bottom=2cm]{geometry}

%Mellemrum mellem linjerne    
\linespread{1.5}

\title{Assignment 2 for TEDI}
\author{Rasmus Bækgaard, 10893}
\date{April 28, 2014}

\title{Middleware and Communication}
\author{Rasmus Bækgaard}
\date{\today}

\begin{document}

\maketitle

\section{Precision Time Protocol}

\subsection{Determinism}:
\begin{itemize}
	\item Synkronisere clocks mellem forskellige enheder.
	\item Standard protokol, IEEE 1588.
	\item Alt er om deadlines (ikke hurtigt -- men før en deadline)
	\item Deadline beregning på relativ event eller absolut synkroniseret clock på L1S4.
	\item Timings, L1S5.
	\item Forskel på real-time (Hard, form, soft), L1S6-7
	\begin{itemize}
		\item Hard: bilen gør ikke det den skal.
		\item Firm: Ikke helt så meget (skade til materiel, men ikke noget slemt).
		\item Soft: Skype mister lidt forbindelse, men du snakker stadig.
	\end{itemize}

	\item Man møder deadline, eller man møder ikke (binært). Det kan ikke nedbrydes (selvom man gerne vil, som ovenover)
	\item Determinism: Det næste stadie vil være unikt og vi ved hvad det vil være.
	\item Non-determinism: man kan gå i flere stadier (og vi ved ikke hvilke).

	\item Ethernet er ikke deterministic, pga. en protocol med et element, som ikke er deterministic.
\end{itemize}

\subsection{Automa}
\begin{itemize}
	\item Hvis man laver et system med en formel bruger man en model, som checker formlen igenne en model-checker.
	\item Deterministics finite automaton, DFA, er en en Turing-machine. L1S11
	\begin{itemize}
		\item $\delta$: Én fra $Q$ og én fra $\Sigma$ giver én ny $Q$
		\item Dobbelt cirkel: slut/accept punkt
		\item L1S12: der er et lige antal '0'er -- derfor tager den imod en \code{string}
		\item $*$ betyder antal gange fra 0 til mange gange.
	\end{itemize}

	\item Timed Finite Automata, TFA:
	\begin{itemize}
		\item Der findes forskellige definationer af dette på nettet, men brug dette.
		\item Tager tid med i beregningerne
		\item $I$ er en initialiser for clock.
		\item $\sim \in \{ =, <, >, \leq, geq\}$
		\item $\neg$ betyder "ikke"
	\end{itemize}

	\item For hvert nyt stadie er clock'en en anden.
	\item S1L16: tegnene på linjerne betyder "given that"

\end{itemize}

\subsection{Precision Time Protocol, PTP}:
Generelt:
\begin{itemize}
	\item En teknologi
	\item Udgivet af IEEE
	\item Synkroniserer clocks på et netværk
	\item Serverne kan have forskellige tider, kaldet også "local time".
	\item Jo tættere på hardware et timestamp er, jo mindre er fejlen
	\subitem Der findes dedikeret hardware til dette.
	\item Synkronisering skal ske continuereligt
	\item PTP kan på mindre end 1 minut nå $\pm$ 50ns præcision. (Udstyr kan synkronisere så godt som hvert sekund!)
	\item Bruges i LTE/5G netværk
	\item Der findes en algoritme der afgører, hvilken clock er bedst egnet som master
	\begin{itemize}
	 	\item "Stratum" er en værdi for den bedste master clock.
	 	\item Hvis master clock fejler, vil den næste blive valgt.
	 \end{itemize} 
	 \item Der findes et PHY niveau under interrupt niveau, som modtager RX og sender TX.
\end{itemize}

Synkronisering:
\begin{itemize}
	\item L1S32-33
	\item Man kender Line delay
	\item Boundary clock (special switch), der vil agere som kopi af (Grand) Master clock (L1S35-37)
	\item Hver boundary switch har et delay, som den forsøger at eliminere selv.
\end{itemize}







\newpage
\section{Real-time Ethernet}

\subsection{Generelt}
\begin{itemize}
	\item Vanilla ethernet er \emph{ikke} deterministisk.
	\item Det kan dog gøres, med de rigtige værktøjer.
	\item Motivationen var, at få det rykket ind til firmaer (til ALT).
	\item Datamængden der sendes er meget lille, responstiden skal være meget lille, men det sker meget ofte, L2S5.
	\item OSI er en måde som en arkitektur skal/burde være.
	\item Ethernet kaldes IEEE 802.3 L2S8.
	\item Pakkerne i en ethernet pakke kaldes et "frame". L2S10
	\begin{itemize}
		\item Det man sender ligger i \code{Payload}
		\item 802.1Q tag \emph{kan} bruges også.
		\item Resten er noget, som blot skal være der
	\end{itemize}

	\item Fordelen ved optic: Ingen EMC forstyrelser
\end{itemize}

\subsection{Single segment}
\begin{itemize}
	\item CSMA/CD, Carrier Send Multiple Access / Collision detection
	\item Deadline gurantees må ikke have collisions!
	\item Man lytter, til der har været en hvis pause på linjen, L2S18
	\item Overførelsel sker, som på L2S20-21
	\begin{itemize}
		\item Når collision sker, har de 2 linjer en back-off algoritme (Many Faces of Ethernet)
	\end{itemize}
\end{itemize}


\subsection{Multi segment}
\begin{itemize}
	\item Bruger switches, L2S25.
	\item Store and foward (L2S29) er bedre ved industriel, mens Cut through er bedre ved ingen støj (et hjem).
	\item Variation i latency er det største problem, L2S30.
	\item Nagle's algoritme er dårlig for Real-time systemer, da en algoritme gør noget (lettere) ukontrolleret.
\end{itemize}

\subsection{Single segment real-time}

\begin{itemize}
	\item Shared medium (Christian gennemgik nogle grafer)
	\item Når man sender data, fyldes pakken ikke helt, men efterlader lidt i begge ender.
\end{itemize}

\subsection{Multi segment real-time}
\begin{itemize}
	\item EtherNet/IP $\rightarrow$ (Internet Protocol)
\end{itemize}












\end{document}