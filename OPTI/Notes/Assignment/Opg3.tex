%!TEX root = Main.tex
\documentclass[Main]{subfiles}

\begin{document}

\section*{Exercise 3}
Consider the function: $F:R \rightarrow R$ where: $f(x) = \dfrac{1}{2}x^2 - cos(x)$.

\paragraph{Sketch the graph of $f$ and find the 1. and 2. deriviative of $f$.}

The sketch:

\begin{figure}[hbtp]
\centering
\includegraphics[width = 0.6 \textwidth]{CH3-1}
\vspace{-15pt}
\caption{Sketch of function $f(x) = \dfrac{1}{2}x^2 - cos(x)$}
\label{fig:ch3-1}
\end{figure}

The first derivative:
\begin{lstlisting}[caption=Derivatives and plot, style=Code-Matlab, label=lst:CH3-1]
syms x;
f = 1/2*x.^2 -cos(x);

figure(1)
ezplot(f);

fm = diff(f)
fmm = diff(fm)
\end{lstlisting}
\begin{align*}
f'(x) &= x+ sin(x).\\
f''(x) &= cos(x) + 1
\end{align*}

\paragraph{Now use the Newton method to find the minimizer of $f$ e.g. use initial value: $x^{(0)} = 0.25$, and calculate the values $x^{(0)}, x^{(1)}, x^{(2)}$ ,\dots 
How many iterations are needed for an accuracy of $\epsilon < 10^{-5}$  on
the minimum.} 

To find new values of $x^{k}$ the following formula is used: $x^{k+1} = x^k - \dfrac{f'(x^k)}{f''(x^k)}$
Insert this into \texttt{Matlab} returns the following output:

\begin{lstlisting}[caption=Newton's Method, style=Code-Matlab, label=lst:CH3-2]
fm = @(x) x+sin(x);
fmm = @(x) cos(x)+1;

x0 = 0.25;
x1 = x0 - (fm(x0)/fmm(x0))
x2 = x1 - (fm(x1)/fmm(x1))
x3 = x2 - (fm(x2)/fmm(x2))
\end{lstlisting}

\begin{align*}
x_1 &= -0.0026 \\
x_2 &= 3.0277 \cdot 10^{-9} \\
x_3 &= 0
\end{align*}

Since the accuracy needed was $10^{-5}$ and $x_2$ has $10^{-9}$ the third step of iteration is enough.









\end{document}