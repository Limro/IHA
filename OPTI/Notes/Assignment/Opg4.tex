%!TEX root = Main.tex
\documentclass[Main]{subfiles}

\begin{document}

\section*{Consider the function $f$ in exercise 2.}

\paragraph{1. Calculate the first 3 steps in steepest descent algorithm i.e. $x^{(0)}, x^{(1)}, x^{(2)}, $ to find the minimizer for $f$ using the initial point $x^{(0)} = (1,1)^T$ .}	 

Using the steppest descent algorithm on exercise 2 on $f$:
\\
$f: \dfrac{1}{2}x^T Qx - x^Tb = \dfrac{1}{2}x^T \cdot$ 
\begin{ArgMat}
2 & -2 \\
-2 & 6
\end{ArgMat}
$-x^T \cdot$
\begin{ArgMat}
0 \\
3
\end{ArgMat}
\\
We start with $x_0 = (1,1)^T$.
\\
The gradient of $f$ is $g(x) = Q \cdot x - b$.
\\
$x_1$ is calculated with
$\alpha_0 = \underset{\alpha \geq 0}{\text{arg min}} f(x^0 - \alpha \nabla f(x^{(0)})) = \Phi(\alpha)$.
\\
Calculate $\Phi(\alpha)$'s FONC to ensure minimum (Set the derivative equal 0) and find $x^{(1)} = x^{(0)}-a_0 \cdot f'(x^{(0)})$

\begin{lstlisting}[caption=Calculate $x^{(1)}$, style=Code-Matlab, label=lst:CH4-1]
syms xx1 xx2 a;
x = [xx1;xx2];

Q = [2 -2;-2 6];
b = [0;3];
f(x) = 1/2*x.' * Q * x - x.' * b;
g(x) = Q*x-b;
x0 = [1;1];

arg = x0 - a*g(x0(1),x0(2));
Phi0(a) = f(arg(1), arg(2))

a0 = solve(diff(Phi0(a)) == 0)
x1 =  x0-a0*g(x0(1),x0(2))
\end{lstlisting}

\begin{align*}
\Phi_0(a) &= 4a + (3a-2) \cdot (a-1)-3\\
a_0 &= \dfrac{1}{6} \\
x^1 &= \begin{bmatrix}
1 \\[0.3em] 
\dfrac{5}{6} 
\end{bmatrix}
\end{align*}

Same with $x^{(2)}$ and $x^{(3)}$:
\begin{align*}
\Phi_1(a) &= \dfrac{5 a}{18} + \bigg(\dfrac{a}{3} - 1\bigg) \bigg(\dfrac{a}{3} - \dfrac{1}{6}\bigg) - \dfrac{5}{4}\\
a_1 &= \dfrac{1}{2} \\
x^2 &= \begin{bmatrix}
\frac{5}{6} \\[0.3em] 
\frac{5}{6} 
\end{bmatrix} 
\\
\Phi_2(a) &= \dfrac{23 a}{18} + \bigg(\dfrac{a}{3} - \dfrac{5}{6}\bigg) \bigg(\dfrac{a}{3} - \dfrac{5}{3}\bigg) - \dfrac{5}{2}\\
a_2 &= \dfrac{1}{6} \\
x^3 &= \begin{bmatrix}
\frac{5}{6} \\[0.3em] 
\frac{7}{9} 
\end{bmatrix}
\end{align*}


\paragraph{2. Does the steepest descent algorithm converge?}
It does.
By plotting it we see a it more clearly

\begin{figure}[hbtp]
\centering
\includegraphics[width = 0.6 \textwidth]{CH4-1}
\vspace{-15pt}
\caption{Sketch of function $f(x) = \dfrac{1}{2}x^2 - cos(x)$}
\label{fig:ch4-1}
\end{figure}



\paragraph{3. Use a fixed-step size algorithm, what step size should be used to insure converges?}
To calulate the steps used with a symmetric  $Q$ the converges can be found as follow:
\begin{align*}
0 < \alpha < \dfrac{2}{\lambda_{max} (Q)}
\end{align*}	

\begin{lstlisting}[caption=Max, style=Code-Matlab, label=lst:CH4-3]
lamMax = max(eigs(Q))
\end{lstlisting}
$lamMax = 6.8284$. 
So as long as $0<\alpha<6.8284$ the algorithm will converge.

\end{document}