%!TEX root = Main.tex
\documentclass[Main]{subfiles}

\begin{document}

\section*{Exercise 6}

\paragraph{Calculate the first 3 steps in conjugate gradient algorithm, $x^{(0)}, x^{(1)}, x^{(2)}$ to find the minimizer for $f$ using the initial point $x = (1,1)^T$.}

The gradient of $f$ is $gr(x) = Q \cdot x-b$:
\begin{lstlisting}[caption=Gradient, style=Code-Matlab, label=lst:CH6-1]
g0 = gr(x0(1), x0(2))
\end{lstlisting}

$ g_0 =  
\begin{bmatrix}
0 \\ 
1
\end{bmatrix} $
\\
\\
Since it's not zero, we continue and find a direction, $d_0 = -g_0$ to find $\alpha_0$ and $x^1$:

\begin{lstlisting}[caption=Finding $x^1$, style=Code-Matlab, label=lst:CH6-2]
d0 = -g0;
a0 = -(g0.'*d0)/(d0.'*Q*d0)
x1 = x0+a0*d0
\end{lstlisting}

\[
x^1=
\begin{bmatrix}
1\\
\frac{5}{6}
\end{bmatrix}
\]
The gradient is now caluculated again:
\begin{lstlisting}[caption=$Gradient_1$, style=Code-Matlab, label=lst:CH6-3]
g1 = gr(x1(1), x1(2))
\end{lstlisting}
\[g_1 = \begin{bmatrix}
\frac{1}{3} \\0
\end{bmatrix}\]
Now we can find $\beta_0$ and the new direction:
\begin{lstlisting}[caption=New direction and $\beta$, style=Code-Matlab, label=lst:CH6-4]
b0 = g1.'*Q*d0/(d0'*Q*d0)
d1 = -g1+b0*d0
\end{lstlisting}
\begin{align*}
b_0 = \frac{1}{9}\\
d_1 = 
\begin{bmatrix}
-\frac{1}{3} \\
-\frac{1}{9}
\end{bmatrix}
\end{align*}
Now we can repeat the algorithm, by finding $\alpha_1$ and $x_2$:

\begin{lstlisting}[style=Code-Matlab, label=lst:CH6-5]
a1 = -(g1.'*d1)/(d1.'*Q*d1)
x2 = x1+a1*d1
\end{lstlisting}
\begin{align*}
a_1 = \dfrac{3}{4} \\
x_2 = 
\begin{bmatrix}
\frac{3}{4} \\
\frac{3}{4}
\end{bmatrix}
\end{align*}

And we have found the minimum.





















\end{document}