\documentclass[10pt, a4]{Memoir}

\usepackage{tcolorbox}
\usepackage{ulem} %math
\usepackage{amsmath}
\usepackage{amsfonts}
\usepackage{amssymb}
\usepackage{graphicx}
\usepackage{enumerate}


%Create a box for theorems
%\begin{theo}[titel] %optional
%tekst
%\end{theo}
\newenvironment{theo}[1][Vigtigt]{%
\begin{tcolorbox}[colback=green!5,colframe=green!40!black,title=\textbf{#1}]
}{%
\end{tcolorbox}
}




%Create a square matrix
%\begin{ArgMat}{2}
%21 & 22 & 23 \\  
%a & b & c
%\end{ArgMat}
%
% Info: http://tex.stackexchange.com/questions/2233/whats-the-best-way-make-an-augmented-coefficient-matrix
%
\newenvironment{ArgMat}{%
$
  \left[\begin{array}{@{}*{100}{r}r@{}}
}{%
  \end{array}\right]
  $
}

\newenvironment{deter}{%
$
  \left|\begin{array}{@{}*{100}{r}r@{}}
}{%
  \end{array}\right|
  $
}


%Create multiple lines with holes
%\begin{SysEqu}
%x_1 && &- &5x_3 &+ &2x_4=& 1 \\
%x_1 &+ &x_2 &+ &x_3 && =& 4 \\
%&&&&&&0 =& 0
%\end{SysEqu}
\newenvironment{SysEqu}{%
$  \setlength\arraycolsep{0.1em}
  \begin{array}{@{}*{100}{r}r@{}}
}{%
  \end{array}$
}

%Create solution for x_1, x_n...
%\begin{solu}
%x_1 &= d \\
%x_2 &= e \\
%x_3 &= s
%\end{solu}
\newenvironment{solu}{%
$
  \setlength\arraycolsep{0.1em}
  \left\{\begin{array}{@{}*{100}{r}r@{}}
}{%
  \end{array}\right.
$
}

\usepackage{lastpage}


\newcommand{\HRule}{\rule{\linewidth}{0.8mm}}

%Tekst i fotter
\newcommand{\footerText}{\thepage\xspace /\pageref{LastPage}}
\newcommand{\ProjectName}{433 MHz styring af AeroQuad}


\chapterstyle{hangnum}




\nouppercaseheads
\makepagestyle{mystyle} 

\makeevenhead{mystyle}{}{\\ \leftmark}{} 
\makeoddhead{mystyle}{}{\\ \leftmark}{} 
\makeevenfoot{mystyle}{}{\footerText}{} 
\makeoddfoot{mystyle}{}{\footerText}{} 
\makeatletter
\makepsmarks{mystyle}{% Overskriften på sidehovedet
  \createmark{chapter}{left}{shownumber}{\@chapapp\ }{.\ }} 
\makeatother
\makefootrule{mystyle}{\textwidth}{\normalrulethickness}{0.4pt}
\makeheadrule{mystyle}{\textwidth}{\normalrulethickness}

\makepagestyle{plain}
\makeevenhead{plain}{}{}{}
\makeoddhead{plain}{}{}{}
\makeevenfoot{plain}{}{\footerText}{}
\makeoddfoot{plain}{}{\footerText}{}
\makefootrule{plain}{\textwidth}{\normalrulethickness}{0.4pt}

\pagestyle{mystyle}

%%----------------------------------------------------------------------
%
%%Redefining chapter style
%%\renewcommand\chapterheadstart{\vspace*{\beforechapskip}}
%\renewcommand\chapterheadstart{\vspace*{10pt}}
%\renewcommand\printchaptername{\chapnamefont }%\@chapapp}
%\renewcommand\chapternamenum{\space}
%\renewcommand\printchapternum{\chapnumfont \thechapter}
%\renewcommand\afterchapternum{\space: }%\par\nobreak\vskip \midchapskip}
%\renewcommand\printchapternonum{}
%\renewcommand\printchaptertitle[1]{\chaptitlefont #1}
\setlength{\beforechapskip}{0pt} 
\setlength{\afterchapskip}{0pt} 
%\setlength{\voffset}{0pt} 
\setlength{\headsep}{25pt}
%\setlength{\topmargin}{35pt}
%%\setlength{\headheight}{102pt}
%\setlength{\textheight}{302pt}
\renewcommand\afterchaptertitle{\par\nobreak\vskip \afterchapskip}
%%----------------------------------------------------------------------




%Sidehoved og -fod pakke
%Margin
\usepackage[left=2cm,right=2cm,top=2.5cm,bottom=2cm]{geometry}
\usepackage{lastpage}



%%URL kommandoer og sidetal farve
%%Kaldes med \url{www...}
%\usepackage{color} %Skal også bruges
\usepackage{hyperref}
\hypersetup{ 
	colorlinks	= true, 	% false: boxed links; true: colored links
    urlcolor	= blue,		% color of external links
    linkcolor	= black, 	% color of page numbers
    citecolor	= blue,
}



%Mellemrum mellem linjerne    
\linespread{1.5}


%Seperated files
%--------------------------------------------------
%Opret filer således:
%\documentclass[Navn-på-hovedfil]{subfiles}
%\begin{document}
% Indmad
%\end{document}
%
% I hovedfil inkluderes således:
% \subfile{navn-på-subfil}
%--------------------------------------------------
\usepackage{subfiles}

%Prevent wierd placement of figures
%\usepackage[section]{placeins}

%Standard sti at søge efter billeder
%--------------------------------------------------
%\begin{figure}[hbtp]
%\centering
%\includegraphics[scale=1]{filnavn-for-png}
%\caption{Titel}
%\label{fig:referenceNavn}
%\end{figure}
%--------------------------------------------------
\usepackage{graphicx}
\usepackage{subcaption}
\usepackage{float}
\graphicspath{{../Figures/}}

%Speciel skrift for enkelt linje kode
%--------------------------------------------------
%Udskriver med fonten 'Courier'
%Mere info her: http://tex.stackexchange.com/questions/25249/how-do-i-use-a-particular-font-for-a-small-section-of-text-in-my-document
%Eksempel: Funktionen \code{void Hello()} giver et output
%--------------------------------------------------
\newcommand{\code}[1]{{\fontfamily{pcr}\selectfont #1}}


% Følgende er til koder.
%----------------------------------------------------------
%\begin{lstlisting}[caption=Overskrift på boks, style=Code-C++, label=lst:referenceLabel]
%public void hello(){}
%\end{lstlisting}
%----------------------------------------------------------

%Exstra space
\usepackage{xspace}
%Navn på bokse efterfulgt af \xspace (hvis det skal være mellemrum
%gives det med denne udvidelse. Ellers ingen mellemrum.
\newcommand{\codeTitle}{Kodeudsnit\xspace}

%Pakker der skal bruges til lstlisting
\usepackage{listings}
\usepackage{color}
\usepackage{textcomp}
\definecolor{listinggray}{gray}{0.9}
\definecolor{lbcolor}{rgb}{0.9,0.9,0.9}
\renewcommand{\lstlistingname}{\codeTitle}
\lstdefinestyle{Code}
{
	keywordstyle	= \bfseries\ttfamily\color[rgb]{0,0,1},
	identifierstyle	= \ttfamily,
	commentstyle	= \color[rgb]{0.133,0.545,0.133},
	stringstyle		= \ttfamily\color[rgb]{0.627,0.126,0.941},
	showstringspaces= false,
	basicstyle		= \small,
	numberstyle		= \footnotesize,
%	numbers			= left, % Tal? Udkommenter hvis ikke
	stepnumber		= 2,
	numbersep		= 6pt,
	tabsize			= 2,
	breaklines		= true,
	prebreak 		= \raisebox{0ex}[0ex][0ex]{\ensuremath{\hookleftarrow}},
	breakatwhitespace= false,
%	aboveskip		= {1.5\baselineskip},
  	columns			= fixed,
  	upquote			= true,
  	extendedchars	= true,
 	backgroundcolor = \color{lbcolor},
	lineskip		= 1pt,
%	xleftmargin		= 17pt,
%	framexleftmargin= 17pt,
	framexrightmargin	= 0pt, %6pt
%	framexbottommargin	= 4pt,
}

%Bredde der bruges til indryk
%Den skal være 6 pt mindre
\usepackage{calc}
\newlength{\mywidth}
\setlength{\mywidth}{\textwidth-6pt}


% Forskellige styles for forskellige kodetyper
\usepackage{caption}
\DeclareCaptionFont{white}{\color{white}}
\DeclareCaptionFormat{listing}%
{\colorbox[cmyk]{0.43, 0.35, 0.35,0.35}{\parbox{\mywidth}{\hspace{5pt}#1#2#3}}}
\captionsetup[lstlisting]
{
	format			= listing,
	labelfont		= white,
	textfont		= white, 
	singlelinecheck	= false, 
	width			= \mywidth,
	margin			= 0pt, 
	font			= {bf,footnotesize}
}

\lstdefinestyle{Code-C} {language=C, style=Code}
\lstdefinestyle{Code-Java} {language=Java, style=Code}
\lstdefinestyle{Code-C++} {language=[Visual]C++, style=Code}
\lstdefinestyle{Code-VHDL} {language=VHDL, style=Code}
\lstdefinestyle{Code-Bash} {language=Bash, style=Code}

%Text typesetting
%--------------------------------------------------------
%\usepackage{baskervald}
\usepackage{lmodern}
\usepackage[T1]{fontenc}              
\usepackage[utf8]{inputenc}         
\usepackage[english]{babel}       

\setlength{\parindent}{0pt}
\nonzeroparskip

%\setaftersubsecskip{1sp}
%\setaftersubsubsecskip{1sp}
 


%Dybde på indholdsfortegnelse
%----------------------------------------------------------
%Chapter, section, subsection, subsubsection
%----------------------------------------------------------
\setcounter{secnumdepth}{3}
\setcounter{tocdepth}{3}


%Tables
%----------------------------------------------------------
\usepackage{tabularx}
\usepackage{array}
\usepackage{multirow} 
\usepackage{multicol} 
\usepackage{booktabs}
\usepackage{wrapfig}
\renewcommand{\arraystretch}{1.5}



%Misc
%----------------------------------------------------------
\usepackage{cite}
\usepackage{appendix}
\usepackage{amssymb}
\usepackage{url,ragged2e}
\usepackage{enumerate}
\usepackage{amsmath} %Math bibliotek


\usepackage{longtable}


\begin{document}
\chapter{OPTI lessons}

\section{Lesson 1} % (fold)
\label{sec:lesson_1}

\subsection{Matrix Games}

\begin{itemize}
	\item Opstil tabel med, hvad R får ud af alle outcomes
	\item Opskriv worst case for hver række
	\item[] Find max at disse. $max[min[a_{ij}]]$
	\item Opskriv worst vase for hver kolonne
	\item[] Find max at disse.  $min[max[a_{ij}]]$
\end{itemize}

\subsection{Proberbility vector}

\begin{align}
x &= [x_1, x_2 \ldots, x_m] \\
\sum x_i &= 1\\
y &= [y_1, y_2 \ldots, y_n] \\
\sum y_i &= 1\\
E[x,y] &= \sum_{j=1}^n \sum_{i=1}^m x_i \cdot a_{ij} \cdot y_i \\
	&= x^T \cdot A \cdot y
\end{align}


% section lesson_1 (end)


\section{Lesson 2}

Linear Programming -- Geometric method and Simplex method.
\\
Slide 3
Maximer profit.
\begin{itemize}
\item Opskriv max. profit med $f(x_1, x_2) = Ax_1 + Bx_2$
\item Opskriv formlerne med $a \cdot x_1 + b \cdot x_2 \leq y_1$
\item Gør det muligt med $x_1 \geq 0, x_2 \geq 0$
\item Hvis der er flere "udgange" for et punkt, skrives $x_{input1}+x_{input2} = y_{output1}+y_{output2}$
\item Udregn
\end{itemize}
Slide 5 -- Definition
\begin{itemize}
\item Ved udregning af $Ax \geq b$ udregnes A's række sum (hver række, $m$'s, sum) og sammenlignes med $b_m$
\end{itemize}
Slide 7 -- theorem
\begin{itemize}
\item Du skal lede ved hjørnerne!
\end{itemize}
Slide 8 -- eksempel
\begin{itemize}
\item Følg slide 3
\item Tegn en graf med variablerne $x_1$ og $x_2$
\item Akserne har max-værdier (conditions)
\item Ingår flere variabler i én formel, så isoler det der er på y-aksen ($x_2$)
\item Kaldes et "konvex set"
\item Sæt extream points in i en tabel og udregn værdierne, og find max
\item I stedet for extream points kan man isolere $x_2$ i "max.-formlen" og finde den værdi, der kun rører i ét punkt (men det er lort!).
\end{itemize}
Slide 9 -- Slack variabler
\begin{itemize}
\item Tilføj slack variabler for at gøre $\leq$ til $=$
\item Basisløsningen er, at de nye basisværdier er det, de er lig med. Således er $x_1, x_2 \ldots = 0$
\end{itemize}
Slide 10 -- Elementary operators on liniar equation systems (pivot)
\begin{itemize}
\item Take one row, multiply with a number and add to another row
\item Interchange to row
\item Multiply 1 row with a number $\not = 0$
\item Opskriv matrix med $x_1, x_2 \ldots y$
\end{itemize}














\end{document}