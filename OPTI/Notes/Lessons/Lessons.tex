\documentclass[10pt, a4]{Memoir}

%Preamble

% Følgende er til koder.
%----------------------------------------------------------
%\begin{lstlisting}[caption=Overskrift på boks, style=Code-C++, label=lst:referenceLabel]
%public void hello(){}
%\end{lstlisting}
%----------------------------------------------------------

%Exstra space
\usepackage{xspace}
%Navn på bokse efterfulgt af \xspace (hvis det skal være mellemrum
%gives det med denne udvidelse. Ellers ingen mellemrum.
\newcommand{\codeTitle}{Code snippet\xspace}

%Pakker der skal bruges til lstlisting
\usepackage{listings}
\usepackage{color}
\usepackage{textcomp}
\definecolor{listinggray}{gray}{0.9}
\definecolor{lbcolor}{rgb}{0.9,0.9,0.9}
\renewcommand{\lstlistingname}{\codeTitle}
\lstdefinestyle{Code}
{
	keywordstyle	= \bfseries\ttfamily\color[rgb]{0,0,1},
	identifierstyle	= \ttfamily,
	commentstyle	= \color[rgb]{0.133,0.545,0.133},
	stringstyle		= \ttfamily\color[rgb]{0.627,0.126,0.941},
	showstringspaces= false,
	basicstyle		= \small,
	numberstyle		= \footnotesize,
%	numbers			= left, % Tal? Udkommenter hvis ikke
	stepnumber		= 2,
	numbersep		= 6pt,
	tabsize			= 2,
	breaklines		= true,
	prebreak 		= \raisebox{0ex}[0ex][0ex]{\ensuremath{\hookleftarrow}},
	breakatwhitespace= false,
%	aboveskip		= {1.5\baselineskip},
  	columns			= fixed,
  	upquote			= true,
  	extendedchars	= true,
 	backgroundcolor = \color{lbcolor},
	lineskip		= 1pt,
%	xleftmargin		= 17pt,
%	framexleftmargin= 17pt,
	framexrightmargin	= 0pt, %6pt
%	framexbottommargin	= 4pt,
}

%Bredde der bruges til indryk
%Den skal være 6 pt mindre
\usepackage{calc}
\newlength{\mywidth}
\setlength{\mywidth}{1.435\textwidth} % Hvis bredden header ikke virker er dette hvad skal ændres!


% Forskellige styles for forskellige kodetyper
\usepackage{caption}
\DeclareCaptionFont{white}{\color{white}}
\DeclareCaptionFormat{listing}%
{\colorbox[cmyk]{0.43, 0.35, 0.35,0.35}{\parbox{\mywidth}{\hspace{5pt}#1#2#3}}}
\captionsetup[lstlisting]
{
	format			= listing,
	labelfont		= white,
	textfont		= white, 
	singlelinecheck	= false, 
	width			= \mywidth,
	margin			= 0pt, 
	font			= {bf,footnotesize}
}

\lstdefinestyle{Code-C} {language=C, style=Code}
\lstdefinestyle{Code-Java} {language=Java, style=Code}
\lstdefinestyle{Code-C++} {language=[Visual]C++, style=Code}
\lstdefinestyle{Code-VHDL} {language=VHDL, style=Code}
\lstdefinestyle{Code-Bash} {language=Bash, style=Code}
\lstdefinestyle{Code-Matlab} {language=Matlab, style=Code}
\lstdefinestyle{Code-Prolog} {language=Prolog, style=Code}
%Speciel skrift for enkelt linje kode
%--------------------------------------------------
%Udskriver med fonten 'Courier'
%Mere info her: http://tex.stackexchange.com/questions/25249/how-do-i-use-a-particular-font-for-a-small-section-of-text-in-my-document
%Eksempel: Funktionen \code{void Hello()} giver et output
%--------------------------------------------------
\newcommand{\code}[1]{{\fontfamily{pcr}\selectfont #1}}

%Seperated files
%--------------------------------------------------
%Opret filer således:
%\documentclass[Navn-på-hovedfil]{subfiles}
%\begin{document}
% Indmad
%\end{document}
%
% I hovedfil inkluderes således:
% \subfile{navn-på-subfil}
%--------------------------------------------------
\usepackage{subfiles}
%Text typesetting
%--------------------------------------------------------
\usepackage[T1]{fontenc} 	% Can use danish characters
\usepackage[utf8]{inputenc} % Input encoding. Can be used on Linux, Mac and Windows         
\usepackage[danish]{babel} 	% Split words accoding to English
\usepackage{lmodern} 		% Font

\setlength\parindent{0pt} 	% No indent
\setlength\parskip{12pt} 	% More than a single line break will give ONE linebreak.

%Margin
\usepackage[left=2cm,right=2cm,top=2.5cm,bottom=2cm]{geometry}

%Margin
\usepackage[left=3cm,right=2cm,top=2.5cm,bottom=2cm]{geometry}

%Mellemrum mellem linjerne    
\linespread{1.5}

\title{Assignment 2 for TEDI}
\author{Rasmus Bækgaard, 10893}
\date{April 28, 2014}

\begin{document}
\chapter{OPTI lessons}

\section{Lesson 1} % (fold)
\label{sec:lesson_1}

\subsection{Matrix Games}

\begin{itemize}
	\item Opstil tabel med, hvad R får ud af alle outcomes
	\item Opskriv worst case for hver række
	\item[] Find max at disse. $max[min[a_{ij}]]$
	\item Opskriv worst vase for hver kolonne
	\item[] Find max at disse.  $min[max[a_{ij}]]$
\end{itemize}

\subsection{Proberbility vector}

\begin{align}
x &= [x_1, x_2 \ldots, x_m] \\
\sum x_i &= 1\\
y &= [y_1, y_2 \ldots, y_n] \\
\sum y_i &= 1\\
E[x,y] &= \sum_{j=1}^n \sum_{i=1}^m x_i \cdot a_{ij} \cdot y_i \\
	&= x^T \cdot A \cdot y
\end{align}


% section lesson_1 (end)


\section{Lesson 2}

Linear Programming -- Geometric method and Simplex method.
\\
Slide 3
Maximer profit.
\begin{itemize}
\item Opskriv max. profit med $f(x_1, x_2) = Ax_1 + Bx_2$
\item Opskriv formlerne med $a \cdot x_1 + b \cdot x_2 \leq y_1$
\item Gør det muligt med $x_1 \geq 0, x_2 \geq 0$
\item Hvis der er flere "udgange" for et punkt, skrives $x_{input1}+x_{input2} = y_{output1}+y_{output2}$
\item Udregn
\end{itemize}
Slide 5 -- Definition
\begin{itemize}
\item Ved udregning af $Ax \geq b$ udregnes A's række sum (hver række, $m$'s, sum) og sammenlignes med $b_m$
\end{itemize}
Slide 7 -- theorem
\begin{itemize}
\item Du skal lede ved hjørnerne!
\end{itemize}
Slide 8 -- eksempel
\begin{itemize}
\item Følg slide 3
\item Tegn en graf med variablerne $x_1$ og $x_2$
\item Akserne har max-værdier (conditions)
\item Ingår flere variabler i én formel, så isoler det der er på y-aksen ($x_2$)
\item Kaldes et "konvex set"
\item Sæt extream points in i en tabel og udregn værdierne, og find max
\item I stedet for extream points kan man isolere $x_2$ i "max.-formlen" og finde den værdi, der kun rører i ét punkt (men det er lort!).
\end{itemize}
Slide 9 -- Slack variabler
\begin{itemize}
\item Tilføj slack variabler for at gøre $\leq$ til $=$
\item Basisløsningen er, at de nye basisværdier er det, de er lig med. Således er $x_1, x_2 \ldots = 0$
\end{itemize}
Slide 10 -- Elementary operators on liniar equation systems (pivot)
\begin{itemize}
\item Take one row, multiply with a number and add to another row
\item Interchange to row
\item Multiply 1 row with a number $\not = 0$
\item Opskriv matrix med $x_1, x_2 \ldots y$
\end{itemize}



\newpage
\section{Lesson 3 -- duality}

\begin{itemize}
	\item Man har maximize og minimize
	\item Primal betegnes med (P)
	\item Duality betegnes med (D) eller (P')
	\item Noter, at $A^T y \geq c$
	\item Byt $c$ og $b$ og transpose $A$
\end{itemize}






\newpage
\section{Lesson 4 -- The general problem}

\begin{itemize}
	\item Der arbejdes i planer
	\item \texttt{Local minimizer} betyder, at det er det laveste punkt for et lille område på en parabel / funktion
	\item \texttt{Global minimizer} er den mindste af dem alle
	\item \texttt{Optimal solution} og \texttt{minimizer} kan være det samme
	\item $(D^2 f)^T = D^2 f$ er symmetrisk
\end{itemize}
Directional derivatives

\begin{itemize}
	\item Er i 3D
	\item Hvis du har et punkt på en overflade og det bevæges i en bestemt retning - hvad sker der så med værdien på overfalden?
	\item $f(x+\alpha d)$
	\item Chain rule: $\dfrac{d}{dx} f(y(x)) = \dfrac{df}{dy} \cdot \dfrac{dy}{dx}$
\end{itemize}
Feasible direction

\begin{itemize}
	\item Blot en mulig retning hvor $x \in \Omega$
\end{itemize}
Quadratisk form -- med matrix

\begin{itemize}
	\item Skrives som $Q(x_1 x_2) = x_1^2 + x_2^2 + 3x_1 x_2$
	\item $Q : \mathbb{R}^2 \rightarrow \mathbb{R}$
	\item $Q(x) = x^T A x$
\end{itemize}

$Q(x) = $
\begin{ArgMat}
x_1 & x_2
\end{ArgMat}
\begin{ArgMat}
a & b \\ c & d
\end{ArgMat}
\begin{ArgMat}
x_1 \\ x_2
\end{ArgMat}

\end{document}