\documentclass{article}
\usepackage{tcolorbox}
\usepackage{ulem} %math
\usepackage{amsmath}
\usepackage{amsfonts}
\usepackage{amssymb}
\usepackage{graphicx}
\usepackage{enumerate}


%Create a box for theorems
%\begin{theo}[titel] %optional
%tekst
%\end{theo}
\newenvironment{theo}[1][Vigtigt]{%
\begin{tcolorbox}[colback=green!5,colframe=green!40!black,title=\textbf{#1}]
}{%
\end{tcolorbox}
}




%Create a square matrix
%\begin{ArgMat}{2}
%21 & 22 & 23 \\  
%a & b & c
%\end{ArgMat}
%
% Info: http://tex.stackexchange.com/questions/2233/whats-the-best-way-make-an-augmented-coefficient-matrix
%
\newenvironment{ArgMat}{%
$
  \left[\begin{array}{@{}*{100}{r}r@{}}
}{%
  \end{array}\right]
  $
}

\newenvironment{deter}{%
$
  \left|\begin{array}{@{}*{100}{r}r@{}}
}{%
  \end{array}\right|
  $
}


%Create multiple lines with holes
%\begin{SysEqu}
%x_1 && &- &5x_3 &+ &2x_4=& 1 \\
%x_1 &+ &x_2 &+ &x_3 && =& 4 \\
%&&&&&&0 =& 0
%\end{SysEqu}
\newenvironment{SysEqu}{%
$  \setlength\arraycolsep{0.1em}
  \begin{array}{@{}*{100}{r}r@{}}
}{%
  \end{array}$
}

%Create solution for x_1, x_n...
%\begin{solu}
%x_1 &= d \\
%x_2 &= e \\
%x_3 &= s
%\end{solu}
\newenvironment{solu}{%
$
  \setlength\arraycolsep{0.1em}
  \left\{\begin{array}{@{}*{100}{r}r@{}}
}{%
  \end{array}\right.
$
}

\usepackage{lastpage}


\newcommand{\HRule}{\rule{\linewidth}{0.8mm}}

%Tekst i fotter
\newcommand{\footerText}{\thepage\xspace /\pageref{LastPage}}
\newcommand{\ProjectName}{433 MHz styring af AeroQuad}


\chapterstyle{hangnum}




\nouppercaseheads
\makepagestyle{mystyle} 

\makeevenhead{mystyle}{}{\\ \leftmark}{} 
\makeoddhead{mystyle}{}{\\ \leftmark}{} 
\makeevenfoot{mystyle}{}{\footerText}{} 
\makeoddfoot{mystyle}{}{\footerText}{} 
\makeatletter
\makepsmarks{mystyle}{% Overskriften på sidehovedet
  \createmark{chapter}{left}{shownumber}{\@chapapp\ }{.\ }} 
\makeatother
\makefootrule{mystyle}{\textwidth}{\normalrulethickness}{0.4pt}
\makeheadrule{mystyle}{\textwidth}{\normalrulethickness}

\makepagestyle{plain}
\makeevenhead{plain}{}{}{}
\makeoddhead{plain}{}{}{}
\makeevenfoot{plain}{}{\footerText}{}
\makeoddfoot{plain}{}{\footerText}{}
\makefootrule{plain}{\textwidth}{\normalrulethickness}{0.4pt}

\pagestyle{mystyle}

%%----------------------------------------------------------------------
%
%%Redefining chapter style
%%\renewcommand\chapterheadstart{\vspace*{\beforechapskip}}
%\renewcommand\chapterheadstart{\vspace*{10pt}}
%\renewcommand\printchaptername{\chapnamefont }%\@chapapp}
%\renewcommand\chapternamenum{\space}
%\renewcommand\printchapternum{\chapnumfont \thechapter}
%\renewcommand\afterchapternum{\space: }%\par\nobreak\vskip \midchapskip}
%\renewcommand\printchapternonum{}
%\renewcommand\printchaptertitle[1]{\chaptitlefont #1}
\setlength{\beforechapskip}{0pt} 
\setlength{\afterchapskip}{0pt} 
%\setlength{\voffset}{0pt} 
\setlength{\headsep}{25pt}
%\setlength{\topmargin}{35pt}
%%\setlength{\headheight}{102pt}
%\setlength{\textheight}{302pt}
\renewcommand\afterchaptertitle{\par\nobreak\vskip \afterchapskip}
%%----------------------------------------------------------------------




%Sidehoved og -fod pakke
%Margin
\usepackage[left=2cm,right=2cm,top=2.5cm,bottom=2cm]{geometry}
\usepackage{lastpage}



%%URL kommandoer og sidetal farve
%%Kaldes med \url{www...}
%\usepackage{color} %Skal også bruges
\usepackage{hyperref}
\hypersetup{ 
	colorlinks	= true, 	% false: boxed links; true: colored links
    urlcolor	= blue,		% color of external links
    linkcolor	= black, 	% color of page numbers
    citecolor	= blue,
}



%Mellemrum mellem linjerne    
\linespread{1.5}


%Seperated files
%--------------------------------------------------
%Opret filer således:
%\documentclass[Navn-på-hovedfil]{subfiles}
%\begin{document}
% Indmad
%\end{document}
%
% I hovedfil inkluderes således:
% \subfile{navn-på-subfil}
%--------------------------------------------------
\usepackage{subfiles}

%Prevent wierd placement of figures
%\usepackage[section]{placeins}

%Standard sti at søge efter billeder
%--------------------------------------------------
%\begin{figure}[hbtp]
%\centering
%\includegraphics[scale=1]{filnavn-for-png}
%\caption{Titel}
%\label{fig:referenceNavn}
%\end{figure}
%--------------------------------------------------
\usepackage{graphicx}
\usepackage{subcaption}
\usepackage{float}
\graphicspath{{../Figures/}}

%Speciel skrift for enkelt linje kode
%--------------------------------------------------
%Udskriver med fonten 'Courier'
%Mere info her: http://tex.stackexchange.com/questions/25249/how-do-i-use-a-particular-font-for-a-small-section-of-text-in-my-document
%Eksempel: Funktionen \code{void Hello()} giver et output
%--------------------------------------------------
\newcommand{\code}[1]{{\fontfamily{pcr}\selectfont #1}}


% Følgende er til koder.
%----------------------------------------------------------
%\begin{lstlisting}[caption=Overskrift på boks, style=Code-C++, label=lst:referenceLabel]
%public void hello(){}
%\end{lstlisting}
%----------------------------------------------------------

%Exstra space
\usepackage{xspace}
%Navn på bokse efterfulgt af \xspace (hvis det skal være mellemrum
%gives det med denne udvidelse. Ellers ingen mellemrum.
\newcommand{\codeTitle}{Kodeudsnit\xspace}

%Pakker der skal bruges til lstlisting
\usepackage{listings}
\usepackage{color}
\usepackage{textcomp}
\definecolor{listinggray}{gray}{0.9}
\definecolor{lbcolor}{rgb}{0.9,0.9,0.9}
\renewcommand{\lstlistingname}{\codeTitle}
\lstdefinestyle{Code}
{
	keywordstyle	= \bfseries\ttfamily\color[rgb]{0,0,1},
	identifierstyle	= \ttfamily,
	commentstyle	= \color[rgb]{0.133,0.545,0.133},
	stringstyle		= \ttfamily\color[rgb]{0.627,0.126,0.941},
	showstringspaces= false,
	basicstyle		= \small,
	numberstyle		= \footnotesize,
%	numbers			= left, % Tal? Udkommenter hvis ikke
	stepnumber		= 2,
	numbersep		= 6pt,
	tabsize			= 2,
	breaklines		= true,
	prebreak 		= \raisebox{0ex}[0ex][0ex]{\ensuremath{\hookleftarrow}},
	breakatwhitespace= false,
%	aboveskip		= {1.5\baselineskip},
  	columns			= fixed,
  	upquote			= true,
  	extendedchars	= true,
 	backgroundcolor = \color{lbcolor},
	lineskip		= 1pt,
%	xleftmargin		= 17pt,
%	framexleftmargin= 17pt,
	framexrightmargin	= 0pt, %6pt
%	framexbottommargin	= 4pt,
}

%Bredde der bruges til indryk
%Den skal være 6 pt mindre
\usepackage{calc}
\newlength{\mywidth}
\setlength{\mywidth}{\textwidth-6pt}


% Forskellige styles for forskellige kodetyper
\usepackage{caption}
\DeclareCaptionFont{white}{\color{white}}
\DeclareCaptionFormat{listing}%
{\colorbox[cmyk]{0.43, 0.35, 0.35,0.35}{\parbox{\mywidth}{\hspace{5pt}#1#2#3}}}
\captionsetup[lstlisting]
{
	format			= listing,
	labelfont		= white,
	textfont		= white, 
	singlelinecheck	= false, 
	width			= \mywidth,
	margin			= 0pt, 
	font			= {bf,footnotesize}
}

\lstdefinestyle{Code-C} {language=C, style=Code}
\lstdefinestyle{Code-Java} {language=Java, style=Code}
\lstdefinestyle{Code-C++} {language=[Visual]C++, style=Code}
\lstdefinestyle{Code-VHDL} {language=VHDL, style=Code}
\lstdefinestyle{Code-Bash} {language=Bash, style=Code}

%Text typesetting
%--------------------------------------------------------
%\usepackage{baskervald}
\usepackage{lmodern}
\usepackage[T1]{fontenc}              
\usepackage[utf8]{inputenc}         
\usepackage[english]{babel}       

\setlength{\parindent}{0pt}
\nonzeroparskip

%\setaftersubsecskip{1sp}
%\setaftersubsubsecskip{1sp}
 


%Dybde på indholdsfortegnelse
%----------------------------------------------------------
%Chapter, section, subsection, subsubsection
%----------------------------------------------------------
\setcounter{secnumdepth}{3}
\setcounter{tocdepth}{3}


%Tables
%----------------------------------------------------------
\usepackage{tabularx}
\usepackage{array}
\usepackage{multirow} 
\usepackage{multicol} 
\usepackage{booktabs}
\usepackage{wrapfig}
\renewcommand{\arraystretch}{1.5}



%Misc
%----------------------------------------------------------
\usepackage{cite}
\usepackage{appendix}
\usepackage{amssymb}
\usepackage{url,ragged2e}
\usepackage{enumerate}
\usepackage{amsmath} %Math bibliotek


\usepackage{longtable}

\title{SMP eksamen}
\date{4. juni, 2014}
\author{Rasmus Bækgaard, 10893}

\begin{document}
\maketitle

\section*{Opgave 1} % (fold)
\label{sec:section_name}

\subsection*{1} % (fold)
\label{sub:1}

Her findes $F_X(x)$ ved indsættelse i formel:
\begin{align}
F_X(x) &= \int_{-\infty}^x f_x(u) du \\
	&= \int_0^x \dfrac{1}{8} x^2 + \dfrac{1}{3} \\
	&= \dfrac{1}{24} \cdot x \cdot \left( x^2 + 8 \right) & \text{Se Matlab kode heruder}
\end{align}

\begin{lstlisting}[caption=Opgave 1 del 1, style=Code-Matlab, label=lst:opg11]
syms x u;
int((1/8)*u^2+(1/3),u,0,x)
\end{lstlisting}

For fordelingsfunktioner gælder, at de er 0 før ved $x < 0$ og 1 efter ovenstående interval: \\
$\uuline{F = 
\begin{cases}
0 & x < 0 \\
\dfrac{1}{24} \cdot x \cdot \left( x^2 + 8 \right) & 0 \leq x \leq 2\\
1 & x > 2
\end{cases}}$
\\
\\
For at beregne $Pr(X > 1)$ gøres følgende: 
\begin{align}
Pr(x > 1) &= 1- Pr(x \leq 1) \\
	&= 1- F_X(1) \\
	&= \uuline{\dfrac{5}{8}}
\end{align}


\subsection*{2} % (fold)

Middelværdien udregnes som følger:

\begin{align}
E[X] &= \int_0^2 x \cdot f(x) dx \\
	&= \int_0^2 x \cdot \dfrac{1}{8} x^2 + \dfrac{1}{3} dx \\
	&= \uuline{\dfrac{7}{6}}
\end{align}

Herefter udregnes variansen som følger:
\begin{align}
E\left[ X^2 \right] &= \int_0^2 x^2 \cdot f(x) dx \\
	&= \int_0^2 x^2 \cdot \dfrac{1}{8} x^2 + \dfrac{1}{3} dx \\
	&= \dfrac{76}{45} \\
Var(X) &= E\left[ X^2 \right] - \left(E[X]\right)^2 \\
	&= \uuline{\dfrac{59}{180}}
\end{align}

\begin{lstlisting}[caption=Opgave 1 del 2, style=Code-Matlab, label=lst:opg12]
syms x
EX = int(x*(1/8*x^2 + 1/3), x, 0, 2)
EX2 = int(x^2*(1/8*x^2 + 1/3), x, 0, 2)

Var = EX2 - EX^2
\end{lstlisting}

\subsection*{3} % (fold)
Tæthedsfunktionen vises for $Y$ ved at bruge transformationssætningen:
\\
Invers:
\begin{align}
y = 2 \cdot x \Leftrightarrow x = \dfrac{y}{2}
\end{align}
Differentier:
\begin{align}
\dfrac{dx}{dy} = \dfrac{1}{2}
\end{align}
Find grænserne:
\begin{align}
0 \leq x \leq 2 \Leftrightarrow 0 \leq \dfrac{y}{2} \leq 2 \Leftrightarrow 0 \cdot 2 \leq y \leq 2 \cdot 2 \Leftrightarrow 0 \leq y \leq 4
\end{align}
Brug af formel:
\begin{align}
f_y(y) &= \left| \dfrac{dx}{dy} \right| \cdot f_X(x) \\
	&= \left| \dfrac{1}{2}\right| \cdot \dfrac{1}{8} x^2 + \dfrac{1}{3} \\
	&= \dfrac{1}{2} \cdot \dfrac{1}{8} \dfrac{y}{2}^2 + \dfrac{1}{3} & \text{Jeg tror du har sat } \dfrac{1}{3} x \text{ ind\dots}\\
	&= \uuline{\dfrac{y^2}{64}+\dfrac{1}{3}} 
\end{align} 

Da tæthedsfunktionen er 0 uden for intervallet følger det, at 
$ f_Y(y) = 
\begin{cases}
\dfrac{y^2}{64}+\dfrac{1}{3} & 0 \leq y \leq 4 \\
0 & \text{ellers}
\end{cases}$


\section*{Opgave 2} % (fold)

\subsection*{1} % (fold)
Middelværdien har en konstant, $B$, og processen er i intervallet $[0 ; 2\pi]$, hvilket svarer til en fuld omgang på en cirkel.
\begin{align}
E\left[X(t)\right] &= E\left[ A \cdot sin( \omega \cdot t + \theta) + B \right] \\
	&= E[A] \cdot E[sin(\omega \cdot t + \theta)] + E[B] \\
	&= 0 \cdot E[sin(\omega \cdot t + \theta)] + E[B] \\
	&= E[B] \\
	&= \uuline{B}
\end{align}
Det følger altså, at middelværdien er $B$, da $B$ ikke er en stokastisk variabel.

\subsection*{2} % (fold)
Autokorrelationensfunktionen kan findes således:
\begin{align}
R_X(t_1, t_2) &= E\left[ X(t_1) \cdot X(t_2) \right] \\
	&= E\left[ \left(A \cdot sin( \omega \cdot t_1 + \theta) + B \right) \cdot \left(A \cdot sin( \omega \cdot t_2 + \theta) + B \right) \right] \\
	&= E\left[ \left(A^2 \cdot sin \left(\omega \cdot t_1 + \theta \right)  \cdot sin \left(\omega \cdot t_2 + \theta \right)\right) + \left(A \cdot sin \left(\omega \cdot t_1 + \theta \right) + B\right)  + \left(A \cdot sin \left(\omega \cdot t_2 + \theta \right) + B\right) + B^2 \right] \\
	&= E\left[A^2] + E\left[sin \left(\omega \cdot t_1 + \theta \right)  \cdot sin \left(\omega \cdot t_2 + \theta \right)\right)\right]
	+ 2 \cdot E[A] \cdot E\left[sin \left(\omega \cdot t_1 + \theta \right) \right] \\ &+ 2 \cdot E[B] + E\left[sin \left(\omega \cdot t_2 + \theta \right) \right]
	+ E\left[B^2 \right] \nonumber \\
	&= Var(A) + Var(B) + E\left[sin \left(\omega \cdot t_1 + \theta \right)  \cdot sin \left(\omega \cdot t_2 + \theta \right)\right] + 0 + 0 + E\left[sin \left(\omega \cdot t_2 + \theta \right) \right] \\
	&\text{Suddenly -- magic\dots} \nonumber \\
	&= \dfrac{1}{2}cos \left(\omega(t_1- t_2)\right) + B^2
\end{align}

Den er der næsten ovenfor$\dots$ \\
Men den ER stationær, da $R_X(t_1, t_2)$ indeholder ledet $t_1 - t_2$.




\newpage
\section*{Opgave 3} % (fold)

\subsection*{1} % (fold)
Parametrene er som følger:\\
\begin{align}
x &= 0.01\\
n &= 10 \\
p &= \text{hypotesen}
\end{align}


\subsection*{2} % (fold)

Vi skal finde sandsynligheden for, at der ved 2 ud af 10 forekommer en 1\% change for hver af dem:

\begin{align}
\text{Sandsynlighed} &= \dfrac{2}{10} \cdot 0.01 \\
	&= 0.002 \\
	&= \uuline{0.2\% \text{ chance}}
\end{align}

\subsection*{3} % (fold)

Vi foretager en hypotesetest for binomialfordeling: 

\begin{align}
H: p &= p_0 \\
	&= \dfrac{15}{100} \\
	&= 0.15
\end{align}

Vi finder nu en p-værdi:
\begin{align}
z &= \dfrac{x - n \cdot p_0}{\sqrt{n \cdot p_0 \cdot(1-p_0)}}\\
	&= \dfrac{15 - 1000 \cdot 0.15}{\sqrt{1000 \cdot 0.15 \cdot(1-0.15)}}
	&= -11.9558 \\
p_\text{værdi} &= 2 \cdot \left| 1- \phi(\left|z\right|) \right|
\end{align}

I et 95\% konfidensinterval skal vi nu gerne finde vores p-værdi inden for dette interval:
\begin{align}
interval &= \dfrac{1}{n+u^2} \left[ x + \dfrac{u^2}{2} \pm u \sqrt{\dfrac{x(n-x)}{n}+\dfrac{u^2}{4}}\right] \\
Min_{interval} &= 0.0091 \\
Max_{interval} &= 0.0246 \\
\end{align}

\begin{lstlisting}[caption=Opgave 3, style=Code-Matlab, label=lst:opg33]
x = 15;
n = 1000;
p0 = 0.15;

z = (x-n*p0)/(sqrt(n*p0*(1-p0)))

val = (2/10) *0.01

u = norminv(1- (0.05/2));

min = 1/(n+u^2) * (x+ u^2/2 - u*sqrt((x*(n-x)/n) + (u^2/4)))
max = 1/(n+u^2) * (x+ u^2/2 + u*sqrt((x*(n-x)/n) + (u^2/4)))
\end{lstlisting}

Her vilde jjeg så have samlignet de $p_\text{værdi}$ og intervallet mellem $Min_{interval} \text{ og } Max_{interval}$, men da jeg ikke kan huske hvordan $p_\text{værdi}$ udregnes, må det stå hen. \\
Hvis det var inden for intervallet ville det stemme over ens med observationerne og visa versa.

\section*{Opgave 4} % (fold)

\subsection*{1} % (fold)

Ved at indtaste data i \texttt{Matlab} findes middelværdien til 330.75 mL. \\
Variansen findes til 4.6184.

\begin{lstlisting}[caption=Opgave 4.1, style=Code-Matlab, label=lst:41]
X = [ 330 334 328 330 330 331 335 330 331 330 ...
    331 330 332 329 335 330 326 331 330 332 ];

mean(X)
var(X)
\end{lstlisting}


\subsection*{2} % (fold)
Tidspres! \\
Jeg vurderer den er korrekt kalibreret, men da jeg ikke ved hvor præcis den skal være, giver det ikke meget mening at udføre 95\% interval. 
Må den vær 0.1 mL ved siden af? 
Eller måske 0.5 mL?
\\
Men gør brug af formlen på Slides 22 i "StatKap4+5+6", og indsæt i formel.

\subsection*{3} % (fold)
Der skal bruges $n \geq \left( \dfrac{1.96 \cdot varians}{B} \right) ^2$. 
Indsæt selv et $B$ (sammen slides, s. 25)


% subsection 3 (end)

% subsection 2 (end)
% section opgave_4 (end)

% subsection 2 (end)

% subsection 1 (end)


% section opgave_3 (end)

%\begin{figure}[hbtp]
%\centering
%\includegraphics[width=0.8 \textwidth]{filnavn-for-png}
%\caption{Titel}
%\label{fig:referenceNavn}
%\end{figure}




\end{document}