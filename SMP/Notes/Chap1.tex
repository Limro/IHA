\documentclass[Main]{subfiles}
\begin{document}

\chapter{Chapter 1}

\begin{itemize}
\item Trial
\subitem Et eksperiment

\item Elementær hændelse
\subitem ét udfald

\item Sammensat hændelse 
\subitem Flere forskellige udfald

\item Simultan hændelse
\subitem Involvere mere end ét eksperiment
\end{itemize}

Det gælder at $N_A+N_B+\ldots+N_n = N$.
\\
Frekvensen af A kan udtrykkes som:
\begin{align*}
r(A) &= \dfrac{N_A}{N}
\end{align*}





\section{Conditional probability}
Lad sige, at man ønsker at få to bestemte udfald (heads og heads) ved at kaste 2 mønter:
\begin{align}
Pr(A,B) = Pr(A) \cdot Pr(B) \label{equ:simu}
\end{align}
Dette er en simultan hændelse.

\begin{table}[H]
\begin{tabular}{ccc}
Kast 1 & Kast 2 & Pr\\ \hline
H & H & $\frac{1}{4}$\\
H & T & $\frac{1}{4}$\\
T & H & $\frac{1}{4}$\\
T & T & $\frac{1}{4}$\\ \hline
& & $\sum = 1$
\end{tabular}
\end{table}

Med en loaded mønt ser det anderledes ud, da $Pr(H) = 0.6$ og $Pr(T)=0.4$
\begin{table}[H]
\begin{tabular}{ccc}
Kast 1 & Kast 2 & Pr\\ \hline
H & H & $0.36$ (Se formel \ref{equ:simu})\\
H & T & $0.24$\\
T & H & $0.24$\\
T & T & $0.16$\\ \hline
& & $\sum = 1$
\end{tabular}
\end{table}


Hvis man ønsker sammensat hændelse for Pr(Præcis ét H) med $Pr(H)=0.5, Pr(T)=0.5$:

\begin{table}[H]
\begin{tabular}{ccc}
Kast 1 & Kast 2 & Pr\\ \hline
H & H & $0.36$ (Se formel \ref{equ:simu})\\
H & T & $0.24$\\
T & H & $0.24$\\
T & T & $0.16$\\ \hline
& & $\sum = 1$
\end{tabular}
\end{table}
På ovenstående vil række 1 og 2 være én heads. 
Plus disse sammen for at få en sammensat hændelse:
\begin{align*}
Pr(2)+Pr(4) = \frac{1}{2}
\end{align*}

Med tre mønter ser det således ud:
\begin{table}[H]
\begin{tabular}{cccc}
Kast 1 & Kast 2 & Kast 3& Pr\\ \hline
H & H & H & $\dfrac{1}{8}$ \\
\rowcolor{gr}
H & H & T & $\dfrac{1}{8}$ \\
H & T & H & $\dfrac{1}{8}$ \\
\rowcolor{gr}
H & T & T & $\dfrac{1}{8}$ \\
T & H & H & $\dfrac{1}{8}$ \\
\rowcolor{gr}
T & H & T&  $\dfrac{1}{8}$ \\
T & T & H & $\dfrac{1}{8}$ \\
\rowcolor{gr}
T & T & T & $\dfrac{1}{8}$ \\ \hline
& & & $\sum = 1$
\end{tabular}
\end{table}
For én heads vil række 4, 6 og 7 opfylde reglen:
$Pr(H,T,T)+Pr(T,H,T)+Pr(T,T,H) = \dfrac{3}{8}$.

Med en loaded mønt hvor $Pr(H) = 0.6, Pr(T)=0.4$ og præcis én mønt:
\begin{align}
Pr(H,T,T) + Pr(T,H,T) + Pr(T,T, H) &=
Pr(H)Pr(T)Pr(T) \nonumber\\
&+Pr(T)Pr(H)Pr(T)+ Pr(T)Pr(T)Pr(H)\\
	&=(0.6 \cdot 0.4 \cdot 0.4)\cdot3\\
	&= 0.096\cdot 3\\
	&= 0.288
\end{align}



\section{Relativ frekvens}
$N$ er det totale antal udfald.
\\
$N_A$ er antallet af ønsket udfald af typen $A$
\\
$Pr(A) \approx \dfrac{N_A}{N}$
\begin{align*}
N_A &= 150\\
N &= 1000 \\
A &= (10 \Omega, 5 W)\\
	&= \dfrac{150}{1000} = 0.15\\
Pr(10 \Omega | 5W) &= \dfrac{150}{360}
\end{align*}

\begin{table}[H]
\begin{tabular}{ccc}
$\Omega$ & W & Pr \\ \hline
\dots	&\dots	&\dots	\\
10 & 5 & $Pr(10 \Omega) \cdot Pr(5 W | 10 \Omega)$\\
& & $Pr(5 W) \cdot Pr(10 \Omega | 5 W)$\\
\end{tabular}
\end{table}

Den totale sandsynlighed af en kolonne eller en række kaldes en \textit{marginal sandsynlighed}.

\begin{align*}
Pr(A,B) &= Pr(A) \cdot Pr(B|A)\\
	&= Pr(B)\cdot Pr(A|B)
\end{align*}

\textbf{Uafhængighed} -- så hvis og kun hvis:
\begin{align*}
Pr(A|B) &= Pr(A)\\
Pr(B|A) &= Pr(B)\\
Pr(A, B) &= Pr(A)\cdot Pr(B|A)\\
	&= Pr(A) \cdot Pr(B)
\end{align*}

\section{Venn diagrammer}

\begin{tikzpicture}
    \draw \firstcircle node[below] {$A$};
    \draw \secondcircle node [above] {$B$};
    \draw \thirdcircle node [below] {$C$};

    % Now we want to highlight the intersection of the first and the
    % second circle:

    \begin{scope}
      \clip \firstcircle;
      \fill[red] \secondcircle;
    \end{scope}

    % Next, we want the highlight the intersection of all three circles:

    \begin{scope}
      \clip \firstcircle;
      \clip \secondcircle;
      \fill[green] \thirdcircle;
    \end{scope}

    % The intersection trick works pretty well for intersections. If you need
    % the set-theoretic difference between two sets, things are a little more
    % complicated:

    % Suppose we want to highlight the part of the first circle that is not 
    % also part of the second circle. For this, we need to clip against the 
    % "complement" of the second circle. The trick is to add a large rectangle
    % that encompasses everything and then use the even-odd filling rule 
    % (see the manual again):

    \begin{scope}[shift={(6cm,0cm)}]
        \begin{scope}[even odd rule]% first circle without the second
            \clip \secondcircle (-3,-3) rectangle (3,3);
        \fill[yellow] \firstcircle;
        \end{scope}
        \draw \firstcircle node {$A$};
        \draw \secondcircle node {$B$};
    \end{scope}
    
    % When using the above, you will notice that the border lines of the
    % original circles are erased by the intersection parts. To solve this
    % problem, either use a background layer (see the manual) or simply draw
    % the border lines after everything else has been drawn.
    
    % The last trick is to cheat and use transparency
    \begin{scope}[shift={(3cm,-5cm)}, fill opacity=0.5]
        \fill[red] \firstcircle;
        \fill[green] \secondcircle;
        \fill[blue] \thirdcircle;
        \draw \firstcircle node[below] {$A$};
        \draw \secondcircle node [above] {$B$};
        \draw \thirdcircle node [below] {$C$};
    \end{scope}
\end{tikzpicture}




















\end{document}