\documentclass[Main]{subfiles}

\begin{document}

\chapter{Stokastiske variabler}

\begin{tabular}{lll}
S	& Population & Kunne være den danske befolkning\\
$\alpha$ 	& Eet element i S & Dronning Magrethe\\
$X(\alpha)$ & Stokastisk variabel & En funktion, der måler en egenskab ved $\alpha$ (f.eks alder)\\
Udfald & Delmængde af S & $\{ \alpha \in S:X(\alpha) \leq x \}$
\end{tabular}
\\
Sandsynligheden for udfald $\{ \alpha \in S:X(\alpha) \leq x \}$ skrives som $Pr(X \leq x)$

\paragraph{Tæthedsfunktion}
\begin{align*}
Pr(x_1 < X \leq x_2) &= \int_{x_1}^{x_2} f_X(x)\cdot dx \\
	&= F_X(x_2) - F_X(x_1)
\end{align*}


\paragraph{Fordelingsfunktion ($F_X$)}
\begin{align*}
Pr(X \leq x) = F_X(x) &= \int_{-\infty}^x f_X(u) \cdot du\\
f_X(x) &= \dfrac{d F_X(x)}{dx}\\
Pr( X > x) &= 1- Pr(X \leq x)\\
	&= 1- F_X(x)
\end{align*}



\paragraph{Middelværdi}
Det vægtede gennemsnit:
\begin{align*}
E[K] &= \sum_{k=0}^n k\cdot Pr(K=k)\\
	&= \sum_{k=0}^n k\cdot Pr(k)
\end{align*}



\paragraph{Middelværdi og varians}
Forventet værdi, $E[X]$:
\begin{align*}
E[X] &= \int_{-\infty}^\infty x \cdot f_X(x) dx\\
E[g(x)] &= \int_{-\infty}^\infty g(x) \cdot f_X(x) dx & \text{Funktion}\\
E[X^2] &= \int_{-\infty}^\infty  x^2 \cdot f_X(x) dx & \text{Mean square}\\
E[(X-\overline{X})^2] &= \int_{-\infty}^\infty (x-\bar{x})^2 \cdot f_X(x) dx  & \text{Varians}\\
	&= Var(X)\\
	&= E[X^2]-(E[X])^2
\end{align*}



\paragraph{Uniform fordeling}
I intervallet $[a,b]$:
\begin{align*}
E[X] &= \frac{1}{2} (b-a)\\
Var(X) &= \frac{1}{12}(b-a)^2
\end{align*}
Binomial fordeling med $n$ trials og sandsynlighedsparameter $p$
\begin{align*}
E[X] &= n \cdot p\\
Var(X) &= n \cdot p \cdot (1-p)
\end{align*}
Gauss fordeling (normalfordeling)
\begin{align*}
f_X(x) &= \dfrac{1}{\sqrt{2 \pi \Omega^2}}\cdot e^{-(x-\mu)^2/{2\omega^2}}
\end{align*}

\section{Transformationssætningen}
Givet er 
\begin{itemize}
\item Funktionen $Y=g(x)$
\item Tætheden $f_X(x)$
\item grænser $\alpha\leq X \leq b$
\end{itemize}

\begin{tabular}{lll}
1 & Inverse & Beregn x = $g^{-1}(y)$\\
2 & Differentier & Beregn $\dfrac{d}{dy}g^{-1}(y) = \dfrac{dx}{dy}(y)$\\
3 & Grænser	& Giver $\alpha \leq X \leq b$, beregn $a_Y  \leq Y \leq b_Y$\\
4 & Formal & $f_Y(y) = \det(\frac{dx}{dy}) f_X(g^{-1} (y)) $
\end{tabular}

\paragraph{Eksempel}
\begin{FunArg}
\dfrac{1}{2}+\dfrac{1}{4}\cdot x & -1 \leq x \leq 1 \\
0 & \text{ellers}
\end{FunArg}
\begin{align*}
g(x) = y &= \dfrac{1}{2}(x+1)\\
y &= \dfrac{1}{2}(x+1) \Leftrightarrow\\
x &= 2y-1\\
\dfrac{dx}{dy} &= \dfrac{d(2y-1)}{dy} = 2\\
-1 \leq X \leq 1 &\Leftrightarrow\\
-1 \leq 2y-1 \leq 1 &\Leftrightarrow\\
0 \leq y \leq 1 \\
f_Y(y) &= \det \dfrac{dx}{dy} f_X(2y-1)\\
	&= 2\cdot (\dfrac{1}{2}+\dfrac{1}{4}(2y-1))\\
	&= y+\dfrac{1}{2}
\end{align*}

























\end{document}