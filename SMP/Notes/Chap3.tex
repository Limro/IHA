\documentclass[Main]{subfiles}

\begin{document}

\chapter{Stokastiske variabler}

\section{Histogram matching}
For at finde $F_y(y)$ fra $F_x(x)$, så skal man finde værdien for $x_i$ og overføre den til $F_y(y)$ (slide Chap. 3, s. 6).



\begin{lstlisting}[caption=Uniform fordeling med random tal, style=Code-Matlab, label=lst:uni]
yrange = -5:0.1:5;
Fy = normcdf(yrange,0,1);

N = 1000;
x = rand(1,N);
Fx = unifcdf(x,0,1);

for i = 1:N
distance = abs(Fx(i)-Fy);
[min_val,min_ix] = min(distance);
y(i) = yrange(min_ix);
end

\end{lstlisting}

Fordelingsfunktionen er per definition integralet fra $-\infty$ til $\infty $.

Slide 23/61 for vigtige relationer.

Givet $f(x,y)$, hvordan beregner man de marginale
sandsynligheder $f_X(x)$ og $f_Y(y)$?

\begin{align*}
f_X(x) &= \int_{-\infty}^{\infty} f(x,y) dy\\
f_Y(y) &= \int_{-\infty}^{\infty} f(x,y) dx
\end{align*}

Hvordan beregnes de betingede tætheder $f(x|y)$ og $f(y|x)$?

\begin{align*}
f(x|y) &= \dfrac{f(x,y)}{f_Y(y)}\\
f(y|x) &= \dfrac{f(x,y)}{f_X(x)}
\end{align*}

Bayes regel:
\begin{align*}
f(y|x) &= \dfrac{g(x|y) \cdot f_Y(y)}{f_X(x)}\\
f(x|y) &= \dfrac{g(x|y) \cdot f_X(x)}{f_Y(y)} & \text{Måske}
\end{align*} 


\section{Korrelation}

Afhænger X og Y af hinanden?\\
Centrer om middelværdien!
\\
Se slide 27/61.
\\
\\
En stærk korrelationskoefficient ligger oven i hinanden.

\section{Symmetrisk Gaussfordeling}

Når $f(x|y) = f_X(x)$, så er $X$ og $Y$ uafhængige. (slide 34/61).
\\\\
Hvis de ikke er uafhængige (slide 36/61), så ligger data IKKE oven i hinanden.
Altså er $f(x|y) \not = f_X(x)$.
Den betingede graf er smallere end den marginale.
\\
\\
Beregn areal under den sorte kurve (slide 37/61): $f(x|y=0) = \dfrac{f(x,y=0)}{f_Y(y=0)}$ og $f(y|x=0.64) = \dfrac{f(x=0.64, y)}{f_X(x=0.64)}$






\section{Hvor er robotten}
Slide. 40-49/61
\\
\\
Vi ved ikke hvor robotten er til start -- den marginale tæthed $f_X(x)$ er derfor uniform.
Dette er \textit{a priori}, før måling, viden om positionen.
\\
\\
Målestøjen er normalfordelt: $ N \sim \mathbb{N}(0.5)$.
Så må $f(y|x) = f(x+n|x) = f_N(n) = f_N(y-x)$.
\\
\\
Dette er også "Kalmers".


\section{Sum af stokastiske variabler}
Slide 50-54.
\\
\\
Hvor mange kombinationer af en funktion kan der forekomme. Summer disse i en graf.
Kaldes en foldning eller \texttt{conv()} i Matlab, $ f_z = f_x \otimes f_y$









\end{document}