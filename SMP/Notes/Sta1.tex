\documentclass[Main]{subfiles}

\begin{document}

\chapter{Hypotese test}

\begin{theo}[Hypotesetest]
Notation: $H: \mu = \ldots $ \\
$ z = \dfrac{x - \mu}{ \frac{\sigma}{\sqrt{n}}}  $: $x$ er gennemsnittet.\\
$ z \sim N(0,1)$: Standard normalfordeling.

\begin{lstlisting}[style=Code-Matlab]
%Sand middelverdi og std. afvigelse
mu = 24;
sigma = 3;
% Samplede data
n = 40;
x = randn(1,n)*sigma + mu;
% Gennemsnit
xhat = mean(x);
% TeststQrrelse
mu_hyp = 50;
z = (xhat-mu_hyp)/(sigma/sqrt(n))
\end{lstlisting}


\begin{align*}
Pr ( Z \leq |z| \cup Z > |z|) & \\
	&= Pr(Z) \leq -55.4453) + Pr(Z > 55.4453) \\
	&= \omega(-55.4453) + \Sigma(1 - \Sigma(55.4453)) \\
	&= (1 - \Sigma(55.4453)) + (1 - \Sigma(55.4453)) \\
	&= 2(1 - \Sigma(55.4453)) \\
	&= 2(1-1) \\
	&= 0
\end{align*}

\begin{lstlisting}[style=Code-Matlab]
normcdf(55.4453) = 1
\end{lstlisting}
\end{theo}

Til at bergene udfald af 2 muligheder bruges Bernoullifordelingen.
\begin{theo}[Bernoullifordelingen]
To udfald: $B = \{0,1\}$.
\\
Sandsynligheder:
\begin{align*}
Pr(B=1) &= p & \text{succes} \\
Pr(B=0) &= 1-p & \text{failure}
\end{align*}

Generel notation: $B_i\sim bernoulli(p)$
\\
\\
Antallet af successer: 
\begin{align*}
X &= \sum_{i=1}^n B_i
\end{align*}
Antalsværdien kan sendes med: $X \sim binominal(n,p)$

\end{theo}

\begin{theo}[Binomialfordeling]
Middelværdi:
\begin{align*}
E[X] &= \sum_{Z \in \{ 0,1\}}Z \cdot P(X=z) \\
	&= 0 \cdot Pr(X=0)+1 \cdot Pr(X=1) \\
	&= p \\
	(&= n \cdot p)
\end{align*}


\begin{align*}
Var(X) &=\sum_{Z \in \{ 0,1\}}(z-p)^2 \cdot P(X=z) \\
	&= (0-p)^2 \cdot Pr(X=0) + (1-p)^2 \cdot Pr(X=1) \\
	&= p(1-p) \\
	(&= n \cdot p(1-p))
\end{align*}

Dette ligner en Gausfordeling, hvis $n \cdot p > 5$ og $n \cdot(1-p) > 5$.  \\
Så vil $Pr(X \leq k) = F_{binomial}(k) \approx \phi\left( \dfrac{k-np}{\sqrt{np(1-p)}}\right)$

Approsimation kan gøres ved følgende:
\begin{lstlisting}[style=Code-Matlab]
%% Approksimativ P-vErdi
n = 580;
p = 1/4;
x = 152; % Observation
lower = n*p-abs(n*p-x)
upper = n*p+abs(n*p-x)
z_lower = (lower-n*p)/sqrt(n*p*(1-p))
z_upper = (upper-n*p)/sqrt(n*p*(1-p))
pval = normcdf(z_lower) + (1 - normcdf(z_upper))
\end{lstlisting}


\end{theo}

\end{document}