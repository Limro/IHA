\documentclass[Main]{subfiles}

\begin{document}

\chapter{Hypotese test}

\begin{theo}[Hypotesetest]
Notation: $H: \mu = \ldots $ \\
$ z = \dfrac{x - \mu}{ \frac{\sigma}{\sqrt{n}}}  $: $x$ er gennemsnittet.\\
$ z \sim N(0,1)$: Standard normalfordeling.

\begin{lstlisting}[style=Code-Matlab]
%Sand middelverdi og std. afvigelse
mu = 24;
sigma = 3;
% Samplede data
n = 40;
x = randn(1,n)*sigma + mu;
% Gennemsnit
xhat = mean(x);
% TeststQrrelse
mu_hyp = 50;
z = (xhat-mu_hyp)/(sigma/sqrt(n))
\end{lstlisting}


\begin{align*}
Pr ( Z \leq |z| \cup Z > |z|) & \\
	&= Pr(Z) \leq -55.4453) + Pr(Z > 55.4453) \\
	&= \omega(-55.4453) + \Sigma(1 - \Sigma(55.4453)) \\
	&= (1 - \Sigma(55.4453)) + (1 - \Sigma(55.4453)) \\
	&= 2(1 - \Sigma(55.4453)) \\
	&= 2(1-1) \\
	&= 0
\end{align*}

\begin{lstlisting}[style=Code-Matlab]
normcdf(55.4453) = 1
\end{lstlisting}
\end{theo}

Til at bergene udfald af 2 muligheder bruges Bernoullifordelingen.







\newpage
\begin{theo}[Bernoullifordelingen]
To udfald: $B = \{0,1\}$.



Sandsynligheder:
\begin{align*}
Pr(B=1) &= p & \text{succes} \\
Pr(B=0) &= 1-p & \text{failure}
\end{align*}

Generel notation: $B_i\sim bernoulli(p)$
\\
\\
Antallet af successer: 
\begin{align*}
X &= \sum_{i=1}^n B_i
\end{align*}
Antalsværdien kan sendes med: $X \sim binominal(n,p)$

\end{theo}








\begin{theo}[Binomialfordeling]
Middelværdi:
\begin{align*}
E[X] &= \sum_{Z \in \{ 0,1\}}Z \cdot P(X=z) \\
	&= 0 \cdot Pr(X=0)+1 \cdot Pr(X=1) \\
	&= p \\
	(&= n \cdot p)
\end{align*}


\begin{align*}
Var(X) &=\sum_{Z \in \{ 0,1\}}(z-p)^2 \cdot P(X=z) \\
	&= (0-p)^2 \cdot Pr(X=0) + (1-p)^2 \cdot Pr(X=1) \\
	&= p(1-p) \\
	(&= n \cdot p(1-p))
\end{align*}

Dette ligner en Gausfordeling, hvis $n \cdot p > 5$ og $n \cdot(1-p) > 5$. 
\\
Standardisering af binomialfordelte data: $z = \dfrac{x-np}{\sqrt{np(1-p)}} \sim N(0,1)$
\\
Så vil $Pr(X \leq k) = F_{binomial}(k) \approx \phi\left( \dfrac{k-np}{\sqrt{np(1-p)}}\right)$

Approsimation kan gøres ved følgende:
\begin{lstlisting}[style=Code-Matlab]
%% Approksimativ P-vErdi
n = 580;
p = 1/4;
x = 152; % Observation
lower = n*p-abs(n*p-x)
upper = n*p+abs(n*p-x)
z_lower = (lower-n*p)/sqrt(n*p*(1-p))
z_upper = (upper-n*p)/sqrt(n*p*(1-p))
pval = normcdf(z_lower) + (1 - normcdf(z_upper))
\end{lstlisting}

\end{theo}








\begin{theo}[Signifikansniveau]
Der skal vælges en "p-værdi" for hypotesetest.
Det betegnes med $\alpha$ og kaldes signifikansniveau.

\begin{itemize}
	\item p-værdi større end $\alpha \Rightarrow$ data strider ikke imod hypotesen.
	\item Typoiske værdier for $\alpha$ er 0.05 eller 0.01
\end{itemize}

\end{theo}









\begin{theo}[Begreber]
\begin{itemize}
	\item Teststørrelse
	\subitem Teststørrelse er en funktion af data, $W = W(x)$.
	\subitem Teststørrelsen er en stokastisk variabel!!!
	\subitem Teststørrelsen bruges til at bestemme graden af overensstemmelse mellem data og hypotese.
	\item Statistisk model
	\subitem Beskriver teststørrelsens sandsynlighedsfordeling.
	\subitem Fordelingen afhænger af en eller flere parametre.
\end{itemize}

Parameteren for en statistisk model er $p$ i $x\sim binomial(580,p)$ som eksempel.

\begin{itemize}
	\item Parameter
\subitem Vi antager, at de samplede data kommer fra en population, hvor den sande værdi af parameteren i den statistiske model er ukendt.
\subitem Hvis vi gentager eksperimentet under identiske betingelser, vil den sande parameter være uændret, selvom vi får andre data.
\item Parameter-skøn eller estimat
\subitem Et skøn af parameteren beregnes på baggrund af de observerede data.
\end{itemize}

\end{theo}






\begin{theo}[Notation]
\begin{itemize}
	\item  Generelt betegner vi den sande parameter $\theta$.
\subitem Bemærk, $\theta$ kan være en vektor.
\item Parameter estimatet betegnes $\hat{\theta} = \hat{\theta}(x)$
\subitem og er altså en funktion af data (x).
\item Estimatet er en stokastisk variabel!!!
\end{itemize}
\end{theo}





\newpage

\begin{theo}[Estimation]
Givet data, hvad kan man sige om parameteren $p$?
\begin{itemize}
	\item Estimat: $\hat{p} = \hat{p}(x) = \dfrac{x}{n}$
	\item Usikkerhed: konfidensinterval
	\item Unbiased: $E[\hat{p}] = E\left[\dfrac{x}{n}\right]=\dfrac{1}{n} E[x] = \dfrac{1}{n}(n \cdot p) = p$
	\item Varians: $Var(\hat{p}) = Var\left(\dfrac{x}{n}\right) = \dfrac{1}{n^2} Var(x)=\frac{1}{n^2}(n \cdot p \cdot (1-p))= \dfrac{1}{n}p \cdot(1-p)$
\end{itemize}
\end{theo}







\begin{theo}[Maximum likelihood]
Tæthedsfunktionen for de observerede data, giver $\hat{p}$ er $f(x|\hat{p}) = \left(\begin{matrix}
	n\\
	x
	\end{matrix}\right) \hat{p}^x (1-\hat{p})^{n-x}$

Dette kan også skrives som:
$x \cdot log(\hat{p}) + (n-x) \cdot log(1-\hat{p})$.
\\
For at finde det maximale punkt differentieres det og sættes lige med 0. 
Herefter isoleres $\hat{p}$ og giver:
$\hat{p} = \dfrac{x}{n}$
\end{theo}



\begin{theo}[Binomialfordeling i Matlab]
\begin{itemize}
	\item Tæthedsfunktion:
	\item[] \begin{lstlisting}[style=Code-Matlab]
	Pr(X = x) = binopdf(x,n,p)
	Pr(X \leq x) = binocdf(x,n,p)
	\end{lstlisting}
\end{itemize}
\end{theo}



\begin{theo}[95\% konfidensinterval]

\begin{itemize}
	\item Definition:
\subitem Sandsynligheden for, at 95\% konfidensintervallet indeholder den sande parameterværdi, skal være 0.95:
$Pr(\theta \text{er indeholdt i intervallet} [\theta_- ; \theta_+]) = 0.95$
\item Bogens overordnede strategi
\subitem Vi ser kun på fordelinger, som har parameter $\theta$, og som kan approksimeres med en normalfordeling.
\subitem Beregn den standardiserede teststørrelse: $z= \dfrac{x-\mu}{\sigma} \sim N(0,1)$
Så gælder der, at $Pr(-1.96 \leq z \leq 1.96) = 0.95$
\subitem Brug dette til at beregne 95\% konfidensintervallet, $[\theta_- ; \theta_+]$
\end{itemize}

Antag binomialfordelte data kan bruges.
Nedre og øvre grænse:
\begin{align*}
p_-(x) &= \dfrac{1}{n+u^2} \left[x+\dfrac{u^2}{2}\pm u \sqrt{\dfrac{x(n-x)}{n} + \dfrac{u^2}{4}}\right]\\
u &= \Phi^{-1}(1- \dfrac{\alpha}{2})
\end{align*}
\begin{lstlisting}[style=Code-Matlab]
u = norminv(1-0.05/2) = 1.96
\end{lstlisting}

Der er flere værdier der har $\alpha > 0.05$.
Disse kaldes kondifensintervallet og skrives som $(1-\alpha) \cdot 100\% $

\end{theo}


\begin{theo}[Poissonfedeling]
Bruges til en beskrive en proces med forskellige ankomsttider.
\begin{itemize}
	\item Opdel tidsakse i $N$ intervaller af længden $\Delta t$
	
	\item Sandsynlighed for overservering af $X = x$ i intervallet $[1;N]: X = \sum_{i=1}^n B_i \sim binomial(N, \lambda \cdot \Delta t)$.
	
	\item[] $Pr(X=x) = \binom{N}{k}(\lambda \cdot \Delta t)^x(1-\lambda \cdot \Delta t)^{N-x}$
	
	\item Observation:
	
	\item[] $N \cdot (\lambda \cdot \Delta t) = konstant = \dfrac{1}{\Delta t} \cdot (\lambda \cdot \Delta t) = t \cdot \lambda = \gamma$
	
	\item[] $Pr(X=x) \dfrac{(\lambda \cdot t)^x}{x!} e^{-\lambda \cdot t} = \dfrac{y^x}{x!}e^{-\gamma} $

	\item Hvis $\gamma ? \lambda \cdot t > 5$ er $X$ cirka normalfordelt.

	\item Approksimativt 95\% konfidensinterval:
	\item[]$[\lambda_-(x);\lambda_+(x)] = \left[\dfrac{1}{t}\left[x+\dfrac{u^2}{2}-u \sqrt{x+\dfrac{u^2}{4}}\right] ; \dfrac{1}{t}\left[x+\dfrac{u^2}{2}+u \sqrt{x+\dfrac{u^2}{4}}\right] \right]$
\end{itemize}
\begin{lstlisting}[style=Code-Matlab]
%TEthedsfunktion:
Pr(X=x) = poisspdf(x,gamma)

%Fordelingsfunktion:
Pr(X=x) = poisscdf(x,gamma)
\end{lstlisting}
\end{theo}

\begin{theo}[Estimat af $\sigma^2$]
$s^2 = \dfrac{1}{n-1} \sum_{i=1}^n (x_i - \bar{x})^2$
\end{theo}
Dette giver $E[s^2] = \sigma^2$


\end{document}