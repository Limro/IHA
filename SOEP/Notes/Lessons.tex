\documentclass[a4, 10pt]{article}

%Preamble

% Følgende er til koder.
%----------------------------------------------------------
%\begin{lstlisting}[caption=Overskrift på boks, style=Code-C++, label=lst:referenceLabel]
%public void hello(){}
%\end{lstlisting}
%----------------------------------------------------------

%Exstra space
\usepackage{xspace}
%Navn på bokse efterfulgt af \xspace (hvis det skal være mellemrum
%gives det med denne udvidelse. Ellers ingen mellemrum.
\newcommand{\codeTitle}{Code snippet\xspace}

%Pakker der skal bruges til lstlisting
\usepackage{listings}
\usepackage{color}
\usepackage{textcomp}
\definecolor{listinggray}{gray}{0.9}
\definecolor{lbcolor}{rgb}{0.9,0.9,0.9}
\renewcommand{\lstlistingname}{\codeTitle}
\lstdefinestyle{Code}
{
	keywordstyle	= \bfseries\ttfamily\color[rgb]{0,0,1},
	identifierstyle	= \ttfamily,
	commentstyle	= \color[rgb]{0.133,0.545,0.133},
	stringstyle		= \ttfamily\color[rgb]{0.627,0.126,0.941},
	showstringspaces= false,
	basicstyle		= \small,
	numberstyle		= \footnotesize,
%	numbers			= left, % Tal? Udkommenter hvis ikke
	stepnumber		= 2,
	numbersep		= 6pt,
	tabsize			= 2,
	breaklines		= true,
	prebreak 		= \raisebox{0ex}[0ex][0ex]{\ensuremath{\hookleftarrow}},
	breakatwhitespace= false,
%	aboveskip		= {1.5\baselineskip},
  	columns			= fixed,
  	upquote			= true,
  	extendedchars	= true,
 	backgroundcolor = \color{lbcolor},
	lineskip		= 1pt,
%	xleftmargin		= 17pt,
%	framexleftmargin= 17pt,
	framexrightmargin	= 0pt, %6pt
%	framexbottommargin	= 4pt,
}

%Bredde der bruges til indryk
%Den skal være 6 pt mindre
\usepackage{calc}
\newlength{\mywidth}
\setlength{\mywidth}{1.435\textwidth} % Hvis bredden header ikke virker er dette hvad skal ændres!


% Forskellige styles for forskellige kodetyper
\usepackage{caption}
\DeclareCaptionFont{white}{\color{white}}
\DeclareCaptionFormat{listing}%
{\colorbox[cmyk]{0.43, 0.35, 0.35,0.35}{\parbox{\mywidth}{\hspace{5pt}#1#2#3}}}
\captionsetup[lstlisting]
{
	format			= listing,
	labelfont		= white,
	textfont		= white, 
	singlelinecheck	= false, 
	width			= \mywidth,
	margin			= 0pt, 
	font			= {bf,footnotesize}
}

\lstdefinestyle{Code-C} {language=C, style=Code}
\lstdefinestyle{Code-Java} {language=Java, style=Code}
\lstdefinestyle{Code-C++} {language=[Visual]C++, style=Code}
\lstdefinestyle{Code-VHDL} {language=VHDL, style=Code}
\lstdefinestyle{Code-Bash} {language=Bash, style=Code}
\lstdefinestyle{Code-Matlab} {language=Matlab, style=Code}
\lstdefinestyle{Code-Prolog} {language=Prolog, style=Code}
%Speciel skrift for enkelt linje kode
%--------------------------------------------------
%Udskriver med fonten 'Courier'
%Mere info her: http://tex.stackexchange.com/questions/25249/how-do-i-use-a-particular-font-for-a-small-section-of-text-in-my-document
%Eksempel: Funktionen \code{void Hello()} giver et output
%--------------------------------------------------
\newcommand{\code}[1]{{\fontfamily{pcr}\selectfont #1}}

%Seperated files
%--------------------------------------------------
%Opret filer således:
%\documentclass[Navn-på-hovedfil]{subfiles}
%\begin{document}
% Indmad
%\end{document}
%
% I hovedfil inkluderes således:
% \subfile{navn-på-subfil}
%--------------------------------------------------
\usepackage{subfiles}
%Text typesetting
%--------------------------------------------------------
\usepackage[T1]{fontenc} 	% Can use danish characters
\usepackage[utf8]{inputenc} % Input encoding. Can be used on Linux, Mac and Windows         
\usepackage[danish]{babel} 	% Split words accoding to English
\usepackage{lmodern} 		% Font

\setlength\parindent{0pt} 	% No indent
\setlength\parskip{12pt} 	% More than a single line break will give ONE linebreak.

%Margin
\usepackage[left=2cm,right=2cm,top=2.5cm,bottom=2cm]{geometry}

%Margin
\usepackage[left=3cm,right=2cm,top=2.5cm,bottom=2cm]{geometry}

%Mellemrum mellem linjerne    
\linespread{1.5}

\title{Assignment 2 for TEDI}
\author{Rasmus Bækgaard, 10893}
\date{April 28, 2014}
\usepackage[left=3cm,right=2cm,top=2.5cm,bottom=2cm]{geometry}

\title{SOEP}
\author{Rasmus Bækgaard, 10893}

\begin{document}

\maketitle

\section*{Lesson 3 -- Risk- and project management}

Slide 9
\begin{itemize}
\item Mange gange skal man lave noget nyt, som man ikke har lavet før.
\item Mange gange vil en kunde have noget, men ingen ved hvad det er. Det skal man finde ud af.
\item Mange gange er man restrikteret af tid. 
	Dette kan gøres nemmere med incremental planning.
\item Man har altid begrænsninger i ressourcer -- folk der ikke kan arbejde for dig, folk der er syge \dots
\end{itemize} 
Slide 10
\begin{itemize}
	\item I modsætning til brobyggeri kan man ikke se, hvor langt man er med software.
	\item I en gruppe skal alle vide hvad de skal gøre -- ellers laver man det samme og slet ikke noget andet.
	\item Hvis man opdager noget der kunne gøre projektet bedre, skal det \textbf{ikke} med, da det vil gøre projektet større.
\end{itemize}
Slide 12
\begin{itemize}
	\item Vær sikker på, at nogen kan bruge dit projekt. Profit / socialt \dots
	\item Er det værd at starte på projektet?
	\item Hvor meget tid vil tage, har du udstyret, hvad skal bruges til det.
	\item Do it!
\end{itemize}
Slide 14
\begin{itemize}
	\item Hvorfor lave projektet og lav en beskrivelse af det.
	\item Hvad koster det er lave, hvad får man igen -- skal man så have det?
	\item Behold alt dataen og se nogle år efter, om det nu kan betale sig.
\end{itemize}
Slide 15-16
\begin{itemize}
	\item Forsøg at nedskrive de fordele og ulemper man kan få ved softwaren -- disse er normalt penge værd.
	\item Forklar hvordan løsningen kan løse problemet
	\begin{itemize}
		\item Husk ikke at ignorere problemet
		\item Løsningen er et hjælpemiddel
	\end{itemize}
	\item Problemmet kaldes gerne "the need(s)".
\end{itemize}
Slide 17
\begin{itemize}
	\item Selve softwaren fylder meget lidt af løsningen.
	\item Store dele af dette er træning af brugerne, udstyr, on-site sikkerhedsprocedure
	\item Når dette er gjort kan der laves aktiviteter, hyres folk, hvornår det kan være færdigt, hvordan det skal laves, hvad det skal laves med \dots
\end{itemize}
Slide 20-21
\begin{itemize}
	\item Bryd det op og aflever små ting med kort mellemrum.
	\item Hvad er risks? Disse er ikke altid de samme.
	\item Hvem er ansvarlig for hvad?
	\item Opfind ikke standarder, men følg dem i stedet.
\end{itemize}
Slide 22
\begin{itemize}
	\item Project risks er skemalægning eller resourcer
	\item Product risks er kvalitet og hvor godt systemet der kører. Det kan være subcontractors' skyld.
	\item Business risks er at organisationen ændrer sig
\end{itemize}
Slide 23-23(2)
\begin{itemize}
	\item Identificer risks.
	\item Planlæg hvad der skal gøres, hvis en risk proc'er.
	\item Techonology risks: Ny teknologi kommer
	\item People risks: Kan sige op, barsel, syge \dots
	\item Tool risks: Programmer der ikke vil virke længere eller vil dø ved opgradering (så opgrader ikke)
	\item Requirement risks: De ændrer sig og man ikke er klar
	\item Estimation risks: Du har undervurderet hvad der skall bruges -- stort problem ved projekter. Vær MEGET pesimistisk!
	\item Tal er svære at arbejde med, så brug navne om sandsynligheden.
	\item Skriv effekterne af enkelte risks ned (catastrophic, serious, tolerable, insignificant)
\end{itemize}
Slide 23 (3) -- Risk planning
\begin{itemize}
	\item Arbejd i grupper, så man har delt viden og projektet ikke stopper der.
	\item Brug noget du ved dur.
\end{itemize}






\newpage
\section*{Lesson 4}

Slide 34 - 35
 \begin{itemize}
 	\item Alle projekter vil have problemer med planlægning
 	\item Skal der laves et nyt projekt, så ansæt en der har lavet det før.
 	\item Der bruges et mix imellem top-down og bottom-up
 \end{itemize}
 Slide 36 - 37, 39
 \begin{itemize}
 	\item COCOMO 2 fortæller, hvor meget der skal bruges til et projekt
 	\item Det giver en rimelig vurdering
 	\item Dette kan dog være svært på et tidligt stadie i projektet pga. størrelsen er ukendt.
 	\item Kun dine egne tal er realistiske i denne sammenhæng
 	\item Enheden $size$ er i 1000-linjer enhed
 \end{itemize}
 Slide 38
 \begin{itemize}
 	\item 
 \end{itemize}





\end{document}