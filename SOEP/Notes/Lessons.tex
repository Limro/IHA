\documentclass[a4, 10pt]{article}

\usepackage{tcolorbox}
\usepackage{ulem} %math
\usepackage{amsmath}
\usepackage{amsfonts}
\usepackage{amssymb}
\usepackage{graphicx}
\usepackage{enumerate}


%Create a box for theorems
%\begin{theo}[titel] %optional
%tekst
%\end{theo}
\newenvironment{theo}[1][Vigtigt]{%
\begin{tcolorbox}[colback=green!5,colframe=green!40!black,title=\textbf{#1}]
}{%
\end{tcolorbox}
}




%Create a square matrix
%\begin{ArgMat}{2}
%21 & 22 & 23 \\  
%a & b & c
%\end{ArgMat}
%
% Info: http://tex.stackexchange.com/questions/2233/whats-the-best-way-make-an-augmented-coefficient-matrix
%
\newenvironment{ArgMat}{%
$
  \left[\begin{array}{@{}*{100}{r}r@{}}
}{%
  \end{array}\right]
  $
}

\newenvironment{deter}{%
$
  \left|\begin{array}{@{}*{100}{r}r@{}}
}{%
  \end{array}\right|
  $
}


%Create multiple lines with holes
%\begin{SysEqu}
%x_1 && &- &5x_3 &+ &2x_4=& 1 \\
%x_1 &+ &x_2 &+ &x_3 && =& 4 \\
%&&&&&&0 =& 0
%\end{SysEqu}
\newenvironment{SysEqu}{%
$  \setlength\arraycolsep{0.1em}
  \begin{array}{@{}*{100}{r}r@{}}
}{%
  \end{array}$
}

%Create solution for x_1, x_n...
%\begin{solu}
%x_1 &= d \\
%x_2 &= e \\
%x_3 &= s
%\end{solu}
\newenvironment{solu}{%
$
  \setlength\arraycolsep{0.1em}
  \left\{\begin{array}{@{}*{100}{r}r@{}}
}{%
  \end{array}\right.
$
}

\usepackage{lastpage}


\newcommand{\HRule}{\rule{\linewidth}{0.8mm}}

%Tekst i fotter
\newcommand{\footerText}{\thepage\xspace /\pageref{LastPage}}
\newcommand{\ProjectName}{433 MHz styring af AeroQuad}


\chapterstyle{hangnum}




\nouppercaseheads
\makepagestyle{mystyle} 

\makeevenhead{mystyle}{}{\\ \leftmark}{} 
\makeoddhead{mystyle}{}{\\ \leftmark}{} 
\makeevenfoot{mystyle}{}{\footerText}{} 
\makeoddfoot{mystyle}{}{\footerText}{} 
\makeatletter
\makepsmarks{mystyle}{% Overskriften på sidehovedet
  \createmark{chapter}{left}{shownumber}{\@chapapp\ }{.\ }} 
\makeatother
\makefootrule{mystyle}{\textwidth}{\normalrulethickness}{0.4pt}
\makeheadrule{mystyle}{\textwidth}{\normalrulethickness}

\makepagestyle{plain}
\makeevenhead{plain}{}{}{}
\makeoddhead{plain}{}{}{}
\makeevenfoot{plain}{}{\footerText}{}
\makeoddfoot{plain}{}{\footerText}{}
\makefootrule{plain}{\textwidth}{\normalrulethickness}{0.4pt}

\pagestyle{mystyle}

%%----------------------------------------------------------------------
%
%%Redefining chapter style
%%\renewcommand\chapterheadstart{\vspace*{\beforechapskip}}
%\renewcommand\chapterheadstart{\vspace*{10pt}}
%\renewcommand\printchaptername{\chapnamefont }%\@chapapp}
%\renewcommand\chapternamenum{\space}
%\renewcommand\printchapternum{\chapnumfont \thechapter}
%\renewcommand\afterchapternum{\space: }%\par\nobreak\vskip \midchapskip}
%\renewcommand\printchapternonum{}
%\renewcommand\printchaptertitle[1]{\chaptitlefont #1}
\setlength{\beforechapskip}{0pt} 
\setlength{\afterchapskip}{0pt} 
%\setlength{\voffset}{0pt} 
\setlength{\headsep}{25pt}
%\setlength{\topmargin}{35pt}
%%\setlength{\headheight}{102pt}
%\setlength{\textheight}{302pt}
\renewcommand\afterchaptertitle{\par\nobreak\vskip \afterchapskip}
%%----------------------------------------------------------------------




%Sidehoved og -fod pakke
%Margin
\usepackage[left=2cm,right=2cm,top=2.5cm,bottom=2cm]{geometry}
\usepackage{lastpage}



%%URL kommandoer og sidetal farve
%%Kaldes med \url{www...}
%\usepackage{color} %Skal også bruges
\usepackage{hyperref}
\hypersetup{ 
	colorlinks	= true, 	% false: boxed links; true: colored links
    urlcolor	= blue,		% color of external links
    linkcolor	= black, 	% color of page numbers
    citecolor	= blue,
}



%Mellemrum mellem linjerne    
\linespread{1.5}


%Seperated files
%--------------------------------------------------
%Opret filer således:
%\documentclass[Navn-på-hovedfil]{subfiles}
%\begin{document}
% Indmad
%\end{document}
%
% I hovedfil inkluderes således:
% \subfile{navn-på-subfil}
%--------------------------------------------------
\usepackage{subfiles}

%Prevent wierd placement of figures
%\usepackage[section]{placeins}

%Standard sti at søge efter billeder
%--------------------------------------------------
%\begin{figure}[hbtp]
%\centering
%\includegraphics[scale=1]{filnavn-for-png}
%\caption{Titel}
%\label{fig:referenceNavn}
%\end{figure}
%--------------------------------------------------
\usepackage{graphicx}
\usepackage{subcaption}
\usepackage{float}
\graphicspath{{../Figures/}}

%Speciel skrift for enkelt linje kode
%--------------------------------------------------
%Udskriver med fonten 'Courier'
%Mere info her: http://tex.stackexchange.com/questions/25249/how-do-i-use-a-particular-font-for-a-small-section-of-text-in-my-document
%Eksempel: Funktionen \code{void Hello()} giver et output
%--------------------------------------------------
\newcommand{\code}[1]{{\fontfamily{pcr}\selectfont #1}}


% Følgende er til koder.
%----------------------------------------------------------
%\begin{lstlisting}[caption=Overskrift på boks, style=Code-C++, label=lst:referenceLabel]
%public void hello(){}
%\end{lstlisting}
%----------------------------------------------------------

%Exstra space
\usepackage{xspace}
%Navn på bokse efterfulgt af \xspace (hvis det skal være mellemrum
%gives det med denne udvidelse. Ellers ingen mellemrum.
\newcommand{\codeTitle}{Kodeudsnit\xspace}

%Pakker der skal bruges til lstlisting
\usepackage{listings}
\usepackage{color}
\usepackage{textcomp}
\definecolor{listinggray}{gray}{0.9}
\definecolor{lbcolor}{rgb}{0.9,0.9,0.9}
\renewcommand{\lstlistingname}{\codeTitle}
\lstdefinestyle{Code}
{
	keywordstyle	= \bfseries\ttfamily\color[rgb]{0,0,1},
	identifierstyle	= \ttfamily,
	commentstyle	= \color[rgb]{0.133,0.545,0.133},
	stringstyle		= \ttfamily\color[rgb]{0.627,0.126,0.941},
	showstringspaces= false,
	basicstyle		= \small,
	numberstyle		= \footnotesize,
%	numbers			= left, % Tal? Udkommenter hvis ikke
	stepnumber		= 2,
	numbersep		= 6pt,
	tabsize			= 2,
	breaklines		= true,
	prebreak 		= \raisebox{0ex}[0ex][0ex]{\ensuremath{\hookleftarrow}},
	breakatwhitespace= false,
%	aboveskip		= {1.5\baselineskip},
  	columns			= fixed,
  	upquote			= true,
  	extendedchars	= true,
 	backgroundcolor = \color{lbcolor},
	lineskip		= 1pt,
%	xleftmargin		= 17pt,
%	framexleftmargin= 17pt,
	framexrightmargin	= 0pt, %6pt
%	framexbottommargin	= 4pt,
}

%Bredde der bruges til indryk
%Den skal være 6 pt mindre
\usepackage{calc}
\newlength{\mywidth}
\setlength{\mywidth}{\textwidth-6pt}


% Forskellige styles for forskellige kodetyper
\usepackage{caption}
\DeclareCaptionFont{white}{\color{white}}
\DeclareCaptionFormat{listing}%
{\colorbox[cmyk]{0.43, 0.35, 0.35,0.35}{\parbox{\mywidth}{\hspace{5pt}#1#2#3}}}
\captionsetup[lstlisting]
{
	format			= listing,
	labelfont		= white,
	textfont		= white, 
	singlelinecheck	= false, 
	width			= \mywidth,
	margin			= 0pt, 
	font			= {bf,footnotesize}
}

\lstdefinestyle{Code-C} {language=C, style=Code}
\lstdefinestyle{Code-Java} {language=Java, style=Code}
\lstdefinestyle{Code-C++} {language=[Visual]C++, style=Code}
\lstdefinestyle{Code-VHDL} {language=VHDL, style=Code}
\lstdefinestyle{Code-Bash} {language=Bash, style=Code}

%Text typesetting
%--------------------------------------------------------
%\usepackage{baskervald}
\usepackage{lmodern}
\usepackage[T1]{fontenc}              
\usepackage[utf8]{inputenc}         
\usepackage[english]{babel}       

\setlength{\parindent}{0pt}
\nonzeroparskip

%\setaftersubsecskip{1sp}
%\setaftersubsubsecskip{1sp}
 


%Dybde på indholdsfortegnelse
%----------------------------------------------------------
%Chapter, section, subsection, subsubsection
%----------------------------------------------------------
\setcounter{secnumdepth}{3}
\setcounter{tocdepth}{3}


%Tables
%----------------------------------------------------------
\usepackage{tabularx}
\usepackage{array}
\usepackage{multirow} 
\usepackage{multicol} 
\usepackage{booktabs}
\usepackage{wrapfig}
\renewcommand{\arraystretch}{1.5}



%Misc
%----------------------------------------------------------
\usepackage{cite}
\usepackage{appendix}
\usepackage{amssymb}
\usepackage{url,ragged2e}
\usepackage{enumerate}
\usepackage{amsmath} %Math bibliotek


\usepackage{longtable}

\usepackage[left=3cm,right=2cm,top=2.5cm,bottom=2cm]{geometry}

\title{SOEP}
\author{Rasmus Bækgaard, 10893}

\begin{document}

\maketitle

\section*{Lesson 3 -- Risk- and project management}

Slide 9
\begin{itemize}
\item Mange gange skal man lave noget nyt, som man ikke har lavet før.
\item Mange gange vil en kunde have noget, men ingen ved hvad det er. Det skal man finde ud af.
\item Mange gange er man restrikteret af tid. 
	Dette kan gøres nemmere med incremental planning.
\item Man har altid begrænsninger i ressourcer -- folk der ikke kan arbejde for dig, folk der er syge \dots
\end{itemize} 
Slide 10
\begin{itemize}
	\item I modsætning til brobyggeri kan man ikke se, hvor langt man er med software.
	\item I en gruppe skal alle vide hvad de skal gøre -- ellers laver man det samme og slet ikke noget andet.
	\item Hvis man opdager noget der kunne gøre projektet bedre, skal det \textbf{ikke} med, da det vil gøre projektet større.
\end{itemize}
Slide 12
\begin{itemize}
	\item Vær sikker på, at nogen kan bruge dit projekt. Profit / socialt \dots
	\item Er det værd at starte på projektet?
	\item Hvor meget tid vil tage, har du udstyret, hvad skal bruges til det.
	\item Do it!
\end{itemize}
Slide 14
\begin{itemize}
	\item Hvorfor lave projektet og lav en beskrivelse af det.
	\item Hvad koster det er lave, hvad får man igen -- skal man så have det?
	\item Behold alt dataen og se nogle år efter, om det nu kan betale sig.
\end{itemize}
Slide 15-16
\begin{itemize}
	\item Forsøg at nedskrive de fordele og ulemper man kan få ved softwaren -- disse er normalt penge værd.
	\item Forklar hvordan løsningen kan løse problemet
	\begin{itemize}
		\item Husk ikke at ignorere problemet
		\item Løsningen er et hjælpemiddel
	\end{itemize}
	\item Problemmet kaldes gerne "the need(s)".
\end{itemize}
Slide 17
\begin{itemize}
	\item Selve softwaren fylder meget lidt af løsningen.
	\item Store dele af dette er træning af brugerne, udstyr, on-site sikkerhedsprocedure
	\item Når dette er gjort kan der laves aktiviteter, hyres folk, hvornår det kan være færdigt, hvordan det skal laves, hvad det skal laves med \dots
\end{itemize}
Slide 20-21
\begin{itemize}
	\item Bryd det op og aflever små ting med kort mellemrum.
	\item Hvad er risks? Disse er ikke altid de samme.
	\item Hvem er ansvarlig for hvad?
	\item Opfind ikke standarder, men følg dem i stedet.
\end{itemize}
Slide 22
\begin{itemize}
	\item Project risks er skemalægning eller resourcer
	\item Product risks er kvalitet og hvor godt systemet der kører. Det kan være subcontractors' skyld.
	\item Business risks er at organisationen ændrer sig
\end{itemize}
Slide 23-23(2)
\begin{itemize}
	\item Identificer risks.
	\item Planlæg hvad der skal gøres, hvis en risk proc'er.
	\item Techonology risks: Ny teknologi kommer
	\item People risks: Kan sige op, barsel, syge \dots
	\item Tool risks: Programmer der ikke vil virke længere eller vil dø ved opgradering (så opgrader ikke)
	\item Requirement risks: De ændrer sig og man ikke er klar
	\item Estimation risks: Du har undervurderet hvad der skall bruges -- stort problem ved projekter. Vær MEGET pesimistisk!
	\item Tal er svære at arbejde med, så brug navne om sandsynligheden.
	\item Skriv effekterne af enkelte risks ned (catastrophic, serious, tolerable, insignificant)
\end{itemize}
Slide 23 (3) -- Risk planning
\begin{itemize}
	\item Arbejd i grupper, så man har delt viden og projektet ikke stopper der.
	\item Brug noget du ved dur.
\end{itemize}






\newpage
\section*{Lesson 4}

Slide 34 - 35
 \begin{itemize}
 	\item Alle projekter vil have problemer med planlægning
 	\item Skal der laves et nyt projekt, så ansæt en der har lavet det før.
 	\item Der bruges et mix imellem top-down og bottom-up
 \end{itemize}
 Slide 36 - 37, 39
 \begin{itemize}
 	\item COCOMO 2 fortæller, hvor meget der skal bruges til et projekt
 	\item Det giver en rimelig vurdering
 	\item Dette kan dog være svært på et tidligt stadie i projektet pga. størrelsen er ukendt.
 	\item Kun dine egne tal er realistiske i denne sammenhæng
 	\item Enheden $size$ er i 1000-linjer enhed
 \end{itemize}
 Slide 38
 \begin{itemize}
 	\item 
 \end{itemize}




\newpage
\section*{Lesson 6}
\code{Z}-notation bruges til matematik og abstraheringer.

\begin{itemize}
	\item Der bruges invariant (der altid er sandt) og skal altid forblive således.
	\item Det forstås bedst ved at tegne billeder -- selvom det virker simpelt.
	\subitem Man glemmer hurtigt små punkter, der blot er tekst.
	\item Når systems udseende og primære funktionalitet er beskrevet kaldes det "statisk".
	\item Hvert punkt skal forstås MEGET bogstaveligt!
	\item Der kan angives states og operationer -- disse kan angives med tal eller bogstaver (helst bogstaver, der giver lidt mening).
	\item Hvis noget ændrer sig, eks. variabel $x$ skrives $\Delta(x)$
	\item Input: $x?$
	\item Output: $x!$
	\item Andre sprog der ligner:

	\begin{itemize}
		\item VDM
		\item B-Method
		\item Event-B
		\item ASM
		\item TLA+ (PlusCal)
	\end{itemize}
\end{itemize}




\newpage
\section*{Lesson 8}

Der findes 3 lag:
\begin{itemize}
	\item Arkitektur styles (højre level design)
	\item Patterns
	\item Idioms (i++)
\end{itemize}

Spg 3:
\begin{itemize}
	\item File types:
		\begin{itemize}
			\item plain text
			\item xml
			\item HTML
			\item json
		\end{itemize}
	\item editing modes
	\begin{itemize}
		\item Plain
		\item WYSIWYG
		\item outline
		\item print
	\end{itemize}
	\item Deplyoment
	\begin{itemize}
		\item Stand-alone
		\item embedded
		\begin{itemize}
			\item Web-browser
			\begin{itemize}
				\item Firefox
				\item Chrome
			\end{itemize}
		\end{itemize}
	\end{itemize}

	\item Platform
	\begin{itemize}
		\item Linux
		\item Max 
		\item Windows
		\item Solaris
		\item Android
	\end{itemize}
\end{itemize}













\end{document}