\documentclass[Main]{subfiles}

\begin{document}

\section*{Part 1}

Devise a way to verify the logarithm program designed in the lecture that computes $k$ with $k = log(n)$ where $n$ is the input and $k = log(n)$ is defined by $2^k \leq n \leq 2^{k+1}$. 
A possible program for the implementation in Promela is shown in \codeTitle \ref{lst:program}.


\begin{lstlisting}[caption=Program in Prolog, style=Code-Prolog, label=lst:program]
int n = 1; 
int k = 0; 

active proctype log() 
{ 
	int m = n; 
	assert(n > 0);

	do
	:: m>=2 -> 
		m = m/2; 
		k = k+1
	:: else -> 
		break
	od;
	
	assert true;
}
\end{lstlisting}

%Note that the assertion 2^k <= n <= 2^(k+1) cannot be expressed directly in Promela. So you have to design a test process in Promela that could do this. See the use of _nr_pr on slide 21 of Lecture 2. 

To test this we add the program in \codeTitle \ref{lst:MyImp}.
This way, the program creates a local variable, \code{k}, which be incremented until it reached \code{n}.
Another variable (\code{value} -- starting at 1) will be multiplied with 2 for each time \code{k} is incremented.
When the proctype finish it will have counted the same as the proctype \code{log}


\begin{lstlisting}[caption=Test implementation, style=Code-Prolog, label=lst:MyImp]
int n = 1;

active proctype logCheck(){
	int k = 0;
	int value = 1;

	do 
	:: k != n -> 
		value = value * 2;
		k++;
	:: else ->
		break;
	od;
	assert true;
}
\end{lstlisting}



\section*{Part 2}
Analyse and explain the algorithm on slide 35 of Lecture 2, also shown in \codeTitle \ref{lst:SecondP}


\begin{lstlisting}[caption=Program to explain, style=Code-Prolog, numbers = left, label=lst:SecondP]
bool turn, flag[2]; 
byte ncrit; 
active [2] proctype user()  
{ 
	assert _pid == 0 || _pid == 1; 
	again:
	flag[_pid] = 1;
	turn = _pid;
	flag[1-_pid] == 0 || turn == 1 - _pid; 
	ncrit++;
	assert ncrit == 1;
	ncrit--;
	flag[_pid] = 0; 
	goto again 
} 
\end{lstlisting}


\subsection{What does it do?}
The code in \codeTitle \ref{lst:SecondP} will start to processes (l. 3), check that they have been given a process ID, \code{pid}, equal 0 or 2 (l. 5).
It will start a \code{label} called "again" (l. 6), set the flag for each process to 1 (l. 7) and set the current turn to the current process' \code{pid}-number (l. 8).
\\
When it reach line 9 it will wait for the other process to get to the same line.
From here one of them will halt and one continue (l. 9).
The one continuing will raise a flag (l. 10), check the flag has been raised (l. 11), lower the flag (l. 12) and allow the other process to continue.
It will then jump back to the \code{label} "again".




\subsection*{How does it work?}

The  trick is performed in line 9 where each process waits for the other process to reach the same line.
Each process set the turn for it self and wait for the other with \code {turn==1-\_pid} or the other process to finish its use of the variable \code{ncrit} which is obtained in line 13.
\\
\\
When the second process has used ncrit, the flag for its process is lowered which will allow the first process to use ncrit.

\subsection*{How do you verify that?}
To be 100\% sure this is the case and neither uses \code{ncrit} at the same time, an \code{atomic\{\}} can be placed around line 10 to 12.
However, since line 11 asserts \code{ncrit} never is more than 1 no matter what, the two processes can't both increment after each other in line 10.


\end{document}