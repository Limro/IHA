\documentclass[Main]{subfiles}

\begin{document}

\section*{Part 1}
\paragraph{Model a client-server network that has peer-to-peer connections.
What is transmitted on the peer-to-peer connections should not be visible to clients and servers outside that connection.}

This is shown in \codeTitle \ref{lst:p2p}.
First we need to create a channel, \code{c}, which we can transmit through.
Then we create a proctype to send from, \code{Client1}, with its own channel, \code{transmitter}.
This channel is send through the first channel, \code{c}, to the proctype \code{Client2}, wait for it to receive the private channel and send the predefined information, \code{data}.

\begin{lstlisting}[caption=Peer-to-peer network, style=Code-C, label=lst:p2p]
chan c = [1] of { byte }; 

active proctype Client1()
{
	byte data = 5;
	chan transmitter = [0] of { byte }; 
	c ! transmitter;
	transmitter ! data;
}

active proctype Client2()
{
	byte dataReceive = 0;
	chan receiver;

	c ? receiver;
	receiver ? dataReceive;
	
	assert( dataReceive == 5 );
}
\end{lstlisting}
This way we have created a peer-to-peer connection with a private channel.




\section*{Part 2}
\paragraph{Model a buffered channel by means of rendezvous channels.}

This is exactly what was done is the first part.
If anything else was wanted, please specify better or reply to the email send by Lasse Brøsted (10769), who asked what the second part meant.



\end{document}